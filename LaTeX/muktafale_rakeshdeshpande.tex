%%%%%%%%%%%%%%%%%%%%%%%%%%%%%%%%%%%%%%%%%%%%%%%%%%%%%%%%%%%%%%%%%%%%%%%%%%%
\chapter*{राकेश देशपांडे }

%%%%%%%%%%%%%%%%%%%%%%%%%%%%%%%%%%%%%%%%%%%%%%%%%%%%%%%%%%%%%%%%%%%%%%%%%%%
\chapter{अनाहत  ध्यान अभ्यासक्रम}

\section*{दिवस १: ध्यानाची मिथके (मेडिटेशन मिथ्स)}

आपल्या मनात विचार येत राहतात, नेहमीच, ध्यान करताना सुद्धा. या विचारांना रोखण्याचा, थांबवण्याचा किंवा जबरदस्तीने दूर करण्याचा प्रयत्न करू नका. तरीही ध्यानाचे खरे उद्दिष्ट म्हणजे निर्विचार होणे. आपण फक्त एक वातावरण-स्थिती निर्माण करू शकतो जिथे मन आपोआप निर्विचार होऊ शकेल. निर्विचार होणे ही एक प्रक्रिया नाही तर त्याचा परिणाम आहे. आपण फक्त बियाणे पेरू शकतो, पाणी देऊ शकतो, पण वनस्पती आपोआप बाहेर येते.

मनाचे निर्विचार होणे आणि शांत राहणे या दोन वेगळ्या गोष्टी आहेत. निर्विचारता फक्त ध्यानादरम्यान काही काळासाठी असते पण शांतता दिवसभर राहते. निर्विचारता जास्त काळ राहू शकत नाही. मनातली शांती म्हणजे काय? कमी मानसिक गप्पागोष्टी. अनावश्यक विचारांची कमतरता. कमी ओव्हर थिंकिंग (अतिचिंतन). नकारात्मक विचार अतिचिंतनामुळे येतात. कामगिरी बिघडते. शांत मन म्हणजे आंतरिक शांति: कमी अतिचिंतन बहुतेक मानसिक समस्यांचे निराकरण करते.

मार्गदर्शित ध्यान (गाइडेड मेडिटेशन) किंवा संगीत ध्यान हे खरे ध्यान नाही. ध्यानाचा मुख्य उद्देश म्हणजे आतल्या बाजूने पाहणे, तरच ध्यान घडते. बाहेरचे आदेश किंवा ऐकणे ही ध्यानाची प्रक्रिया होऊ शकत नाही. बाह्य जगापासून संपर्क तोडण्याची गरज आहे. खर्‍या ध्यानात आपल्याला अंतर्मुखीपणे हृदयाच्या ठोक्यांवर, श्वासावर, 'अनहत नाद' यांवर लक्ष केंद्रित करावे लागते. मार्गदर्शित ध्यान खरंतर विश्रांती आहे, शांत आणि ताजेपणा अनुभवण्यासाठी. योगनिद्रा म्हणून ही एक विश्रांती आहे. हे 'प्रत्याहार' च्या अंतर्गत येते पण 'धारणा, ध्यान, समाधी' मध्ये नाही.

चालताना, धावताना, वाहन चालवताना ध्यान करणे, या सर्व गोष्टी बकवास आहेत. हे अनुपस्थित-मन (एब्सेंट माइंडेड) असणे आहे. हे 'फ्लो-स्टेट' आहे का? ध्यान फक्त एकाच ठिकाणी स्थिर बसून केले जाते. ध्यानाचे उद्दिष्ट मानसिक पॅटर्न (साइकोलॉजिकल पॅटर्न्स) मिटवणे आहे. निश्चित वेळ आणि जागा असली पाहिजे आणि खाज सुटणे किंवा त्रास होणे यासारखे विचार आले तरी एकाच ठिकाणी निष्क्रिय बसण्याची शक्ती असली पाहिजे. जर तुम्हाला झोप येत असेल तर याचा अर्थ तुम्ही व्यवस्थित झोप घेतली नाही. ७-८ तासांची झोप अत्यावश्यक आहे.

२० मिनिटे ध्यान करा, फक्त सर्व इंद्रियांद्वारे (स्पर्श, चव, दृष्टी, वास, ऐकणे) निरीक्षण करा. हे अगदी हळू हळू करा. उठू नका. डोळे उघडे ठेवा. मोजणी करू नका. दुसरे कोणतेही ध्यान करू नका.

\section*{दिवस २: ध्यान घडत आहे का याचा शोध}

ध्यानाचे उद्दिष्ट म्हणजे शक्य तितके निर्विचार मन मिळवणे. हे कधी घडेल? ती कोणती अवस्था आहे? मुख्य मुद्दा असा आहे की हे घडावे म्हणून मनावर काम करू नये. पंखा थांबवण्यासाठी तुम्ही पंख्याचे पंख थांबवता का? नाही, तुम्ही स्विच चालू/बंद करता. तर मला मन बंद करण्यासाठी एक बटण/स्विच शोधावे लागेल. जर आपल्याला अभौतिक (सूक्ष्म) गोष्टींचे परिणाम हवे असतील तर आपण त्यावर थेट प्रवेश मिळवू शकत नाही, आपल्याला फक्त भौतिक (स्थूल) गोष्टींवरच काम करावे लागते. मूर्तिपूजा असेच आहे. खर्‍या देवतेची (अभौतिक) पूजा करण्यासाठी तुम्ही त्याच्या प्रतिनिधी मूर्तीची (भौतिक) पूजा करता.

विचार अभौतिक आहेत आणि त्यांचे भौतिक रूप म्हणजे श्वास. मन निर्विचार करण्यासाठी मनावर काम करू नका तर श्वासावर काम करा.

प्रयोग:
\begin{itemize}
 \item सरळ बसा
 \item मनात ५ वेळा ओंकार
 \item श्वासाला आवश्यक ओंकार लांबी देऊन शांतता येते
\end{itemize}

श्वास हळू करण्याच्या पद्धतींना ध्यान म्हणतात. एवढेच.

निर्विचार होण्यासाठी तुम्हाला श्वास घेणे थांबवावे लागेल का? नाही. आपण श्वासाचा वेग कमी करू शकतो. तेही कठीण आहे. तर दुसरा अप्रत्यक्ष मार्ग असा आहे की, जर आपण शरीर स्थिर केले तर श्वास हळू होतो. तर आपल्याला आपले शरीर स्थिर करावे लागेल.

निर्विचारता $\rightarrow$ हळू-श्वास $\rightarrow$ स्थिर-शरीर. तर उलटे करून बघा.

ध्यानादरम्यान एका क्षणासाठी श्वास थांबतो, मन निर्विचार होते, ती 'शून्यावस्था'. मग आपल्याला त्या क्षणाची जाणीव होते, तोही एक विचार... मग आपण परत विचारांकडे येतो. ध्यानाचे उद्दिष्ट 'शून्यावस्था' मिळवणे आहे. हा क्षण ५ सेकंदांपर्यंत वाढू शकतो, नंतर अधिक... अभ्यास आणि तज्ञतेवर आधारित... श्वास खरोखरच थांबतो. आत्मसाक्षात्कारासाठी हे आवश्यक आहे, तिथे निर्विचार असताना जागरूकता असते (विरोधाभासी, निर्विचारता आणि जागरूकता एकत्र राहू शकत नाहीत, बरोबर ना, त्यामुळे व्यक्त करता येत नाही). साधारण मानवांसाठी 'शून्यावस्थे'चा थोडा काळ मानसिक शांतीसाठी पुरेसा आहे.

'शून्यावस्था' झोपेसारखी आहे. त्यांचा श्वास चालू असतो पण आपल्याला त्याची जाणीव नसते. पण तिथे, झोपेत, 'मी' अजूनही अस्तित्वात आहे. तर ते खरोखर ध्यान नाही, पण अगदी जवळचे आहे. झोप आपोआप येते, पण ध्यान आपण चालवतो आणि मग घडू देतो. तर आपण ध्यानाकडे परत स्पर्श करत नाही. ध्यानात तुम्हाला शरीराचे नियंत्रण असावे लागते, दात स्पर्श करणे. झोपेत शरीराचे नियंत्रण नसते. हे सर्व हस्तमुद्रा फक्त शरीराच्या नियंत्रणासाठी आहेत, इतर कारणे फारशी नाहीत. स्थिरं सुखं आसनम्.

आणखी एक समान अवस्था म्हणजे स्वप्न पाहत असताना तुम्हाला जाणीव असते की तुम्ही स्वप्नात आहात!

तुम्ही फक्त शरीर स्थिर करू शकता, मग ध्यान घडते. सर्व काही नैसर्गिकपणे, प्रत्येकासाठी. सार्वत्रिक नियम.

योगाचे दोन व्यापक दृष्टिकोन आहेत:
\begin{itemize}
 \item अष्टांग योग, पतंजली यांचा, मुख्यतः मानसिक-आध्यात्मिक उन्नतीशी संबंधित आहे. आसनांबद्दल फक्त थोडक्यात बोलतो. मुख्य उद्दिष्ट शरीर स्थिर करणे आहे जेणेकरून नंतरच्या टप्प्यांत ध्यान घडेल. स्थिरम् सुखम् आसनम्
 \item  हठयोग, स्वात्मराम यांचा १४व्या शतकातील, मुख्यतः शारीरिक, आसने, मुद्रा, बंध इत्यादी
\end{itemize}

व्यायाम: डोळे उघडे, सरळ बसा, पाठीला आधार नको. २० मिनिटे स्थिर. खाज सुटू नका. फक्त पापण्या हलवणे आणि लाळ गिळणे मान्य.

\section*{दिवस ३: ध्यानातील अडथळे}

ध्यान: स्थिर राहा $\rightarrow$ श्वास हळू होतो $\rightarrow$ निर्विचार होता, एवढेच.

विचार तरी येतच. निर्विचार होण्यास वेळ लागतो, कधीकधी. शरीर बांधण्यासारखे, वेळ लागतो, परिणाम नंतर येतात. पहिल्या दिवसापासूनच काही फायदे मिळू लागतात जेव्हा आपण ध्यान सुरू करतो. फायदे निर्विचार होण्याव्यतिरिक्त इतर सुद्धा होऊ शकतात, जसे विश्रांती, लक्ष केंद्रित होणे इत्यादी.

अनुभव पुनरावृत्त होत नाही. तुम्हाला प्रत्येक वेळी ध्यानाचा अनुभव मिळत नाही. हे तुमच्या मानसिक स्थितीवर अवलंबून असते, जी दररोज बदलते. दररोज वेगवेगळ्या गोष्टी घडतात, त्यामुळे ध्यानाचा अनुभव बदलतो. ध्यान बसण्याव्यतिरिक्त बाकीचा वेळ सुद्धा ध्यान आहे. ही एक जीवनशैली आहे.

इतरांसारखा अनुभव अपेक्षा करणे, जसे हवेत तरंगणे! असे नाही की तुमचे ध्यान काम करत नाही. सर्व लोक वेगळे आहेत. तुमच्या मानसिक अवस्था वेगळ्या आहेत. त्यामुळे ध्यानाचे अनुभव वेगळे असतात. ध्यानाच्या अनुभवाची कोणती विशिष्ट पाककृती नाही.

ध्यान कंटाळवाणे वाटू लागते. ते टाळण्याचा प्रयत्न करा. तुम्हाला ते लवकर संपवायचे वाटते. जर तुम्हाला कोणताही आध्यात्मिक अभ्यास (स्पिरिच्युअल प्रॅक्टिस) ३ महिने केल्यानंतरही तसेच वाटत असेल आणि अजूनही कंटाळा येत असेल, तर ती साधना तुमच्यासाठी नाही.

ध्यानाची तंत्रे काही वर्षांनी कंटाळवाणी वाटू शकतात, पण तुम्हाला तीच तंत्रे चालू ठेवावी लागतील, कोणताही बदल करू नये. तंत्रे वारंवार बदलू नका. तीच वेळ पाळली पाहिजे, दिवसातून दोनदा. दृढ इच्छाशक्तीने केले पाहिजे.

दिवस ३ चा व्यायाम:
एक दृश्य असेल. त्यातून ध्यानासाठी काय वापरता येईल हे शोधा. दृश्य: भारत-पाकिस्तान क्रिकेट सामना. शेवटच्या चेंडूवर भारताला ४ धावांची गरज आहे. संपूर्ण स्टेडियम चिंतेत आहे, श्वास रोखून धरला आहे. अचानक दिवे निघून जातात. काय झाले ते तुम्हाला कळत नाही. बस्स. यातून ध्यानासाठी काय वापरता येईल?
उत्तर: जेव्हा आपण एखाद्या गोष्टीवर लक्ष केंद्रित करतो तेव्हा श्वास रोखला जातो. तर ध्यानात एखाद्या गोष्टीवर लक्ष केंद्रित करा, मग श्वास हळू होतो (स्थिर बसण्याव्यतिरिक्त). डोळे बंद करून २० मिनिटे निश्चल स्थिरपणे बसा, श्वास, हृदयाची धडधड, अनहत नाद यासारख्या अंतर्गत घटकांवर लक्ष केंद्रित करा. साक्षीभाव.


\section*{दिवस ४: ध्यान मदत करत आहे का?}
जर ध्यान काम करत असेल तर तुम्हाला हे फायदे दिसतील:
\begin{itemize}
 \item आंतरिक स्थिती बाह्य स्थितीमुळे प्रभावित होऊ नये. कोणीतरी तुम्हाला रागावू शकते पण सर्वांना नाही. प्रभावित होणे की नाही ही तुमची निवड आहे. आपण बाह्य वातावरणाला आंतरिक अवस्थेशी जोडतो. सकारात्मक किंवा नकारात्मक, बाहेरच्या गोष्टी तुम्हाला कसे वाटते हे ठरवत नाहीत, ती तुमची निवड आहे.
 \item स्वीकार (एक्सेप्टन्स). घे किंवा सोड, त्यात बदल करण्याचा प्रयत्न करू नका. काय तुमच्या नियंत्रणात आहे आणि काय नाही हे ठरवा. परिस्थिती सहन करणे नाही तर परिस्थिती बदलवता येईल की नाही हे ठरवणे.
 \item  अनासक्ती (डिटॅचमेंट). जास्त आसक्ती हानीकारक आहे. जास्त अवलंबन. व्यक्तीला स्थिरता मिळते आणि स्पष्टपणे विचार करण्यासाठी थोडी अनासक्ती मिळते. वस्तू, ध्येय इत्यादींची आसक्ती सुद्धा वाईट आहे. जास्त नुकसान टाळण्यासाठी 'स्टॉप लॉस' ठेवा, तिथे तुम्हाला ते सोडावे लागेल.
 \item  दिवसाचा बहुतेक वेळ आपण वर्तमानात राहतो. भूतकाळ आणि भविष्याच्या चिंतेत न राहता सध्याच्या क्षणात जगण्याची क्षमता वाढते.
 \item  आपल्यात अधिक सकारात्मकता येते. आकर्षणाचा नियम (लॉ ऑफ अॅट्रॅक्शन). सकारात्मक विचार (पॉझिटिव्ह थिंकिंग). प्रतिज्ञा (अफर्मेशन्स). दिवसात आपल्याला हजारो विचार येतात. आपल्याला शेवटी फक्त २०% मिळतात. ध्यानानंतर ते २०% सकारात्मक असले तरी उर्वरित ८०% मध्ये सुद्धा ते सकारात्मक होऊ लागतात. गोष्टी प्रत्यक्षात येऊ लागतात (मॅनिफेस्ट होतात). संपूर्ण अस्तित्व आपोआप सकारात्मक कंपने देते. प्रत्येक गोष्टीत शंका घेऊ नका. कोणत्याही शक्यतेसाठी मन मोकळे ठेवा.
 \item  इतरही बदलतात. खरंतर ते बदलून जातात. ते निघून जातात. तुमच्या आयुष्यात सकारात्मक लोक येतात. तुम्हाला तुमच्यासारखे लोक आजूबाजूला मिळतात.
 \item  भावना नियंत्रणात येतात. रागावणे, दुःखी होणे, घाबरणे यासारख्या भावना तुमच्या नियंत्रणात येतात आणि तुम्ही त्यांना योग्य दिशा देऊ शकता.
\end{itemize}
