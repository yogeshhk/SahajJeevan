%%%%%%%%%%%%%%%%%%%%%%%%%%%%%%%%%%%%%%%%%%%%%%%%%%%%%%%%%%%%%%%%%%%%%%%%%%%
\chapter*{राकेश देशपांडे}

%%%%%%%%%%%%%%%%%%%%%%%%%%%%%%%%%%%%%%%%%%%%%%%%%%%%%%%%%%%%%%%%%%%%%%%%%%%
\chapter{अनाहत  ध्यान अभ्यासक्रम}
\section*{दिवस १: ध्यानाविषयी असलेली मिथके (मेडिटेशन मिथ्स)}
ध्यान किंवा ध्यानधारणा या विषयावर अनेक गैरसमज आणि मिथके समाजात प्रचलित आहेत. विशेषतः ध्यान करताना मन पूर्णपणे रिकामे असावे, सतत शांत असावे किंवा काही विशेष पद्धतींचा अवलंब केल्याशिवाय ध्यान साध्य होत नाही अशा कल्पना बऱ्याच जणांच्या मनात असतात. या मिथकांचे विश्लेषण करून, ध्यानाचा खरा अर्थ, त्याची प्रक्रिया आणि त्याचा परिणाम याबद्दल अधिक स्पष्ट समज निर्माण करणे आवश्यक आहे. खाली काही प्रमुख मिथके आणि त्यांचे सविस्तर स्पष्टीकरण दिले आहे.

ध्यानात विचार थांबवायला लागतात असे अनेकांना वाटते. परंतु आपल्या मनात विचार सतत निर्माण होत असतात; ध्यान करताना देखील हे विचार थांबत नाहीत. ध्यान म्हणजे विचारांना जबरदस्तीने थांबवणे किंवा दडपणे नव्हे. ध्यानाचे मुख्य उद्दिष्ट निर्विचार अवस्था प्राप्त करणे असले तरीही ती अवस्था एका प्रक्रियेमुळे निर्माण होत नाही, तर ती एका योग्य वातावरणामुळे आपोआप घडते. जसे आपण एखादे बियाणे जमिनीत पेरतो, त्याला पाणी घालतो, सूर्यप्रकाश मिळू देतो; त्यानंतर ते बियाणे आपोआप अंकुरते आणि हळूहळू मोठे होते. त्याचप्रमाणे, आपण मनासाठी योग्य वातावरण तयार करू शकतो आणि त्यातून विचार आपोआप निवळत जातात.

निर्विचार अवस्थेमुळे शांती मिळते, अशी समजूत अनेकांना असते. परंतु मन निर्विचार झाले तरी शांत राहतेच असे नाही. विचार नसलेली अवस्था ही ध्यान करत असतानाच काही काळ अनुभवता येते, पण दिवसभर टिकणारी शांती ही मानसिक संतुलनावर, जागरूकतेवर आणि जीवनशैलीवर अवलंबून असते. मनातले विचार कमी झाले तरी शांती कायम राहील याची खात्री नाही. शांतता म्हणजे मनात अनावश्यक विचार, अतिचिंतन, नकारात्मक भावना यांचा अभाव. जेव्हा मन सतत विचार करत राहते, त्याला इंग्रजीत ‘चॅटर’ म्हणतात, आणि त्यातूनच मानसिक थकवा, अस्वस्थता आणि नकारात्मक भावना निर्माण होतात. ‘ओव्हर थिंकिंग’ (अतिचिंतन) मुळे मन अस्थिर होते आणि कामगिरीवर त्याचा परिणाम होतो. मानसिक शांती म्हणजे केवळ विचार थांबणे नाही; तर मनाची स्थिरता आणि सजगता टिकवणे आहे. जेव्हा अतिचिंतन कमी होते, तेव्हा अनेक मानसिक समस्या आपोआप सुटू लागतात आणि मन शांततेकडे वाटचाल करते.

‘गाइडेड मेडिटेशन’ (मार्गदर्शित ध्यान) किंवा ‘म्युझिक मेडिटेशन’ (संगीत ध्यान) हे सुद्धा ध्यान असते, असा एक गैरसमज आहे. ध्यानाचा मुख्य उद्देश म्हणजे अंतर्मुख होऊन स्वतःकडे पाहणे. बाह्य आदेश किंवा आवाज ऐकून त्यावर प्रतिक्रिया देणे हे ध्यानाचे स्वरूप नाही. ध्यान म्हणजे बाह्य जगाशी संपर्क तोडून, आपल्या अंतर्मनात श्वास, हृदयाचे ठोके किंवा ‘अनाहत नाद’ (अंतर्गत नाद) यावर लक्ष केंद्रित करणे होय. मार्गदर्शित ध्यान हे प्रत्यक्ष ध्यान नसून विश्रांतीची प्रक्रिया आहे. काही वेळ मनाला शांत करण्यासाठी त्याचा उपयोग होतो, परंतु त्यातून खरी ध्यान अवस्था प्राप्त होत नाही. योगनिद्रा ही विश्रांतीची अवस्था असली तरी ती ध्यानाच्या ‘धारणा, ध्यान, समाधी’ या टप्प्यांमध्ये येत नाही. ती ‘प्रत्याहार’ या प्रारंभिक टप्प्याशी संबंधित असते.

‘वॉकिंग मेडिटेशन’ (चल ध्यान) किंवा चालत असताना, धावत असताना, वाहन चालवत असताना ध्यान साध्य होते, अशी एक समजूत आहे. प्रत्यक्षात ही अवस्था अनुपस्थित मनाची (एब्सेंट माइंडेडनेस) असते. मन विचारांमध्ये गुरफटलेले असते आणि जागरूकतेचा अभाव असतो. काही वेळा याला ‘फ्लो स्टेट’ (फ्लो अवस्था) म्हणतात, परंतु ते ध्यान नाही. ध्यानासाठी निश्चित वेळ आणि स्थिर जागा आवश्यक असते. खाज सुटणे, त्रास होणे, बेचैनी जाणवणे काहीही विचार आले तरी त्यावर प्रतिक्रिया न देता स्थिर बसून राहण्याची क्षमता विकसित करावी लागते. जर ध्यान करताना झोप येत असेल, तर याचा अर्थ शरीराला पुरेशी विश्रांती मिळालेली नाही. दिवसातून किमान सात ते आठ तासांची गाढ झोप आवश्यक आहे, जेणेकरून ध्यान करताना जागरूक राहणे शक्य होईल.

पहिल्या दिवसासाठी साधना: आज २० मिनिटे ध्यान करण्याचा सराव करायचा आहे. ध्यान करताना सर्व इंद्रियांचा वापर करून आजूबाजूच्या वस्तू निरीक्षण करायच्या आहेत.  स्पर्श, चव, दृष्टी, वास आणि ऐकणे या इंद्रियांच्या माध्यमातून वातावरण अनुभवायचे आहे. हे निरीक्षण अगदी हळू हळू आणि शांतपणे करावे. प्रत्येक इंद्रियावर साधारणपणे ५ मिनिटे तरी लक्ष द्यावे. ध्यान करताना उठू नये, डोळे उघडे ठेवून निरीक्षण करावे. कोणतीही मोजणी करू नये किंवा इतर कोणतीही ध्यानपद्धती वापरू नये. आजचा उद्देश फक्त जागरूकतेने इंद्रियांद्वारे निरीक्षण करणे आणि स्वतःच्या उपस्थितीचा अनुभव घेणे हा आहे.

%%%%%%%%%%%%%%%%%%%%%%%%%%%%%%%%%%%%%%%%%%%%%%%%%%%%%%%%%%%%%%%%%%%%%%%%%%%%%%%%%%%%%%%
\section*{दिवस २: ध्यान घडत आहे का याचा शोध}
ध्यानाचा मुख्य उद्देश म्हणजे शक्य तितके निर्विचार मन प्राप्त करणे हा आहे. पण हा निर्विचार मनाचा अनुभव नेमका कधी येतो? कोणत्या अवस्थेत तो जाणवतो? या प्रश्नांची उत्तरे शोधताना एक महत्त्वाचा मुद्दा समोर येतो – ध्यान घडावे म्हणून मनावर जबरदस्तीने काम करणे योग्य नाही. आपण पंखा थांबवण्यासाठी त्याच्या पंखांना हाताने थांबवतो का? नाही. आपण फक्त त्याचा स्विच बंद करतो. त्याप्रमाणे, मन थांबवण्यासाठी त्यावर थेट नियंत्रण ठेवण्याचा प्रयत्न करण्याऐवजी त्यासाठी योग्य परिस्थिती निर्माण करणे आवश्यक असते. जसे आपण एका बटण किंवा स्विचच्या मदतीने पंखा थांबवतो, तसेच ध्यानात मन थांबावे यासाठी बाह्य सहाय्य किंवा साधने शोधण्याचा प्रयत्न करणे चुकीचे आहे.
आपल्याला जर अभौतिक किंवा सूक्ष्म गोष्टींचे परिणाम अनुभवायचे असतील, तर त्या गोष्टींवर थेट नियंत्रण ठेवणे शक्य नसते. त्याऐवजी आपण भौतिक किंवा स्थूल गोष्टींवर काम करून त्या सूक्ष्म गोष्टींवर अप्रत्यक्ष परिणाम घडवू शकतो. उदाहरणार्थ, मूर्तिपूजा हे याचे उत्कृष्ट उदाहरण आहे. खरी देवता ही सूक्ष्म आणि अभौतिक असते. पण तिच्या पूजेकरिता आपण तिच्या प्रतिनिधी रूपाची – म्हणजेच मूर्तीची – पूजा करतो. त्या भौतिक रूपाद्वारे आपण सूक्ष्म अनुभव मिळवण्याचा प्रयत्न करतो.
विचार हे अभौतिक स्वरूपाचे असले तरी त्यांचे भौतिक रूप म्हणजे श्वास होय. त्यामुळे मन निर्विचार करण्यासाठी मनावर थेट काम करण्याऐवजी श्वासावर काम करणे अधिक परिणामकारक ठरते. श्वासावर लक्ष केंद्रित केल्यास विचार आपोआप कमी होतात आणि मन शांत होत जाते.
ध्यान साधनेसाठी खालील प्रयोग सोपा आणि परिणामकारक आहे:
\begin{itemize}
 \item आरामशीर आणि सरळ बसा.
 \item मनात शांतपणे 'ॐकार' हा मंत्र उच्चारा.
 \item श्वास घेताना आणि सोडताना त्याला 'ॐ' च्या लांबीप्रमाणे वेळ द्या; त्यामुळे मनात शांततेची अनुभूती निर्माण होईल.
 \end{itemize}
श्वास हळू करण्याच्या या साध्या पद्धतींनाच ध्यान म्हणतात. ध्यान ही फार मोठी, गुंतागुंतीची प्रक्रिया नाही; श्वास हळू करून मन शांत करणे एवढेच त्याचे सार आहे.
निर्विचार होण्यासाठी श्वास पूर्ण थांबवावा लागतो का? नाही. आपण श्वासाचा वेग कमी करू शकतो, त्याला शांततेकडे नेऊ शकतो. सुरुवातीला हे अवघड वाटू शकते, परंतु त्याचा दुसरा मार्ग म्हणजे शरीर स्थिर ठेवणे. जेव्हा आपण शरीर स्थिर करतो, तेव्हा श्वास नैसर्गिकरीत्या मंदावतो आणि त्यातून मन शांत होण्याची प्रक्रिया सुरू होते.
म्हणूनच, विचारपूर्वक पाहिल्यास निर्विचारता ही पुढील क्रमाने घडते:

निर्विचारता→हळू श्वास→स्थिर शरीर

या क्रमाचा उलटा विचार करून पहा – जर शरीर स्थिर केले, तर श्वास हळू होतो आणि श्वास हळू झाल्यास मन निर्विचार होण्याच्या दिशेने जाते.

ध्यान करताना एका क्षणासाठी श्वास थांबतो आणि मन निर्विचार होते. त्या क्षणाला 'शून्यावस्था' असे म्हणतात. त्या क्षणाची जाणीव होते, पण जाणीव हीही एक विचारच असल्याने पुन्हा मन विचारांकडे वळते. तरीही ध्यानाचे उद्दिष्ट हे त्या क्षणाचा अनुभव घेणे आणि तो क्षण जास्त काळ टिकवणे हे आहे. अभ्यास आणि सातत्य यामुळे हा क्षण पाच सेकंदांपर्यंत टिकू शकतो आणि पुढे अधिक काळ वाढू शकतो. अशा वेळी श्वास खरोखरच थांबतो. आत्मसाक्षात्कारासाठी ही अवस्था अत्यावश्यक मानली जाते. आश्चर्य म्हणजे, निर्विचारता आणि जागरूकता या विरोधाभासी गोष्टी असल्या तरी ध्यान करताना त्या दोन्ही एकाच क्षणी अनुभवता येतात. सामान्य मनुष्यांसाठी शून्यावस्थेचा थोडकाच अनुभवही मानसिक शांतीसाठी पुरेसा ठरतो.

शून्यावस्था ही झोपेसारखी असते. झोपेतही श्वास चालू असतो, पण आपल्याला त्याची जाणीव नसते. तरीही झोपेत ‘मी’ अस्तित्वात असतो, त्यामुळे ती अवस्था खरी ध्यानाची नाही, पण त्यासारखी अनुभवली जाते. झोप आपोआप येते, तर ध्यान हे आपण स्वतः चालवतो आणि योग्य परिस्थिती निर्माण करून घडू देतो. ध्यान करताना शरीरावर नियंत्रण ठेवणे आवश्यक असते – जसे दात हलू नयेत किंवा विशिष्ट मुद्रेत बसावे. झोपेत शरीरावर नियंत्रण नसते. काही जण ज्या हस्तमुद्रांचा उपयोग करतात त्यामागे मुख्य हेतू शरीर स्थिर ठेवण्याचा असतो; अन्य कारणे फारशी नसतात. म्हणूनच, शांत आणि स्थिर बसणे हे ध्यानासाठी अत्यावश्यक आहे. यासंबंधी प्राचीन ग्रंथात म्हटले आहे – “स्थिरं सुखं आसनम्” – म्हणजेच स्थिर आसन हेच खरे सुख देणारे असते.

याशिवाय आणखी एक समान अवस्था म्हणजे स्वप्न पाहताना तुम्हाला जाणीव असते की तुम्ही स्वप्नात आहात. ही अवस्था ध्यानासारखी वाटली तरी ती ध्यान नाही; तरीही ती ध्यानाच्या जवळ जाणारी अवस्था आहे.

प्रत्यक्ष ध्यान घडवण्यासाठी फक्त शरीर स्थिर करणे पुरेसे असते. त्यानंतर ध्यान आपोआप घडते. ही प्रक्रिया प्रत्येकासाठी समान आणि नैसर्गिक आहे. हा एक सार्वत्रिक नियम आहे, जो सर्वांवर लागू होतो.

योगाच्या परंपरेत दोन व्यापक दृष्टिकोन आढळतात:

\begin{itemize}
 \item अष्टांग योग – पतंजली ऋषी यांनी मांडलेला हा मार्ग मुख्यतः मानसिक आणि आध्यात्मिक उन्नतीशी संबंधित आहे. आसनांबद्दल येथे फार थोडक्यात चर्चा होते. स्थिर आसन हे पुढील टप्प्यांतील ध्यानासाठी आवश्यक असते. “स्थिरं सुखं आसनम्” हे त्याचे मूलभूत तत्त्व आहे.
 \item हठयोग – स्वात्मराम यांनी १४व्या शतकात विकसित केलेला हा मार्ग मुख्यतः शारीरिक साधनांवर भर देतो. विविध आसने, मुद्रा, बंध इत्यादींच्या माध्यमातून शरीर नियंत्रण आणि स्वास्थ्य यावर लक्ष केंद्रित केले जाते.
 \end{itemize}
 
दुसऱ्या दिवसासाठी दिलेली साधना खालीलप्रमाणे आहे: डोळे उघडे ठेवा आणि सरळ बसा. पाठीला आधार घेऊ नका. वीस मिनिटे स्थिर राहा. खाज सुटली तरी ती सहन करा आणि त्यावर लक्ष न देता बसून राहा. फक्त पापण्या हलवणे आणि लाळ गिळणे यासारख्या नैसर्गिक हालचाली मान्य आहेत; अन्य कोणतीही हालचाल टाळावी. अशा प्रकारे स्थिर बसण्याचा सराव ध्यानाच्या पुढील टप्प्यांसाठी मजबूत आधार निर्माण करतो.


%%%%%%%%%%%%%%%%%%%%%%%%%%%%%%%%%%%%%%%%%%%%%%%%%%%%%%%%%%%%%%%%%%%%%%%%%%%%%%%%%%%%%%%%%%%%
\section*{दिवस ३: ध्यानातील अडथळे}
ध्यान ही एक साधी पण अतिशय सूक्ष्म प्रक्रिया आहे. याचा मूलभूत क्रम असा आहे – स्थिर बसल्यावर श्वास हळूहळू मंदावतो आणि त्यानंतर मन निर्विचार होण्याच्या दिशेने जाऊ लागते. एवढेच ध्यानाचे मुख्य स्वरूप आहे. याशिवाय काही नाही. परंतु या प्रक्रियेमध्ये काही अडथळे आणि गैरसमज निर्माण होतात. त्यांचे सविस्तर विवेचन खाली दिले आहे.

ध्यान करत असताना विचार येणे ही नैसर्गिक बाब आहे. कोणालाही एका क्षणात निर्विचार मन मिळतेच असे नाही. जसे शरीर सुदृढ करायला वेळ लागतो आणि त्याचे परिणाम हळूहळू जाणवू लागतात, तसेच ध्यानातही सुरुवातीला प्रयत्न करावे लागतात आणि नंतर त्याचे फायदे मिळू लागतात. ध्यान सुरू केल्याच्या पहिल्याच दिवसापासून काही सकारात्मक परिणाम जाणवू शकतात. हे परिणाम निर्विचारतेच्या पलीकडेही असू शकतात – जसे विश्रांती जाणवणे, लक्ष केंद्रित होणे, मानसिक थकवा कमी होणे इत्यादी.

ध्यानाचा अनुभव प्रत्येक वेळी तसाच मिळेलच असे नाही. एका दिवशी मन अधिक शांत असते, तर दुसऱ्या दिवशी विचारांचा पूर येतो. यामागे तुमच्या मानसिक अवस्थेचा, शरीराच्या स्थितीचा आणि त्या दिवशी घडलेल्या घटनांचा मोठा प्रभाव असतो. त्यामुळे ध्यानाच्या अनुभवात बदल होणे स्वाभाविक आहे. याशिवाय ध्यान हे फक्त आपण ध्यानाला बसतो त्या वेळेपुरते मर्यादित नाही; ज्या क्षणी आपण जागरूक राहतो, स्वतःचे निरीक्षण करतो, तेव्हाही ध्यानाची प्रक्रिया सुरू असते. ही एक जीवनशैली आहे, याची जाणीव कायम ठेवा.

ध्यान करताना इतरांना आलेल्या अनुभवाची अपेक्षा करणे हा एक मोठा अडथळा ठरतो. काहींना ध्यान करताना हवेत तरंगल्यासारखे वाटते, काहींना प्रकाश दिसतो, तर काहींना वेगळेच अनुभव येतात. परंतु तुम्हालाही त्याच प्रकारचा अनुभव यायला हवा, अशी अपेक्षा करणे चुकीचे आहे. प्रत्येक व्यक्तीचे मन, त्याची मानसिक अवस्था आणि जीवनानुभव वेगळे असतात, त्यामुळे ध्यानाचा अनुभवही वेगळा असतो. ध्यानासाठी कोणतेही विशिष्ट सूत्र किंवा पाककृती नसते; प्रत्येकाने आपापल्या मार्गाने, आपल्या क्षमतेनुसार साधना करावी.

ध्यान करताना काही वेळाने कंटाळा येणे हे सामान्य आहे. काहींना ध्यान लवकर संपावेसे वाटते, त्यातून सुटका व्हावी असे वाटते. जर तुम्ही कोणताही आध्यात्मिक अभ्यास (स्पिरिच्युअल प्रॅक्टिस) सतत तीन महिन्यांपासून करत असाल आणि तरीही तो कंटाळवाणा वाटत असेल, तर कदाचित ती साधना तुमच्यासाठी योग्य नाही. मनाचा थकवा आणि असह्यतेमुळे ध्यान टाळण्याचा प्रयत्न केला जातो. पण याचा अर्थ ध्यान निरुपयोगी नाही; उलट यामुळे आपल्या मनाची सहनशक्ती आणि सातत्य तपासले जाते.

काळ जसजसा पुढे जातो तसतशी ध्यानाची तंत्रे कंटाळवाणी वाटू लागतात. परंतु तरीही त्याच तंत्रांचा सातत्याने अवलंब करणे आवश्यक असते. वेळोवेळी ध्यानाची पद्धत बदलणे, नवे प्रयोग करणे किंवा नवीन मार्ग शोधणे यामुळे मन विचलित होण्याची शक्यता वाढते. म्हणून तीच वेळ आणि तीच पद्धत पाळणे आवश्यक आहे. दिवसातून दोन वेळा निश्चित वेळ राखून, दृढ इच्छाशक्तीने साधना करणे हाच यावर उपाय आहे. सातत्य आणि निष्ठा यामुळेच साधना परिपक्व होते.

तिसऱ्या दिवसासाठी ध्यानाचा अभ्यास खालीलप्रमाणे करा. मन शांत करण्यासाठी आणि अंतर्गत घटकांवर लक्ष केंद्रित करण्यासाठी एक कल्पनाचित्र मनात आणा आणि त्यातून ध्यानासाठी काय उपयोग करता येईल हे शोधा.

दृश्य: भारत आणि पाकिस्तान यांच्यातील क्रिकेट सामना सुरू आहे. शेवटच्या चेंडूवर भारताला विजयासाठी चार धावा हव्या आहेत. संपूर्ण स्टेडियममध्ये तणाव आहे. प्रत्येक प्रेक्षकाचा श्वास रोखलेला आहे. अशा क्षणी अचानक दिवे निघून जातात आणि काय झाले हे कोणालाही समजत नाही. काही क्षण पूर्ण अंधार आणि गोंधळ असतो. एवढेच.

या दृश्यावर विचार करा आणि स्वतःला विचारा – यातून ध्यानासाठी काय उपयोग करता येईल? याचे उत्तर असे आहे की, जेव्हा आपण एखाद्या गोष्टीवर पूर्ण लक्ष केंद्रित करतो, तेव्हा आपोआप श्वास मंदावतो किंवा थांबल्यासारखा होतो. ध्यान करतानाही अशाच प्रकारे अंतर्गत घटकांवर लक्ष केंद्रित केल्यास श्वास हळूहळू शांत होतो. त्यामुळे ध्यानासाठी बाह्य परिस्थितीपेक्षा अंतर्गत निरीक्षण अधिक महत्त्वाचे आहे.

ध्यान करताना डोळे बंद करून व निश्चल बसून २० मिनिटे श्वास, हृदयाची धडधड आणि ‘अनहद नाद’ यांसारख्या अंतर्गत घटकांवर लक्ष केंद्रित करा. यामुळे हळूहळू साक्षीभाव निर्माण होतो. हा साक्षीभाव म्हणजे बाह्य घटनांकडे न पाहता स्वतःच्या अस्तित्वाकडे जागरूकतेने पाहण्याची प्रक्रिया आहे. यामुळे मन शांत होते, विचारांचा वेग मंदावतो आणि ध्यान अधिक सखोल होते.
ध्यानातील अडथळे समजून घेतल्यावर आपण त्यावर सहज मात करू शकतो. सातत्याने अभ्यास करत राहिल्यास मनाची शांती आणि जागरूकता हळूहळू वाढते. ध्यान हे एक साधे पण प्रभावी साधन आहे, जे प्रत्येकासाठी उपलब्ध आहे – फक्त सातत्य आणि निष्ठेची गरज आहे.

%%%%%%%%%%%%%%%%%%%%%%%%%%%%%%%%%%%%%%%%%%%%%%%%%%%%%%%%%%%%%%%%%%%%%%%%%%%%%%%%%%%%
\section*{दिवस ४: ध्यान मदत करत आहे का?}
ध्यान ही अशी प्रक्रिया आहे जी हळूहळू तुमच्या विचारांवर, भावनांवर आणि जीवनाकडे पाहण्याच्या दृष्टिकोनावर परिणाम करते. ध्यान नियमितपणे करत राहिल्यास त्याचे काही स्पष्ट संकेत दिसून येतात. खाली ध्यानाचे परिणाम आणि त्यातून मिळणारे फायदे तपशीलाने दिले आहेत.
जर ध्यान प्रभावीपणे कार्य करत असेल तर तुम्हाला खालील गोष्टी अनुभवायला मिळतील:
\begin{itemize}
\item
 \textbf{आंतरिक स्थिती बाह्य परिस्थितीवर अवलंबून राहात नाही:}
 ध्यानामुळे तुम्हाला समजते की बाहेर काय घडते यावर तुमची अंतर्गत शांती अवलंबून नाही. उदाहरणार्थ, कोणीतरी तुमच्यावर रागावले तरी त्याचा परिणाम सर्वांवर होत नाही. कोणावर प्रतिक्रिया द्यायची आणि कोणावर नाही हे ठरवणे हे तुमच्या हातात असते. आपण बाह्य घटनांशी मन जोडतो आणि त्यावरून आपली भावना ठरवतो; परंतु ध्यानाने आपल्याला निवड करण्याची क्षमता मिळते. बाहेर सकारात्मक किंवा नकारात्मक परिस्थिती असली तरी त्या परिस्थितीशी तुमचे मन कसे जोडायचे हे तुम्ही ठरवू शकता.
\item
 \textbf{स्वीकार (एक्सेप्टन्स):}
 ध्यानामुळे वास्तव स्वीकारण्याची क्षमता वाढते. परिस्थिती बदलणे शक्य नसेल तर तिला स्वीकारणे आणि जे तुमच्या नियंत्रणात आहे त्यावर लक्ष केंद्रित करणे शिकता येते. स्वीकार म्हणजे हार मानणे नव्हे, तर परिस्थितीचे विश्लेषण करून ती बदलता येते का याचा विचार करणे होय. जे बदलू शकत नाही त्यावर ऊर्जा खर्च न करता समजून घेणे ही ध्यानाची देणगी आहे.
\item
 \textbf{अनासक्ती (डिटॅचमेंट):}
 जास्त आसक्ती मानसिक अस्थिरता निर्माण करते. कोणत्याही गोष्टीशी किंवा व्यक्तीशी अत्याधिक जोडले गेल्यास स्पष्ट विचार करणे कठीण होते. थोडीशी अनासक्ती मनाला स्थिर ठेवते आणि योग्य निर्णय घेण्यास मदत करते. वस्तू, पद, ध्येय यांसोबत अतिरेकी अवलंबित्व ठेवणे टाळा. जसे आर्थिक नुकसान टाळण्यासाठी 'स्टॉप लॉस' ठेवतो, तसेच मानसिक हानी टाळण्यासाठी आसक्तीतून बाहेर पडणे आवश्यक आहे. ध्यानामुळे हळूहळू अनासक्ती निर्माण होते आणि मन हलके होते.
\item
 \textbf{वर्तमान क्षणात राहण्याची क्षमता:}
 ध्यानामुळे आपले मन भूतकाळाच्या आठवणींमध्ये किंवा भविष्याच्या चिंता-कल्पनांमध्ये अडकून राहत नाही. दिवसाचा बहुतांश वेळ आपण सध्याच्या क्षणावर लक्ष केंद्रित करू लागतो. अशा प्रकारे जाणीवपूर्वक जगण्याची क्षमता वाढते. यामुळे मन शांत राहते आणि जीवनातील लहान लहान गोष्टींचा आनंद घेता येतो.
\item
 \textbf{सकारात्मकता वाढते:}
 ध्यानामुळे मनातील नकारात्मक विचार कमी होतात आणि सकारात्मक विचारांचा प्रभाव वाढतो. आकर्षणाचा नियम (लॉ ऑफ अट्रॅक्शन), सकारात्मक विचार (पॉझिटिव्ह थिंकिंग), आणि प्रतिज्ञा (अफर्मेशन्स) यांचा उपयोग ध्यानामुळे सहज शक्य होतो. दिवसात आपल्याला हजारो विचार येतात, त्यापैकी फक्त सुमारे २०\% विचारांचे परिणाम प्रत्यक्षात येतात. ध्यानामुळे हे २०\% विचार सकारात्मक असले तरी उर्वरित ८०\% विचारही हळूहळू सकारात्मक दिशेने जाऊ लागतात. त्यामुळे जीवनातील संधी वाढतात आणि गोष्टी प्रत्यक्षात यायला लागतात (मॅनिफेस्ट होतात). प्रत्येक गोष्टीवर शंका न घेता मन मोकळे ठेवण्याची सवय लागते.
\item
 \textbf{इतर लोकांमध्येही सकारात्मक बदल:}
 ध्यानामुळे तुमच्या आसपासच्या लोकांमध्येही बदल जाणवू लागतो. काही नकारात्मक किंवा अस्थिर लोक हळूहळू दूर जाऊ शकतात आणि त्यांच्या जागी तुमच्यासारखे विचार करणारे, सकारात्मक आणि शांत लोक तुमच्या आयुष्यात येतात. तुमच्या ऊर्जेचा प्रभाव इतरांवरही पडतो.
\item
 \textbf{भावना नियंत्रणात येणे:}
 राग, भीती, दुःख यांसारख्या भावना आपल्यावर नियंत्रण ठेवण्याचा प्रयत्न करतात. ध्यानामुळे या भावना निरीक्षणात आणता येतात आणि त्यांना योग्य दिशेने वळवता येते. भावनांना दडपून न ठेवता त्यांचा स्वीकार करून त्यावर काम करण्याची ताकद मिळते. यामुळे मन स्थिर राहते आणि निर्णय घेणे सुलभ होते.
\end{itemize}
ध्यान ही एक प्रक्रिया आहे जी एका दिवसात पूर्ण परिणाम देत नाही, पण सातत्याने अभ्यास केल्यास जीवनाच्या प्रत्येक अंगावर सकारात्मक परिणाम घडवते. या फायद्यांचा अनुभव घेतल्यावर तुमचे मन अधिक स्थिर, जागरूक आणि शांत होईल. ध्यानाचा सराव करत राहिल्यास तुम्हाला जीवनाकडे पाहण्याचा दृष्टिकोनच बदललेला जाणवेल – आणि हीच खरी प्रगती आहे.

%%%%%%%%%%%%%%%%%%%%%%%%%%%%%%%%%%%%%%%%%%%%%%%%%%%%%%%%%%%%%%%%%%%%%%%%%%%%%
\section*{दिवस ५: ध्यानाचे नियम व सूचना (रूल्स / टिप्स)}
उत्तम आणि फलदायी ध्यान साध्य करण्यासाठी काही स्पष्ट नियम, सूचनात्मक मार्गदर्शक तत्त्वे आणि शिस्तबद्ध जीवनपद्धतीचे पालन करणे अत्यावश्यक आहे. ध्यान (मेडिटेशन) ही केवळ एका आसनावर बसून डोळे मिटण्याची क्रिया नसून, ती आपल्या संपूर्ण दिवसाच्या विचारशैलीशी, आहाराशी, नैतिकतेशी आणि मानसिक स्वच्छतेशी निगडित असलेली एक सूक्ष्म पण प्रभावी साधना आहे. यामध्ये मनाला स्थिर करणे, विचारांवर संयम ठेवणे आणि बाह्य जगाच्या प्रभावातून अंतर्मनाकडे वळणे यांचा समावेश होतो. पुढे दिलेले मार्गदर्शक मुद्दे सविस्तर आणि स्पष्टपणे समजावून सांगितले आहेत जेणेकरून प्रत्येक वाचकास ते सहज अंगीकारता येतील.
ध्यानाला बसण्यापूर्वीचा तयारीचा काळ अतिशय महत्त्वाचा असतो. ध्यानाच्या आधी थोडी हलकी शारीरिक हालचाल आणि श्वसनाशी संबंधित व्यायाम (एक्सरसाईझ) केल्यास शरीर आणि मन दोन्ही शिथिल होतात. त्यामुळे उर्जेचा प्रवाह समतोल राहतो आणि स्थिर बसण्याची क्षमता वाढते. साधारण दोन–तीन मिनिटे सौम्य हालचाली करणे, त्यानंतर दीर्घ व सम श्वासोच्छ्वास करणे आणि शेवटी काही क्षण शांत बसणे या क्रमाने ध्यानाची सुरुवात करणे अधिक फलदायी ठरते. या तयारीचे मुख्य उद्दिष्ट म्हणजे शरीरातील अनावश्यक ताण कमी करणे आणि मनाला एकाग्रतेसाठी तयार करणे होय. अशा प्रकारे ध्यानापूर्वी केलेली तयारी पुढील ध्यानास सुरळीत आणि प्रभावी बनवते.
ध्यान करताना वेळ आणि कालावधी निश्चित ठेवणेही अत्यंत उपयुक्त ठरते. दररोज एकाच वेळेला, शक्यतो दोन वेळा आणि प्रत्येकी सुमारे वीस मिनिटे अशी नियमित साधना करावी. जेवणाच्या किमान दोन तास आधी ध्यान करण्याचा सल्ला दिला जातो. कारण पचनप्रक्रियेला अधिक ऑक्सिजन (ऑक्सिजन) लागतो आणि जेवणानंतर ध्यान करताना श्वसनाची लय मंदावू शकते किंवा शरीरावर अतिरिक्त ताण येऊ शकतो. तसेच, झोपण्याच्या अगोदर लगेच ध्यान टाळावे, कारण ध्यानामुळे मन ताजेतवाने होते आणि त्यामुळे झोप लागण्यास अडथळा निर्माण होऊ शकतो. संध्याकाळी शक्यतो सात वाजण्यापूर्वी ध्यानास बसणे अधिक हितावह ठरते. दुपारच्या जेवणाच्या आधीचा वेळही ध्यानासाठी अत्यंत योग्य मानला जातो. ध्यान करताना पोट रिकामे असणे आवश्यक आहे आणि शरीराला विचलित करणाऱ्या गरजा – जसे की शौच-विसर्जनाची हाक – पूर्ण झालेल्या असाव्यात. म्हणूनच, जागे झाल्यानंतर लगेच ध्यानास बसणे टाळावे. शरीर आणि मन स्थिर करण्यासाठी थोडा वेळ घेऊन मग ध्यानास सुरुवात करावी.
संपूर्ण दिवसातील जीवनशैली ध्यानाच्या गुणवत्तेला आकार देते. आहार सात्त्विक असावा, आणि येथे ‘सात्त्विक’ याचा अर्थ केवळ शाकाहारी एवढाच नसून स्वच्छ, पचनास सुलभ आणि मन शांत ठेवणारा असा असावा. आपल्या प्रकृतीला अनुरूप असलेला, ताज्या आणि अल्प मसाल्याचा आहार मनःशांतीस पोषक ठरतो. ध्यानाच्या आधी आपण कोणते विचार करतो, कोणते काम करतो आणि कोणते वातावरण तयार करतो याचा थेट परिणाम ध्यानावर होतो. त्याचप्रमाणे ध्यानानंतर मिळालेली शांती आपल्या पुढील कृतींवर प्रभाव टाकते. त्यामुळे पूर्वकृती आणि उत्तरकृती – म्हणजे ध्यानापूर्वी व ध्यानानंतरच्या कृती – जागरूकतेने आणि शिस्तीने निवडणे आवश्यक आहे.
ध्यानाच्या साधनेस नैतिकतेची भक्कम पायाभरणी आवश्यक असते. तत्त्वनिष्ठ (एथिकल) वर्तन म्हणजे असत्य न बोलणे, भ्रष्टाचार (करप्शन) टाळणे, वेळेचे काटेकोर पालन करणे, समयपालनशील राहणे (पंक्चुअल असणे) आणि इतरांच्या हिताचा आदर राखणे होय. अनैतिक वर्तनामुळे आपल्या मनःसाक्षीला (कॉन्शन्स) वेदना होतात आणि अंतर्मनातील खळबळ ध्यानात एकाग्रतेला अडथळा निर्माण करते. म्हणूनच, दैनंदिन जीवनात प्रामाणिकपणा, स्वच्छता आणि नैतिकतेचा अवलंब करणे हीच ध्यानसाधनेची मजबूत आधारभूमी आहे.
आपण दिवसभर जे काही ग्रहण करतो – श्रवण, वाचन, दर्शन – त्या सर्व गोष्टी मनावर नोंदवल्या जातात. त्यामुळे सामाजिक माध्यमांवरील (सोशल मीडीया) किंवा मोबाईलवरील (मोबाईल) आपण पाहतो त्या ‘आशया’ची (कंटेंटची) निवड सजगपणे करणे गरजेचे आहे. प्रौढांसाठी असलेले (अडल्ट्स) किंवा हिंसात्मक स्वरूपाचा (वायलन्स), तसेच चुकीचा, नकारात्मक (नेगेटिव्ह) किंवा खिन्न करणारा (डिप्रेसिंग) आशय टाळावा. आपल्या आसपास चालणारी संभाषणे, वाचनासाठी निवडलेली पुस्तके किंवा पाहण्यासाठी घेतलेले मनोरंजन यामध्येही हीच जागरूकता ठेवावी. मनाला अस्थिर करणाऱ्या किंवा उत्तेजित करणाऱ्या बाह्य घटकांपासून थोडे अलिप्त राहण्याचा प्रयत्न करणे आवश्यक आहे.
एकाग्रतेस अडथळा आणणाऱ्या सर्व गोष्टींपासून शक्य तितकी दूरी राखावी. अस्थिर करणाऱ्या घटना, लोक, आठवणी, ठिकाणे आणि वस्तू यापासून निवडकपणे अलिप्त राहण्याचा सराव करणे हे मानसिक स्थैर्यास मदत करते. लोक (पीपल), आठवणी (मेमरीज), ठिकाणे (प्लेसेस), वस्तू (थिंग्स) यांपासून आवश्यकतेनुसार थोडे दूर राहणे, म्हणजेच विवेकी माघार (विथड्रॉ) घेणे हे कमजोरीचे लक्षण नसून परिपक्वतेचे प्रतीक आहे. मनाचा नियंत्रण (माइंड कंट्रोल) मिळवण्यासाठी बाह्य आवाज कमी करणे आवश्यक असते; त्यानंतरच अंतर्मनातील सूक्ष्म ध्वनी स्पष्ट ऐकू येतो.
पाचव्या दिवसासाठी एक साधा पण प्रभावी सराव येथे दिला आहे. डोळे मिटल्यावर विचार एकामागून एक येत राहतात. कधीकधी हे विचार परस्परांशी संबंध नसलेले असतात. या विचारांच्या प्रवाहात मन एका निवडक विचारावर स्थिर ठेवण्याचा प्रयत्न करणे ही एक-विचार धारणेची साधना आहे. आपण हा विचार ध्यान सुरू करण्यापूर्वी ठरवू शकतो किंवा ध्यानाच्या सुरुवातीला जो पहिला विचार स्वाभाविकरीत्या उगवतो तोही स्वीकारू शकतो. त्यानंतर सुमारे वीस मिनिटे त्या विचाराशी सखोल संबंध ठेवण्याचा प्रयत्न करावा. हा विचार एखादी मधुर आठवण असू शकतो किंवा एखादा शोधक प्रश्न असू शकतो; उदाहरणार्थ, महाविद्यालयीन सहलीची आठवण (कॉलेज ट्रिप). एकदा विचार निवडला की त्याची शाखा फोडू नये, म्हणजे संबंधित दुसऱ्या आठवणींमध्ये भटकू नये. त्या विचाराच्या केंद्राशी राहणे आवश्यक आहे. हळुवार श्वसन, मऊ लक्ष आणि संयमाने मनाला परत आणण्याची सवय यांचा आधार घेऊन ही साधना अधिक सुलभ होते.
या प्रकारच्या एका विचारावर केंद्रित असलेल्या ध्यानाला काही परंपरांमध्ये अधिभौतिक ध्यान (ट्रान्सेंडेन्टल मेडिटेशन) असे संबोधले जाते. महार्षी महेश योगी यांनी ऋषिकेशजवळील ‘चौरासी कुटिया’ येथे या परंपरेला लोकप्रियतेचा नवा आयाम दिला. त्याचप्रमाणे, ईश्वरनिष्ठ नामाचा जप करण्याची साधना, म्हणजे नामस्मरण (नेम चॅन्टिंग), हीही याच तत्त्वाशी साधर्म्य राखते. एका ध्वनीबिंदूवर किंवा एका अर्थबिंदूवर मन स्थिर ठेवणे आणि त्यावर एकाग्र राहणे, हे याचे मुख्य उद्दिष्ट असते.
एकूण सांगायचे तर, ध्यान म्हणजे केवळ वेळेचा तास नव्हे तर जगण्याची एक संपूर्ण पद्धत आहे. योग्य तयारी, निश्चित वेळेची शिस्त, सात्त्विक आहार, तत्त्वनिष्ठ वर्तन, सजग आशय-निवड आणि आवश्यक तिथे अलिप्त राहण्याची तयारी या सर्वांचा संगम झाला की ध्यानाची गुणवत्ता नैसर्गिकपणे उंचावते. अशी साधना दिवसेंदिवस परिपक्व होत जाते आणि अंतर्मनात स्थैर्य, शांतता आणि स्पष्टता निर्माण करते. हाच योग, हाच लाभ आणि हाच आपल्या दैनंदिन जीवनाचा सत्त्वगुणी आधार ठरतो.

%%%%%%%%%%%%%%%%%%%%%%%%%%%%%%%%%%%%%%%%%%%%%%%%%%%%%%%%%%%
\section*{सहावा दिवस : अवचेतन मनात कसे प्रवेश करावा?}
भारतीय संस्कृतीत प्राचीन ग्रंथांना अतिशय महत्त्व दिले गेले आहे. विशेषतः वेद, उपनिषदे आणि पुराणांमध्ये जीवन, धर्म, अध्यात्म आणि मानसिक साधना याविषयी सविस्तर माहिती आढळते. पुराणांमध्ये युगांची आणि भगवान विष्णूच्या विविध अवतारांची माहिती दिली आहे. त्या ग्रंथांनुसार काळाच्या चार मुख्य टप्प्यांना युग असे म्हणतात आणि प्रत्येक युगात मानवतेच्या आध्यात्मिक प्रवासाला दिशा देणारे विशिष्ट अवतार प्रकट झाले आहेत. ही युगे पुढीलप्रमाणे आहेत:

सत्य युगामध्ये भगवान विष्णूचे चार अवतार वर्णिले गेले आहेत – मत्स्य, कूर्म, वराह आणि नरसिंह. या अवतारांनी सृष्टीचे संरक्षण, धर्माची स्थापना आणि अन्यायावर विजय मिळवण्याचे कार्य केले.

त्रेतायुगामध्ये वामन, परशुराम आणि राम हे अवतार प्रकट झाले. युगधर्म टिकवणे, अहंकार नष्ट करणे आणि न्यायाची पुनर्स्थापना करणे हे या अवतारांचे मुख्य कार्य मानले गेले आहे.

द्वापर युगामध्ये भगवान कृष्ण आणि बुद्ध हे दोन महत्त्वाचे अवतार सांगितले आहेत. कृष्णाने प्रेम, भक्ती आणि योग यांचा मार्ग दाखविला तर बुद्धांनी करुणा आणि ध्यान यांचे तत्त्वज्ञान दिले.

कलियुगात भविष्यात प्रकट होणारा कल्की अवतार सांगितला जातो. या अवताराचे उद्दिष्ट अधर्माचा नाश करून धर्म आणि सत्य यांची पुनर्स्थापना करणे असे मानले गेले आहे.

या सर्व युगांमध्ये चांगुलपणा आणि वाईटपणा यांचे मिश्रण असते. कोणतेही युग पूर्णपणे शुभ किंवा पूर्णपणे अशुभ नसते. प्रत्येक युगात काही सकारात्मक मूल्ये आणि काही नकारात्मक प्रवृत्ती असतात. म्हणूनच जीवनात चांगुलपणावर विश्वास ठेवणे, नैतिकतेचा स्वीकार करणे आणि सद्गुणांचा अभ्यास करणे आवश्यक आहे. मात्र प्रत्येक परिस्थितीत सर्व नैतिक नियमांचे पालन करणे शक्य किंवा आवश्यक नसते. जीवनातील अनुभव आणि परिस्थिती यानुसार विवेकाने निवडक नैतिकतेचा स्वीकार करावा लागतो. उदाहरणार्थ, वाईट लोकांशी व्यवहार करताना त्यांच्या कृतींना उत्तर देण्यासाठी “डोळ्यास डोळा” (ठकासी ठक, उद्धटासी उद्धट) अशा प्रकारे प्रत्युत्तर देणे कधी कधी आवश्यक ठरते. परंतु याचा अर्थ हिंसाचार किंवा प्रतिशोध नव्हे; शेवटी आपले मन शांत आणि स्थिर राहिले पाहिजे. हीच मानसिक स्थिरता आणि संतुलन जीवनाचे खरे अंतिम ध्येय आहे.

जीवनाचा सर्वात महत्त्वाचा उद्देश म्हणजे स्वतःला जाणून घेणे. स्वतःचे खरे स्वरूप ओळखणे, आपल्या विचारांचा, सवयींचा आणि अंतर्मनाचा अभ्यास करणे हेच आत्मज्ञानाचे खरे ध्येय आहे. त्याशिवाय कोणतेही बाह्य ध्येय किंवा यश जीवनाला अंतिम अर्थ देत नाही. आपल्या मनावर काळाच्या ओघात विविध मानसिक सवयी, विचार आणि भावना यांच्या थरांचे आवरण चढते. या थरांमध्ये अडकून आपण आपले खरे स्वरूप विसरतो. त्यामुळे हे मानसिक थर विसरून जाणे आणि अंतर्मनाशी संपर्क साधणे आवश्यक ठरते.

तथापि, जर आपल्याला जीवनात इतर ध्येये किंवा उद्दिष्टे असतील तर त्यात काहीही चूक नाही. जीवनाचा प्रवाह जसा आहे तसा स्वीकारणे हेही अध्यात्माचा भाग आहे. प्रत्येकाने कठोर ध्येय ठेवण्याची आवश्यकता नाही. जीवनातील घटना आणि अनुभव स्वीकृत करणे, त्यातून शिकणे आणि मनाच्या शांततेला प्राधान्य देणे अधिक महत्त्वाचे आहे.

जेव्हा आपण मनापासून कोणत्यातरी गोष्टीसाठी इच्छा करतो, तेव्हा त्या इच्छेला “आकर्षणाचा नियम” ([Law of Attraction]) असे म्हणतात. ही इच्छा वारंवार मनात दृढ करणे म्हणजे “अफर्मेशन” ([Affirmation]) होय. सतत त्या इच्छेचा पुनरुच्चार करत राहिल्यास अवचेतन मनावर त्याचा ठाम प्रभाव पडतो आणि त्या इच्छेला बळ मिळते. जेव्हा ही इच्छा प्रत्यक्षात रूपांतरित होते, तेव्हा त्याला “मॅनिफेस्टेशन” ([Manifestation]) असे म्हणतात.

एकदा कोणतीही गोष्ट आपल्या अवचेतन मनात गेली की ती चांगली असो वा वाईट असो, ती प्रक्रिया पूर्ण झाली असे समजले जाते. अवचेतन मनाने त्या गोष्टीला स्वीकारले आहे आणि ती आपल्या मानसिक संरचनेचा भाग बनते. एवढेच साधे पण प्रभावी सत्य आहे.

आध्यात्मिक साधना आपल्या बाह्य मनाला थोडे मवाळ आणि संयमी बनवते. यामुळे आपल्या मनातील संकल्प आणि हेतू सहजपणे अवचेतन मनापर्यंत पोहोचू शकतात. काही विशेष साधना थेट अवचेतन मनाला बळकटी देण्यासाठी उपयोगी ठरतात. त्यापैकी एक प्रभावी साधना म्हणजे “त्राटक”. त्राटक साधनेमुळे मनाची एकाग्रता वाढते, अंतर्मनाशी नाते दृढ होते आणि मानसिक स्पष्टता प्राप्त होते.

सहाव्या दिवसाचे ध्यान म्हणजे त्राटक साधना होय. त्राटक करताना डोळे उघडे ठेवून एका ठराविक बिंदूकडे किंवा ज्योतीकडे लक्ष केंद्रित करायचे असते. सतत एकाग्रतेने ध्यान करत राहिल्यास अवचेतन मन बळकट होते आणि मानसिक स्थिरता प्राप्त होते.

\subsection*{बिंदु त्राटक}
बिंदु त्राटक साधनेसाठी ८ इंच x ८ इंच आकाराचा पांढरा कागद घ्यावा. त्यावर २ इंच व्यासाचा काळा गोल बिंदू तयार करावा. हा कागद डोळ्यांसमोर, नेत्रसामोरी लावून बसावे. त्या बिंदूकडे साधारण तीन फूट अंतरावरून लक्ष केंद्रित करावे. शक्यतो डोळे न मिचकावता ध्यान करणे आवश्यक आहे. ही साधना प्रगत असल्याने सुरुवातीला फक्त ५ मिनिटे करावी आणि नंतर हळूहळू जास्तीत जास्त २० मिनिटांपर्यंत वेळ वाढवावा. साधना संपल्यानंतर डोळे स्वच्छ पाण्याने धुऊन घेणे आवश्यक आहे. ध्यान करताना काळ्या बिंदूच्या जागी प्रकाशमान बिंदू दिसू लागतो. काही दिवसांनी तो बिंदू अस्पष्ट होत जाऊन पूर्णपणे नाहीसा होतो. ही अवस्था म्हणजे आपण अवचेतन मनात प्रवेश केला आहे, असे समजले जाते. साधनेदरम्यान श्वास हळूहळू मंदावतो किंवा त्यात बदल जाणवतो. संध्याकाळच्या मावळत्या प्रकाशात ही साधना करणे अधिक योग्य ठरते. सुरुवातीच्या काळात ही साधना दररोज न करता विश्रांती घेणे आवश्यक आहे.

\subsection*{बिंदु त्राटक - भाग दुसरा}
साधनेचा दुसरा टप्पा म्हणजे बिंदूचा आकार हळूहळू कमी करत नेणे. पहिल्या महिन्यात १ इंच व्यासाचा बिंदू तयार करावा. पुढच्या महिन्यात त्याचा आकार अर्धा करावा आणि तिसऱ्या महिन्यात अतिशय लहान बिंदू तयार करावा. अशा क्रमाने साधना करत गेल्यास लक्ष केंद्रित करण्याची क्षमता वाढते आणि मनाची एकाग्रता अधिक प्रभावी होते.

\subsection*{प्रतिबिंब त्राटक}
प्रतिबिंब त्राटकासाठी ८ इंच x ८ इंच आकाराचा आरसा वापरावा. तो साधारण तीन फूट अंतरावर ठेवावा. आरशात स्वतःच्या चेहऱ्याकडे पाहताना दोन्ही भुवयांच्या मधोमध लक्ष केंद्रित करावे. या साधनेमुळे स्वतःच्या प्रतिमेशी खोल नाते निर्माण होते. ध्यान करत राहिल्यास हळूहळू चेहरा विरघळल्यासारखा दिसतो आणि तो नाहीसा झाल्यासारखा भासतो. ही अवस्था अंतर्मनाशी नाते जुळल्याचे लक्षण मानली जाते.

\subsection*{ज्योती त्राटक}
ज्योती त्राटक साधनेसाठी मेणबत्ती हा एकमेव प्रकाश स्रोत म्हणून वापरावी. इतर सर्व प्रकाश बंद ठेवावा. सुरुवातीला ५ मिनिटे ध्यान करावे आणि हळूहळू वेळ २० मिनिटांपर्यंत वाढवावा. ही साधना दिवसभरात दोन वेळा करावी. डोळे उघडे ठेवून ध्यान करत राहिल्यास मनाची एकाग्रता आणि अंतर्मनाशी संपर्क साधण्याची क्षमता अधिक मजबूत होते.

या सर्व साधना संयमाने आणि नियमितपणे केल्यास मनाची एकाग्रता, मानसिक स्थिरता आणि अवचेतन मनाशी असलेले नाते दृढ होते. हळूहळू मन शांत होते, विचार स्पष्ट होतात आणि अंतर्गत शक्तीशी संपर्क साधण्याचा मार्ग मोकळा होतो. अवचेतन मनाच्या या साधनेद्वारे आत्मज्ञानाकडे जाणारा प्रवास सुकर आणि प्रभावी होतो.


%%%%%%%%%%%%%%%%%%%%%%%%%%%%%%%%%%%%%%%%%%%%%%%%%%%%%%%%%%%
\section*{सातवा दिवस : या जीवनाचा खरा शोध – स्वतःच्या स्वरूपाची जाणीव आणि ध्यान}
या जीवनाचा मुख्य उद्देश काय आहे, याचा विचार आपण अनेक वेळा करतो; परंतु या प्रश्नाचे खरे उत्तर म्हणजे स्वतःच्या खऱ्या स्वरूपाचा शोध घेणे हेच आहे. आपल्या अस्तित्वाचा पाया कोणता, आपले मन, शरीर याच्या पलीकडे असलेले स्वरूप कोणते, याची जाणीव होणे हाच या शोधाचा केंद्रबिंदू आहे. हा शोध केवळ बौद्धिक पातळीवर न राहता प्रत्यक्ष अनुभवाने साध्य होतो आणि त्या अनुभवाचा मार्ग म्हणजे ध्यान ([मेडिटेशन]) ही साधना होय.

ध्यानाची प्राथमिक व्याख्या अशी सांगितली जाते की विचारशून्यता प्राप्त करणे आणि मनाचा लय होणे. याचा अर्थ विचारांचे, भावना-आकर्षणांचे प्रवाह थांबवून मन शांत करणे. परंतु ध्यान इतक्यावरच थांबत नाही. विचार थांबले, मन लय पावले तरी पुढे आत्म्याचा गूढ शोध घेण्याची खरी आध्यात्मिक यात्रा सुरू होते. या अवस्थेला आपण आध्यात्मिक ध्यान ([स्पिरिच्युअल मेडिटेशन]) असे म्हणतो, कारण येथे मनाच्या मर्यादा ओलांडून आत्मस्वरूपाशी थेट संपर्क साधला जातो.

या अवस्थेत आपण शून्यावस्थेची अनुभूती घेऊ शकतो. शून्यावस्था म्हणजे मनाच्या कल्पना, विचार आणि भास यापलीकडील स्थिती. त्यात आपले आत्मस्वरूप प्रकट होते. आत्मा हा मनाच्या पलीकडे असतो. मन हे वेळ आणि अवकाश ([टाइम अँड स्पेस]) निर्माण करणारे असते. मन जिथे कार्यरत असते तिथेच वेळेचा आणि अवकाशाचा अनुभव निर्माण होतो. परंतु मन जिथे थांबते किंवा अस्तित्वात नसते, तिथे वेळ आणि अवकाशही नसते. याचा अर्थ नाद, ध्वनी किंवा कंपन ([साउंड]) हे वेळ आणि अवकाशाच्या बंधनांपलीकडे असते.

अवकाशाच्या पलीकडे असणे म्हणजे त्या नादाला कोणतेही ठिकाण नाही, कोणतीही सीमा नाही, कोणताही आकार नाही. तो सर्वत्र अस्तित्वात असतो; मात्र कोणत्याही एका स्थानावर त्याचे बंधन नसते. त्याला आकाररहित ([फॉर्मलेस]) आणि सर्वव्यापी असे वर्णन करता येते. अशा नादाचा अनुभव घेताना आपण त्याच्या उपस्थितीत असतो, परंतु तो कोणत्याही एका ठिकाणी मर्यादित नसतो.
त्याचप्रमाणे वेळेच्या पलीकडे असणे याचा अर्थ असा की कारण–परिणाम या सिद्धांताचा ([कॉजेशन थिअरी], कार्य-कारण भाव) त्यावर परिणाम होत नाही. सामान्यतः कोणतीही घटना घडण्यासाठी कारण आणि त्यानंतर परिणाम असा अनुक्रम असतो. परंतु वेळेच्या पलीकडे असलेल्या नादावर या अनुक्रमाचा परिणाम होत नाही. त्यामुळे त्याला सुरुवात नाही आणि शेवटही नाही. तो कोणत्याही एका कालखंडाशी किंवा घटनेशी जोडलेला नाही. या अवस्थेला ‘स्थलकालातीत’ ([ट्रान्सेंडिंग टाइम अँड स्पेस]) अशी संज्ञा दिली जाते.

या अवस्थेची अनुभूती घेतल्यावर ध्यान हे केवळ मन शांत करण्याचे किंवा तणाव कमी करण्याचे साधन राहात नाही. त्यापलीकडे जाऊन ध्यान आत्मस्वरूपाच्या अखंड अस्तित्वाशी एकरूप होण्याचा, स्वतःला जाणण्याचा आणि विश्वाशी समरस होण्याचा मार्ग बनतो. मनाच्या पलीकडे असलेला नाद हा प्रत्येक क्षणी, प्रत्येक ठिकाणी असलेला सूक्ष्म अनुभव असतो. तो वेळ आणि अवकाशाच्या बंधनांपासून मुक्त होण्यास मदत करतो आणि आपल्याला अस्तित्वाच्या गाभ्याशी जोडतो.

या गूढ अवस्थेची जाणीव होणे हेच खऱ्या अर्थाने जीवनाच्या शोधाची सुरुवात आहे. जेव्हा आपण विचारांपासून मुक्त होतो, तेव्हा आत्म्याची अनुभूती संभवते. त्या क्षणी ध्यान ही प्रक्रिया केवळ मानसिक शांततेपुरती मर्यादित राहत नाही, तर ती अस्तित्वाच्या परम सत्याशी जोडणारा सेतू ठरते. स्वतःला जाणून घेण्याची, स्वतःच्या स्वरूपाशी एकरूप होण्याची आणि अंतर्मनातील शाश्वत अस्तित्वाशी संवाद साधण्याची ही एक अद्भुत यात्रा आहे.

ही अनुभूती जसजशी गहिरी होत जाते, तसतसे जीवनाकडे पाहण्याचा दृष्टीकोन बदलतो. बाह्य घटनांपासून मन मागे घेत आत्म्याच्या उपस्थितीत स्थिर राहण्याची क्षमता विकसित होते. अशा ध्यानातून मिळणारे समाधान, शांती आणि स्थिरता ही जीवनाच्या प्रत्येक क्षेत्रात सकारात्मक परिवर्तन घडवणारी ठरते. त्यामुळे ध्यान ही केवळ एक मानसिक क्रिया नाही, तर जीवनाचा खरा शोध घेण्याचा आणि आत्म्याशी एकरूप होण्याचा दिव्य मार्ग आहे.
