%%%%%%%%%%%%%%%%%%%%%%%%%%%%%%%%%%%%%%%%%%%%%%%
\chapter{परिचय}

जीवन कठीण आहे किंवा आपल्याला तसे वाटते तरी. 

पण खरी गोष्ट ही आहे की जीवन तेवढेच कठीण आहे जेवढे आपण ते समजतो-ठरवतो.

आपल्यापैकी बहुतेकजण दररोज अनेक कामांमध्ये आणि धावपळीत असतात, हे काम, त्यानंतर ते काम, काहीतरी  तडजोड अशा अनेक नाट्यमय प्रसंगांना सामोरे जात असतात. तुकोबारायांनी म्हटल्याप्रमाणे ‘रात्रंदिन आम्हा युद्धाचा प्रसंग’. पण नीट लक्ष दिले तर असे दिसेल की या संघर्षांपैकी बहुतेक गोष्टी काल्पनिक आणि उगाचच ओढून ताणून बोलावल्यासारख्या असतात.

आपण खरंतर अगदी साधे प्राणी आहोत. अन्न, वस्त्र, निवारा, आणि नातेसंबंध हे एवढेच आपल्याला आनंदी राहण्यासाठी पुरेसे आहेत. अन्न  नैसर्गिकपणे मिळते. वस्त्र म्हणजे कापड, तेही मिळते. निवारा म्हणजे छत, तेही आता बहुतेक सर्वांच्या डोक्यावर असते. नातेसंबंध म्हणजे अपेक्षांशिवाय एकमेकांच्या सहवासाचा आनंद घेणे. ते पण असतील तर बहारच. पण आपण मात्र या साध्या गरजांच्या पलीकडे, उगाचच काही काल्पनिक गरजा जोडल्या आहेत: करिअर, बॉस-सहकारी; नवीन यंत्रे (गॅजेट्स), सॉफ्टवेअर आणि सोशल मीडिया; गाड्या आणि छान कपडे, पर्स, लॅपटॉप बॅग, परदेशी सहल, आणि बरेच काही. 

मी असे म्हणत नाही की आपण अगदी आदिमानवाच्या काळात परत जावे, पण हे लक्षात ठेवणे महत्त्वाचे आहे की काय अगदी आवश्यक आहे आणि काय काल्पनिक आहे.

जेव्हा आपल्याला कळते की काहीतरी काल्पनिक आहे, तेव्हा आपण त्या गरजेला नाकारण्याचे ठरवू शकतो; जर ती चांगल्या-महत्वाच्या  उद्देशाची पूर्तता करत नसेल. जर ती जीवन अधिक कठीण बनवत असेल, तर ती सोडली-त्यागली जाऊ शकते! जीवन कठीण बनवणाऱ्या गोष्टी काढून टाकल्याने, आपल्याकडे जे उरते ते 'सहज जीवन'.

मला जेव्हा पट्टीचा पोहणारा बनायचे होते तेव्हा मी एक महत्त्वाचा धडा शिकलो.  मला वाटत होते की खूप लांब आणि वेगाने पोहणे यासाठी लागतं ते म्हणजे फक्त अधिक जोर लावणे आणि खूप वेळ तो प्रयत्न कष्टप्रद वाटलं तरी करत राहणे. म्हणून मी पाण्यात वेड्यासारखा हातपायमारत बसायचो, आणि मग थकून जायचो. पण जेव्हा मला हे उमगले की पाणी सुद्धा  तुम्हाला वर ढकलत असते आणि ते तरंगण्यास मदत करू शकते, तेव्हा मात्र त्यातून पोहणे खूप सोपे व्हायला लागले. मी मनात शांत आणि अतिप्रयत्नात शिथिल झालो, गोष्टींना जबरदस्तीने करण्याचा प्रयत्न थांबवला आणि कमी प्रयत्नाने चांगले पोहायला शिकलो, अगदी माशासाऱ्याखे. 
असं बघा, आयुष्य हा पण एक प्रवाह आहे. आयुष्य म्हणजे पाणी, आणि आपण मात्र त्याला हातपाय मारत राहतो, खूप धडपड करतो, जबरदस्तीने पुढे जायचा प्रयत्न करतो. शेवटी पदरात पडते  ती तगमग, आणि फुकाचा संघर्ष. खरं शहाणपण हेच आहे की आपल्याला असे विनाकारण कष्टदायक पोहोण्याची गरजच नाहीये, तर गरज आहे तरंगायला शिकायची. गोष्टींना जबरदस्तीने वळवायचं नाही, तर त्यांना सहजपणे घडू द्यायचं आहे. असं झालं की आपण जास्त आरामात पुढं जातो आणि आयुष्याचा गोडवा अधिक खुलतो. कदाचित ज्ञानोबारायांनीच म्हटले आहे की, ‘सहजी नीटू जाहला’. 
अशी कल्पना करा की तुम्ही सकाळी जागे होता आणि लगेच तुम्हाला आवडणारं काम करता. तुम्ही फक्त जगण्यासाठी नाही, तर जगण्याचा आनंद लुटण्यासाठी जगता. ज्या लोकांना तुम्ही मनापासून आवडता, त्यांच्यासोबत वेळ घालवता, आणि त्या प्रत्येक क्षणाला पुरेपूर जगता. त्या क्षणात तुम्ही इतके रमता की भविष्याची काळजी तुम्हाला बोचत नाही, आणि भूतकाळातील चुका तुमच्यावर ओझ्यासारख्या ओझं बनून राहात नाहीत. कुण्याकाळचे पाणीही डोळ्यात येत नाही (कवी: संदीप खरे). 

कल्पना करा, तुमच्याकडे काही मोजके जवळचे मित्र आणि जिवलग कुटुंबीय आहेत. तुम्ही त्यांच्यासोबत मोकळेपणाने, भरपूर वेळ घालवता. त्यांच्याकडून तुम्हाला कसल्याही अपेक्षा नसतात. म्हणूनच ते तुम्हाला कधी निराश करत नाहीत. उलट, ते जे काही करतात ते तुम्हाला अगदी योग्यच वाटतं. तुम्ही त्यांना त्यांच्या मूळ स्वरूपासाठी प्रेम करता, त्यांचं व्यक्तिमत्त्व जसं आहे तसं मान्य करता. त्यामुळे तुमची नाती कधीच गुंतागुंतीची होत नाहीत.

अजून कल्पना करा, की तुम्हाला एकांताचा आनंद घेता येतो. एकटं राहूनही तुम्ही समाधानी असता, तुमच्या स्वतःच्या विचारांसोबत, निसर्गाच्या सहवासात, एखाद्या पुस्तकात हरवून, किंवा कदाचित काहीतरी नवीन निर्माण करताना. हे सर्व स्वप्नवत, कदाचित आभासी आणि अशक्यप्राय वाटतं  ना? 

हेच खरं साधं, सहज जीवन आहे. आता सहज म्हणजे अजिबात प्रयत्न न करणं असा त्याचा अर्थ नाही. उलट, प्रयत्न होतात पण ते भारासारखे वाटत नाहीत. ते हलके वाटतात, सहजतेने घडतात. आणि शेवटी, हीच सहजतेची जाणीव महत्त्वाची ठरते.

आणि सगळ्यात खास म्हणजे, असं जीवन फक्त कल्पनेत नाही, तर प्रत्यक्ष जगता येणं  पूर्णपणे शक्य आहे.

सहज जीवनाच्या मार्गात जर कोणी अडथळा आणत असेल, तर ते दुसरं कोणी नसून आपलं स्वतःचं मन असतं.


%%%%%%%%%%%%%%%%%%%%%%%%%%%%%%%%%%%%%%%%%%%%%%%
\chapter{सहज जीवनासाठी मार्गदर्शक तत्त्वे}

जीवनात सहजता शोधायची असेल तर त्याचे नियम, तत्वे  कठोर नसावीत, तर काळानुरूप लवचिक आणि प्रवाही-जिवंत असावीत. इथे सांगितलेली तत्त्वे-नियम काही दगडावरची रेघ नाही, तर त्या सौम्य सूचना आहेत. आणि विशेष लक्ष द्यायचं म्हणजे ही तत्त्वे जाणीवपूर्वक ‘नकारात्मक’ स्वरूपात मांडलेली आहेत. कारण इथे कोणीही आपल्याला “काय करावे” हे शिकवायला आलेलं नाही. त्याऐवजी हे मार्गदर्शन फक्त एवढंच सांगतं की “काय करू नये”. कारण बऱ्याच वेळा आपल्या जीवनात जी घाईगडबड, गुंतागुंत आणि थकवा निर्माण होत असतो, तो आपण केलेल्या अनावश्यक प्रयत्नांमुळे असतो. हे प्रयत्न टाळले तरच सहजतेचा मार्ग खुला होतो. आणि मग उरलेला भाग, म्हणजे काय करायचं, कोणत्या दिशेने चालायचं, हे पूर्णपणे तुमच्या हाती आहे.

आयुष्य हा एक जणू नकाशा आहे, त्या नकाशावर चालण्याचा मार्ग तुम्हालाच निवडायचा आहे. वळणाचा की सरळ बरा? (कवी: सौमित्र)

\textbf{मार्गदर्शक तत्त्वे:}
\begin{itemize}
\item इतरांना कोणतीही हानी पोहोचवू नका. कारण दुखावणे म्हणजे आपल्या स्वतःच्याच मनःशांतीवर घाव घालणे होय.
\item कोणतेही लक्ष्य (गोल्स) कठोर, अतिनिश्चित अथवा अपरिवर्तनीय ठेवू नका, तसेच कोणत्याही योजना (प्लॅन्स)  आखीवरेखीव बनवू  नका. आयुष्य नेहमीच बदलत्या वाऱ्यासारखं असतं, त्याच्याशी जुळवून घ्यावं लागतं, लवचिकतेने.
\item अपेक्षा ठेवू नका. कारण अपेक्षा ही निराशेची जननी असते.
\item खोट्या किंवा दिखाऊ गरजा निर्माण करू नका. जे खरेच आवश्यक आहे, ते आपल्यापर्यंत आपोआप येते.
\item ज्या गोष्टी तुम्हाला आवडत नाहीत, त्या जबरदस्तीने करू नका. तिरस्कारातून केलेले काम नेहमीच ओझं बनतं.
\item घाईगडबड करू नका. घाई म्हणजे चुका, आणि चुका म्हणजे पुन्हा दुरुस्त्या.
\item अनावश्यक कृती टाळा. कमी करणे म्हणजेच कधी कधी जास्त साध्य करणे असते.
\end{itemize}


\textbf{काही संभाव्य सकारात्मक मार्गदर्शक तत्त्वे:}
\begin{itemize}
\item दयाळूपणे वागा. दया म्हणजे दुसऱ्याच्या डोळ्यांतून-भूमिकेतून अजमावण्याची-पाहण्याची ताकद.
\item उत्साहाने जगा. आयुष्य हे एका रंगमंचावरील नाटकासारखं आहे, तुम्हालाच रस नसेल तर इतर-प्रेक्षक सुद्धा उठून जातात.
\item समाधान शोधा. कारण समाधान म्हणजे खऱ्या श्रीमंतीचं मोजमाप.
\item हळूहळू चला. सावकाश चालण्यातही एक प्रकारची खोल शांतता असते.
\item संयम ठेवा. थांबण्याची ताकद ही धावण्यापेक्षा जास्त असते.
\item वर्तमान क्षणात जगा. कारण भूतकाळ केवळ आठवण आणि भविष्य फक्त कल्पना आहे.
\item बेरीजेपेक्षा वजाबाकीला प्राधान्य द्या. अनावश्यक गोष्टी काढून टाकल्यावरच आयुष्याचं सौंदर्य दिसतं.
\end{itemize}

ही तत्त्वे म्हणजे जीवन साधं करण्याचा एक सौम्य प्रयत्न आहे. थोडं थांबा, विचार करा आणि मगच पुढचं पाऊल टाका, एवढं पुरेसं आहे सहजतेसाठी.
%%%%%%%%%%%%%%%%%%%%%%%%%%%%%%%%%%%%%%%%%%%%%%%%%%%%%%%
 \chapter{वू वेई आणि निष्क्रियता तत्व}
‘ताओ’वादामध्ये एक संकल्पना आहे, तिचं  नाव ‘वू वेई’. ही आपल्या नेहमीच्या पाश्चिमात्य-आधुनिक विचारपद्धतीला सरळ सोप्या भाषेत पटवून देणे अवघड जाते. साधारणपणे तिचा अर्थ “न-करणे” किंवा “निष्क्रियता” असा आहे. पण मला मात्र (हे एका वाक्यात सांगायचं झालं तर) केव्हा ‘काहीच न करणे’ योग्य ठरते आणि कधी ‘कृती करणे गरजेचे असते’ हे ओळखणे, असा अर्थ अभिप्रेत आहे असे वाटते.
आपल्या आधुनिकजीवनशैलीत मात्र याचा स्वीकार करणे बिलकुल सोपे नाही. कारण आपली संपूर्ण संस्कृतीच काहीतरी सतत “करत राहणे”, ‘बिझी असणे’, कार्यमग्न असणे, या तत्त्वाभोवती फिरत आलेली आहे. आपल्याला नेहमी काहीतरी कृती करत राहिलं पाहिजे असं वाटतं. एखादी कृती न केल्यास किंवा थांबल्यास मनात एक विचित्र अस्वस्थता, चिंतेची रुखरुख जाणवते. या सततच्या “काहीतरी करत राहण्याच्या” सवयीमुळे आपण आयुष्यात बऱ्याच अडचणी उभ्या करतो. आपण फक्त “न-करणे” या साध्या अवस्थेला घाबरतो, आणि त्या भीतीपोटी अनावश्यक श्रम स्वतःवर लादतो.
पण थोडा विचार करा, खरंच काहीच न करणे शक्य आहे का? अक्षरशः म्हणाल तर नाही. आपण काम करत नसू, तरी आपण बसलेले असतो, आडवे पडलो असतो किंवा उभे राहिलो असतो. म्हणजे काहीतरी घडतच असतं. पण साधारणतः “कृती” म्हटल्यावर आपल्याला जे सुचतं ते म्हणजे एखादं काम, जे कोणत्या तरी ध्येयाकडे (लक्ष्य , गोल्स) किंवा हेतूकडे (पर्पझ) नेणारं असतं. जर आपणच तो हेतू किंवा ध्येय बाजूला ठेवलं, तर ती कृती अनावश्यक ठरते. उलट ती केल्यामुळे आयुष्य अधिकच गुंतागुंतीचं होऊ शकतं.
म्हणूनच, जर आपण लक्ष्यं आणि उद्दिष्टं कमी केली, त्यांना सुलभ, सुकर आणि सोपं केलं, तर आयुष्यातील अनेक गोष्टींना आपोआपच करण्याची गरज उरत नाही.
हा विचार मनाशी पक्का करणे मात्र फार कठीण आहे. कारण आपल्याला नेहमी “उत्पादक” (प्रोडक्टीव्ह) राहायची सवय जडलेली असते. “निष्क्रिय” (इनॅक्टिव्ह) या शब्दालाच आपल्या समाजाने इतके नकारात्मक अर्थ-भाव चिटकवले आहेत की आपल्याला काहीच न करणे म्हणजे आळस, अपयश किंवा वेळ वाया घालवणे असेच वाटते. आळशीपणाला आपली संस्कृती नेहमी हिणवते. म्हणून मग आपण नकळत अनावश्यक ध्येयं ठरवतो, कामं करतो, केवळ हे दाखवण्यासाठी की आपण काहीतरी करतोय.
पण जरा कल्पना करा, जर आपण स्वतःची किंमत आपल्या यशापयशाने किंवा कामगिरीवरून मोजणं थांबवलं, तर? आपल्या फक्त “आहे’पणाला” (बीइंग) नेहमीच “करत राहण्यापेक्षा” (डुईंग) अधिक महत्त्व असणार नाही का?
चला, एक छोटा प्रयोग करून बघा. काहीच न करण्याचा प्रयत्न करा. अगदी फक्त पाच मिनिटांसाठी. त्या पाच मिनिटांत आपण कासावीस होतो , बेचैन होतो, मनात येतं की नवीन टॅब उघडावा, ईमेल तपासावा, एखादा लेख वाचावा, कुणाला फोन करावा, किंवा एखादं छोटं काम उरकावं. आणि हे सगळं फक्त पाच मिनिटांत! मग जरा विचार करा, जर पूर्ण दिवस काहीच केलं नाही, तर आपलं काय हाल होईल?
जर आपण खोट्या-काल्पनिक गरजा, कृत्रिम ध्येयं, नुसत्या अपेक्षा आणि बनावट उद्दिष्टं दूर केली, तर आपल्या रोजच्या कृतींपैकी अर्ध्याहून अधिक कृतींचं अस्तित्वच नष्ट होईल. मग आपल्यासमोर एक असं रिकामं स्थान उरेल, जे फक्त खरी गरज, नैसर्गिक प्रवृत्ती आणि खरी सुंदरता यांनी भरून काढता येईल.
अशा त्या “काहीच न करण्याच्या” रिकामेपणातच खरी समृद्धी दडलेली आहे.
%%%%%%%%%%%%%%%%%%%%%%%%%%%%%%%%%%%%%%%%%%%%%%%%%%%%%%%
\chapter{खऱ्या आणि साध्या गरजा}

तर मग खरंच आवश्यक काय आहे? हा प्रश्न विचारताना आपल्याला समजते की माणसाच्या खऱ्याखुऱ्या गरजा अगदी मोजक्या आहेत,  अन्न, कपडे, निवारा आणि नातेसंबंध. या चार गोष्टींपलीकडे उरलेलं बरंचसं केवळ आपल्या समाजाच्या कल्पनेतून, आणि थोडंफार आपल्या अहंकारातून उभं राहिलेलं आहे.

या मूलभूत गरजांपैकी एकही गरज प्रत्यक्षात फारशी क्लिष्ट नाही.

उदाहरणार्थ, अन्न घेऊया. कोणी म्हणेल की अन्न मिळवणे हे फारच गुंतागुंतीचे काम आहे. पण मसानोबू फुकुओकाचे “वन स्ट्रॉ रिव्होल्यूशन” वाचले की डोळे उघडतात. तो दाखवतो की फक्त एक एकर जमिनीवरही एका कुटुंबासाठी पुरेसं अन्न उगवता येतं, आणि त्यासाठी निसर्गात फारसा हस्तक्षेप करण्याची गरजही नाही. तण जसं वाढतं तसं वाढू द्या, कीटकनाशकांचा वापर करू नका, जमिनीची सतत नांगरट करू नका, आणि प्राणी, कीटक, सरडे शेतामध्ये मोकळेपणाने फिरू द्या. हे चित्र बघितलं तर लक्षात येतं,  अन्न उगवणं खरं तर अवघड नाही.

याचा अर्थ असा अजिबात नाही की आपण सगळे उद्या आपापली नोकरी सोडून जमिनीवर शेती करायला लागणार आहोत. पण इतकं लक्षात ठेवायला हरकत नाही की आपल्या अन्नाची खरी गरज समाजाने क्लिष्ट करून टाकली आहे. आज अन्न हे पोषण देणाऱ्या साध्या गोष्टीऐवजी  सामाजिक पत -स्थान-उंची-दर्जा (“स्टेटस सिम्बॉल”) बनलं आहे. अन्न महागडं, ब्रँडेड किंवा दिखाऊ असलं तरच त्याला किंमत आहे असं मानलं जातं. पण खरं म्हणजे साधं अन्न पुरेसं असतं, आणि जीवन साधं करण्याची ताकद इथूनच सुरू होते, जे नको त्यांची वजाबाकी करून.

निवाऱ्याचंही अगदी तसंच झालं आहे. घर ही बहुतांश लोकांसाठी आयुष्यातील सर्वात मोठी गुंतवणूक असते. एक सुंदर, मोठं, सजवलेलं घर सुद्धा आज “स्टेटस सिम्बॉल” मानलं जातं. पण मूळात निवारा म्हणजे एवढंच असतं  की,  हवामान, पाऊस, उन्हापासून आपलं रक्षण करणारी जागा, छत. ते एखाद्या माणसासाठी छोटं शेड असू शकतं किंवा काही कुटुंबांसाठी जरा मोठं आश्रयस्थान. त्याला भव्यतेची गरज नाही. हवं तर ते अगदी साधं असू शकतं.

कपड्यांबद्दलही हेच खरे. एक काळ होता जेव्हा कपडे ही केवळ अंग झाकण्याची गरज होती. पण आज कपडे हे इतके गुंतागुंतीच्या पद्धतीने, पुन्हा, “स्टेटस सिम्बॉल” झाले आहेत की त्यांचा खऱ्या गरजेशी संबंध उरलेलाच नाही. खरेतर आपल्याला फक्त शरीर झाकायला काहीतरी हवे. गांधीजींनी दाखवून दिलं होतं,  हाताने कातलेलं साधं कापड पुरेसं आहे. आपण आता सगळे फक्त पंचा परिधान करायला लागणार नाही, हे खरं. पण किमान हे लक्षात ठेवायला हवं की आपल्या कपड्यांपैकी किती गरजेपुरतं आहे आणि किती फक्त कल्पनेतून निर्माण झालंय.

नातेसंबंध या चारही गरजांमध्ये सर्वात गुंतागुंतीचे आहेत. कारण माणसं हे गुंतागुंतीचे “प्राणी” आहेत. आपल्याला ‘आपलेपण’ हवं असतं, इतरांच्या नजरेत चांगले दिसावं असं वाटत असतं, आपल्याला आकर्षक भासायचं असतं. आणि म्हणूनच नातेसंबंध ही भावनांचा, अपेक्षांचा आणि परस्परसंवादांचा एक असा गुंतागुंतीचा तिढा बनतो की तो सहजासहजी सुटतच नाही.

पण हे सर्व खरंच इतकं कठीण असायलाच हवं असं काही बिलकुल नाही. जेव्हा मी एका मित्राला भेटतो. त्या क्षणी उरलेलं जग बाजूला ठेवून मी फक्त त्या भेटीत उपस्थित राहतो. आम्ही बोलतो, विनोद करतो, आणि एकमेकांकडून काही अपेक्षा बाळगत नाही. मग आम्ही निरोप घेतो तेव्हा मनात कुठलीही कटुता किंवा पुन्हा कधी भेटणार याची चिंता राहत नाही. इतकं साधंही नातं असू शकतं.

नक्कीच, माझं लग्न किंवा माझ्या मुलांबरोबरचे नातेसंबंध इतके सरळ नाहीत. पण मी शिकतोय की अपेक्षा कमी केल्या, मागण्या घटवल्या तर जे उरतं ते म्हणजे प्रत्येक नात्याचं खरं सौंदर्य,  प्रत्येक व्यक्तीला जशी आहे तशी स्वीकृती. अजून मी पूर्णपणे इथे पोचलो नाहीये, पण शिकतोय. अनावश्यक अपेक्षांची वजाबाकी केली की जे उरतं ते म्हणजे नात्याचं सार.

समाजाशी आपलं नातंही असंच असतं. नोकरीच्या माध्यमातून आपलं योगदान ठरवलं जातं. पण ती नोकरी आपल्या आयुष्याचा मोठा भाग खाते, आणि ताण, निराशा निर्माण करते. का? कारण आपण बनावट गरजा पूर्ण करण्यासाठी अधिकाधिक तास काम करतो. जर आपण आपल्याच गरजा कमी केल्या आणि कमी गोष्टींमध्ये समाधान मानायला शिकलो, तर खरं म्हणजे फार थोडं काम केलं तरी जगता येतं.

मग उरतो भरपूर वेळ, समाजासाठी साध्या पण महत्त्वाच्या मार्गाने काहीतरी देण्यासाठी. आपण धर्मादाय संस्थांमध्ये स्वयंसेवक म्हणून काम करू शकतो, आपल्या शेजाऱ्यांना मदत करू शकतो, काहीतरी सुंदर निर्माण करू शकतो. आपल्याला हवं तर आपण चांगलं काम करू शकतो आणि ते केल्यावर लगेच सोडून देऊ शकतो,  बक्षीस, कौतुक किंवा मोबदल्याची अपेक्षा न ठेवता. किंवा फक्त उपलब्ध राहू शकतो, म्हणजे इतरांना गरज असेल तेव्हा आपण सतत आपल्या ध्येयांच्या मागे धावत नसू.

शेवटी गोष्ट अगदी सरळ आहे. आपल्या खऱ्याखुऱ्या गरजा अगदी साध्याच आहेत. बाकीचं सगळं आपण स्वतःहून ओढवून घेतलेलं ओझं आहे.


%%%%%%%%%%%%%%%%%%%%%%%%%%%%%%%%%%%%%%%%%%%%%%%%%%%%%%%
 \chapter{आपल्या गरजा कमी करा}
मी आधीच म्हटल्याप्रमाणे, आपल्या खऱ्याखुऱ्या गरजा फारशा नसतात, त्या अगदी साध्या आणि मर्यादित असतात. पण आधुनिक समाजात आपण स्वतःभोवती नवनवीन गरजा जोडत गेलो आहोत. आलिशान घर हवे, भारी कपडे हवेत, महागडी गाडी हवी, गॅजेट्स-कॉम्प्युटर,उंची शौक, नेहमी बाहेर खाणे, परदेशी सहली, करमणूक (एंटरटेनमेंट), परदेशी शिक्षण आणि अजून बऱ्याच गोष्टींसाठी आपल्याला नोकरीची गरज भासते. म्हणजे, मूलभूत गरजा पूर्ण करण्यापेक्षा अधिकाधिक खर्च करण्यासाठीच आपण आयुष्य झिजवत असतो.
पण जर आपण आपल्या गरजा हळूहळू कमी करायला सुरुवात केली, आणि "कमी मिळालं तरी मी आनंदी राहीन" अशी सवय लावली, तर मग जीवन सोपं होऊ लागतं. गरजा कमी म्हणजे खर्च कमी. खर्च कमी म्हणजे कष्ट आणि धडपडही कमी आणि कमी धडपडीमुळे आयुष्यात थोडा श्वास घ्यायला, हसतखेळत राहायला जास्त वेळ मिळतो.
जेव्हा आपल्याकडे कमी गरजा असतात, तेव्हा यशस्वी व्हायचे दडपणही कमी असते. सतत पुढची उंची गाठण्याची हाव लागत नाही. त्यामुळे मन जास्त शांत राहतं, काळजी कमी होते, कारण काळजी करण्यासारखं फारसं उरतच नाही.
गरजा कमी करण्याची ही प्रक्रिया रातोरात साध्य होत नाही. ही सजगपणे (माइंडफुल) आणि हळूहळू करायची गोष्ट आहे. पहिल्यांदा आपले खर्च नीट पहा. आठवड्यात आपण कोणकोणत्या गोष्टींवर किती-किती पैसे खर्च करतो, कोणत्या सवयी जोपासतो, हे स्वतःला विचारा. आणि मग त्यातील कोणत्या गोष्टी खरंच आवश्यक आहेत याचा विचार करा.
थोडं थोडं करून अनावश्यक गोष्टी बाजूला काढायला लागा. उदाहरणार्थ, रोजच्या रोज ‘सी-सी-डी‘ किंवा ‘स्टारबक्स’ मधली महागडी कॉफी खरंच हवी का? की तुम्ही घरी स्वतः छान कॉफी बनवू शकता, किंवा त्याऐवजी साधं पाणी पिऊनही आनंदी राहू शकता? महागड्या फराळाची (स्नॅक्स) खरंच गरज आहे का? की फळं, सुकामेवा यातून अधिक आरोग्यदायी आनंद घेऊ शकता? महागड्या करमणुकीमध्ये (एंटरटेनमेंट) भाग घ्यायलाच पाहिजे का? की आपल्या मुलांसोबत खेळणं, उद्यानात मित्रांसोबत वेळ घालवणं यातूनही तितकाच खरा आनंद मिळू शकतो. खरंच त्या महागड्या जिमचं सदस्यत्व हवं का? की मग जोडीदारासोबत रोज फेरफटका मारणं, बाहेरच पुश-अप्स करणं, सूर्यनमस्कार घालणं पुरेसं ठरू शकतं?
हळूहळू मोठ्या खर्चांकडे पाहायला लागा. खरंच दोन गाड्यांची आवश्यकता आहे का? मोठ्या गाडी (एस-यू-व्ही) ऐवजी एखादी लहान, कमी किमतीची वापरलेली गाडी पुरेशी ठरू शकत नाही का? गाडी सोडून सार्वजनिक वाहतूक म्हणजे बस-रिक्षा (पब्लिक ट्रान्सपोर्ट) किंवा जवळच्या अंतरांसाठी सायकलने काम भागवता येईल का? इतक्या मोठ्या घराची गरज आहे का? की मग कमी किमतीचं, उष्णतेसाठी वा थंडीसाठी कमी खर्चाचं घर पुरेसं होऊ शकतं? शिक्षणाचा खर्च इतका महाग असायलाच हवा का? की मग स्वशिक्षणाने, मोफत उपलब्ध साधनांचा वापर करून आपण ज्ञान मिळवू शकतो?
मी इथे असं सांगत नाही की या सगळ्या गोष्टी तात्काळ सोडून द्या. माझा हेतू एवढाच आहे की आपण स्वतःला प्रश्न विचारावा, खर्च कुठे होतोय हे ओळखावं, आणि हळूहळू अनावश्यक गोष्टी कापत नेऊन फक्त प्रमुख आणि अत्यावश्यक गोष्टींवर लक्ष केंद्रित करावं.
खरंतर, आयुष्यातला खरा आनंद देणाऱ्या गोष्टींसाठी फार खर्च करावा लागत नाही. माझ्यासाठी गरजेच्या आणि मला खऱ्या अर्थाने सुख देणाऱ्या गोष्टी या अगदी साध्या आहेत :
\begin{itemize}
 \item एक चांगलं पुस्तक, जे सहजपणे ग्रंथालयात मिळतं.
 \item लिहिण्यासाठी वही किंवा साधा कॉम्पुटर-लॅपटॉप
 \item रोजचं बाहेर पायी चालणं.
 \item माझ्या पत्नीसोबतचा साधा, पण मनाला उल्हसित करणारा चहा.
 \item माझ्या मुलांसोबत खेळण्यातला निरागस आनंद.
 \item एखाद्या मित्रासोबत धावायला जाणं.
 \end{itemize}
मूलभूत गरजा, म्हणजे अन्न, कपडे, निवारा यांच्यापलीकडे, खरंतर आनंदी राहायला मला एवढंच लागतं. आणि लक्षात घ्या, या सर्वांची आर्थिक किंमत फारच कमी आहे.
गरजा कमी करा. साधेपणात समाधान शोधा. मग बघा, जीवनासाठी लागणारा प्रयत्न, संघर्ष, धडपड कित्येक पटींनी कमी होतो. आयुष्य हलकंफुलकं आणि सुखद होतं.

%%%%%%%%%%%%%%%%%%%%%%%%%%%%%%%%%%%%%%%%%%%%%%%%%%%%%%%
\chapter{हानी करू नका आणि दयाळूपणे जगा}
हा माझ्या आयुष्याचा मूलभूत नियम आहे. तो अगदी साधा दिसतो, पण त्याने मला आतापर्यंत खूप आधार दिला आहे. खरं सांगायचं तर, या नियमामुळे माझं जीवन जरा हलकंफुलकं झालं, भार कमी झाला. त्याचे परिणाम अगदी रोजच्या जगण्यात दिसतात. उदाहरणार्थ, 
\begin{itemize}
 \item नातेसंबंध अधिक सोपे, सुसंवादी आणि समाधान देणारे झाले.
 \item लोक आपोआपच माझ्याशी थोडं अधिक प्रेमळ, सौम्य आणि दयाळूपणे वागतात.
 \item दयाळू व्यक्ती म्हणून लोकांच्या मनात प्रतिमा निर्माण झाली की, समाजात अनेक दारे आपोआप उघडू लागतात.
 \item माझ्या मनात रोजच्या आयुष्यातील समाधानाची आणि आनंदाची पातळी वाढली.
 \item माझ्या आजूबाजूच्या लोकांच्या चेहऱ्यावरही थोडं अधिक हसू उमटलं, कारण आनंद संसर्गजन्य असतो.
 \end{itemize}
सहज जीवन (एफर्टलेस लाईफ)  या तत्त्वज्ञानाचा पहिला आणि सर्वात महत्त्वाचा नियम म्हणजे,  ‘हानी करू नका’. हा नियम पहिल्यांदा मांडला जातो, कारण तो उरलेल्या सर्व नियमांवर छाप टाकतो. धरून चालू की, “घाई करू नका” हा नियम पाळताना कुणाला हानी होणार असेल, तर अशावेळी तुम्ही “घाई करू नका” या नियमाला मागे सारून “हानी करू नका” या नियमाला अग्रक्रम द्यायला हवा.
कारण, जेव्हा आपण कुणाला हानी पोहोचवतो, तेव्हा समस्या ही फक्त त्या क्षणापुरती मर्यादित राहत नाही. ती पाण्यावर दगड टाकल्यावर उठणाऱ्या वलयांसारखी सर्वत्र पसरते. त्यामुळे तुमचं स्वतःचं जीवन गुंतागुंतीचं होतं, आणि ज्यांना तुम्ही त्रास दिला आहे, त्यांचं जीवनसुद्धा कठीण होतं. त्यानंतर तुम्हाला चुका दुरुस्त करण्याची, भरपाई करण्याची आणि क्षमा मागण्याची वेळ येते. हा प्रवास फार लांबणारा आणि नीरस असतो, आणि खरं तर तो आधीच टाळता येण्यासारखा होता.
दैनंदिन जीवनात हे नियम कसे लागू होतात? काही उदाहरणे अशी, 
\begin{itemize}
 \item इतरांना जाणीवपूर्वक मारू नका किंवा त्यांच्यावर हिंसा करू नका.
 \item प्रदूषण करून इतरांच्या आरोग्यावर परिणाम करू नका.
 \item मद्यपान करून गाडी चालवू नका, किंवा बेफिकीरपणे असे काहीही करू नका ज्यामुळे इतरांना इजा होऊ शकेल.
 \item प्राणी किंवा प्राणिजन्य पदार्थ खाऊ नका, कारण त्यातूनही हानीच होते.
 \item इतरांना अन्यायकारक व दडपशाहीच्या परिस्थितीत कामाला लावू नका, किंवा अशा मजुरांच्या, विशेषतः बालमजुरांच्या, श्रमातून निर्माण झालेल्या वस्तूंचा वापर टाळा.
 \item इतरांना संकटात टाकणारी चुकीची किंवा दिशाभूल करणारी माहिती पसरवू नका.
 \item चोरी करू नका, किंवा इतरांचे हक्काचे सामान बळकावू नका.
 \item कुणाच्या जगण्याला आधार देणारी साधने, संसाधने रोखून धरू नका.
 \item तुमच्या डोळ्यासमोर इतरांना त्रास दिला जात असेल तर शांत राहून किंवा निष्क्रिय राहून पाठींबा देऊ नका.
 \item जसं वागणं तुम्हाला स्वतःला नकोसं वाटेल, तसं इतरांशी करू नका.
 \item तुमचे वैयक्तिक विचार, धार्मिक श्रद्धा किंवा मतं इतरांवर जबरदस्ती लादू नका.
 \item खोटं बोलून विश्वास तोडू नका.
 \item खरी गरज नसताना वस्तू विकत घेऊ नका,  कारण प्रत्येक अनावश्यक खरेदीतून पर्यावरणाला हानी होते.
 \end{itemize}
अनेक वेळा “हानी करू नका” हा नियम सोपा नसतो. तुम्हाला बसून विचार करावा लागतो की कोणती कृती (किंवा कृती न करणे) कमी हानीकारक आहे. योग्य निर्णय काढणं नेहमीच सहज होत नाही, पण प्रयत्न मात्र करावा लागतो.
या तत्त्वाचा उजवा, सकारात्मक पैलू म्हणजे,  ‘दयाळूपणे जगा’ (बी कंपॅशनेट). हा नियम प्रत्यक्षात आपल्या विचार करण्याच्या सवयींचं रूपांतर करायला लावतो. उदाहरणार्थ, इतरांशी न्याय करणे. टीका करणे अथवा दोष काढणे याऐवजी, दयाळूपणा म्हणजे इतरांना समजून घेण्याचा प्रामाणिक प्रयत्न करणे. त्यांच्या मनात शिरून पाहणं, त्यांच्या वेदनांशी सहानुभूती दाखवणं आणि त्यांच्या त्रासात मनापासून दिलासा देणं.
दयाळूपणे जगणं हे एवढे मोठे आणि सखोल तत्व-चिंतन आहे की त्यावर स्वतंत्र पुस्तक लिहिता येईल. दलाई लामांचे “आनंदाची कला” (द आर्ट ऑफ हॅपिनेस) हे पुस्तक मी इथे सुचवेन. थोडक्यात सांगायचं तर, दयाळूपणा म्हणजे समजूतदारपणा, सहानुभूती, इतरांचे दुःख हलकं करण्याची इच्छा आणि त्यांच्या जीवनात आनंद वाढवण्याचा सातत्यपूर्ण प्रयत्न. अंगीकारून तर बघा. 

%%%%%%%%%%%%%%%%%%%%%%%%%%%%%%%%%%%%%%%%%%%%%%%%%%%%%%%
\chapter{कोणतेही लक्ष्य किंवा योजना ठरवू नका}
ठराविक, कवेत मावतील एवढीच, साध्य करता येणारी, ध्येये असावीत, ही कल्पना आपल्या संस्कृतीत जणू काही जन्मतःच रोवलेली आहे. अगदी लहानपणापासून आपण ऐकत आलो आहोत की “ध्येयाशिवाय जीवन म्हणजे दिशाहीन प्रवास.” मी स्वतः अनेक वर्षे ध्येयांच्या मागे धावत राहिलो. खरं सांगायचं तर माझ्या आधीच्या लिखाणाचा मोठा भाग हा ध्येये कशी ठरवायची, त्यांची आखणी कशी करायची आणि ती पूर्ण करण्यासाठी काय करायला हवं याभोवतीच फिरत होता.
परंतु आजकाल माझं जीवन पूर्णपणे वेगळ्या वाटेने चाललयं. आता मी प्रामुख्याने ध्येयांशिवाय जगतो आणि गंमत म्हणजे, त्यात मला एक विलक्षण स्वातंत्र्य आहे असं वाटतंय. लोकांना नेहमी शिकवलं जातं की ध्येये न ठेवल्यास आयुष्य हातातून निसटून जाईल. पण माझा अनुभव अगदी उलट आहे. ध्येये नसली तरी तुम्ही काहीना काही साध्य करणं कधी थांबवत नाही; उलट, तुम्ही स्वतःला कृत्रिम मर्यादांमध्ये अडकवणं थांबवता.
लोकप्रिय समजुतीत असं मानलं जातं की,  “जर तुम्हाला कुठे जायचं आहे हे ठाऊक नसेल, तर तुम्ही कुठेच पोहोचणार नाही”. पहिल्यांदा ऐकल्यावर हे अगदीच तर्कसंगत वाटतं. पण जर शांत बसून विचार केला, तर ही गोष्ट किती उथळ आहे हे लक्षात येतं. एक साधा प्रयोग करून बघा. घराबाहेर पडा आणि एखाद्या निव्वळ सहज सुचलेल्या दिशेने चालत राहा. चालताना मनात आलं की दिशा बदला. वीस मिनिटं, किवा एखादा तास असे भटकून बघा. तुम्ही नक्कीच कुठेतरी पोहोचाल. फरक फक्त इतकाच की तुम्हाला आधीपासून ठाऊक नसेल की तुम्ही नेमकं कुठे उतरणार आहात. आर्टिफिशील इंटेलिजन्स मधील ‘रिइन्फोर्समेंट लर्निंग’ या शाखेत त्यास शोधभ्रमण (एक्सप्लोरेशन) म्हणतात. त्यांची अचंभित करणारी उत्तरे सापडू शकतात. 
यातूनच खरा धडा मिळतो. तुमचं मन खुलं ठेवलं, तर तुम्ही कधीच न कल्पिलेल्या ठिकाणी जाऊन पोहोचता. ध्येयांशिवाय जगलात, तर नवीन प्रदेशांची सफर करता. अनपेक्षित शिकवणी मिळते. नवनवीन अनुभवांच्या धबधब्यात तुम्ही स्वतःला झोकून देता. जिथे जायची कधी कल्पना केली नसेल, तिथे स्वतःला वेगळ्या रूपात गवसलेलं बघता. हे या विचारसरणीचं खरं सौंदर्य आहे. पण खरं सांगायचं तर, हा बदल स्वीकारणं अवघड असतं. कारण ध्येय ठरवण्याची सवय ही आपल्या अंगात पिढ्यान्पिढ्या रुजलेली आहे.
आज मी बहुतेक वेळा ध्येयांविना जगतो. कधीतरी नकळत एखादं ध्येय डोक्यात येतं. पण मी ते धरून बसत नाही; मी त्याला जाऊ देतो. खरं म्हणजे, ध्येयांशिवाय जगणं हे सुद्धा माझं कोणतं तरी ठरवलेलं ध्येय नव्हतंच. ही तर एक जीवनशैली आहे जी मला चालता चालता सापडली आहे. यात एक गोड दिलासा आहे, एक मुक्तता आहे, आणि सगळ्यात महत्त्वाचं म्हणजे,  यात मला माझ्या आवडीचं काम मनसोक्त करता येतं.

\section*{तीन महत्त्वाच्या नोंदी}

अनेकांना माझ्या “लक्ष्य नाही” (नो गोल्स) या प्रयोगाबद्दल आक्षेप असतो. लोकांना हा विचार फार पटत नाही किंवा फारच अतिरेकी वाटतो. म्हणूनच, या प्रयोगामागचा विचार स्पष्ट होण्यासाठी, आधी तीन महत्त्वाच्या नोंदी मांडतो.

\subsection*{पहिली नोंद ही “लक्ष्य” या शब्दाच्या व्याख्येबद्दल आहे.}  

मी लक्ष्य म्हणजे “काहीही करायची इच्छा” असे मानत नाही. इच्छा असणे हे माणसाच्या जगण्याचा नैसर्गिक भाग आहे. इथे माझा मुद्दा हा आहे की आपण आधीच डोक्यात ठरवलेले परिणाम, म्हणजेच “पूर्वनिर्धारित परिणाम” किंवा ठराविक गंतव्यस्थान (डेस्टिनेशन), यांना सोडून द्यावे. उदाहरण घ्या. तुम्ही चालायला लागलात, पण कुठे जायचे हेच माहीत नाही. तरीही तुम्ही म्हणता, “मी चालण्याचे लक्ष्य ठेवले आहे!”. पण खरं पाहता, त्या चालण्यात ठराविक ठिकाण नाही. उलट, जर तुम्ही दुकानात जाण्यासाठी चालायला निघालात, तर तो प्रवास स्पष्ट लक्ष्य असलेला आहे. त्यामुळे जेव्हा लोक म्हणतात, “तुम्ही काहीतरी करत आहात, म्हणजे तुमची लक्ष्ये आहेतच!”, तेव्हा माझे उत्तर असते, “होय, मी करत आहे, पण मला ते कुठे नेईल याची मला कल्पना नाही, आणि प्रामाणिकपणे सांगायचे तर मला त्याची चिंता नाहीसुद्धा.” आणि हे सांगून ठेवतो,  हे “गॉचा सिंड्रोम”  नावाच्या वृत्तीचे लक्षण आहे. यात लोक शिफारसी स्वतः वापरून बघण्याऐवजी माझ्या म्हणण्यातल्या लहानमोठ्या विरोधाभासांवर बोट ठेवण्याचा खेळ खेळतात.

\subsection*{दुसरी नोंद अशी की, तुम्हाला हा प्रयोग करून पाहण्याची अजिबात गरज नाही.}

जर लक्ष्यांशिवाय जगणे तुम्हाला अगदी हास्यास्पद, अतिरेकी किंवा “जगण्याला अर्थ नाही” असे काहीसे वाटत असेल, तर निश्चिंत रहा,  हा प्रयोग तुमच्यासाठी नाही. याबाबत तुम्ही माझ्याशी असहमत असलात तरी त्यात मला काही फरक पडत नाही. हा प्रयोग माझ्यासाठी काम करतो, पण कदाचित तुमच्यासाठी अजिबात करणार नाही. आणि तेही अगदी ठीक आहे. या पुस्तकात इतर बर्‍याच गोष्टी आहेत ज्या तुमच्यासाठी उपयुक्त ठरू शकतात. आणि कोण जाणे, कदाचित भविष्यात एखाद्या दिवशी तुम्ही या विचाराकडे परत याल, आणि त्याकडे वेगळ्या नजरेने पाहाल.

\subsection*{तिसरी आणि शेवटची नोंद, मला सुरुवातीला लक्ष्यांची खरंच गरज होती का?}

अनेक लोक म्हणतात, “तुला आता लक्ष्यांची गरज नाही, कारण तू आधीच बरंच काही साध्य करून बसला आहेस. जेव्हा माणूस एवढा पुढे गेला, तेव्हाच तो ‘लक्ष्य नाही’ म्हणू शकतो.” हे म्हणणे ऐकायला वाजवी वाटते. तुम्हाला हवे असल्यास तुम्ही हा मुद्दा मान्य करून बसू शकता. पण एक पर्याय असा आहे की,  एकदा स्वतःच प्रयत्न करा. एकदा जगून बघा, काम करून बघा, अगदी लक्ष्ये न ठेवता. आणि मग बघा काय होते ते.
मला स्वतः सुरुवातीला खरंच लक्ष्यांची गरज होती का? हे आता मागे वळून कसे तपासणार? भूतकाळाकडे जाऊन प्रयोग करून बघणे शक्य नाहीच. पण माझा अंदाज असा आहे की जर मी अगदी सुरुवातीपासून हा “लक्ष्यांशिवाय” प्रयोग केला असता, तर मी आज जिथे आहे तिथे पोहोचणार नाही कदाचित. पण याचा अर्थ असा नाही की मी वाईट ठिकाणी असतो. उलट, मी कुठेतरी दुसर्‍या सुंदर, समाधानकारक ठिकाणी असतो. जगण्याच्या प्रवासात शेवटचं ठिकाण मोजण्यापेक्षा चालण्याचा अनुभवच जास्त महत्त्वाचा असतो.


\section*{लक्ष्यांची समस्या}
भूतकाळात, मी वर्षाची सुरुवात होताच एखादे मोठे लक्ष्य किंवा किमान दोन-तीन तरी ठरवायचो. मग त्या वार्षिक ध्येयाला गाठण्यासाठी महिन्यागणिक लहान उप-लक्ष्ये आखली जात. पुढे त्याचे अजून विघटन करून आठवड्यागणिक, आणि शेवटी दिवसागणिक कृतीच्या पायऱ्या तयार होत. त्या कृतींवर माझा दिवस केंद्रित करण्याचा मी जीव तोडून प्रयत्न करायचो.
पण खरे सांगायचे झाले तर हे गणित इतक्या सरळसोटपणे कधीच जुळून येत नाही. हे सगळ्यांनाच ठाऊक आहे. आयुष्यात अचानक काही कामे घुसतात, वेळेचा ताण येतो, कधी टाळाटाळ तर कधी आळस हावी होतो. आणि मग होते काय, मासिक आणि साप्ताहिक लक्ष्ये मागे सरकतात, थांबून राहतात किंवा पूर्णपणे रुळावरून घसरतात. त्यातून मनात निराशा घर करून बसते कारण आपल्यात शिस्त नाही, आत्मसंयम नाही, असे वाटू लागते. मग आपण पुन्हा एकदा बसतो, आपली जुनी ध्येये उघडतो, त्यांचा आढावा घेतो, काहीतरी सुधारणा करून नवी ध्येये ठरवतो. उप-लक्ष्यांची आणि क्रियापद्धतीची ताजी यादी तयार होते.
कधीतरी एखादे ध्येय खरंच गाठले जाते आणि त्या क्षणी आपण स्वतःला हिरो समजतो, अप्रतिम वाटते. पण वस्तुस्थिती अशी की बहुतांश वेळा ध्येये गाठली जात नाहीत आणि त्याचे ओझे आपण स्वतःच्या डोक्यावर घेतो. चूक आपलीच आहे, आपणच निकामी ठरलो, अशी अपराधी भावना वाढत जाते.
खरे गुपित येथे आहे: समस्या तुमच्यात नाही, समस्या या संपूर्ण यंत्रणेच्या (सिस्टीम) रचनेत आहे. लक्ष्य-केंद्रित व्यवस्था ही प्रत्यक्षात अपयशासाठीच घडवलेली आहे.
कारण जरी तुम्ही पुस्तकात लिहिल्याप्रमाणे, नियमाप्रमाणे, अगदी बिनचूक वागलात तरीही ही प्रणाली आदर्श नाही. ध्येये तुमच्या क्षमतेभोवती भिंत उभी करतात, संधी कमी करतात. कधी तुम्हाला एखादी गोष्ट मनापासून करायची इच्छा नसते, पण फक्त ध्येय ठरवल्यामुळे तुम्हाला स्वतःला ओढूनताणून ती करावीच लागते. तुमचा मार्ग आधीच आखलेला असल्याने नवीन संधींचा शोध घेण्याची जागा उरत नाही. योजना ठरवलेली असल्याने तिचे पालन करावेच लागते, जरी त्या क्षणी तुमच्या मनाचा ओढा एखाद्या वेगळ्याच विषयाकडे असेल तरी.
काही ध्येय-प्रणाली लवचिक असतात, यात शंका नाही. पण खरी गोष्ट अशी आहे की ध्येयरहित आयुष्याइतके मोकळे, लवचिक आणि विस्तारासाठी उघडे काहीच नाही.
 \section*{लक्ष्यांशिवाय जगणे}
तर, लक्ष्यांशिवायचे जीवन नक्की कसे दिसते? व्यवहारात ते लक्ष्यांवर आधारलेल्यापेक्षा पूर्णपणे वेगळे असते. साधारणपणे आपल्याला लहानपणापासूनच “वार्षिक लक्ष्य”, “महिन्याचे उद्दिष्ट”, “आठवड्याचे प्लॅन” आणि अगदी “दिवसाचे टू-डू (कामांची सूची)” अशी यादी ठेवायला शिकवले जाते. पण इथे चित्र उलट आहे, तुम्ही वर्षासाठी ध्येय ठरवत नाही, महिन्यासाठी मापदंड आखत नाही, आठवड्याचा अजेंडा बांधत नाही आणि रोजची यादी तयार करत नाही.
तुम्ही तपासणी (ट्रॅकिंग) या संकल्पनेचा मागोवा घेत बसत नाही, प्रत्येक कृतीला टप्प्याटप्प्याने मोजत नाही. अगदी टू-डू लिस्ट हवीच असे बंधनसुद्धा नाही. हो, आवडेल तर थोड्या आठवणी वहीत लिहून ठेवल्या तरी काही बिघडत नाही. पण त्या न लिहिल्या, तरी आयुष्य विस्कळीत होत नाही.
मग प्रश्न येतो, असं असताना दिवस कसा गेला म्हणायचा? तुम्ही दिवसभर सोफ्यावर लोळत बसता का? अजिबात नाही. तुम्ही जे काही खऱ्या अर्थाने आतून आवडते, ज्याबद्दल आतून ज्वाला पेटते ते शोधता आणि मग ते करता. “ध्येय नाही” म्हणजे “काहीही काम नाही” असं नव्हे. उलट, यातूनच नव्या निर्मितीला वाव मिळतो. तुम्ही काहीतरी घडवू शकता, नवे तयार करू शकता, आपल्या आवडीच्या दिशेने वाहू शकता.
हे प्रत्यक्ष अनुभवात अधिकच अद्भुत होते. सकाळी जागे झाल्यावर नेमके ध्येयाच्या चौकटीत न अडकता तुम्ही थेट आपल्या आवडीकडे वळता. माझ्यासाठी ते बहुधा लेखन असते. पण हे फक्त लेखनापुरते मर्यादित नाही. ते इतरांना मदत करणे असू शकते, अद्वितीय लोकांशी संबंध जोडणे असू शकते, पत्नीबरोबर वेळ घालवणे असू शकते, किंवा मुलांबरोबर खेळण्यात रमणे असू शकते. संधी आणि शक्यता या अमर्याद आहेत, कारण मी मोकळा आहे, बांधलेला नाही.
विस्मयकारक गोष्ट म्हणजे शेवटी मी बहुतेक वेळा लक्ष्य ठरवले असते त्यापेक्षा जास्त साध्य करतो. कारण रोजच मी अशा गोष्टी करत असतो ज्याबद्दल मी उत्साही असतो, जिवंत वाटतो. पण खरं सांगायचं तर “साध्य करणे” हे अंतिम ध्येय नाहीच. खरी गोष्ट म्हणजे मी सतत मला आवडणारे करतोय, हेच मोलाचे आहे.
अशा वाटा मला नेहमीच अप्रतिम, आश्चर्यकारक आणि आनंददायी ठिकाणी पोहोचवतात. गंमत म्हणजे सुरुवातीला मला त्या ठिकाणी पोहोचायचे आहे हे कधीच माहीत नसते. पण शेवटी ते ठिकाण समोर उभे राहते आणि हसू येते, "वा! हे कुठून आलं?"
म्हणून, तुम्ही कोणताही मार्ग धरला, कुठेही पोहोचलात तरी ते सुंदरच असते. कुठलाही मार्ग वाईट नसतो, कुठलाही शेवट निरर्थक नसतो. प्रत्येक मार्ग फक्त वेगळा असतो, आणि \textbf{वेगळेपण म्हणजेच अद्भुतता}.
न्याय करू नका. तुलना करू नका. फक्त त्या क्षणाचा अनुभव घ्या.
आणि शेवटी नेहमी लक्षात ठेवा, प्रवास हेच सर्वस्व आहे. गंतव्य हे फक्त तात्पुरते एक ठिकाण आहे.
%%%%%%%%%%%%%%%%%%%%%%%%%%%%%%%%%%%%%%%%%%%%%%%%%%%%%%%
\chapter{कसल्याच अपेक्षा नकोत }
आपल्या ताणाचा, निराशेचा, रागाचा आणि चिडचिडीचा मूळ उगम अनेकदा एखाद्या अगदी लहान घटनेत लपलेल्या अपेक्षांमुळेच असतो. आपण मनात काही ठरवून ठेवतो आणि ते तंतोतंत तसे घडले नाही तर मनाला दु:ख, अस्वस्थता आणि चिडचिड वाटते; या साऱ्याचे मूळ हेच अपेक्षा असतात, हे लक्षात घेतले पाहिजे. जर आपण अपेक्षा ओळखल्या आणि त्यांना व्यवस्थित हाताळले नाही तर त्या अपेक्षा काळानुसार मोठ्या आक्रामकतेने करतात आणि अंतःकरणात सतत अस्वस्थता निर्माण करतात.
आपण मनात वेगवेगळ्या अपेक्षांची चित्रं रंगवतो: लोकांनी कसे वागावे, आपले जीवन कसे असावे, रस्त्यावर कोणती गाडी कशी चालवावी, असे अनेक प्रकारचे कल्पनिक नियम मनाने तयार करतो. परंतु या अनेक कल्पना आणि बांधलेल्या नियमांचे बहुतेक भाग फक्त कल्पनाविलास आहेत; ते वास्तविकतेशी जुळत नाहीत. वास्तविकता आणि कल्पनेतला हा ताळमेळ न बसल्यास मन हेच मान्य न करणे सुरू ठेवते आणि ‘‘जग असंचं नसतं तर किती चांगले झाले असते’’ असा विचार सतत उठत राहतो.
या अवास्तव अपेक्षांप्रमाणेच आपला अंतःकरण ओझे बनते; अशा वेळी एक सोपा, कल्पक आणि प्रभावी उपाय प्रयोगात आणता येऊ शकतो. त्यासाठी असा प्रतिकात्मक प्रयोग करा: आपल्या मनात असणाऱ्या सर्व अपेक्षा एकत्र गुंडाळून प्रतीकात्मकरीत्या पाण्यात विसर्जित करण्याचा विचार करा. स्वतःच्या आयुष्याविषयी, स्वतःविषयी, जोडीदाराविषयी, मुलांविषयी, कार्यालयातील सहकाऱ्यांविषयी (ऑफिस) किंवा नोकरीविषयी इत्यादी सर्व अपेक्षांची कल्पना करा, त्यांना एकत्र करा आणि जवळच्या समुद्रात, नदीत किंवा तलावात प्रतीकात्मकपणे सोडून देण्याचा निर्णय घ्या.
हा प्रतिकात्मक क्रियाकलाप अगदी सोपा आहे: तुम्ही त्या अपेक्षा पाण्यावर तरंगताना पाहता; लाटा त्या अपेक्षांना हळूहळू वाहून नेतात; प्रवाहाबरोबर त्या अपेक्षा दूर जातात आणि शेवटी अदृश्य होतात. हे प्रत्यक्ष पाण्यात इंधन टाकण्यासारखे काही नसून, मनाचे एक व्यायाम आहे,  अपेक्षांना पाहून ओळखणे, त्यांना स्वतःवर ओढवून न घेणे, आणि हळूवारपणे सोडणे. या प्रक्रियेत स्वच्छ पाणी आपले प्रतीक आहे,  ते आपली आंतरिक शुद्धता आणि स्वीकारण्याची क्षमता वाढवते; आपण फक्त त्या अपेक्षांना सोडतो आणि पुढे चालतो.
या प्रयोगानंतर अपेक्षाविरहित जीवन कसे वाटते हे जाणून घेण्याचा प्रयत्न करा. अपेक्षांशिवाय आयुष्य अधिक हलके आणि शांतभरलेले वाटते कारण आपल्याला जग जसा आहे तसा स्वीकारता येतो. माणसांची वागणूक त्यांची आहे असे समजून घेतल्यावर आपण त्यांना आपल्या अपेक्षांच्या चौकटीत बसवण्यासाठी दबाव देत नाही; घटना जशा घडतात तशा स्वीकारण्याची क्षमता वाढते. त्यामुळे निराश होण्याची एक प्रमुख कारणे स्वतःपासूनच कमी होतात आणि राग, चीड किंवा मानसिक त्रास आल्यास ते अनुभवूनही त्यांना पकडून ठेवण्याची आवश्यकता कमी होते.
हे अपेक्षांविरहित राहून निवांत जीवन म्हणजे सुटकेचा सुगम मार्ग असा अर्थ नाही की आपण कृती करणे थांबवावे. उलट, आपल्या मूल्यांनुसार आणि विवेकाच्या आधारे आपण निश्चितपणे कृती करतो आणि समाजावर, कुटुंबावर, कार्यक्षेत्रावर सकारात्मक परिणाम घडवण्याचा प्रयत्न करतो. परंतु या कृतींचा परिणाम किंवा जगाने दिलेला प्रतिसाद कसा असेल, हे ठरवण्याचे काम आपण स्वतःस देऊ नये. उदाहरणार्थ, आपण जर अधिक चांगले कार्य केले तर आपल्याला प्रशंसा (प्रेझ) किंवा कौतुक (ऍप्रिसिएशन) मिळायला हवे अशी अपेक्षा ठेवली तर ती अपेक्षा पूर्ण न झाल्यास पुन्हा निराशा निर्माण होऊ शकते. म्हणून चांगले काम फक्त त्या कर्माच्या आनंदासाठी, कर्तव्यबोधासाठी आणि मूल्यांसाठी करावे; त्यापलीकडे काहीही मागत बसण्याची सवय न बाळगावी.
आपल्या विचारांकडे आणि मनाच्या प्रवाहाकडे सातत्याने लक्ष देणे आवश्यक आहे. जेव्हा मनात पुन्हा कोणतीही अपेक्षा जन्म घेते, तेव्हा स्वतःला दोष देण्याऐवजी ती अपेक्षा ओळखा, पहा आणि सौम्यपणे ती दूर करण्यासाठी प्रयत्न करा. एक साधी पद्धत म्हणजे हसत-हसत त्या अपेक्षेचे प्रतिकात्मक विसर्जन करणे,  ‘‘ही माझी अपेक्षा आहे’’ असे जाणीवपूर्वक मान्य करून तिला वाहून जाण्याचा मार्ग दाखवणे. या प्रक्रियेत स्वतःवर क्रूर न होता, शांतपणे आणि सुसंयमाने वागायला शिका.
जर तुम्हाला असे जाणवले की मनात सतत ‘‘हे असं नाही तर तसं असावे’’ या शैलीचे विचार येत आहेत, तर सावध राहा. विशेषतः कोणी व्यक्ती तुमच्या अपेक्षांच्या विरोधात वागल्यास ते लक्षात घ्या की तो त्या व्यक्तीचा वर्तन आहे; ते तुमची बनवलेली अपेक्षा आहे. अशा वेळेस त्या अपेक्षांनाही सोडून देण्याचे प्रशिक्षण उपयुक्त ठरते. वास्तविकतेला स्वीकारताना पुढे चालण्याचा दृष्टिकोन आत्मसात केल्यास वैयक्तिक सुखशांती वाढते आणि जीवन अधिक स्थिर बनते.
निसर्गातील वाहते पाणी धूळ, मळ आणि अशुद्धता वाहून नेते; त्याचप्रमाणे आपल्या मनातील जुन्या, अवास्तव आणि ओझेभरलेल्या अपेक्षाही आपण सोडून दिल्यास सहज वाहून जातील. फक्त पाणी मिळविण्याइतकाच ती अपेक्षा सोडण्याची तयारी व संयम आवश्यक आहे. एकदा आपण या प्रक्रियेत पारंगत झालो की आपण आधीच सुंदर आणि अद्भुत असलेल्या या जगात अधिक हलक्या मनाने, अधिक मोकळेपणाने आणि कोणत्याही अवास्तव कल्पनांच्या ओझ्याशिवाय जगू शकतो.
%%%%%%%%%%%%%%%%%%%%%%%%%%%%%%%%%%%%%%%%%%%%%%%%%%%%%%%
\chapter{नियंत्रणाचा भ्रम}
जेव्हा आपल्याला असे वाटते की आपल्या हातात सर्व काही नियंत्रणात आहे, तेव्हा प्रत्यक्षात आपण एक अतिशय मोठ्या गैरसमजात अडकलेलो असतो. ही मानवी प्रवृत्ती खरोखरच विलक्षण आहे, आपण किती वेळा स्वतःला पटवतो की आपणच काहीतरी नियंत्रणात ठेवले आहे. पण खरी गोष्ट अशी आहे की नियंत्रण हा केवळ एक भ्रम आहे, जणू मृगजळाप्रमाणे हातात येतच नाही.
आपण दररोज योजना आखतो, स्वप्ने रंगवतो, नीट आखीवरेखीव योजना करतो, पण शेवटी काय घडते? गोष्टी आपल्या मनात जशा रचलेल्या असतात तशाच घडतात का? अजिबात नाही. म्हणूनच जुनी म्हण आहे: “देवाला हसवायचं असेल, तर योजना करा” (Make God laugh, make plans). आपण आपल्या योजनांवर मेहनत करतो, ध्येय गाठण्यासाठी घाम गाळतो, आणि तरीही ध्येय हुकते. मग आपण कितीदा भविष्यातील घटनांना आपल्या हातात बसवण्याचा प्रयत्न करतो, जेव्हा की ते भविष्य आपल्या हुकुमातच नसते.
पाच वर्षांपूर्वी कुणाला कल्पना होती का की जग इतक्या वेगाने बदलेल, अमेरिकेत बराक ओबामा राष्ट्राध्यक्ष होतील, शेअर बाजार (स्टॉक मार्केट) कोसळेल, मंदी येईल, प्रचंड भूकंप आणि समुद्री महाभयंकर लाटांचा तडाखा (त्सुनामी) बसेल, आणि आपण आज जे करत आहोत तेच करत बसू? अर्थातच नाही. भविष्य आपल्यासाठी नेहमीच अंधारात लपलेले असते; त्याला जाणून घेणंही कठीण, आणि नियंत्रित करणं तर अजूनच अशक्य. तरीदेखील आपल्याला वाटतं की आपण ते पेलू शकतो, पण वेळोवेळी आपण फसलो जातो.
तरीही आपण या नियंत्रणाच्या भ्रमाला घट्ट धरून ठेवतो. जग हे गुंतागुंतीचे, अनिश्चित आणि अनेकदा गोंधळलेले आहे. आपण त्याला वश करण्यासाठी जी किमया करतो तीच आपल्याला थकवून टाकते.
आपण नियंत्रणाचा हा हट्ट अनेक प्रकारे दाखवतो. उदाहरणार्थ:
\begin{itemize}
\item आपण आपल्या मुलांना आपल्या मर्जीप्रमाणे घडवण्याचा आटापिटा करतो. आपण समजतो की ती मातीचे पुतळे आहेत आणि आपण शिल्पकार आहोत. पण खरं तर माणसं ही अत्यंत गुंतागुंतीची असतात, त्यांच्या स्वभाव, विचार आणि स्वप्नांची गुंफण आपल्याला कधीच नीट कळत नाही.
\item आपण आपल्या रोजच्या जगण्यातल्या छोट्यातल्या छोट्या गोष्टींची नोंद ठेवण्याचा प्रयत्न करतो. जसे की खर्च, व्यायाम, खाणं, केलेली कामं, वेबसाइटवर येणारे अभ्यागत, चाललेली पावलं, धावलेले मैल. ही यादी कधी संपतच नाही. आणि आपण समजतो की हे सर्व ट्रॅकिंग म्हणजेच खऱ्या परिणामांवर नियंत्रण आहे, पण ते खरे नसते.
\item आपण आपल्या कर्मचाऱ्यांना किंवा सहकाऱ्यांना आपल्या मनासारखं चालवण्याचा हट्ट धरतो. पण त्यांच्या वेगवेगळ्या स्वभाव, प्रेरणा, लहरी, सवयी आपल्याला पूर्णपणे समजतातच कुठे?
\item आपण प्रकल्प, प्रवास, पार्टी, अगदी रोजचा दिवसही वेडसरपणे नियोजित करतो, जणू कार्यक्रमांचे शेवटचे परिणाम आपल्या हातात आहेत.
\end{itemize}
पण खरं सांगायचं तर, हा सगळा खेळ म्हणजे फक्त एक भ्रम आहे. जर आपण याचा त्याग केला, तर मग उरते काय? या गोंधळलेल्या जगात मग जगायचं कसं?
यासाठी एका साध्या उदाहरणाचा विचार करा. माशाचा. समुद्र कितीही वादळी, प्रचंड आणि अव्यवस्थित असला तरी मासा पोहतच राहतो. त्याला हा भ्रम नसतो की तो समुद्राला नियंत्रित करू शकतो किंवा इतर माशांना वश करू शकतो. तो फक्त प्रवाहासोबत चालतो, येणाऱ्या लाटांना सामोरं जातो. तो खातो, लपतो, प्रजनन करतो, पण कुठे थांबायचं किंवा कुठे पोहोचायचं याचा अट्टाहास करत नाही.
आपण त्या माशापेक्षा अधिक शहाणे नाही. उलट आपलं विचारचक्रच आपल्याला या नियंत्रणाच्या मृगजळात अडकवतं. म्हणूनच एकच उपाय आहे,  हा विचार बाजूला ठेवा. मासा व्हायला शिका.
जेव्हा आपण गोंधळाच्या वावटळीत अडकतो, तेव्हा त्याला वश करण्याची गरजच सोडून द्या. त्या क्षणाचा अनुभव घ्या, त्यात बुडून जा. प्रवाहावर ताबा मिळवायचा प्रयत्न करू नका, फक्त त्याला प्रतिसाद द्या.
असं जगणं म्हणजे पूर्णपणे वेगळं तत्त्वज्ञान आहे. उदाहरणार्थ:
\begin{itemize}
\item ध्येयांच्या मागे धावणं थांबवा. त्याऐवजी ज्या गोष्टीत मन रमते, ज्या गोष्टीत हृदय धडधडतं, त्या गोष्टी करा.
\item नियोजनाच्या फाईली भरत बसू नका. त्याऐवजी थेट कृती करा.
\item भविष्यातील चित्र रंगवण्यापेक्षा वर्तमान क्षणात जगायला शिका.
\item इतरांना नियंत्रित करण्याचा हट्ट सोडून द्या. त्याऐवजी प्रेम, दयाळूपणा आणि आपुलकी दाखवा.
\item आपण ज्या मूल्यांना जपतो, त्यावर विश्वास ठेवा. परिणामांच्या हव्यासापेक्षा कृतीत मूल्यं जगणं अधिक महत्वाचं आहे.
\item प्रत्येक पाऊल हलक्या मनाने, संतुलनाने, वर्तमान क्षणात टाका. पुढची हजार पावलं (हजार पावले — thousand steps) नियोजित करण्याऐवजी मनाशी ज्या गोष्टी जुळतात त्यांना मान द्या.
\item जगाला जसं आहे तसं स्वीकारा. त्याच्यामुळे त्रास, राग किंवा ताण न घेता.
\item कधीही निराश होऊ नका. कारण तुम्ही अपेक्षाच ठेवल्या नाहीत. जे येतंय ते हसत स्वीकारा.
\end{itemize}
काहींना हा जीवनमार्ग निष्क्रिय किंवा सुस्त वाटेल. कारण आपली संस्कृती आपल्याला आक्रमक, ध्येयवादी आणि परिणामाभिमुख राहायला शिकवते. म्हणून जर हा मार्ग तुम्हाला पटत नसेल, तर हरकत नाही. बरेच लोक आयुष्यभर या नियंत्रणाच्या भ्रमासोबतच जगतात. अज्ञानामुळे त्यांना जे दुःख मिळतं ते त्यांना चालून जातं.
पण जर तुम्ही हे तत्त्व अंगीकारायला शिकलात, तर हा अनुभव जगातील सर्वात मोठा दिलासा ठरतो आणि खरीच मुक्ती देतो.

%%%%%%%%%%%%%%%%%%%%%%%%%%%%%%%%%%%%%%%%%%%%%%%%%%%%%%%
 \chapter{गोंधळासोबत जगणे}
आपण आतापर्यंत जीवनातील ध्येये, आखलेल्या योजना आणि आधीपासून उभ्या केलेल्या अपेक्षा यांना कसे सोडून द्यावे, याबद्दल विचार केला आहे. पण खरी गंमत इथेच आहे. माझा अजूनही चाललेला अभ्यास हा आहे की, जेव्हा आपण नियंत्रणाच्या भ्रमाला पूर्णपणे सोडून देतो आणि फारशी योजना न आखता गोष्टी नैसर्गिकरीत्या घडू देतो, तेव्हा पुढचे पाऊल कसे टाकायचे हा प्रश्न समोर येतो.
ध्येय किंवा योजनांशिवायचे जीवन म्हणजे नक्की काय असते? आपण दररोज ज्या गोंधळाशी झगडत असतो, त्याच्याशी आपली मैत्री कशी करावी? हा प्रश्न खूप महत्त्वाचा आहे, कारण गोंधळ टाळता येत नाही, त्याच्याशी राहणं शिकावं लागतं.
माझ्याकडे या प्रश्नांची सर्वसमावेशक उत्तरे नाहीत. खरं सांगायचं तर अशा प्रकारच्या प्रश्नांची ठाम उत्तरे कोणीही देऊ शकणार नाही. पण एक गोष्ट मात्र निश्चित आहे की, मी हळूहळू खूप काही शिकत आहे आणि अनुभवत आहे. हे शिकणे म्हणजे केवळ सिद्धांत नव्हे, तर प्रत्यक्ष जगण्यात आलेले अनुभव आहेत.
अलीकडेच मी "वर्ल्ड डॉमिनेशन समिट" (जागतिक वर्चस्व परिषद) या कार्यक्रमाला पोर्टलँड (Portland) या शहरात गेलो होतो. आश्चर्याची गोष्ट म्हणजे, तिथे जाताना मी नेहमीसारख्या खूप योजना केलेल्या नव्हत्या. फक्त काही ठराविक गोष्टी होत्या. मला एक भाषण द्यायचे होते, दोन-तीन छोट्या सत्रांचे संचालन करायचे होते, तसेच एक सायकल फेरफटका (बाईक टूर) ठरलेला होता. याशिवाय विमानाचे तिकीट आणि हॉटेलची खोली आरक्षित होती. पण बाकीच्या आठवड्याचा बहुतांश भाग मी जाणीवपूर्वक मोकळा ठेवला होता.
हा अनुभव किती वेगळा ठरला हे सांगायलाच हवे. भाषण देणे मजेशीर होतेच, टूरदेखील मनापासून आवडला. पण खरी जादू अनपेक्षित भेटींमध्ये आणि घडामोडींमध्ये दडलेली होती. अगदी अनोळखी लोकांशी अचानक संवाद साधणे, कधीच न भेटलेल्या व्यक्तींशी वेळ घालवणे, आणि गर्दी ज्या दिशेने वाहते त्या प्रवाहात स्वतःला सहजतेने सोडून देणे, हे सारे अनुभव मनाला ताजेतवाने करणारे ठरले. पुढच्या क्षणी काय होणार आहे याचा मला अजिबात अंदाज नव्हता. हे नक्कीच थोडे भीतीदायक होते, परंतु त्याचवेळी एक अजब मुक्तीचा अनुभव देणारे होते.
याच धर्तीवर, नुकतेच मी गुआम (Guam) या बेटावर एक महिना घालवला. तिथे माझे अनेक मित्र आणि नातेवाईक होते, त्यांच्याशी भेट झाली. पण राहण्याची जागा सोडली तर माझ्याकडे कोणतीही ठोस योजना नव्हती. आम्ही कसा प्रवास करणार, दररोज काय करणार, याबद्दल काहीच निश्चित ठरलेले नव्हते. पहिल्या काही दिवसांत हे अनिश्चितपणं थोडं भीतीदायक वाटलं. पण जसजसा वेळ गेला तसतसे जाणवले की, आपण काहीही ठरवलं नाही तरी गोष्टी घडतात, आपण आपोआप जुळवून घेतो, आणि अखेरीस सगळं छान पार पडतं.
मग मोठा प्रश्न उरतो, गोंधळासोबत खरंच कसं जगायचं?
 त्याचं उत्तर इतकंच आहे की, आपण त्या गोंधळाला आलिंगन देऊन स्वीकारायला शिकतो. स्वीकाराच्या भावनेतून आलेली ही तयारी आपल्याला शांतता, स्थैर्य आणि नव्या अनुभवांना खुलं मन देते.

%%%%%%%%%%%%%%%%%%%%%%%%%%%%%%%%%%%%%%%%%%%%%%%%%%%%%%%
 \chapter{योजनांशिवाय दैनंदिन जगणे}
माणसाच्या आयुष्यात अनेकदा असं घडतं की आपण जीवन अधिकाधिक गुंतागुंतीचं बनवतो. कठोर वेळापत्रक, काटेकोर योजना आणि बंधनकारक उद्दिष्टे (गोल्स) यांच्या ओझ्याखाली आपण स्वतःलाच दाबून टाकतो. पण मी मात्र नेहमी प्रयत्न करतो की माझं दैनंदिन जीवन शक्य तितकं साधं, मोकळं आणि स्वच्छंद राहावं.
दर सकाळी उठल्यावर माझा पहिला प्रश्न स्वतःलाच असतो, “आज मला खऱ्या अर्थाने आनंद आणि उत्साह कोणत्या गोष्टीमुळे मिळेल?” हा प्रश्न अत्यंत सोपा दिसतो, पण त्यामध्ये प्रचंड गूढता दडलेली असते. गंमत म्हणजे, या प्रश्नाचं उत्तर प्रत्येक दिवशी बदलतं. कधी ते एखाद्या सर्जनशील कार्याशी संबंधित असतं, कधी साध्या निसर्गभ्रमंतीशी, तर कधी एखाद्या प्रिय व्यक्तीसोबत घालवलेल्या वेळेशी.
हो, मान्य आहे की जीवनात काही जबाबदाऱ्या असतात. त्या पार पाडल्याशिवाय पर्यायच नसतो. पण माझ्या बाबतीत त्या जबाबदाऱ्या बहुतेकदा अशाच गोष्टी असतात, ज्या मला आवडतात, ज्या मला प्रेरणा देतात आणि ज्यामुळे माझं मन आनंदी होतं. अर्थात, कधी कधी काही कामं अशीही येतात जी मनाला फारशी रुचत नाहीत. अशा प्रसंगी मी प्रथम त्यांना टाळता येईल का याचा विचार करतो. जर ते शक्य नसेल, तर मग शक्य तितक्या शांत आणि स्थिर मनाने ती पूर्ण करण्याचा प्रयत्न करतो.
माझा खरा प्रयत्न असा असतो की मी प्रत्येक क्षण जाणीवपूर्वक, म्हणजेच संपूर्ण भान ठेवून जगावा. इंग्रजीत याला “जाणीवपूर्वक वर्तमानात राहणे” (कॉन्शसली इन द मोमेंट) असं म्हणतात. या तत्त्वाचा अर्थ असा की आपल्याला त्या क्षणाचा संपूर्ण अनुभव घ्यायचा असतो, विचार करायचा असतो की “मी खऱ्या अर्थाने कोणत्या गोष्टीबद्दल उत्कट आहे?” आणि पुढे स्वतःलाच विचारायचं असतं, “मी माझ्या मूल्यांना खरे राहून या क्षणाला कशा प्रकारे सामोरे जाऊ शकतो?” हेच खरे तर सजगता (माइंडफुलनेस) आहे.
दुर्दैवाने, आपल्या आजूबाजूला पाहिलं तर लक्षात येतं की आपल्यापैकी बरेच जण हे भान ठेवून जगत नाहीत. आपण नकळत काळजी, अस्वस्थता किंवा स्वप्नांच्या पाठीमागे धावत असतो. त्यामुळे खरी शांती आणि समाधान आपल्याला अनुभवायला मिळत नाही.
माझ्या जीवनातील सर्वांत मूलभूत मूल्य म्हणजे करुणा (कंपॅशन). करुणा ही फक्त एकच भावना नसून तिचे अनेक रंग आहेत, प्रेम, दयाळूपणा, सहानुभूती आणि कृतज्ञता. प्रत्येक नवीन परिस्थितीसमोर उभा राहिलो की मी स्वतःला विचारतो, “या प्रसंगाला मी करुणेच्या दृष्टीकोनातून कसं हाताळू शकतो?” मला वाटतं, हा प्रश्न प्रत्येकाने स्वतःला विचारायला हवा. कारण या प्रश्नाचं चिंतन केल्याने आपण योग्य कृती करतोच, पण त्याचबरोबर आपल्या अंतःकरणातील उबही जपली जाते.
हे मान्य करावंच लागेल की मी अजूनही शिकत आहे. मी हे तत्त्व पूर्णपणे आत्मसात केलं आहे, असं अजिबात म्हणू शकत नाही. उलट, माझ्या जाणिवा मला सांगतात की पुढील अनेक वर्षं मला याचं विविध मार्गांनी चिंतन, प्रयोग आणि सराव करावा लागेल. खरं तर हा प्रवास कधीच थांबणार नाही, कारण जाणीवपूर्वक आणि करुणेने जगण्याची कला म्हणजे आयुष्यभर चालणारा शोधप्रवास आहे.

%%%%%%%%%%%%%%%%%%%%%%%%%%%%%%%%%%%%%%%%%%%%%%%%%%%%%%%
 \chapter{योजना का भ्रम आहेत}
आपल्या दैनंदिन जीवनात आपण प्रत्येक गोष्टीसाठी एखादी योजना आखतो. ही योजना कशी पार पडेल, काय काय घडेल आणि आपण ते सगळं कसं नियंत्रित करू याचा आपण अंदाज बांधतो. पण खरं तर या योजना नेमक्या काय असतात? त्या म्हणजे फक्त एक गोड पण खोटा भास, नियंत्रणाचा (कंट्रोल) एक भ्रम.
योजनांशिवाय जगणं हे बहुतेक लोकांना अवास्तव, अगदी वेडेपणासारखं वाटेल. कोणी म्हणेल, “अरे, कसली ही विचित्र गोष्ट! जर योजना नसेल तर सगळं बिघडून जाईल!” पण खरी गोष्ट अशी आहे की आपण जी योजना करतो, त्या फक्त आपल्या मनाचे खेळ असतात.
एक सोपं उदाहरण घ्या. तुम्ही ठरवलं आहे की आज सकाळी तुम्ही एक अहवाल (रिपोर्ट), किंवा एक ब्लॉग लेख (ब्लॉग पोस्ट), किंवा पुस्तकाचा एखादा अध्याय लिहिणार आहात. त्यानंतर, सकाळी अकरा वाजता तुम्हाला एखाद्या सहकाऱ्याशी किंवा व्यवसाय भागीदाराशी (बिझनेस पार्टनर) भेटायचं आहे. कागदावर ही योजना व्यवस्थित दिसते, सकाळी नऊ ते अकरा हा वेळ लेखनासाठी राखून ठेवलेला आणि अकरा वाजता भेट.
आता मानू या की या घटना योजनेप्रमाणेच घडल्या. तरीही हे नेहमी घडतं असं नाही. अनेकदा अचानक एखादा फोन येतो, नवं काम समोर येतं, किंवा एखादा छोटा अडथळा संपूर्ण नियोजन बिघडवून टाकतो. अशा वेळेस लक्षात येतं की नियंत्रण (कंट्रोल) हा फक्त एक नाजूक काचेचा बुडबुडा आहे, जो क्षणात फुटू शकतो.
कधी कधी आपण नशिबवान असतो, आणि आखलेल्या योजना जशा आहेत तशाच पार पडतात. समजा तुम्ही सकाळी नऊ वाजता लिहायला बसलात. कदाचित तुम्ही आधीच त्याची रूपरेषा (आउटलाइन) तयार केली असेल. पण जसजसं तुम्ही प्रत्यक्ष लिहायला सुरुवात करता, तसतसे नवीन विचार मनात उगवू लागतात. हे विचार मूळ योजनेत मुळीच नसतात.
लेखन करताना अशा समस्या आणि संधी समोर येतात, ज्यांचा तुम्हाला आधी विचारही आलेला नसतो. खरं म्हणजे लेखन ही अशी गोष्ट आहे की ती आधीपासून पूर्ण आखून ठेवता येत नाही. ती प्रक्रिया फक्त लिहायला लागल्यानंतरच उलगडते, कारण लिहिताना आपण प्रत्यक्षात विचार करीत असतो. आणि विचार कसा घडेल हे आपल्यालाच ठाऊक नसतं; मग दुसऱ्याचा विचार आपण कसा ओळखणार?
यातूनच एक मजेदार प्रक्रिया घडते. लिहिता लिहिता असे काही विचार आणि अभिव्यक्ती उमटतात, ज्यांचा आधी कधीही विचार झालेला नसतो. जर आपण मन उघडं ठेवलं, तर हे अनपेक्षित विचार अनेकदा विलक्षण आणि तेजस्वी (ब्रिलियंट) स्वरूप धारण करतात. उलट, जर आपण आधी आखलेल्या रूपरेषेला घट्ट चिकटून बसलो, तर या नवीन संधींकडे आपण डोळेझाक करण्याची शक्यता अधिक असते.
यानंतर घड्याळ अकरा वाजल्याचा गजर देतं, आणि तुमच्या भेटीची वेळ होते. तुम्ही सहकाऱ्याला किंवा भागीदाराला भेटता, आणि संभाषण (कॉन्व्हर्सेशन) सुरू होतं. पण इथेही तेच दिसून येतं. संभाषण कधीही योजनेप्रमाणे चालत नाही. तुम्ही कदाचित एखादा अजेंडा (अजेंडा) तयार केला असेल, पण बोलताना एखादा नवा विचार उद्भवतो. तो विचार दुसऱ्या व्यक्तीच्या मनात आणखी एका नवीन विचाराची ठिणगी पेटवतो. अशा रीतीने एकामागोमाग नवनवीन कल्पना जन्म घेत राहतात.
शेवटी, या भेटीतून असे प्रकल्प, सहकार्ये आणि कल्पना समोर येतात, ज्यांचा आधी कधीही विचारसुद्धा केलेला नसतो. आणि ह्याच क्षणांत खरी सौंदर्यपूर्णता दडलेली असते.
म्हणून बघा, जरी दोन घटना योजनाबद्ध रीतीने वेळेवर घडल्या, तरी त्यांच्या मूळ प्रवाहातला गाभा हा पूर्णपणे अनियोजित आणि अनियंत्रित असतो. जितकं आपण या गोंधळाला (कॅओस) स्वीकारतो, तितक्या नवीन, तेजस्वी शक्यता आपल्या आयुष्यात येतात. आणि जितकं आपण नियोजनाच्या चौकटीत स्वतःला बांधून ठेवतो, तितकं आपण आपली खरी क्षमता मर्यादित करतो.
शेवटी निष्कर्ष असा की, योजना म्हणजे फक्त एक गोड गोळी आहे, जी आपण स्वतःला खाऊ घालतो. पण खरी जिंदगी ही सदैव अनपेक्षित घटनांच्या रंगमंचावरच खुलत असते आणि नाचत राहते.
%%%%%%%%%%%%%%%%%%%%%%%%%%%%%%%%%%%%%%%%%%%%%%%%%%%%%%%
 \chapter{उलगडणाऱ्या क्षणासाठी मुक्त राहा}
आपण बर्‍याचदा असा समज करून घेतो की आयुष्य आपल्याला हवे तसे आपल्या मुठीत घट्ट पकडून ठेवता येते. ही एक प्रकारची भ्रमित धारणा आहे. आपण स्वतःला वारंवार पटवून देतो,  “सगळं माझ्या नियंत्रणाखाली आहे.” पण वास्तवाकडे पाहिलं तर लक्षात येतं की जग कधीच आपल्याच्या पूर्ण नियंत्रणात नसतं. जीवन हे सतत बदलणारं, प्रवाही आणि गतिशील असं आहे. मग प्रश्न उरतो,  जर आपण या बदलत्या प्रवाहाला उघड्या मिठीत घेतलं, त्याला विरोध न करता स्वीकारलं, तर काय होईल?
जर आपण स्वतःला प्रत्येक क्षणाच्या उलगडण्यासमोर पूर्णपणे मोकळं ठेवलं, तर अमर्याद शक्यता समोर येतात. या शक्यता आपल्या नेहमीच्या आखणीपेक्षा, काटेकोर नियोजनापेक्षा (प्लॅनिंग) अधिक गूढ आणि विलक्षण असतात. जीवनाचा खरा सौंदर्यभाव याच ठिकाणी दडलेला आहे. हे सौंदर्य शब्दांत पूर्णपणे सांगता येत नाही; ते अनुभवायचं असतं, क्षणोक्षणी उमलतं.
एखादा साधा प्रयोग करून पाहा. पुढील एक तासासाठी सर्व नियोजन, सर्व कार्ययादी (टू-डू लिस्ट) जाणीवपूर्वक बाजूला ठेवा. मग अनुभव घ्या की त्या वेळेत नेमकं काय उलगडतं. जे क्षण तुम्हाला आनंद देतात, ज्यातून तुम्हाला रोमांच मिळतो आणि जे तुमच्या मूल्यव्यवस्थेशी खोलवर जोडलेले आहेत, त्यांचा विचार करा. नुसता विचारच नाही, तर त्या गोष्टींना प्रत्यक्ष कृतीत आणा.
जेव्हा आपण त्या गोष्टी करायला लागतो ज्या आपल्याला खरोखरच उत्साहित करतात आणि आपल्या जीवनमूल्यांशी सुसंगत असतात, तेव्हा आपल्या आत नवीन विचार जन्म घेतात. त्या कृतीतून नवनवीन दिशा सापडतात. लोकांशी संवाद साधा, पण कोणताही ठराविक कार्यक्रम किंवा कार्यसूची (एजेंडा) न ठेवता. अशा सहज संवादातून कितीतरी ताज्या कल्पना उगम पावतात. लोकांशी, कल्पनांशी आणि स्वतःच्या विचारांशी मुक्त संवाद साधताना अनेक संधी आपल्या दारात येतात.
हे सगळं ऐकताना कदाचित थोडं धूसर, आभाळासारखं भासेल. परंतु वस्तुस्थिती अशी आहे की हे विचार जितके हलके वाटतात, तितकेच ते ठोस आहेत. आपण जेव्हा एखादा कार्यक्रम किंवा योजना (प्लॅन) आखतो, तेव्हा आपण समजतो की जणू दगडावर कोरल्याप्रमाणे काहीतरी निश्चित ठरवलं आहे. पण जीवन खरं तर प्रवाही असतं. आपण मात्र स्वतःला पटवून देतो की ते कणखर, अढळ आणि कॉंक्रिटसारखं आहे.
जर आपण हे मान्य केलं की जीवन सतत बदलणारं आणि तरल आहे, तर त्या तरलतेलाच आपण आपली खरी ताकद बनवू शकतो. मग आपण प्रवाहासोबत सहज वाहू शकतो. बदलत्या लाटा आणि प्रवाहांना (करंट्स) थोपवण्याऐवजी आपण त्यांच्याशी खेळू शकतो. अशा वेळी आपण जगाला आपल्या छोट्या छोट्या योजनांशी जुळवून घेण्याचा हट्ट सोडतो आणि उलट डोळे खरोखरच उघडून पाहायला शिकतो.
माझ्याकडे सर्व प्रश्नांची उत्तरे नाहीत. आणि खरं सांगायचं तर, जर मी असा दावा केला की या मार्गाने जगल्यावर नेमकं काय होईल हे मला ठाऊक आहे, तर मी स्वतःला फसवेन. कारण कुणालाही ठाऊक नाही की या मुक्त प्रवाहासोबत चालताना पुढचा क्षण काय घेऊन उभा राहील.
म्हणूनच मी सरळपणे कबूल करतो,  “मला माहित नाही काय होईल” (“आय डोन्ट नो व्हॉट विल हॅपन”). आणि हाच साधा वाक्यांश अमर्याद शक्यता घेऊन येतो. याचा खोलवर विचार केला तर लक्षात येईल की कदाचित हाच स्वीकार, हाच खुला दृष्टिकोन ही खरी मुक्तता आहे.

%%%%%%%%%%%%%%%%%%%%%%%%%%%%%%%%%%%%%%%%%%%%%%%%%%%%%%%
\chapter{खोट्या गरजा निर्माण करू नका}
आपले आयुष्य अनेकदा ठरावीक कामांनी, पाळायच्या नियमांनी आणि इतर लोकांच्या अपेक्षांनी व्यापलेले असते. परंतु थोडा वेळ थांबून, शांतपणे विचार केला तर अनेक वेळा आपण ज्या “गरजा” म्हणून ओळखतो त्या खरंतर खरी गरज नसून आपले मन किंवा सामाजिक वातावरणाने निर्माण केलेले अपेक्षित वर्तन असतात. या वेगळ्या प्रकारच्या गरजा ओळखणे आणि त्यांचा अभ्यास करणे म्हणजे आपल्या आतल्या नियंत्रित पद्धतींना शहाणपणाने तपासणे होय.
थोडंसं स्वतःकडे बघा: तुमच्याकडे कोणत्या प्रकारच्या गरजा सातत्याने सतावतात? कदाचित वारंवार इलेक्ट्रॉनिक पत्रव्यवहार तपासण्याची सवय, म्हणजे ईमेल (ईमेल) किंवा इनबॉक्स (इनबॉक्स) नेहमी रिकामी ठेवण्याचा आग्रह. कदाचित तुम्हाला सर्व वेबसाइटवरील वैयक्तिक लेखमाला वाचूनच वाटावे, म्हणजे ब्लॉग्स (ब्लॉग्स) वाचणे गरजेचे वाटणे; कदाचित घरातील पलंग, बिछाना किंवा घराचे तुकडे नेहमी परिपूर्ण आणि नीटनेटके ठेवण्याची जबाबदारी स्वतःवर घ्यावीशी वाटते; किंवा बाहेर निघताना एखादा असा पोशाख परिधान करायचा जो जणू एखाद्या फॅशन मासिकातून उतरलेला दिसेल.
या भावनांमध्ये कौटुंबिक किंवा व्यावसायिक दबावही मिसळलेला असतो. मुलांवर प्रत्येक लहानशी चूक नसती तरही ओरडून दंड देण्याची इच्छा, सहकाऱ्यांना नियंत्रणाखाली ठेवण्याचा नेहमीचा आग्रह, कोणत्याही भेटीला आवर्जून उपस्थित राहण्याबद्दलची बंधने, या सगळ्याही आपण “गरज” समजून स्वीकारतो. आर्थिक चातुर्याची किंवा सामाजिक प्रतिष्ठेची तीव्र इच्छा, अधिक पैसा कमवण्याची आस, महागडी गाडी बाळगण्याची हौस, यांमधूनही अनेक वेळा आपण खऱ्या गरजा आणि नाटकांमधील फरक विसरतो. उदाहरणार्थ, भागातले लोक जर तुमच्या घराच्या अंगणात हिरवळीचा आदर्श न पाहिले तर ताबडतोब न्यायाची भाषा सुरू करतात; काहीजणी महागड्या बीएमडब्ल्यू गाड्यांना (बीएमडब्ल्यूज) प्राधान्य देतात; नवीनतम फ्लॅगशिप स्मार्टफोन नसेल तर तेंव्हा आपल्याला तंत्रज्ञानप्रेमी म्हणून मिळणारी प्रतिष्ठा कमी झाल्यासारखी वाटते, म्हणजे आयफोन नसल्यास तुमची तंत्रज्ञानी म्हणूनची प्रतिष्ठा (गीक क्रेड) कमी होते आणि प्रतिष्ठेचे प्रतीक (स्टेटस सिम्बॉल) हातून नातंसे वाटते.
या गरजा नेहमीच समाजाने घाललेल्या नियमांमुळे निर्माण होत नाहीत; अनेकदा आपण स्वतःच अंतर्बाह्य दबाव निर्माण करतो. वेबवरील अद्यतनांची यादी सतत तपासण्याची भावना, म्हणजे आरएसएस फीड्स तपासण्याची सवय, बातम्या वाचण्याच्या संकेतस्थळांना वारंवार भेट देणे, संदेश तपासणे किंवा ट्विटरसाठी खाती,  म्हणजे ट्विटरवरील खाती (ट्विटर अकाउंट्स) सतत पाहण्याचा आग्रह, यांचे मुख्य कारण म्हणजे आपण स्वतःवर घालून घेतलेला अकार्यक्षम नियम असतो. अनेकदा असे करताना आपल्याला वाटते की हे न केल्यास सामाजिक किंवा व्यावसायिक पातळीत काही वाईट होईल, परंतु वास्तविकतेत त्यामुळे बर्‍याच वेळा कोणताही प्रतिकूल परिणाम होत नाही.
आयुष्यातील अनेक अपेक्षा आणि कठोर योजनाही या बनावट गरजांचीच बेरीज असतात. आपण आयुष्यभरात साध्य करायच्या गोष्टींची विस्तृत यादी तयार करतो किंवा एका वर्षात पूर्ण करावयाच्या उद्दिष्टांचा काटेकोर आराखडा बैठवतो, आणि तो एकही घटक साध्य न झाल्यास मनाला चैन लागत नाही. परंतु इथे विसरले जाते की काही गोष्टी साध्य न झाल्यातून जगाचा शेवट होतो असं नाही; जीवनाची मोठी दिशा व सामर्थ्य कायम राहते. या दृष्टिकोनातून विचार केल्यास आपल्याला अनेक अनावश्यक दबाव स्वतःवर का निर्माण करायचे हे प्रश्नच निर्माण होतात.
या बनावट गरजांच्या मुळाशी बहुतेकदा घामटलेली भीती असते, नेमकी कोणती ही गोष्ट हरवणार आहे, इतरांनी कमी समजू शकतात, अपयशाची भीती, किंव्हा स्वतःचा मूल्य कमी होण्याची शक्यता अशा असंख्य भीतींचा संच दडलेला असतो. जेव्हा आपण या भीतींचा चिरनिरिक्षण करतो आणि त्यांना नावाने ओळखतो, तेव्हा त्यांना अपवास मिळण्याची शक्यता कमी होते. मनोविश्लेषण आणि ध्यान यांसारख्या साधनांनी आपल्याला या भीतींपर्यंत पोहोचणे, त्यांची कारणं शोधणे आणि निर्णय घेणे सोपे होते.
हे परिवर्तन एकदा साधण्यासाठी छोटा व्यावहारिक प्रयोग करणे फायदेशीर ठरते. सुरुवातीला स्वतःला अनुमती द्या की तुम्ही एखादी निव्वळ, बनावटीची गरज एक तासासाठी बाजूलाच ठेवणार आहात. त्या तासात तुमच्या मनात काय घडते, शांतपणा येतो का, चिंता वाढते का, किंवा मूळ कामकाजात अधिक एकाग्रता येते का, हे निरीक्षण करा. नंतर अनुभवावरून हा कालावधी एक दिवस आणि अधिक आत्मविश्वास आल्यास एक आठवडा असा वाढवा. या प्रयोगात तुमच्यावर जर कोणताही थेट नकारात्मक परिणाम झाला नाही तर पुढे तुम्ही तो अनुभव अधिक काळपर्यंत ठेवू शकता. तुमच्या मनात सतत विचार करा: हा आग्रह सोडल्याने माझ्या जीवनात काय बदल होईल? मला जास्त मोकळा वेळ मिळेल का? माझ्या कामाची गुणवत्ता सुधारेल का? माझं मन हलकं होईल का? किंवा प्रतिकूल काही घडेल का आणि ते घडण्याची प्रत्यक्ष शक्यता किती आहे? जर प्रतिकूल काही घडले, तर त्यावर उपाय काय असू शकतो?
या प्रक्रियेत एक ध्यानशील दृष्टिकोन अत्यंत उपयुक्त ठरतो. जेव्हा तुम्ही गरजेच्या मागे दडलेल्या भीतींचे निरीक्षण कराल, त्यांचे श्वासांशी, शरीरातील संवेदनांशी, विचारांशी जुळवून पाहाल, तेव्हा त्या भीतींच्या शक्तीत तोटा येऊ लागतो. साधे श्वास-निरीक्षण, विचारांच्या पलीकडे पाहण्याचा सराव आणि स्वतःशी शांत संवाद हे पावले या कार्यात उपकृत ठरतात. त्याचबरोबर, व्यावहारिक दृष्ट्या तुम्ही कोणतीही गरज पूर्णतः बंद न करता मोकळ्या अंतराने हळूहळू कमी करता, तर बदल अधिक सुसंवादी आणि टिकाऊ राहतो.
सोप्या भाषेत सांगायचे तर बनावट गरजा दूर करण्यासाठी अगदी मोठे बदल आवश्यक नसतात; थोडीसा परवानगी, हळूहळू सराव आणि भीतींना समोरून पाहण्याचे धैर्य पुरेसे असते. सोडण्याच्या या प्रक्रियेतून जे मोकळेपणाचे अनुभव मिळतात, ते केवळ तात्पुरते नसून दीर्घकालीन मानसिक आरामासाठी आधार देणारे असतात. या अनुभवातून माणूस स्वतःच्या खरी गरजा, मूल्ये आणि प्राथमिकता अधिक स्पष्टपणे ओळखू लागतो.
शेवटी, हे लक्षात घ्या की गरजा कमी करणे म्हणजे आयुष्याच्या क्षितिजाला मर्यादा आणणे नाही; उलट, अनावश्यक थकव्यापासून मुक्त होऊन आपण आपल्या जीवनातील महत्त्वाच्या बाबींवर अधिक वेळ आणि ऊर्जा देऊ शकतो. खोट्या गरजांना ओळखून, त्यांना नियंत्रित करणे आणि हळूहळू तुटविणे ही एक अशी साधना आहे जी अंतर्मुखी स्वातंत्र्यापर्यंत घेऊन जाते आणि या स्वातंत्र्यातूनच खऱ्या अर्थाने मानसिक शांती आणि आयुष्याची खोली मिळते.
 
%%%%%%%%%%%%%%%%%%%%%%%%%%%%%%%%%%%%%%%%%%%%%%%%%%%%%%%
 \chapter{उत्कट राहा आणि नावडत्या गोष्टींपासून स्वतःला दूर ठेवा}
आपल्या अनेक जणांच्या आयुष्यात मोठा काळ अशा गोष्टींवर निघून जातो ज्यांना ते आवडत नाहीत, ज्यामुळे त्यांना मानसिक भार अनुभवावा लागतो किंवा जे त्यांना सतत त्रास देतात. घरापासून शाळेपर्यंत, शाळेपासून कामापर्यंत, समाजाने आणि संस्कारांनी आपल्याला असे शिकवले आहे की काही गोष्टी आवडत असली तरीही त्या करणे आवश्यक आहे; त्यांना न्याय्य किंवा सद्गुण म्हणून पाहिलं जातं. परंतु या पारंपरिक समजुतीवर थांबून विचार केला तर प्रश्न उद्भवतो, जर एखादी क्रिया मनात तळमळ, तिटकारा किंवा कंटाळा निर्माण करत असेल, तर का तेच करावे? या प्रश्नाला निष्कर्षापर्यंत नेणं गरजेचं आहे कारण जीवनाच्या अनेक वर्षांचा मध्यभाग आपण असाच अनिच्छेने घेतलेल्या कर्तव्यांनी घालवतो.
जेव्हा एखाद्या कामामुळे अंतःकरणाला त्रास होत असेल, तेव्हा त्या कामाला थांबवण्याचे किंवा बदलण्याचे वेगवेगळे मार्ग असू शकतात. काहीवेळा तो मार्ग अत्यंत सोपा असतो, उद्या पासून किंवा ठरलेल्या तारखेपासून ती क्रिया थांबवायची, काही वेळा त्या कामाचे स्वरूप बदलून त्यात आनंद आणायचा किंवा जबाबदाऱ्या कमी करायच्या. तरीही काही प्रसंगात मोठा निर्णय घ्यावा लागतो, नोकरी बदलावी लागते, वास्तव्यातील रहिवासी व्यवस्था बदलावी लागते किंवा जीवनशैलीत महत्त्वाचा बदल करावा लागतो. असे टोकाचे निर्णय घेण्यासाठी धाडस आणि स्वप्रेरणा आवश्यक असतात; तरी निर्णयाचे अंतिम अधिकार आणि परिणाम सर्वस्वी तुमच्या हाती असतात.
व्यक्तिगत अनुभवातून सांगायचे तर मी अनेकदा स्वतःच्या आयुष्यात अशाच अनैच्छिक किंवा मनाला ओझं ठेपणाऱ्या गोष्टींना नक्कीच सोडले आहे. काही वेळा मी एखादी गोष्ट सरळ थांबवली, तर काही वेळा मी मोठे बदल स्वीकारले, जसे की नोकरी बदलणे किंवा इतर देशात किंवा शहरात स्थलांतर करणे. प्रत्येक वेळी जेव्हा मी एखादी त्रासदायक जबाबदारी किंवा अनावश्यक बंधन काढून टाकतो, तेव्हा तात्काळ माझ्या मनाचा भार घटतो आणि मुक्ततेची जाणीव होऊ लागते. मुक्तता ही फक्त बाह्य आराम नव्हे तर अंतर्गत शांततेचे आणि प्रेरणेचे स्त्रोत बनते.
व्यक्तिगत उदाहरणे देताना मी हेही नमूद करतो की मला गाडी चालवायला आवडत नाही, त्यामुळे मी आपणास हव्या त्या प्रकारे रहायला जास्त सोयीस्कर असलेल्या ठिकाणी स्थलांतर केले; माझे कुटुंब आणि मी सध्या अशा प्रकारच्या ठिकाणी आहोत की आम्ही साधारण जीवनात गाडीसारख्या गोष्टींवर अवलंबून नाही. आर्थिक व्यवस्थेबाबतही मी प्रयत्नपूर्वक सोप्या नियमांपासून सुरूवात केली: मी माझे आर्थिक व्यवहार स्वयंचलित केले, म्हणजे नियमित बचत, बिल भरणे आणि गुंतवणूक यांसारख्या गोष्टी स्वयंचलित पद्धतीने चालू केल्या (ऑटोमेटेड). असे करून माझ्या दररोजच्या निर्णयांमधून न जाणवता घालवणारा वेळ व ऊर्जा आता माझ्या आवडत्या गोष्टींसाठी उपलब्ध झाल्या.
इंटरनेट किंवा सामाजिक माध्यमांवर असताना लोकांच्या टिप्पण्यांचे (कॉमेंट्स) व्यवस्थापन करणे मला अख्ख्या प्रक्रियेचा कंटाळा वाटवू लागले, त्यामुळे मी त्या व्यवस्थापनाच्या पद्धतींमध्ये बदल केला आणि जिथे आवश्यक तिथे टिप्पण्या बंद केल्या किंवा त्या विभागातून स्वतःला मोकळे केले. माझ्या वेबसाईटवर येणाऱ्या जाहिरातींमुळे सुरुवातीला काही आर्थिक फायद्यामुळे त्रास सहन केला, परंतु त्यांच्या प्रमाणाने वैयक्तिक मानसिक त्रास वाढला म्हणून मी जाहिराती काढून टाकण्याचा निर्णय घेतला. या सर्व गोष्टींमुळे माझा वेळ आणि मानसिक ऊर्जा अनावश्यक घटना कमी करण्याकडे वळले.
वाचनाच्या बाबतीतही मी शिस्तीतून वाचण्याच्या ऐवजी हवे असलेले वाचन निवडतो: एखादं पुस्तक वाचनात कंटाळा आला तर मी तो बंद करून दुसरं पुस्तक घेण्यापासून परावृत्त राहत नाही. अशा प्रकारे मी माझ्या जीवनातील पुनरावृत्ती करणारी, कंटाळवाण्या आणि ऊर्जा चोरणारी कामे शक्यतो स्वयंचलित केली किंवा कायमची काढून टाकली. परिणामी हात मोकळे झाले आणि त्या मोकळ्या वेळेत मी माझ्या आवडीच्या कामांसाठी अधिक वेळ, विचार आणि उत्साह देऊ शकतो.
जेव्हा तुम्ही तुमच्या काळात आणि उर्जेतून फक्त आवडत्या गोष्टींनाच प्राधान्य द्याल, तेव्हा तुमचे संबंध, काम आणि सर्जनशीलता अधिक गतीने वाढतात. एखाद्या प्रकल्पाशी सुरुवातीला उत्साह असतानाच तो स्वीकारणे नेहमीच लाभदायक असते; पण ज्यावेळी त्या प्रकल्पाप्रती तुमची आवड कमी होते, त्या वेळी त्याचा समारोप किंवा त्याला सोडून देणे हे धाडसही असू शकते आणि ते काहीशा प्रकारे बुद्धिमत्तेचे लक्षण आहे. आपण सर्वकाही केवळ सुरुवात केल्यावरच पूर्ण करावे, अशी धारणा नेहमीच बंधनकारक ठरते; माझ्या अनुभवातून मला उलट हे समजले की जीवनात ज्या गोष्टींनी मनापासून समाधान मिळते त्यांना प्राधान्य देणेच खरी समजदारी आहे.
माझे दिवस आता त्या लोकांसोबत आणि त्या क्रियाकलापांसह भरलेले असतात जे मला प्रत्यक्षात प्रिय आहेत, मित्र आणि कुटुंबासोबत वेळ घालवणे, बारीक वाचन, नियमित धावणे, लेखन आणि इतरांना मदत करणे. त्याचबरोबर मी स्वतःसाठी एकांताचे वेळा राखून ठेवतो, कारण शांत एका वेळेमुळेच विचारांची स्पष्टता आणि नवीन सर्जनशील कल्पना जन्माला येतात. या सगळ्यामुळे माझे जीवन संतुलित आणि अर्थपूर्ण झाले आहे.
आरोग्य आणि तंदुरुस्तीच्या बाबतीतही मी नियमापेक्षा आवडीला जास्त महत्त्व देतो. जे आहार मला आवडतात आणि जे माझ्यासाठी निरोगी आहेत, त्यांना मी प्राथमिकता देतो, म्हणजेच निरोगी आहार (हेल्दी फूड) हा माझ्यासाठी केवळ ट्रेंड नाही तर जीवनशैली आहे. तंदुरुस्ती (फिटनेस) साठी मी ज्या क्रियाकलापांत आनंद अनुभवतो त्या निवडतो: खेळ, धावणे, उड्या मारणे, मुलांसोबत खेळणे, टेकड्यांवर चढून पळणे, पोहणे किंवा बास्केटबॉल खेळणे, यामुळे व्यायाम (एक्सरसाईज) हा माझ्यासाठी कर्तव्य नसून आवड आणि आनंद बनतो. अशा प्रकारचे हालचाली केवळ शरीरच नव्हे तर मनही सशक्त करतात.
एक क्षणासाठी विचार करा: जर तुम्ही तुमचे अमूल्य दिवस आणि ऊर्जा सतत अशा गोष्टींवर खर्च केली नाहीत ज्या तुम्हाला आवडत नाहीत, तर तुमची खरी क्षमता किती उजळून निघेल! तुमचे काम अधिक कौशल्यपूर्ण, अधिक भावपूर्ण आणि अधिक उपयोगी बनेल; आणि सुमारेच्या लोकांना, समाजाला व स्वतःलाही त्याचा तात्काळ व दीर्घकालीन फायदा दिसेल. म्हणूनच, आयुष्यात कोणती कामे तुम्हाला उरतात आणि कोणती कामे तुम्हाला कमी करायची आहेत हे तपासून बघा, त्रासदायक बंधने हळूहळू दूर करा, आणि उरलेला वेळ उत्तम गोष्टींसाठी वापरा, हीच माझी वैयक्तिक शिकवण आणि लाखो अनुभवातून मिळालेले तत्त्व आहे.

%%%%%%%%%%%%%%%%%%%%%%%%%%%%%%%%%%%%%%%%%%%%%%%%%%%%%%%
 \chapter{घाई करू नका, हळूहळू चालत राहा आणि पूर्णपणे उपस्थित राहा}
घाई करू नका. आयुष्य नेहमी धावपळीने न जगता, हळूहळू चालत राहा. प्रत्येक क्षणात स्वतःला पूर्णपणे उपस्थित ठेवा. हे केवळ शब्द नाहीत, तर शांत, समाधानी आणि संतुलित जीवन जगण्यासाठीची एक मूलभूत साधना आहे.
आपल्या रोजच्या जीवनात आपण बहुतेक वेळा धावत असतो. एखाद्या दिवसात जितकी कामे मावू शकतील त्यापेक्षा जास्त कामे स्वतःवर लादतो आणि मग ती शक्य तितक्या वेगाने उरकण्याचा प्रयत्न करतो. परिणामी आपल्या आयुष्यात श्वास घेण्याइतकीही मोकळी जागा शिल्लक राहत नाही. विश्रांतीचा क्षण सुद्धा गाठता येत नाही. जेवण असो, काम असो किंवा आपल्या जवळचे माणसं असोत, आपण जणू त्यात प्रत्यक्ष उपस्थितच राहत नाही. शरीर एका ठिकाणी असते, पण मन कुठेतरी दुसरीकडे भटकत असते. अशा अवस्थेत जीवनाचा खरा आनंद आपल्या हातातून निसटून जातो.
घाई करण्याचा अर्थ केवळ वेगाने वागणे नाही, तर स्वतःसाठी आणि इतरांसाठी त्रास निर्माण करणे देखील होय. घाईघाईत वाहन चालवले तर अपघात घडणे अपरिहार्य ठरते. कामाच्या ठिकाणी धावपळ केली तर चुका होतात, छोट्या समस्या मोठ्या संकटात रूपांतरित होतात, आणि त्याचवेळी आपण स्वतःलाही इजा करतो व इतरांनाही अडचणीत ढकलतो. घाईत असताना सावधपणा कमी होतो, लक्ष विचलित होते, आणि म्हणूनच योग्य निर्णय क्षणभर उशीराने लक्षात येतात.
घाई ही केवळ वैयक्तिक नुकसान घडवत नाही, तर आपल्या सभोवतालच्यांनाही ताण देते. उदाहरणार्थ, जेव्हा मी “उशीर होऊ नये” म्हणून माझ्या कुटुंबाला दारातून पटकन बाहेर पडण्यास प्रवृत्त करतो, तेव्हा माझी पत्नी, जी शांतपणे, वेळ घेऊन तयारी करणारी असते, तिच्यावर अनावश्यक दडपण येते आणि ती तणावग्रस्त होते. त्याचप्रमाणे, जेव्हा आपण ऑफिसमध्ये एखादे काम फार वेगाने करण्याचा प्रयत्न करतो, तेव्हा आपले सहकारीही नकळत घाईत येतात आणि त्यांच्यावर अतिरिक्त ताण वाढतो. अशाप्रकारे जीवनातील प्रत्येक गोष्ट जणू एका अदृश्य पण जड दडपणाखाली येते.
म्हणूनच घाई टाळा आणि जीवनाला हळुवारपणे सामोरे जा. हाच खरंतर “सहज आणि निरागस जीवनाचा ताल” (इफर्टलेस लाइफ चा खरा ताल) आहे. गंमत अशी की बहुतांश लोकांना हे अजिबात सोपे वाटत नाही. ऑफिसमध्ये एका टेबलावरून दुसऱ्या टेबलाकडे किंवा घरात एका खोलीतून दुसऱ्या खोलीत शांतपणे, हळूहळू जाणे ही संकल्पना बहुतेक लोकांसाठी अगदीच नवी आणि अपरिचित असते.
अन्न खाण्याच्यावेळीही हळूहळू खाण्याची सवय लावा. जेवताना फक्त जेवणावर लक्ष केंद्रित करा. त्या क्षणी “वाचन” (रीडिंग), “इंटरनेट ब्राउझिंग”, “टेलिव्हिजन पाहणे” किंवा गप्पा मारणे या गोष्टी टाळा. सुरुवातीला हे कदाचित अवघड वाटेल, कारण ही आपली सवय मोडावी लागते. पण नंतर हळूहळू लक्षात येते की अशा प्रकारे खाल्ल्यास आपण अन्नाच्या प्रत्येक पैलूकडे सजग होतो, त्याची चव, त्याचा पोत, ते कुठून आले आहे, आपण किती खात आहोत, आपले पोट खरोखर किती भरले आहे. या पद्धतीने खाल्ल्यास वजन नियंत्रणात राहते, आपल्या जवळ असलेल्या अन्नाबद्दल कृतज्ञतेची भावना जागृत होते आणि आपण जे खात आहोत त्याची खरीखुरी प्रशंसा करता येते.
गाडी चालवतानासुद्धा हळूहळू चालवून बघा. तुम्ही स्वतः अधिक सुरक्षित राहाल, इतरांनाही धोका कमी निर्माण कराल, तुमच्यातील ताण कमी होईल आणि सर्वांत महत्त्वाचे म्हणजे प्रवासातील प्रत्येक क्षणाचा खरा आनंद घेता येईल.
हळुवार जीवन म्हणजे स्वतःवर लादलेली अनावश्यक उद्दिष्टे, बेत आणि क्रिया हळूहळू कमी करणे होय. ही एकप्रकारे वजाबाकी आहे जी आयुष्याला मोकळीक आणि श्वास घेण्याइतकी जागा देते. मात्र ही वजाबाकी एका दिवसात होत नाही; वेळ लागतो. म्हणूनच, ही प्रक्रिया सुद्धा हळूहळूच करणे आवश्यक आहे. स्वतःला हा अधिकार द्या की तुम्ही शांतपणे, सावकाशपणे आणि संयमाने अनावश्यक गोष्टी आयुष्यातून दूर कराल.

%%%%%%%%%%%%%%%%%%%%%%%%%%%%%%%%%%%%%%%%%%%%%%%%%%%%%%%
\chapter{अनावश्यक कृती टाळा}
आपण दररोज केलेल्या अनेक कृती खऱ्या अर्थाने आवश्यक नसतात. हे विधान ऐकायला धाडसी वाटू शकते, परंतु आयुष्यातील निरीक्षण आणि अनुभव दाखवतात की, ही गोष्ट खरोखर खरी आहे. आपण नेहमीच्या जीवनशैलीत इतके बंधनात्मक कार्य करतो की, त्यामागील खर्‍या गरजांकडे लक्षच जात नाही.
उदाहरण म्हणून मसानोबू फुकुओकाचे नाव घेता येईल. ते एक क्रांतिकारी जपानी शेतकरी होते, ज्यांचा उल्लेख मी मागील धड्यात केला आहे. त्यांनी “खऱ्या गरजा, साध्या गरजा” या तत्त्वावर आपले विचार मांडले. फुकुओकाने पारंपरिक तसेच आधुनिक शेतीचा सखोल अभ्यास केला. वर्षानुवर्षे निरीक्षण केल्यानंतर त्याने असे धाडसी निष्कर्ष काढले की, शेतकरी, मग तो पारंपरिक पद्धतीने काम करणारा असो किंवा आधुनिक तंत्रज्ञानाचा वापर करणारा असो, खूपशी अनावश्यक कृती करतो.
त्याने नांगरणी करणे, मशागत करणे, तण काढणे, खत घालणे, छाटणी करणे, कीटकनाशक फवारणे, या सर्व कृतींचा अभ्यास करून त्यांना “अनावश्यक” असे घोषित केले. फुकुओकाने या सर्व कृती कमी केल्यावर त्याला जाणवले की प्रत्यक्षात पूर्ण करण्यासाठी फार कमी काम उरते. हे तत्त्व फक्त शेतीपुरते मर्यादित नाही, तर आपल्या दैनंदिन जीवनातील प्रत्येक क्षेत्रात लागू होते.
आपण जे करतो त्यातील बराचसा भाग आपण फक्त परंपरेमुळे, “हे करणे आवश्यक आहे” या गैरसमजामुळे, किंवा इतर कृतींनी निर्माण केलेल्या समस्यांचे निराकरण करण्यासाठी करतो. जर आपण प्रत्येक कृतीकडे विचारपूर्वक लक्ष दिले आणि स्वतःला प्रश्न विचारला की, “ही गोष्ट खरोखरच आवश्यक आहे का?”, तर आपण अनावश्यक कृतींपासून स्वतःला वाचवू शकतो.
सामान्य बोलण्यापेक्षा उदाहरणे अधिक स्पष्ट करतात. खाली काही ठोस उदाहरणे दिली आहेत:
\begin{itemize}
 \item प्रत्येक ईमेल, फेसबुक मेसेंजर किंवा ट्विट-ला उत्तर देणे अनिवार्य नाही. अनेकदा आपण फक्त असभ्य वाटू नये म्हणून किंवा सामाजिक दबावामुळे उत्तर देतो. प्रत्यक्षात, फार कमी लोक खरोखर दुखावले जातात जर आपण उत्तर दिले नाही. त्यामुळे, आवश्यकतेनुसारच उत्तर द्या, फक्त तीच प्रतिक्रिया द्या जी खरोखर गरजेची आहे.
\item घरात अनावश्यक वस्तू जास्त जमा केल्यास, त्या वस्तूंची साफसफाई, देखभाल आणि साठवणूक अतिरिक्त ओझे निर्माण करते. जर आपण नको असलेल्या वस्तू दूर केल्या (ज्याला आजकाल डी-क्लटरिंग म्हणतात) आणि नवीन वस्तू खरेदी करण्याचे प्रमाण कमी केले, तर घर सांभाळण्याचे श्रम आपोआप कमी होतात.
\item पालक म्हणून आपण आपल्या मुलांसाठी अनेकदा जास्त हस्तक्षेप करतो. मात्र, मुलांना स्वतंत्रपणे खेळायला, शिकायला आणि शोध घेण्याची संधी दिल्यास, ते अधिक आत्मनिर्भर होतात. प्रत्येक क्षणी आपण किंवा इलेक्ट्रॉनिक्स त्यांच्याजवळ असणे आवश्यक नाही. पालनपोषणातील अनावश्यक हस्तक्षेप कमी करून आपण मुलांना अधिक स्वातंत्र्य, सर्जनशीलता आणि शिकण्याची क्षमता देऊ शकतो.
\item जर आपण अंगणातील गवत उपटण्याऐवजी नैसर्गिकरित्या वाढू दिले आणि त्याच गवतामध्ये भाज्या पेरल्या, तर अंगणाची देखभाल खूप कमी होते. हे शेजारच्या रूढीशी विसंगत वाटू शकते, परंतु अशा पद्धतीने काम केल्यास आपली जीवनशैली अधिक सोपी आणि सुलभ होते.
\item जर आपण डोक्याचे मुंडन केले, तर केसांच्या देखभालीसाठी लागणाऱ्या असंख्य कृतींचा ओझा आपोआप कमी होतो.
\item जर आपण घरून काम करू शकतो किंवा कामाच्या ठिकाणाजवळ राहतो, तर रोजचा प्रवास संपतो आणि वेळ वाचतो.
\item जर आपण आपल्या ब्लॉग (लेख) वरून कॉमेंट्स (टिप्पणी) ची सुविधा काढली, तर कॉमेंट्स तपासण्याची आणि नियंत्रण करण्याची आवश्यकता उरत नाही.
\end{itemize}
ही काही उदाहरणे फक्त आरंभिक आहेत. प्रत्यक्षात, अशा अनावश्यक कृतींची संख्या अमर्याद आहे. तरीही, लक्षात ठेवण्याजोगे मार्गदर्शक तत्त्व एकच आहे, “अनावश्यक काहीही करू नका”. या तत्त्वाचे पालन केल्यास दिवस अधिक हलका, वेळ अधिक शिल्लक आणि मन अधिक शांत राहते.

%%%%%%%%%%%%%%%%%%%%%%%%%%%%%%%%%%%%%%%%%%%%%%%%%%%%%%%
 \chapter{समाधान शोधा}
जगातला जवळजवळ प्रत्येक माणूस, त्यात माझ्या ओळखीतील लोकांचाही समावेश आहे, सतत काहीतरी अधिक चांगले, अधिक मोठे किंवा अधिक आकर्षक शोधत असतो. अनेकदा त्यांच्या अपेक्षा संपत नाहीत. कुणाला उत्तम जीवन हवे असते, कुणाला चांगले कपडे, कुणाला दर्जेदार गाडी, कुणाला प्रतिष्ठापूर्ण नोकरी, तर कुणाला राहण्यासाठी अधिक आरामदायी आणि सुंदर जागा हवी असते.
ही गोष्ट मला अनोखी वाटत नाही, कारण मी स्वतः माझ्या आयुष्याचा मोठा भाग ह्या शोधात घालवला आहे. परंतु, जेव्हा मी थोड्या थोड्या प्रमाणात समाधानी होण्याचा सराव केला आणि त्या क्षणांचा खरा आस्वाद घेण्यास शिकलो, तेव्हा माझ्या जीवनात खऱ्या अर्थाने बदल घडू लागले. जीवन अधिक समृद्ध आणि अर्थपूर्ण होऊ लागले, आणि व्यक्तिमत्व व जीवनशैलीमध्ये सुधारणा दिसू लागली.
माझ्या अनुभवातून काही उदाहरणे पुढीलप्रमाणे आहेत:
\begin{itemize}
 \item मला जेव्हा स्पष्टपणे जाणवले की माझ्या पत्नीबरोबर गप्पा मारणे, मुलांबरोबर खेळणे आणि स्वतःसोबत शांत वेळ घालवणे ह्याच गोष्टी माझ्या आनंदासाठी पुरेशा आहेत, तेव्हा मला बाहेरील एंटरटेनमेंट (मनोरंजन) किंवा शॉपिंग (खरेदी) यांची गरज भासू लागली नाही. ह्या बदलामुळे मी अनावश्यक खर्च कमी करू शकलो, गैरजरुरी खरेदी थांबवली आणि आर्थिक दृष्ट्या स्थिरता मिळवली, अगदी कर्जमुक्तही झालो.
\item जेव्हा मी साध्या, घरच्या घरी बनवलेल्या अन्नात समाधान मानायला शिकलो, तेव्हा मला सतत हॉटेलमध्ये किंवा बाहेर जाऊन खाण्याची गरज भासू लागली नाही. अधूनमधून बाहेर खाणे चालते, परंतु ती गरज नसून केवळ एक निवड बनली. ह्यामुळे माझे शरीर हलके झाले आणि वजनही संतुलित राहू लागले.
\item जेव्हा मी माझ्या आजूबाजूच्या छोट्या-छोट्या गोष्टींकडे नव्याने लक्ष देण्यास शिकलो, तेव्हा पक्ष्यांचे किलबिलणे, झाडांचे हिरवेपण, चालत्या रस्त्यावरची साधी हालचाल ह्यांतही मला आश्चर्य आणि आनंद दिसू लागला. ह्या सजगतेमुळे माझी गाडीवर अवलंबून राहण्याची सवय कमी झाली. आज मी गाडी वापरून फारसे जात नाही, आणि चालणे व सायकलिंग ह्यामुळे माझा शरीर निरोगी राहतो. यामुळे मी ग्लोबल वॉर्मिंग (जागतिक उष्णतेतील वाढ) मध्ये माझा निसर्गपूरक वाटा कमी करू शकतो.
\item सर्वांत मोठा बदल म्हणजे, “आणखी हवे”, “आणखी चांगले हवे” या कधीही न संपणाऱ्या इच्छांच्या चक्रातून मी स्वतःला मुक्त केले. मी पूर्ण मनाने जाणवले की माझ्याकडे जे काही आहे तेच पुरेसे आहे. आज मी पूर्वीपेक्षा अधिक समाधानी, अधिक स्थिर आणि खरोखर आनंदी आहे.
 \end{itemize}
समाधान मिळवणे ही एखाद्या रात्रीत होणारी जादू नसते. हे थोड्या थोड्या प्रमाणात, हळूहळू आपल्या जीवनात फुलत जाते. जर आजपासून काही सोप्या व उपयुक्त सवयी अंगीकारल्या तर जीवनात हळूहळू समाधान येऊ लागते:
\begin{itemize}
 \item आत्ता क्षणभर थांबा आणि आपल्या आजूबाजूला नीट पाहा. कदाचित तुम्ही घरी बसलेले असाल किंवा रस्त्यावर चालत असाल. लक्षात घ्या की तुमच्या आजूबाजूला असलेली बरीच साधी वस्त्रे, सहवास, अन्न व नैसर्गिक सौंदर्य आनंदासाठी पुरेसे आहेत. खरं तर आनंदासाठी फार फार काही लागत नाही. जीवनातील मूलभूत गरजा म्हणजे अन्न पोट भरण्यासाठी, डोक्यावर छप्पर ठेवण्यासाठी, अंग झाकण्यासाठी कपडे, सोबत करण्यासाठी इतर माणसे, काहीतरी अर्थपूर्ण काम आणि समाधानाची वृत्ती. बाकी सर्व अपेक्षा फक्त अतिरिक्त आहेत.
\item तुम्हाला काहीतरी अर्थपूर्ण काम करायचे असेल तर लगेच नोकरी बदलण्याची घाई करू नका. जिथे आहात तिथेच सुरुवात करा. इतरांना मदत करा, जशी शक्य आहे तशी. आपल्या सहकाऱ्यांना यशस्वी होण्यासाठी मार्गदर्शन करा. मित्रांच्या गरजेच्या वेळी त्यांचा पाठीराखा बना. आपल्या घरच्यांबरोबर वेळ घालवा आणि त्यांना प्रोत्साहित करा. गरजू लोकांसाठी स्वयंसेवा करा. समाजात लहान किंवा मोठे, पण सकारात्मक बदल घडविण्याचा प्रयत्न करा.
\item माणसांशिवाय जीवनात रिक्तता निर्माण होते. तुम्हाला सोबत हवी आहे का? तर शेजाऱ्यांशी गप्पा मारा, नवी मैत्री जोडा. स्वयंसेवा करताना लोकांशी जुळा आणि नवीन नातेसंबंध तयार करा. कार्यालयात सहकाऱ्यांबरोबर संवाद साधा. प्रत्येक व्यवहारात नम्रता, विचारशीलपणा आणि सकारात्मकता ठेवा.
\item आपल्या आयुष्यातील आशीर्वादांची, म्हणजेच कृतज्ञतेची कारणे रोज मोजायला शिका. तुमच्याकडे जे आहे ते केवळ थोडे नाही, तर अत्यंत मौल्यवान आहे, हे सतत स्वतःला आठवून द्या.
\item जेव्हा तुम्ही स्वतःला “मला अजून काही हवे आहे” असा विचार करताना पकडता, तेव्हा थोडा विराम घ्या आणि तुमच्याकडे जे काही आहे त्याचे कौतुक करा.
\item जागरूकता म्हणजे फक्त ध्यानधारणा (मेडिटेशन) नव्हे, तर प्रत्येक कृतीत उपस्थित राहणे. खाणे, अंघोळ करणे, चालणे, काम करणे, भांडी धुणे, गप्पा मारणे, लिहिणे, वाचन ह्या सगळ्या साध्या कृतींमध्येही अधिक सजगतेने सहभागी व्हा. ह्या कृतींतून जीवनाचा नवा अर्थ सापडतो.
\item तुमची सजगता आणखी वाढवायची असेल तर दररोज थोडा वेळ शांत बसून ध्यानधारणा करा. बसून श्वासावर लक्ष केंद्रित करा आणि मनाला शांततेत बुडू द्या.
 \end{itemize}
जेव्हा तुम्ही खरे समाधान शोधता, तेव्हा तुम्हाला जाणवते की आनंदासाठी फार थोडकंच पुरेसं आहे. त्यासाठी मोठ्या प्रयत्नांची गरज नसते. अशा वेळी जीवन हलकं, सोपं आणि पूर्वीपेक्षा अधिक सुखदायक व अर्थपूर्ण वाटू लागते.


%%%%%%%%%%%%%%%%%%%%%%%%%%%%%%%%%%%%%%%%%%%%%%%%%%%%%%%
 \chapter{यशाची हाव आणि मान्यता मिळवण्याची भूक सोडा}
चिनी तत्वज्ञ लाओत्से याने एक अतिशय विचारप्रवर्तक विधान केले आहे:
\begin{quote}
 "यश हे अपयशाइतकेच धोकादायक आहे.
 आशा ही भीतीइतकीच पोकळ आहे."
यश धोकादायक कसे, अपयशासारखेच?
 कारण तुम्ही शिडीवर वर चढलात किंवा खाली उतरला,
 तुमचे स्थान कधीच स्थिर नसते.
 पण जर तुम्ही दोन्ही पाय जमिनीवर रोवले,
 तर तुम्ही नेहमीच संतुलनात राहाल.
 \end{quote}
आपल्या मनात "यश" हा शब्द लहानपणापासूनच रोवला जातो. बालवाडीतल्या स्पर्धेतून ते शालेय शिक्षण, महाविद्यालय, परीक्षा, क्रीडा, नोकरी, करिअर, सगळे जणू या संकल्पनेभोवतीच फिरते. लहानपणापासून समाज आपल्याला शिकवतो की यशस्वी होणे हेच जीवनाचे अंतिम ध्येय आहे. पण कधी आपण थांबून विचार करतो का, यश म्हणजे नेमकं काय? त्याची व्याख्या कोण करतो? ते एवढं महत्त्वाचं का आहे? आणि जर ते मिळालंच नाही तर खरोखर काही बिघडतं का?
कधी कधी यश मिळूनही मन भरत नाही. अजून हवेसे वाटते. किंवा मिळालेल्या यशाचे महत्त्व वेळेनुसार कमी होत जाते. आपल्याला जाणवते की एवढा परिश्रम, एवढी धडपड, एवढा खटाटोप करून जे मिळवले ते तितकेसे मोलाचे नाही. मग आपण आयुष्याचा मोठा भाग वाया घालवला नाही का, असा प्रश्न मनाला पडतो.
यश मिळवण्याची हाव आणि लोकांनी आपल्याला "सुप्रसिद्ध" (फेमस) किंवा "यशस्वी" (सक्सेसफुल) म्हणावे, ही भूक आपल्याला अनेकदा अशा गोष्टींकडे ढकलते ज्या खऱ्या तर अनावश्यक असतात. मोठं घर, महागडी गाडी, नवनवीन फॅशनचे कपडे, आधुनिक साधने आणि उपकरणे (गॅजेट्स), जगभर फिरणे, प्रतिष्ठित नोकरी, कामगिरींची लांबलचक यादी, आणि इंटरनेटवर हजारो अनुयायी (फॉलोअर्स) जमवणे, हे सगळं कशासाठी? फक्त जगासमोर आपण मोठे, श्रेष्ठ, विशेष आहोत हे दाखवण्यासाठी!
पण खरी गोष्ट अशी आहे की इतर लोक आपल्या यशाकडे फार लक्ष देत नाहीत. ते त्यांच्या स्वतःच्या काळजीत, अडचणीत, यश-अपयशात इतके व्यस्त असतात, की तुमचं यश त्यांच्यासाठी फक्त एक क्षणभराचं नजरेस पडणारं दृश्य असतं.
म्हणूनच ही भूक सोडायला शिका. "मला यश हवंय", "लोकांनी आपल्याला मान द्यावा" – या सततच्या तगमगीला बाजूला ठेवा. हो, आपल्याला समवयस्कांपुढे छान दिसावंसं, चांगलं काम करावंसं वाटतंच; परंतु तेच जीवनाचा मुख्य उद्देश बनवणं धोकादायक आहे.
पाय जमिनीवर घट्ट रोवा. जीवनात संतुलन शोधा. समाधान शोधा. "यश" हा शब्द मनातून काढून टाका. कारण खऱ्या जगण्याची ताकद ही यशात नसून साधेपणात आणि समाधानात आहे. स्वतःला तसेच स्वीकारणे आणि त्यातून अंतःशांती अनुभवणे हेच खरे परिपूर्ण जीवन आहे.


%%%%%%%%%%%%%%%%%%%%%%%%%%%%%%%%%%%%%%%%%%%%%%%%%%%%%%%
 \chapter{वजाबाकीला प्राधान्य द्या}
आपल्या स्वभावात एक अत्यंत वैशिष्ट्यपूर्ण आणि सर्वसाधारणपणे दिसणारा कल असतो, आयुष्यात सतत नवनवीन गोष्टी जोडत राहण्याचा.
 आपण जणू सतत एखाद्या धावपळीच्या स्पर्धेतच आहोत. अधिकाधिक गोष्टी मिळवायच्या आहेत, अजून नवी कामं करायची आहेत, नवे छंद जोपासायचे आहेत, नवे मित्र जोडायचे आहेत, आणि शक्य तितकं जास्त मिळवण्याची अखंड धडपड सुरू असते.
पण हे विसरतो की जेव्हा आपल्या जीवनात एखादी नवी गोष्ट सामील होते, तेव्हा तिच्यासोबत नवे प्रयत्न, नवे कष्ट, नवे व्यवस्थापन आणि नवी जबाबदारीसुद्धा येते.
 प्रत्येक नवे साधन, नवा मित्र, नवा छंद, नवे ध्येय, यापैकी प्रत्येकाला स्वतंत्रपणे वेळ, ऊर्जाशक्ती आणि लक्ष द्यावं लागतं. हळूहळू या सगळ्या गोष्टींनी आपण एवढे वेढले जातो की आयुष्यच ओझ्यासारखं वाटू लागतं.
 श्वास घ्यायलासुद्धा वेळ नाही, असं जाणवतं. अशा वेळी कोणत्या गोष्टी कमी करायच्या, कोणत्या टाकून द्यायच्या, हे ठरवणं खूपच अवघड होतं.
म्हणून एक साधं पण गहिरे तत्त्व लक्षात ठेवा: आयुष्यात नवी गोष्ट जोडण्याआधी सावधगिरी बाळगा. शक्य असेल तेव्हा नवी भर घालण्यापेक्षा वजाबाकीकडे कल ठेवा.
उदाहरणार्थ, जेव्हा एखादं नवं ऑनलाईन सामाजिक जाळं (सोशल नेटवर्क) बाजारात येतं, तेव्हा लगेच खाते उघडण्याऐवजी थोडा विचार करा.
 जोडण्यापेक्षा, आधीपासून चालू असलेल्या निरर्थक ऑनलाईन सवयी कमी करण्याकडे लक्ष द्या.
नवे मित्र जोडताना, नवे प्रकल्प सुरू करताना, नवी जबाबदारी स्वीकारताना नेहमी विचारपूर्वक पाऊल टाका.
 जर काही गोष्टी तुमच्या जीवनात मूल्य वाढवत नसतील, तर त्या ठेवून गोंधळ का वाढवायचा? त्यातून स्वतःला सावकाश मुक्त करा.
हे सदैव लक्षात ठेवा की वजाबाकी ही एक सखोल, विचारपूर्वक आणि हळूहळू करायची प्रक्रिया आहे.
 नवीन गोष्ट जोडणं मात्र फार सहज घडतं. कारण "हो" म्हणणं सोपं असतं, पण त्याचे आयुष्यावर होणारे दूरगामी परिणाम आपण अनेकदा दुर्लक्षित करतो.
 म्हणून जेव्हा कधी नवी गोष्ट करण्याचा मोह होईल, तेव्हा थोडा वेळ घ्या, शांतपणे विचार करा आणि जे काही शक्य आहे ते सावकाश कमी करत जा.
आयुष्याचा कुशल कारागीर व्हा, सजग निवड करणारा क्युरेटर व्हा.
 जशी कलादालनातील प्रदर्शनं नीट निवडून, छाननी करून आणि योग्य पद्धतीने मांडलेली असतात, तशीच आपल्या जीवनातील गोष्टी निवडा.
 अनावश्यक गोष्टी हळूहळू कापून टाका, जोपर्यंत तुमच्याकडे फक्त त्या गोष्टी उरतात ज्या तुम्हाला खऱ्या अर्थाने प्रिय आहेत, ज्या अत्यावश्यक आहेत आणि ज्या तुम्हाला खरा आनंद देतात.

%%%%%%%%%%%%%%%%%%%%%%%%%%%%%%%%%%%%%%%%%%%%%%%%%%%%%%%
 \chapter{विचारसरणी बदलणे आणि अपराधीपणातून मुक्त होणे}
जेव्हा लोक प्रथमच प्रयत्नरहित जीवन (एफर्टलेस लाईफ) ही संकल्पना ऐकतात, तेव्हा त्यांना अनेकदा ती विचित्र किंवा धोकादायक वाटते; इतकेच नाही तर त्वरित नकारात्मक भावना उत्पन्न होतात. या संकल्पनेचा मूळ आशय असा आहे की ध्येयांना पूर्ण करण्यासाठी वारंवार होणाऱ्या तणावपूर्ण प्रयत्नांच्या ऐवजी ध्येयांचे स्वरूप आणि अपेक्षांचे प्रमाण याचा परिमाणात्मक विचार करावा, जबरदस्त नियंत्रण थोडे शिथिल करावे आणि जीवनातील सततच्या धावपळीच्या पळण्या-पूसण्या ऐवजी क्रियाकलाप थोडे कमी, परंतु अधिक गुणात्मक पद्धतीने करावेत. हा बदल तात्काळ आळशीपण किंवा निष्क्रियता नव्हे, तर जास्त नैसर्गिक आणि सहजतेने जगण्याचा एक मार्ग आहे.
आपली सामाजिक आणि कौटुंबिक संस्कृती "जास्त कर", "जास्त मिळव", "जास्त कष्ट कर" या तत्त्वांना प्रामुख्याने महत्त्व देते आणि हाच संदेश लहानपणापासून सतत आपल्यावर भिनवला जातो. शाळा, कामाच्या ठिकाणीची स्पर्धा, समाजातील मुल्यनिर्धारण आणि अगदी कुटुंबीयांचे अपेक्षित वर्तन, हे सर्व एकत्र मिळून ‘जास्त करणे’ हा आदर्श तयार करतात. त्यामुळे कमी करणे म्हणजे निकम्मेपणा किंवा आळस आणि ‘निष्क्रियता’ अशीच व्याख्या अनेकांच्या मनात आपोआप उगवते; हाच समाजशास्त्रीय शास्त्रज्ञादर्शी वर्तनाचा परिणाम आहे.
परंतु विचार करण्यासारखे आहे की या शिकवणीमागील तत्त्व कुठे आणि कधी प्रभावी ठरते आणि कुठे ती अपुरे किंवा त्रासदायक आहे. आपण जर सर्व क्षणांमध्ये सतत प्रयत्न करत राहिलो तर काय होईल, व्यक्तीची अंतर्गत शांती टिकेल का, संबंध मजबूत राहतील का, आणि आतून समाधान मिळेल का? माझ्या वैयक्तिक जीवनाच्या अनुभवातून आणि अनेकांसोबत झालेल्या संवादातून मला कळाले आहे की सततची धावपळ हि नेहमीच समाधान देणारी नसते; अनेकवेळा ती थकवा, असमाधान आणि अंतर्गत गोंधळ निर्माण करते.
जर तुम्हाला ह्या संकल्पनेबद्दल त्वरित नकारात्मक भावना येत असतील तर त्या भावनांकडे दुर्लक्ष करणे चुकीचे ठरेल; त्यांचा अभ्यास करणे आवश्यक आहे. स्वतःला विचार करा, माझे हे विचारसरणीचे तुकडे खरे आहेत का, किंवा मी फक्त सांस्कृतिक स्पर्धेच्या दबावाखाली आहे का? तसेच, अंदाजांवर कायम राहण्यापेक्षा थोडा प्रयोग करून पाहणे अधिक उपयुक्त ठरते: स्वतःच्या दैनंदिन जीवनात काही वेळेस प्रयत्न कमी करून बघा, आणि त्या बदलाचे परिणाम नोंदवून स्वतःला पुरावा मिळवा.
असे प्रयोग सुरू केल्यावर सुरुवातीला अनेकांना अपराधीपणाची तीव्र भावना येते; मन विचारते, "मी कमी करतोय म्हणजे मी आळशी झालोय का?" किंवा "मी माझे कर्तव्य बाजूला ठेवतो आहे का?" ही शंका नैसर्गिक आहे आणि त्याकडे संवेदनशीलतेने पाहिले पाहिजे. परंतु वेळोवेळी जेव्हा त्या कमी केलेल्या प्रयत्नांमुळे परिणाम अधिक स्थिर, अधिक स्पष्ट किंवा अधिक आनंददायी दिसू लागतात, तेव्हा मन स्वतःहून हलके होते आणि अपराधीपणाची ती भावना कमी होऊ लागते.
कमी करणे म्हणजे आळस नव्हे; ते एक जागरूक निर्णय असू शकतो, जीवनातील अनावश्यक दडपण कमी करून अधिक सजगतेने व समाधानीपणे आयुष्य घालवण्याचा निर्णय. ह्या पद्धतीने जगताना मनाची गुणवत्ता बदलते, आपल्याला स्वतःच्या अनुभवांकडे लक्ष देण्यासाठी वेळ मिळतो आणि संबंध, आनंद व शांतता या पैलूंना नव्याने पोत मिळतो. अशा प्रकारे ‘कमी करणे’ जीवनातील एक शाश्वत आणि समतोलपूर्ण दृष्टिकोन बनतो.
या संपूर्ण प्रक्रियेचा शेवट असा आहे की आपण जुन्या विचारसरणीला अंधपणे नाकारत नाही तर स्वत:ला परवानगी देत आहोत, अनुभव करून बघण्याची, परीक्षणाची आणि नंतर निर्णय घेण्याची परवानगी. हेच परिपक्व, अर्थपूर्ण आणि समाधानकारक जीवन आहे, अर्थातच अर्थपूर्ण आणि समाधानकारक जीवन (गुड लाईफ) असे म्हणता येईल आणि होय, या विचारसरणीच्या तटस्थ विश्लेषणानंतर किंवा प्रत्यक्ष अनुभवानंतर अनेकांना दिसते की हेच खरे प्रयत्नरहित जीवन (एफर्टलेस लाईफ) आहे.
%%%%%%%%%%%%%%%%%%%%%%%%%%%%%%%%%%%%%%%%%%%%%%%%%%%%%%%
\chapter{पाण्यासारखे बना}
प्रसिद्ध लढाऊ कलांचा कलाकार ब्रूस ली (मार्शल आर्टिस्ट) यांनी जीवन आणि लढाऊ कलांत अनुभवलेल्या सङ्गतीतून लवचिकतेचा एक गहन आणि निर्णायक धडा दिला आहे. त्यांच्या सूक्ष्म आणि प्रभावी वचनाने हे दाखवले आहे की व्यक्तीचा मनोवृत्तीचा स्वरूप ज्या प्रकारे पाण्याचे रूप बदलते तसाच बदलता येऊ शकतो; हा विचार साधा वाटला तरी त्याचा जीवनावर खोल परिणाम होतो. त्यांच्या त्या शब्दांचा अर्थ समजून घेणे आणि त्यानुसार वागणे म्हणजे केवळ तत्त्वज्ञान पाळणे नव्हे, तर रोजच्या जीवनात जीवनशैली बदलण्याचे एक व्यावहारिक सूत्र आहे.
\begin{quote}
 भेगांमधून मार्ग काढणाऱ्या द्रवासारखे स्वतःला वागवा; आडमुठे थांबून न राहता परिस्थितीशी जुळून घ्या आणि मग मार्ग आपोआप उघडून येईल.
 मन आतून कठीण न ठेवता मृदू ठेवा; बाह्य जग आपल्यासाठी उघडले जाण्यास अधिक सुलभ होईल.
 मन रिकामे ठेवा, आकाररहित व्हा — पाण्यासारखे बना.
 जेव्हा पाणी कपात टाकले जाते, तेव्हा ते त्या कपाच्या आकारात रूपांतरित होते (यू पुट वॉटर इन अ कप, इट बिकम्स द कप).
 जेव्हा ते बाटलीत टाकले जाते, तेव्हा ते त्या बाटलीच्या स्वरूपात बदलते (यू पुट वॉटर इंटु अ बॉटल, इट बिकम्स द बॉटल).
 जेव्हा ते चहाच्या पितळीत टाकले जाते, तेव्हा ते त्या पितळ्याच्या रूपात वर्तते (यू पुट इट इन अ टी-पॉट, इट बिकम्स द टी-पॉट).
 पाणी सौम्यपणे वाहू शकते किंवा प्रचंड उर्जेने धडक देऊ शकते; परिस्थितीनुसार त्याची प्रवृत्ती बदलते (नाऊ वॉटर कॅन फ्लो ऑर इट कॅन क्रॅश).
 पाणी बना, मित्रा (बी वॉटर, माय फ्रेंड).
 \end{quote}
वरील संदेश प्रथमदर्शनी साधा दिसतो, परंतु त्याचा संदेश अत्यंत व्यावहारिक आणि जीवनाची दिशा बदलणारा आहे. याचा मूळ आशय असा आहे की आयुष्यातील प्रत्येक क्षण आपल्याला नियोजनाबाहेरची परिस्थिती हाताळावी लागते; त्यामुळे आधीपासून ठाम योजना, एकाच मार्गावर अढळ मन किंवा कठोर अपेक्षा ही अडथळा ठरू शकतात. परिस्थिती सतत बदलत राहते, आणि त्याच बदलाशी जुळवून घेण्याची क्षमता जोपासणे म्हणजे खऱ्या अर्थाने सशक्त होणे.
ठाम अपेक्षा आणि आखीवरेखीव योजना जर मनात घट्ट बसवल्या तर लवचिकतेचा काय अनुभव राहतो? मन एका विशिष्ट मार्गावर अडकले असल्याने नवीन संधी किंवा वेगळ्या दिशेने जाण्याची शक्यता कमी होते. दगडासारखे कठोर विचार आणि वागणूक ठेवणे म्हणजे बदलांना प्रतिकार करणे; परंतु पाण्यासारखे मृदू व लवचिक मन ठेवण्याचा अर्थ म्हणजे प्रवाहानुसार जुळवून घेणे, नव्या परिस्थितीत स्वतःला सहज रुजू करणे आणि आवश्यक ते बदल स्वीकारणे.
भावनिक स्तरावरही याचे महत्त्व खूप मोठे आहे. योजना बिघडल्यावर किंवा अपेक्षित परिणाम न आल्यावर येणारा राग, वैताग किंवा निराशा यांचे मूळ बऱ्याचदा आपण केलेल्या ठाम अपेक्षांमध्ये असते. अपेक्षा सोडायला शिकलो तर या नकारात्मक प्रतिक्रियांचा भार कमी होतो आणि आपण अधिक स्पष्ट, शांत व विचारप्रवण मनाने निर्णय घेऊ शकतो. बदल स्वीकारल्यावर मनातली ऊर्जा रागकचाटी किंवा खिन्नतेवर खर्च न करता पुढे जाण्यात वापरता येते.
याच प्रक्रियेत जीवनाची खरी सुंदरता उलगडते. जेव्हा आपल्याकडे एकच ठाम मार्ग नसतो, तेव्हा प्रत्येक वळण नवीन संधी, नवीन अनुभव आणि नव्या दिशेने जाण्याची संधी घेऊन येते. अनिश्चिततेत सामर्थ्य शोधणे म्हणजे प्रत्येक घटनेतून शिकून पुढे जाणे; अनुभवांना स्वीकारून त्यानुसार आपले मार्ग बदलणे म्हणजे बुद्धिमत्तेचे आणि सहजतेचे जीवन.
आध्यात्मिक दृष्टिकोनातून पाहिल्यास हा विचार ध्यान आणि स्मृतीयोगाच्या तत्त्वांशी घट्ट जोडतो. मन रिकाम्या अवस्थेत ठेवण्याचा तात्पर्य म्हणजे विचारांमधून अनावश्यक पकड सोडणे, श्वासावर केंद्रित राहून साक्षीभाव निर्माण करणे आणि घडणाऱ्या घटनेशी न्याय्य, निर्विकार दृष्टी ठेवणे. रोजच्या ध्यानसाधनेतून, श्वासावर लक्ष केंद्रीत करून आणि येणाऱ्या-निघणाऱ्या विचारांना जडले न करता बघण्याचा सराव केल्यास हा लवचिकतेचा भाव स्वाभाविकपणे वाढतो.
शेवटी लक्षात ठेवण्यासारखे महत्वाचे बिंदू म्हणजे तरीही आयुष्यात स्थैर्यही गरजेचे असते; पाण्यासारखे बनताना त्याचा अर्थ अनन्याश्रयी होणे नाही, तर परिस्थितीशी जुळवून घेण्याची आणि आवश्यक तेव्हा ठामपणे उभे राहण्याची क्षमता बाळगणे हा आहे. त्यामुळे ब्रूस लीचे हे वचन नेहमी स्मरणात ठेवा, “पाण्यासारखे बना” (बी लाइक वॉटर) आणि जीवनाच्या प्रवाहात शांतपणे, जागरूकपणे व लवचिकतेने पुढे चला.

%%%%%%%%%%%%%%%%%%%%%%%%%%%%%%%%%%%%%%%%%%%%%%%%%%%%%%%
\chapter{प्रत्येक कृतीला समान महत्त्व द्या}
माझी एक झेन ध्यानपरंपरेतील भिक्षुणी आणि जवळची मैत्रीण आहे, सुसान ओ’कॉनेल. ती सॅन फ्रान्सिस्को झेन केंद्र (सॅन फ्रान्सिस्को झेन सेंटर) मध्ये उपाध्यक्ष म्हणून कार्यरत आहे, आणि तिचा पाठीमागचा जीवनप्रवास हा ध्यानधारणा आणि साधनांच्या माध्यमातून घडलेला आहे; त्याचबरोबर ती पूर्वी चित्रपट व दूरदर्शन अभिनेत्री (मूव्ही अँड टीव्ही ऍक्ट्रेस) म्हणूनही काम करीत होती. अलीकडे तिने मला एक अतिशय सखोल आणि उपयोगी धडा दिला. तिचे ते भाषण इतके विचारप्रवर्तक आणि जाणीवपूर्वक होते की मी लगेचच त्या कल्पना माझ्या दैनंदिन जीवनात लागू करून बघू लागलो.
सुसान म्हणते की तिला आपल्या प्रत्येक दिवसातील प्रत्येक कृतीला आणि प्रत्येक क्षणाला समान महत्त्व द्यायचे असते. बहुतांश लोक जशाप्रमाणे काही कामांना अधिक महत्त्व देतात आणि त्यांवरच आपले लक्ष केंद्रीत करतात, तशीच आपली सवय असते; परंतु त्या दृष्टीकोनामुळे इतर छोट्या-मोठ्या तसेच सक्तीच्या कामांकडे दुर्लक्ष होते किंवा त्यांना कमी महत्त्व समजले जाते. अशाप्रकारे आपण जीवनाच्या अनेक सूक्ष्म क्षणांना न पाहता जातो, आणि त्या क्षणांमधील अनुभव कुठेतरी गळून पडतो.
सुसानच्या दृष्टीने ध्यानधारणा करणे, एखादे महत्त्वाचे व्यावसायिक कार्य करणे, रस्त्यात एखाद्या अनोळखी व्यक्तीशी विनम्रपणे गप्पा मारणे, गाडी पार्क केलेल्या जागेपर्यंत (पार्किंग) चालून जाणे, किंवा फक्त गरम सूपचा एक वाडगा उबदारपणे घ्यावा, प्रत्येक कृतीला तितकाच महत्त्व आहे. कोणतीही कृती दुसऱ्या कृतीपेक्षा जास्त किंवा कमी महत्त्वाची नाही; प्रत्येक अनुभवाला आपण समान ओघाने आणि सावधगिरीने सामोरे जावे, असे तिचे मत आहे.
तिने अधिक खोलवर जाऊन सांगितले की फक्त कृतीच नाही तर त्या कृतींच्या मध्ये येणाऱ्या सूक्ष्म क्षणांनाही तितकंच महत्त्व द्यावे, म्हणजेच “मधली जागा” (स्पेसेस बिटवीन). ही मधली जागा म्हणजे काय हे समजावून घेण्यासाठी साधे उदाहरण घ्या: तुम्ही एखादा इलेक्ट्रॉनिक पत्र वाचून संपवता आणि लगेचच सहकाऱ्याशी बोलायला सुरुवात करता, त्या असलेल्या संक्रमणाच्या क्षणाला एक वेगळे स्थान आहे; किंवा जेवल्यानंतर ताट सिंकमध्ये ठेवण्यासाठी हात वाढवताना तो क्षणही एक स्वतंत्र जागा आहे. बहुतेक वेळा आपण हे छोटे-छोटे संक्रमणाचे क्षण लक्षातच घेत नाही आणि ते नकळत निघून जातात.
या ‘मधल्या जागांना’ आपण जशी एखाद्या मोठ्या, महत्त्वाच्या कामाला महत्व देतो तसाच दिला तर काय बदल घडू शकेल, हे विचार करा. जर एक दिवस असेलच ज्यात प्रत्येक कृती आणि प्रत्येक मधला क्षण सजगतेने आणि पूर्ण जाणीवेत अनुभवले गेलेले असतील, तर तो दिवस कसा असेल? माझ्या व्यक्तिगत अनुभवात आणि सुसानच्या शिकवणीनुसार असा दिवस अधिक सजग, अधिक मऊ गतीचा, समतोलयुक्त आणि मानसिकदृष्ट्या स्थिर राहतो. अंगावरील ताण कमी होतो, परिश्रम कमी शनिग्रहण करणारे वाटू लागतात, आणि मन जास्त शांत व प्रसन्न राहते.
हा प्रयोग प्रत्यक्षात करून बघण्यासारखा आहे. एक तास इतका प्रयत्न करा की जो काही तुम्ही करतो ते सर्व काही पूर्णपणे जाणीवपूर्वक करा, सामान नीट ठेवताना स्वच्छ आणि ठाम हालचाल, एका खोलीतून दुसऱ्या खोलीत जाताना पाय टाकण्यापासून ते दरवाजा उघडण्यापर्यंतची प्रत्येक कृती, फोन उचलताना होणारी प्राथमिक प्रतिक्रिया, एखाद्याशी संवाद सुरू करणे, हे सर्व इतक्या जाणीवपूर्वक करा की प्रत्येक कृती स्वतःमध्ये एक पूर्ण क्रिया असल्याचा अनुभव येईल. प्रत्येक कृतीला समतुल्य महत्त्व दिल्यास मनाच्या नाट्यमयतेत घट होते आणि आपल्याकडून भावनांचा वाया जाणारा उर्जा खर्च थांबतो.
याचा आणखी एक प्रत्यक्ष लाभ असा आहे की आपण जे काही अत्यधिक महत्त्वाचे समजतो, त्यावर कधी कधी आपण नाटकीयतेचा अधिक आव आणतो; त्या नाटकीयतेमुळे त्या गोष्टींचा प्रभाव कमी वाटू लागतो. जेव्हा प्रत्येक क्रियेला समान तीव्रतेने आणि शिस्तबद्ध भावनेने घेतले जाते, तेव्हा निरर्थक भावनांचा वापर कमी होतो, आपले संतुलन टिकून राहते आणि दृष्टीकोन व्यापक व स्थिर राहतो.
निश्चितपणे, ह्या कल्पनेचा सराव मीही अजून करीत आहे आणि त्यात मला रोज नवीन काहीतरी समजत असते; तरीही हा धडा मौल्यवान ठरला आहे. जर तुम्ही हा दृष्टिकोन आत्मसात केला तर मला खात्री आहे की तुमच्या दैनंदिन जीवनात त्याचा सकारात्मक परिणाम दिसेल. आणि जर त्या बदलाचा श्रेय द्यायचे तर तो सुसान ओ’कॉनेल (सुसान ओ’कॉनेल) यांना द्यावे, असे मला म्हणायला हरकत नाही.

%%%%%%%%%%%%%%%%%%%%%%%%%%%%%%%%%%%%%%%%%%%%%%%%%%%%%%%
 \chapter{साधा आहार}
मी माझ्या दैनंदिन जीवनशैलीत हळूहळू बदल केले आणि त्यातून माझं आरोग्य सुधारत गेलं. शरीर अधिक हलकं-फुलकं झालं, आकार नीटस आणि सुदृढ दिसू लागला, मन प्रसन्न व आनंदी राहू लागलं. हे सगळं एखाद्या मोठ्या खर्चिक उपचारामुळे किंवा कठोर शिस्तीमुळे नव्हे, तर एका साध्या, समजूतदार आणि बुद्धिमान आहार पद्धतीमुळे शक्य झालं. 
इथे "डाएट" हा शब्द ऐकल्यावर आपल्याला लगेच डोक्यात येतं की तो एखादा कठोर, टोकाचा, तडजोड न मानणारा आणि नियमबद्ध कार्यक्रम असेल. पण माझा आहार मात्र अगदी याच्या उलट आहे. मी स्वतःला भरपूर मोकळीक देतो. कधी कधी चुकलो तरी मी ते सहज मान्य करतो आणि स्वतःवर रागवत नाही. त्यामुळे माझ्या आहारात साधेपणा आहे, पण माझा दृष्टिकोन लवचिक आणि जीवनाशी सुसंगत आहे.
\section*{मी काय खातो}
मी मुख्यतः नैसर्गिक, न वाळवलेले आणि प्रक्रिया न केलेले वनस्पतीजन्य पदार्थ खातो. मात्र हे नेहमीच काटेकोरपणे पाळतो असं नाही, कधी अपवाद होतोच. माझ्या दैनंदिन आहाराचं मिश्रण साधारणपणे असं असतं:
हिरव्या पालेभाज्यांचा मोठा भाग माझ्या थाळीत असतो. 
कडधान्यं, त्यात सोयाबीनसकट सर्व प्रकाराचा समावेश असतो. 
बदाम, अक्रोड, जवस, क्विनोआसारख्या बिया व सुकामेवा हे माझ्या आहारात नित्याचे घटक आहेत. 
मी अख्खे धान्य वापरतो, जसं की स्टील-कट ओट्स, ब्राऊन तांदूळ, तसेच इतर पूर्ण धान्यांचे पदार्थ. 
फळं मात्र मी कोणत्याही मर्यादेशिवाय खातो. 
पेयांमध्ये कधी कधी द्राक्षासव (वाइन), कॉफी किंवा चहा घेतो.
कधी कधी मी थोडेसे प्रक्रिया केलेले पदार्थही खातो, जसं की ऑलिव्ह तेल, शेंगदाणा बटर (नट बटर), टोफू, व्हिनेगर इत्यादी. मला आवडणाऱ्या पदार्थांत अवोकाडो, काळा हरभरा, मसूर, बदाम, नारळाचं दूध, बेरी फळं, रताळी आणि अंकुरलेली धान्यं यांचा नेहमी समावेश असतो.
माझ्या ताटात अनेकदा झणझणीत चिली असतं, टोफू आणि भाज्यांचं क्विनोआसोबत स्टर-फ्राय असतं, ब्राऊन राईसवर काळ्या हरभऱ्याचं आणि ताहिनी सॉसचं रुचकर मिश्रण असतं, केल भाजीसोबत विविध पदार्थ असतात. कधी कधी स्टील-कट ओट्सवर कच्चे बदाम, बेरीज, जवस पावडर आणि थोडंसं दालचिनी टाकून मी न्याहारी करतो. म्हणजेच, माझं खाणं साधं असलं तरी त्याची चव अप्रतिम असते.
\section*{मी काय कमी खातो}
मी जे पूर्णपणे बंद केलं आहे ते म्हणजे मांस आणि सर्व प्राणीजन्य पदार्थ. बाकी काही गोष्टी मात्र मी बंद केल्या नाहीत, परंतु जुन्या सवयींच्या तुलनेत त्यांचं प्रमाण खूपच कमी झालं आहे.
मी गोड पदार्थ फारच क्वचित खातो. प्रक्रिया केलेली धान्यं, साखरयुक्त पेयं आणि तेलकट तळलेले पदार्थही आता माझ्या आहारात फारच कमी आलेले असतात. कधी कधी या पदार्थांची मजा घेतो, पण त्याचं प्रमाण नेहमी मर्यादित ठेवतो.
\section*{मी प्राणीजन्य पदार्थ का टाळले?}
मांस, दूध, अंडी हे मी आरोग्याच्या कारणामुळे बंद केलेलं नाही. तर ते मी नैतिक कारणामुळे बंद केलं आहे. लाखो निरपराध आणि भावनाशील प्राण्यांना केवळ आपल्या जिभेच्या चवीसाठी यातना देणं, त्यांचा वध करणं—यात काहीच न्याय नाही असं मला ठामपणे वाटतं. अर्थात हे माझं वैयक्तिक मत आहे आणि तुम्हीही ते तुमच्या आहारातून बाद करावं असा आग्रह मी अजिबात धरणार नाही. तुमचा आहार “साधा” आणि पौष्टिक होण्यासाठी जे करता येईल ते तुमच्या शरीरानुसार आणि आरोग्यानुसार ठरवा. 
काही लोक म्हणतात की मासे किंवा दही आरोग्यासाठी चांगले असतात. हो, यात शंका नाही, पण त्याशिवायही आपण पोषक, आरोग्यदायी आणि चविष्ट आहार घेऊ शकतो हे मी स्वतः अनुभवलं आहे.
प्राणीजन्य पदार्थ खाण्याचं एकमेव खरं कारण म्हणजे आपली रसनातृप्ती. पण माझ्या मनाला हे पटत नाही. ज्यांना हे आवडतं त्यांना मी कधीही पाण्यात पाहत नाही किंवा त्यांच्यावर टीका करत नाही, मात्र मी स्वतः त्या प्रवृत्तीत सहभागी व्हायला इच्छित नाही.
\section*{निष्कर्ष}
हा आहार अजिबात कठीण नाही. उलट तो तुलनेने स्वस्त आहे, पोषणाने समृद्ध आहे, फार किचकट तयारी लागत नाही आणि सर्वात महत्त्वाचं म्हणजे तो खायला फारच स्वादिष्ट आहे.
वनस्पतीजन्य, साधे आणि प्रक्रिया न केलेले पदार्थ खाल्ले तरी आपण उत्तम आरोग्य राखू शकतो. यासाठी फार ताण घेण्याची किंवा कठोर शिस्तीचा भार पेलण्याची काहीही गरज नाही. हा आहार आनंदाने करा, आणि कधी कधी कमी आरोग्यदायी पदार्थ थोड्या प्रमाणात खाल्ले तरी काही हरकत नाही.
मी स्वतः अनुभवून लक्षात घेतलं की जेव्हा मी प्राणीजन्य पदार्थ पूर्णपणे बंद केले, तेव्हा उरलेल्या पदार्थांचा आनंद अधिक वाढला. माझ्या ऊर्जा पातळीत (एनर्जी लेव्हल) लक्षणीयरीत्या वाढ झाली.
\section*{काही पाककृती}
लोकांना नेहमीच काही सोप्या आणि उपयुक्त पाककृतींची मागणी असते. म्हणून मी आधी शेअर केलेल्या काही कृतींचे दुवे येथे देत आहे:
\begin{itemize}
 \item स्टील-कट ओट्स: \url{http://zenhabits.posterous.com/my-favorite-healthy-breakfast}
 \item स्क्रॅम्बल्ड टोफू: \url{http://zenhabits.posterous.com/leos-healthy-scrambled-tofu}
 \item वेजी चिली: \url{http://zenhabits.net/health-tip-try-eating-vegetarian/}
\item ताहिनी सॉस (बीन्स, केल, ब्राऊन राईस सोबत): \url{http://www.livestrong.com/recipes/i-am-attentive-spice-tahini-saue/}
 \end{itemize}
%%%%%%%%%%%%%%%%%%%%%%%%%%%%%%%%%%%%%%%%%%%%%%%%%%%%%%
 \chapter{सहजसुंदर पालकत्व}
आपल्यापैकी ज्यांना लेकरं आहेत, त्यांच्यासाठी पालकत्व ही आयुष्यातली सर्वात दमछाक करणारी जबाबदारी असते. खरं सांगायचं झालं तर पालक होणं, मुलं वाढवणं अजिबात सोपं नसतं आणि मी हे प्रकरण लिहिताना खोटं सांगणार नाही की ते सहजशक्य आहे.
परंतु मला वैयक्तिक अनुभव असा आला आहे की आपण आयुष्याप्रमाणेच पालकत्वामधील अनावश्यक ओझं कमी केलं, तर ते खूपच हलकं, आनंददायी आणि सुंदर होऊ शकतं. म्हणून या प्रकरणात आपण दररोजच्या पालकत्वातील सवयी पाहूया आणि "कमी करायच्या" या तत्त्वांमुळे आपल्याला मिळणारी सवलत समजून घेऊया.
पहिली गोष्ट म्हणजे, आपण आपल्या लेकरांना अक्षरशः भरगच्च वेळापत्रकात अडकवून ठेवतो. शाळा, गृहपाठ, अभ्यास या सोबतच खेळ, नृत्यवर्ग, संगीतवर्ग, उन्हाळी शिबिरे, मित्रमैत्रिणींच्या भेटी, वाढदिवस समारंभ आणि अजून कितीतरी उपक्रम. मुलं दिवसभर धावत असतात आणि आपण पालकही त्यांच्या मागे धावतो. पण जर आपण त्यांना कमी उपक्रम दिले, त्यांच्या वेळापत्रकात मोकळीक ठेवली, त्यांना थोडा कंटाळा येऊ दिला आणि त्यांनी स्वतःच्या कल्पनेतून मनोरंजन शोधलं, तर त्यांना श्वास घेण्यासाठी जागा मिळते आणि आपलंही ओझं हलकं होतं.
दुसरी गोष्ट म्हणजे, घर स्वच्छ ठेवण्याबाबतची आपली अती चिंता. आपण कायम मुलांच्या खोलीत स्वच्छता आहे की नाही, खेळणी पसरली आहे का, कपड्यांचा ढीग जमलाय का, यावर चिडतो, रागावतो आणि स्वतःला त्रास करून घेतो. पण जर आपण त्या अपेक्षा कमी केल्या, फक्त स्वतः नीटनेटका राहून उदाहरण दिलं आणि त्यांना त्यातून जे शिकायचं ते शिकू दिलं, तर घरातला ताणही कमी होतो आणि आपला मानसिक त्रासही घटतो.
तिसरी गोष्ट म्हणजे, मुलांच्या यशाबद्दल आपण अती काळजी घेतो. म्हणूनच आपण त्यांच्या उज्ज्वल भविष्यासाठी पायाभरणी करण्याचा अट्टाहास करतो. पण जर आपण त्या अपेक्षा सोडल्या, "ते मोठं होऊन काय बनतील?" हे सततचे प्रश्न विचारणं थांबवलं आणि त्यांनी निवडलेला मार्ग स्वीकारला, तर घरातील हवा अधिक मोकळी आणि आयुष्य अधिक हलकं होतं.
चौथी गोष्ट म्हणजे, मुलांनी आदर्श वर्तन करावं, शिस्त पाळावी आणि आपण सांगतो तेच करावं अशी आपली धारणा असते. परंतु वास्तव वेगळं असतं. मुलं नेहमी तसं वागत नाहीत आणि त्यांनी या अपेक्षा पूर्ण करण्याचा प्रयत्न केला तरी ते प्रचंड ताण अनुभवतात. जर आपण त्या अपेक्षा सोडून त्यांना जसे आहेत तसे स्वीकारलं, तर घरातले वातावरण अधिक मुक्त आणि आनंदी होतं.
पाचवी गोष्ट म्हणजे, मुलांच्या शिक्षणासाठी आपण ठरवलेला आराखडा. मला स्वतःला जाणवलं की माझ्या जुन्या सगळ्या समजुती निरर्थक होत्या. म्हणूनच ईव्हा आणि मी आमच्या मुलांना "अन-स्कूलिंग" म्हणजेच "शाळेबाहेरचं शिकणं" या पद्धतीनं वाढवत आहोत. आम्ही पारंपरिक वरून-खाली (टॉप-डाऊन) शिक्षणपद्धती सोडली आहे. त्याऐवजी आम्ही मुलांना त्यांच्या आवडीनुसार शिकू देतो. ते स्वतः प्रश्न विचारतात, स्वतः उत्तरे शोधतात, स्वतः शिकतात. म्हणजेच ते मोठं झाल्यावर जसं आपण शिकतो, तसं ते लहानपणापासून शिकतात. या पद्धतीमुळे माझ्या पालकत्वातील जबाबदारी हलकी झाली आहे. मुलांना खरंतर दोनच गोष्टी शिकायच्या आहेत, स्वतः शिकण्याची कला आणि प्रश्न सोडवण्याची क्षमता.
मी स्वतः अजून हे धडे पूर्णपणे आत्मसात केलेले नाहीत. मी अजूनही प्रयोग करतो आहे. पण आतापर्यंतचे अनुभव अत्यंत आनंददायी आहेत. मला जाणवलं आहे की पालक म्हणून आपल्याला फारसं काही करावंच लागत नाही. सर्वात मोठं काम म्हणजे त्यांना सुरक्षित ठेवणं आणि आपणच त्यांच्या वाढीला अडथळा न आणणं.
जपानी शेतकरी मसनोबू फुकुओका यांनी "नैसर्गिक शेती" या तत्त्वज्ञानात जसं सांगितलं आहे, की "कमीत कमी हस्तक्षेप म्हणजे अधिक पोषक वातावरण." पालकत्वातही हाच नियम लागू होतो. जितकं कमी नियंत्रण आपण ठेवतो, तितकी लेकरं बहरतात.
याचा अर्थ असा नाही की मुलांकडे दुर्लक्ष करायचं. मी त्यांच्यासोबत वेळ घालवतो, पण तो वेळ असतो मुक्त, बंधनमुक्त आणि अपेक्षाविरहित. मी स्वतः उदाहरण देतो, पण त्यांनी ते अनुकरण करावं अशी अपेक्षा बाळगत नाही. ते कसेही वागले तरी मी त्यांना निरपेक्ष प्रेम देतो. त्यांना स्वतः वाढू देतो, शिकू देतो, कमी हस्तक्षेप करतो आणि निकालांच्या अपेक्षा सोडतो. आणि माझा अनुभव असा आहे की या पद्धतीमुळे मुलं अधिक आनंदी, स्वतंत्र आणि निडर होतात.
%%%%%%%%%%%%%%%%%%%%%%%%%%%%%%%%%%%%%%%%%%%%%%%%%%%%%%
 \chapter{सहजसुंदर नातेसंबंध}
नातेसंबंध ही आपल्या जीवनातील कदाचित सर्वात गुंतागुंतीची गोष्ट आहे. कार्यालयातील सहकाऱ्यांसोबतची भांडणं, जिवाभावाच्या जोडीदारासोबतचे सुखदुःखाचे क्षण, किंवा लेकरं वाढवताना मिळणारे आनंद आणि निराशा,  प्रत्येक नातं हे अनेक पदराचं, जुन्या आठवणींचं आणि त्या आठवणींनी निर्माण झालेल्या भावनांचं मिश्रण असतं.
प्रश्न असा आहे की, हे नातेसंबंध सोपे कसे करायचे? उत्तर साधं आहे, वर्तमानात जगायचं आणि भूतकाळात झालेले अन्याय किंवा कटू आठवणी विसरायच्या. आपल्या जवळच्या लोकांबद्दलच्या अवाजवी अपेक्षा बाजूला ठेवायच्या आणि त्यांना जसे आहेत तसे स्वीकारायचं.
कल्पना करा, तुम्ही सकाळी उठला आणि काल रात्रीच्या एखाद्या किरकोळ वादावरून तुमचा पत्नीवर अजूनही राग आहे. पण तुम्ही हवं तर दुसरा पर्याय निवडू शकता,  जाग आल्यावर तिचं सुंदर चेहरा पाहून मनात म्हणू शकता, "ही माझ्यासोबत आहे, हे किती मोठं सौभाग्य!" अशा प्रकारे तुम्ही तिचं अस्तित्व कृतज्ञतेने स्वीकारू शकता. भूतकाळातले राग फक्त तेव्हाच टिकतात जेव्हा आपण स्वतः त्यात अडकतो. जर आपण वर्तमान क्षणात राहायला शिकलो, तर भूतकाळ आपोआप विरघळून जातो.
शेवटी आपल्याकडे फक्त आत्ताचा क्षण असतो. समोरचा माणूस हा आपल्यासारखाच श्वास घेत असलेला, प्रेम मिळवण्याची इच्छा बाळगणारा जीव आहे. पुढच्या वेळी मित्र किंवा प्रियजनांशी बोलताना हे लक्षात ठेवा. काय घडलं होतं पूर्वी किंवा पुढे काय होईल हे विसरून फक्त त्या व्यक्तीसोबत असण्याचा आनंद घ्या. त्यांचं बोलणं लक्षपूर्वक ऐका आणि त्या क्षणी त्यांच्या जवळ आहात याबद्दल कृतज्ञ राहा.
सगळ्यात महत्त्वाचं म्हणजे, इतरांकडून अपेक्षा करणं सोडून द्यायचं शिका. या अपेक्षाच आपल्याला सतत राग, निराशा आणि त्रास देतात. उदाहरणार्थ, तुमचा सहकारी तुम्हाला त्रास देतो कारण तुमची अपेक्षा असते की तो वेगळा, अधिक चांगला असावा. पण तो तसाच आहे. त्याने बदलावा अशी इच्छा धरून तुम्हाला काहीही मिळणार नाही, फक्त त्रास होणार. म्हणून त्याला जसा आहे तसा स्वीकारा आणि त्या वर्तमानातच काम करा.
याचा अर्थ असा नाही की प्रत्येकाच्या उद्धट वर्तनाचा भार गप्प बसून घ्यायचा. याचा अर्थ एवढाच की मनात "तो वेगळा असावा" अशी अपेक्षा धरायची नाही. त्याच्या उद्धटपणाला शांतपणे, समतोलपणे हाताळायचं. कदाचित यातून तुम्हाला माणुसकीबद्दल काही नवीन शिकायलाही मिळेल.
हे इथे लिहिणं किंवा सांगणं खूप सोपं आहे, पण वास्तव हे आहे की अपेक्षा सोडणं, किंवा न करणं खूप कठीण आहे. पण त्याची सुरुवात होते जागरूकतेपासून, म्हणजेच "माइंडफुलनेस". आपल्या मनात अपेक्षा आहेत आणि त्या अपेक्षा आपल्याला त्रास देत आहेत, हे मान्य करणं ही पहिली पायरी आहे. ही खरंच कठीण आहे. पण त्याहून कठीण म्हणजे त्या अपेक्षा खरोखर सोडणं. त्यासाठी खोल श्वास घ्यावा लागतो आणि मनाशी म्हणावं लागतं "आत्ताचा हा क्षण असाच असणार आहे, आणि तो परिपूर्ण आहे."
अपेक्षा आणि त्यातून होणारा त्रास सर्वत्र असतो. वाहतुकीत एखादा चालक नियम पाळत नाही तेव्हा आपल्याला राग येतो कारण आपली अपेक्षा असते की सर्वांनी शिस्तीत गाडी चालवावी. पण वास्तव असं आहे की रस्त्यावर नेहमीच काही उद्धट चालक असतात. तशीच, रांगेत उभं असताना जर काउंटरवरील व्यक्ती हळू काम करत असेल तर आपली अपेक्षा असते की त्याने झपाट्याने करावं, आणि आपण चिडतो. आपलं मूल चुकीचं वागलं तर आपल्याला वाटतं त्यांनी नेहमी आदर्श वागणूक दाखवली पाहिजे. मित्र भेटायला आला नाही, तर आपली अपेक्षा तुटते.
या सगळ्या अपेक्षांचा शेवटी आपल्याला काहीही उपयोग होत नाही. फक्त दुःख मिळतं. अपेक्षा सोडल्या की प्रत्येक नातं साधं आणि सोपं होतं.
%%%%%%%%%%%%%%%%%%%%%%%%%%%%%%%%%%%%%%%%%%%%%%%%%%%%%%
 \chapter{सहज काम}
काम म्हणजे नेहमी कंटाळवाणं, त्रासदायक आणि खांद्यावरचं ओझं असंच असतं का? अजिबात नाही! काम खेळासारखंही होऊ शकतं, आणि जेव्हा ते खेळ बनतं, तेव्हा ते सहजसुंदर आणि "एफर्टलेस"म्हणजेच प्रयत्नविरहित वाटतं.
उदाहरणादाखल हे पुस्तक लिहिण्याचं काम घ्या. मी ठरवलं की हे काम गंभीर जबाबदारी म्हणून न पाहता मजा म्हणून करायचं. नवीन कल्पना मांडायच्या, अलीकडे मी ज्या गोष्टींचा सराव करतोय त्यावर विचार मांडायचे आणि हे सर्व खुलेपणाने जगासमोर ठेवायचं. हे करताना मला अपार आनंद मिळाला. त्याचा परिणाम असा झाला की हे पुस्तक मी माझ्या इतर पुस्तकांच्या तुलनेत जास्त वेगाने लिहिलं आणि लिहिण्याची प्रक्रिया पूर्वीपेक्षा अधिक सोपी गेली.
तर मग प्रश्न येतो,  काम खेळासारखं कसं बनवायचं?
१. जे काम आपल्याला आतून आनंद आणि ऊर्जा देतं तेच करा. जर आपण केवळ कंटाळवाणं आणि जबरदस्तीने केलं जाणारं काम केलं, तर त्यातून आनंद मिळणं कठीण असतं. पण जर काम आपल्याला प्रेरणा देणारं असेल, तर त्यात वेळ कसा जातो हे समजतही नाही.
२. जे काम करताय ते कोणाच्यातरी सोबत करा. एकटं काम करणं जड जातं, पण आपल्याला आवडणाऱ्या व्यक्तीबरोबर काम करणं, त्याला गट-प्रकल्प (ग्रुप प्रोजेक्ट) मध्ये बदलणं किंवा "अकाउंटॅबिलिटी पार्टनर" मिळवणं हे कामात रंगत आणतं.
३. आपली प्रगती इतरांशी समाजमाध्यमांवर शेअर करा. हे प्रत्येक प्रकल्पासाठी शक्य नसेल, पण जेव्हा आपण आपला प्रवास "ऑनलाईन" जगासमोर मांडतो, तेव्हा फीडबॅक मिळाल्यामुळे ऊर्जा टिकते.
४. काम लहान लहान सत्रांत विभागून करा. उदाहरणार्थ, या पुस्तकातील प्रत्येक प्रकरण मी मुद्दाम लहान ठेवलं आहे. त्यामुळे मी ते एका बसण्यात लिहून पूर्ण करू शकतो. कधी कधी तर एका बसण्यात दोन-तीन प्रकरणं पूर्ण होतात. त्यामुळे प्रकरण लिहिणं कधीच जड जात नाही आणि प्रचंड डोंगरासारखं काम डोक्यावर असल्यासारखं वाटत नाही.
५. कामाला स्पर्धेचं स्वरूप द्या. दोन किंवा अधिक लोकांमध्ये थोडीशी चुरस (चॅलेंज) निर्माण केली की कोणतंही काम मजेदार बनतं. उदाहरणार्थ, मला बास्केटबॉल खेळायला खूप आवडतं. मी तासन्‌तास खेळतो, पण त्यावेळी मला अजिबात वाटत नाही की मी व्यायाम करतोय, ते फक्त खेळणं असतं.
६. कंटाळा आला की काम थोडं बाजूला ठेवा. खेळ खेळताना आपण स्वतःला कधीच जबरदस्ती करत नाही. कंटाळा आला की सहज बाजूला होतो. कामाबाबतही हीच पद्धत असावी. जबरदस्ती करून करायचं नाही. कंटाळा आला की सोडा, आणि पुन्हा उत्साह आला की सुरू करा.
नक्कीच हे सगळं तेव्हाच शक्य होतं जेव्हा आपल्या कामावर काही प्रमाणात नियंत्रण असतं. कधी कधी तसं नसतं. पण तरीही आपण कामातील आनंददायी भागावर लक्ष केंद्रित करू शकतो आणि नीरस भागांना छोट्या छोट्या खेळात बदलू शकतो. उदाहरणार्थ, "पुढच्या दहा मिनिटांत मी किती शब्द लिहू शकतो?" किंवा "मी किती ग्राहकांना हसतमुख निरोप देऊ शकतो?"
जर तुमचं काम तुम्हाला अजिबात आवडत नसेल, म्हणजे त्यात रोज "सहजसुंदरता" सापडत नसेल, तर लक्षात ठेवा: आपण नोकरीत अडकून बसलेलो नसतो. कधी कधी तसं वाटतं, विशेषतः जेव्हा कुटुंब आपल्यावर (विशेषतः आपल्या पगारावर) अवलंबून असतं. मलाही अनेकदा तसंच वाटलं. पण मी हळूहळू बदल केला, नवीन संधी शोधल्या आणि माझी खरी आवड कुठे आहे ते समजून घेतलं.
तुम्हाला जे करायला आवडतं, जे तुम्हाला खेळासारखं (काम करतोय असा वाटत नाही) वाटतं, तेच काम करायचा प्रयत्न करा. पण त्यासाठी त्यात कुशल व्हावं लागतं. कारण एकदा तुम्ही त्यात प्रवीण झालात की लोक त्यासाठी तुम्हाला मोबदला देतील. सुरुवातीला ते बाजूला, म्हणजेच "साइड हसल" (नोकरी करत करत वेळ काढून) म्हणून करा. खेळासारखं करत राहा आणि हळूहळू त्यात तज्ज्ञ व्हा. कारण सरावाशिवाय प्रावीण्य येत नाही.
एकदा प्रावीण्य आलं की त्यातून उपजीविकेचा मार्ग शोधा. तुम्हाला जे करायला आवडतं त्यातून इतरांना मदत करण्याचे मार्ग शोधा. थोडा सर्जनशील विचार करावा लागेल, पण आजकाल "इंटरनेट सर्च" केल्यावर असे अनेक लोक दिसतात जे हाच मार्ग निवडून पैसे कमावत आहेत.
शेवटी सांगायचं तर, कसलंही काम असो, त्याला खेळात बदलण्याचा मार्ग नक्की सापडतो. सगळं आपल्या मनःस्थितीवर अवलंबून असतं. आणि एकदा ते काम-खेळ झालं की ते सहजसुंदर आणि "एफर्टलेस" होतं.
%%%%%%%%%%%%%%%%%%%%%%%%%%%%%%%%%%%%%%%%%%%%%%%%%%%%%%
 \chapter{तक्रारींचं कृतज्ञतेत रूपांतर}
आपलं जीवन जर सततच्या तक्रारींनी भरलेलं असेल, तर ते जीवन अजिबात सहजसुंदर (एफर्टलेस) राहात नाही. सततच्या तक्रारींमुळे जीवनातील प्रत्येक क्षण संघर्षमय वाटतो. प्रत्येक घटना, प्रत्येक माणूस आणि प्रत्येक प्रसंग यात फक्त कठीणता, कुरूपता आणि अन्याय दिसू लागतो. पण गंमत अशी की, फक्त विचारातला, मनःस्थितीतील एक छोटासा बदल संपूर्ण चित्र पालटू शकतो.
आपली आजची सर्वात मोठी तक्रार एकदा मनात आणा. आता थोडं थांबा आणि स्वतःला प्रामाणिकपणे विचारा या तक्रारीत कृतज्ञतेचा एखादा धागा सापडू शकेल का? आश्चर्याची गोष्ट म्हणजे, उत्तर बहुधा होकारार्थीच असेल. काही ठोस उदाहरणे पाहूया:
“बास्केटबॉल खेळताना माझा कोपरा दुखावला.” पहिल्या क्षणी हे ऐकल्यावर त्रास आणि वेदना जाणवतात. पण जर वेगळ्या नजरेतून पाहिलं, तर लक्षात येतं की  “अरे वा! मी इतकं सक्रिय जीवन जगतोय की मला खेळताना दुखापत होण्याइतकी संधी मिळाली आहे.” हीच खरी कृतज्ञतेची दृष्टि आहे.
“माझा बॉस दिवसभर खूप चिडचिड करत होता.” अशावेळी साधारणतः मनात येतं, “काय नशीब खराब आहे!” पण जर विचारांची दिशा बदलली तर असंही म्हणता येईल,  “या प्रसंगामुळे मला संयमी राहण्याची, वर्तमानात जगण्याची, मानवी स्वभाव समजून घेण्याची आणि जीवनाला अधिक सखोलतेने पाहण्याची अप्रतिम संधी मिळाली आहे.”
“आज माझी नोकरी गेली.” ही बातमी ऐकताक्षणी मन हादरतं, हृदय धडधडू लागतं आणि भविष्य अंधारमय वाटतं. पण जर आपण थोडं शांत होऊन खोल श्वास घेतला, तर हाच प्रसंग एक नवीन दार उघडू पाहतंय हे लक्षात येईल. “अरे वा! आता मला त्या सर्व नोकरीच्या नवीन वाटा शोधता येतील, ज्या मी आतापर्यंत भीतीमुळे किंवा वेळेअभावी शोधूच शकलो नव्हतो.”
तक्रारींचं कृतज्ञतेत रूपांतर ही प्रक्रिया खरंच अद्भुत आहे. तक्रारीतून कृतज्ञतेकडे वळताना मन जगाकडे द्वेष, राग किंवा अन्यायाच्या भावनेने पाहणं थांबवते आणि प्रेमळ नजरेने स्वीकारायला शिकते. “माझ्याकडे नाही” या विचारापासून मुक्त होऊन “माझ्याकडे जे आहे त्याचं कौतुक” या भावनेत जीवन वाहू लागतं.
तक्रार कृतज्ञतेत बदलण्याची प्रक्रिया तशी साधीच आहे. आणि ती टप्प्याटप्प्याने साधता येते. कसं ते पाहूया:
१. सर्वप्रथम हे ओळखा की तुम्ही तक्रार करत आहात. आपल्या मनात सतत चालणाऱ्या नकारात्मक संवादाकडे बारकाईने लक्ष द्या आणि ते ओळखायला शिका.
२. यानंतर हेही जाणून घ्या की तुमच्या मनाला गोष्टी आत्ता आहेत त्यापेक्षा वेगळ्या असाव्यात असं वाटतंय. पण विचार केल्यास लक्षात येईल की हे व्यर्थ आहे, कारण वस्तुस्थिती आपल्याला हवी तशी बदलत नाही. पण मन मात्र सतत एका वेगळ्या जगाची अपेक्षा करत राहतं.
३. तिसऱ्या टप्प्यात वस्तुस्थिती स्वीकारा. जग जसं आहे तसं आहे, त्याला स्वीकारा. प्रत्येक गोष्ट आपल्या आवडीप्रमाणे चालेलच असं नाही. पण हे विसरू नका की आपणास आवडो वा न आवडो, जीवनाचा प्रवाह थांबत नाही, तो पुढे जात राहतो.
४. आता शेवटचा टप्पा म्हणजे - कृतज्ञ व्हा. कारण तुमच्याकडे तक्रार करण्याचीही संधी आहे म्हणजे तुम्ही अजूनही जिवंत आहात. जीवन जगणं हीच एक विलक्षण आणि चमत्कारिक देणगी आहे. प्रत्येक घटनेत एक रुपेरी किनार (सिल्व्हर लाईनिंग) असतेच, जर आपण ती शोधायला आणि पाहायला शिकलो तर?
ही प्रक्रिया एकदाच करून पाहिली, तरी ती एखाद्या ताजेतवाने करणाऱ्या श्वासासारखी वाटते. पण जर हा सराव आपण दररोज, किंबहुना दिवसभराच्या व्यवहारात केला, तर जीवनाचा संपूर्ण दृष्टिकोन बदलतो. हळूहळू तक्रारी नाहीशा होतात आणि आयुष्य आधीपेक्षा खूपच आनंदी, हलकं आणि समाधानकारक वाटायला लागतं.
%%%%%%%%%%%%%%%%%%%%%%%%%%%%%%%%%%%%%%%%%%%%%%%%%%%%%%
 \chapter{संघर्ष सोडून देणं}
आपण सतत संघर्ष करत असतो म्हणून आपलं आयुष्य अनेकदा सहजसुंदर (सहज जीवन) वाटत नाही. पण खरं पाहिलं तर संघर्ष हा आपण आपल्या मनात, विचारात तयार करतो.
आपण हा संघर्ष “निर्माण” करण्याची तशी पहिली तर बरीच कारणं आहेत. आपल्या जीवनाला अर्थ (कृत्रिम का असेना) देण्यासाठी, आपण काहीतरी साध्य केलं आहे याची स्वतःला जाणीव करून देण्यासाठी, आपली कथा, गोष्ट नाट्यमय करण्यासाठी (मग ती भलेही फक्त आपल्याच डोक्यात का असेना) किंवा कदाचित फक्त ते आपल्या अंगवळणी पडलंय म्हणून असेल.
“संघर्ष सोडून देणं सोपं नाही” हे खरं आहे. पण जसं एखाद्या घट्ट गाठीतला दोर हळूच सोडवला कि वाटतं, तसं जीवनातल्या संघर्षाचं ओझं कमी झालं की मन हलकं, मुक्त, आणि जास्त सहज (एफर्टलेस) होतं.
उदाहरण घ्यायचं झालं तर, समजा तुमची लहान मुलगी भाज्या खायला तयार नाहीये आणि तुम्ही तिला जबरदस्तीने भाज्या भरवायला जाताय. हा संघर्ष काही साध्य करतोय का? अजिबात नाही. मुलगी भाज्या अजूनच नावडत्या समजायला लागते. पण जर तुम्ही स्वतः भाज्या आनंदाने खाल्ल्या, मुलीला भाज्या रंगीत, मजेशीर स्वरूपात दिल्या, तर हा खेळ मुलीला हळूहळू आवडायला लागतो. इथे जबरदस्ती सोडून देणं, आणि “खेळकर पद्धतीने प्रेरणा देणं”  ही खरी कला ठरते. तिच्यावर भाज्या खाण्याची बळजबरी करण्याची इच्छा सोडणं म्हणजे अनावश्यक संघर्ष टाळणे.
याच तत्त्वाचा उपयोग प्रत्येक नात्यात करता येतो. आपल्या संघर्षाचं मोठं कारण म्हणजे आपल्या दुसऱ्याकडूनच्या अपेक्षा (एक्स्पेक्टेशन्स). या अपेक्षा पण “काल्पनिक आदर्शावर” अवलंबून आहेत. त्या अपेक्षा दुसऱ्यांवर लादून उपयोग नाही. त्याऐवजी काय करावं? प्रेरणा द्या, गोष्टी मजेशीर बनवा, आणि सर्वांत महत्त्वाचं म्हणजे काय महत्वाचं आहे यावरून स्वतःचं हटू देऊ नका - आणि ते म्हणजे “नातं”. संघर्ष हा नात्यांपेक्षा मोठा असू शकत नाही.
संघर्ष हा नेमका कधी निर्माण होतो? तर जेव्हा आपण एखादी गोष्ट “फक्त या एका मार्गानेच घडली पाहिजे” असं ठरवतो. आणि जेव्हा वास्तव त्यापासून दूर जातं, आपण ते वास्तव जबरदस्तीने वाकवायचा प्रयत्न करतो. यालाच संघर्ष म्हणतात. पण जसं पाणी अडथळ्यांभोवती वाहून स्वतःसाठी मार्ग शोधून काढतं,तसं पर्यायी मार्ग शोधला तर संघर्ष मिटतो. लवचिक बना, परिस्थितीशी जुळवा, आणि बदलांना स्वीकारा. हे जीवन जगण्याचं खरं तंत्र आहे.
%%%%%%%%%%%%%%%%%%%%%%%%%%%%%%%%%%%%%%%%%%%%%%%%%%%%%%
 \chapter{इतरांशी निभावून नेणं}
जेव्हा आपण आपल्या आयुष्यात काही बदल करायला जातो तेव्हा एक यक्ष प्रश्न समोर उभा ठाकतो, “मी तर आयुष्य सोपं करायला जातोय, पण माझ्या आयुष्यात असलेल्या लोकांना ते नको असेल तर? त्यांना माझं आयुष्य सोपं करणं पटलं नाही किंवा रुचलं नाही तर?”
हा प्रश्न इतका सामान्य आहे, पण त्याचा उत्तर मात्र इतकं सरळसोपं, साधं नाहीये. 
पण अश्या परिस्थितीमध्ये, जर तुमच्या सोपं करण्याच्या प्रक्रियेत तुमचे सहचारी, परिवार, सहकारी, मित्र किंवा अजून इतर कोणीही आडवे येत असतील तर ते अडथळे दूर करण्यासाठी तुम्ही काही मार्ग अवलंबू शकता. 
मी स्वतःला भाग्यवान समजतो की माझी पत्नी ईव्हा या प्रवासात माझ्या बरोबरीने चालते आहे. तिने स्वतःच्या बऱ्याचश्या वस्तू कमी केल्या, जीवन बरंचसं साधं केलं, आणि जरी ती माझ्यासारखी टोकाची किमान जीवनशैलीवाली (मिनिमलिस्ट) नसली, तरी खूप मोठा बदल तिने साधला आहे. आणि मला तिचा प्रचंड अभिमान वाटतो.
हे मात्र योगायोगाने झालं नाही. सुरुवातीपासून मी तिला या प्रवासात सामील केलं, तिच्यावर जबरदस्ती केली नाही, तिचं म्हणणं ऐकलं, तिच्या गतीने तिला बदलू दिलं. आणि त्याहून महत्त्वाचं म्हणजे, तिला पण मी जे करेन त्यात मला यश आणि समाधान मिळायला हवंच होतं. 
मुलांना देखील थोडंफार या प्रवासात सहभागी करून घेता आलं, पण त्याच वेळी त्यांच्या वेगळ्या जीवनपद्धतीला मान्यताही द्यावी लागली. सगळेच कुटुंबीय वा ओळखीचे सहकारी मदत करणारे नसतात, कधी कधी काही लोकं तर उघड विरोध करतात. अशा वेळेस वेगळा दृष्टिकोन घेणं हाच एक उपाय ठरतो.
\section*{साधं तंत्र}
हे सगळं हाताळताना काही छोटे नियम माझ्या उपयोगी पडले, ते तुम्हाला सांगतो. कदाचित तुमचे अनुभव वेगळे असू शकतात किंवा तुमची पद्धत थोडी वेगळी असू शकते:
\begin{enumerate}
 \item स्वतः आदर्श व्हा: इतरांना काही सुचवायचं, सांगायचं असेल किंवा तुम्ही म्हणताय ते करवून घ्यायचं असेल तर, तुम्ही स्वतः ते कृतीतून दाखवणे आवश्यक आहे. तुम्ही साधेपणाने जगत आहात हे मुलांना, जोडीदाराला, मित्रांना वा सहकाऱ्यांना स्पष्ट दिसलं, तर तेही हळूहळू त्यात रस घेऊ लागतात.
 \item  महत्व आणि फायदे स्पष्ट सांगा: जेव्हा तुम्ही साधी राहणी जोपासताय, तेव्हा ती दुसऱ्यांना असं करणं तुमच्यासाठी का आवश्यक होतं, त्यामुळे तुम्हाला काय फायदा झाला हे दाखवून द्या (नुसतं सांगून नाही तर प्रात्यक्षिक दाखवून). त्यांच्याशी बोलून यामागची तुमची प्रेरणा काय आहे हे सांगितलं कि त्यांना त्याचं महत्व कळेल. आणि जेव्हा त्यांना हे दिसेल कि तुम्ही जे करताय ते करून तुम्ही किती आनंदी आणि समाधानी आहात, तेव्हा त्यांना त्या गोष्टीची भीती वाटणे कमी होईल आणि तुमच्यासोबत मार्गक्रमण करायला ते उद्युक्त होतील.
 \item मदत मागा: “मी एकटा हे करू शकत नाही, मला तुझी मदत हवी आहे” असं प्रामाणिकपणे सांगणं खूप महत्त्वाचं असतं. अगदी हेच मी माझ्या पत्नी इवा सोबत केलं. मी तिला तिचं समर्थन हवंय, तिची मदत हवीये हे सांगितलं. आणि फक्त मानसिक मदत नाही तर शारीरिक मदत सुद्धा. जे आपल्यावर खरंच प्रेम करतात, ते शक्य तेवढी मदत करतातच. त्यांना तुम्हाला आनंदी पाहायचंय, आणि जेव्हा तुम्ही त्यांची मदत तुम्हाला आनंदी होण्यासाठी कशी मदत करतीये हे सांगाल तेव्हा ते कुठलाही आडपडदा ना ठेवता संपूर्ण मदत करतील.
 \item  दुसऱ्यांना शिकवा: दुसऱ्यांना शिकवण्याची सगळ्यात चांगली पद्धत म्हणजे तुम्ही स्वतः तसे वागून, बोलून, करून दाखवा. पण त्याव्यतिरिक्तही तुम्ही वाचलेलं पुस्तक, एखादं छान लेखन, किंवा \textbf{ब्लॉग} (Blog,  ब्लॉग) वाचायला किंवा त्याबद्दल चा माहितीपट पाहायला सुचवू शकता. पण त्यांनी हे वाचावेच किंवा पाहावंच असं त्यांच्यावर ना लादता फक्त साधनं उपलब्ध करून द्या, पुढे त्यांची उत्सुकताच घडवून आणेल. पण जर त्यांना त्यात काही आवड जाणवली नाही तर मात्र त्याबद्दलचे बोलणे थांबवा.
 \item  यशस्वी होण्यात मदत करा: वर उहापोह केल्याप्रमाणे जर कोण्या जवळच्या व्यक्तीला तुम्ही उद्युक्त करू शकला, किंचित का असेना, तर ते जे काही करत आहेत त्यावर टीका करू नका. “मी जितकं चांगलं आणि व्यवस्थित करतोय तितकंच तुम्ही पण केला पाहिजे” असं म्हणू नका. उलटपक्षी ते जे काही करत आहेत त्यात त्यांना प्रोत्साहन द्या. त्या प्रयत्नांना सांघिक रूप द्या म्हणजे त्यांना वेगळं नाही वाटणार. 
 \item  इतरांवर नियंत्रण नको: या सगळ्या प्रक्रियेत एक खूप वारंवार वैफल्य आणणारा क्षण तेव्हा येतो जेव्हा लोकं बाकीचे काय करत आहेत आणि कसं करत आहेत हे नियंत्रित करायचा, किंवा त्यांना बदलायचा प्रयत्न करतात. जोडीदार असो वा मुलं, आपण त्यांना संपूर्णपणे नियंत्रित करू शकत नाही. त्याऐवजी प्रोत्साहन, प्रेरणा आणि आधार देणं हेच खरं काम आहे. नियंत्रण ठेवायची जी गरज उत्पन्न होतीये मनात ती सोडून द्या. अवघड आहे पण आवश्यक आहे. आणि हे नियंत्रण ठेवायची भावना, गरज एकदा सुटली ना कि मग तुम्ही पण या प्रक्रियेचा आनंद लुटू शकाल. 
 \item  मर्यादा आखा: एकदा नियंत्रणाची प्रयत्न कमी झाला कि मग घरातल्या इतर लोकांसोबत, ज्यांचं ध्येय वेगळं आहे, आयुष्याचे मार्ग वेगळे आहेत; त्यांच्यासोबत सौहार्दाने कसं राहायचं याचे मार्ग तुम्हाला शोधायला हवेत. जर तुम्हाला साधेपणाने जगायचं असेल, पण घरातले इतरांना तसं नको असेल, तर सामंजस्याने मधला मार्ग शोधायला हवा. उदा., घरातली तुमची वस्तू तुम्ही कमी करा, किंवा जागा वाटून घ्या. आणि त्यांची जागा, वस्तू या मध्ये दखलंदाजी करू नका.
 \item   धीर धरा: तुम्ही बदललात म्हणून इतर लगेच बदलतील असं अपेक्षित ठेवू नका. त्यांना वेळ लागेल, किंवा ते कदाचित बदलणारच नाहीत, किंवा तुम्हाला जे हवा आहे त्याला समर्थन देणार नाहीत किंवा त्यात मदतही करणार नाहीत. पण हेही लक्षात ठेवा,  धीर धरलात तर शक्यता कायम उरते.
 \item जिथे शक्य आहे तिथे बदल घडवा: सगळं बदलणं तुमच्या हातात नाही हे सगळ्यात आधी तुम्हाला स्वतःला सांगायला लागेल. जिथे तुमची सत्ता आहे, जिथे इतरांना तुम्ही करत असलेल्या बदलाबद्दल सकारात्मक भाव आहेत, तिथेच लक्ष केंद्रित करा. बाकीच्या गोष्टी हळूहळू येतीलच. इतरांसोबत आयुष्य व्यतीत करण्याची हीच तर खासियत आहे, नियंत्रण सोडायला लागतं पण त्यांच्या सहवासाचं गोड फळ पण मिळतं, जे मी कधीही सोडलं नाही. 
 \item समर्थन शोधा: तुम्हाला घरच्यांकडून मदत मिळाली नाही तर मित्र, समाज, किंवा \textbf{ऑनलाइन कम्युनिटी} (Online community,  ऑनलाईन कम्युनिटी) यांचा आधार घ्या. तुमच्यासारखेच साधेपण शोधणारे अनेक लोक आहेत. त्यांना तुमची आव्हानं, प्रगती, विफलता हे सगळं सांगा आणि त्यांची मदत घ्या.
\end{enumerate}
\section*{सरावाची संधी}
अनेकदा आयुष्यात काही गोष्टी आपल्या नियंत्रणाबाहेर असतात. उदा., किशोरवयीन मुलांना पालकांचे नियम पाळावेच लागतात, किंवा नोकरदार माणसाला ऑफिसचं वातावरण बदलता येत नाही. अशा वेळी निराश न होता हे क्षण शिकण्याची संधी मानावीत.
प्रत्येक अवघड प्रसंग म्हणजे एक संधी. कोणी तर तुम्हाला नियंत्रित करायचा प्रयत्न करत असेल किंवा सहकार्य करत नसेल, किंवा अडचणी निर्माण करत असेल, तर एक सोपी पण प्रभावी पद्धत म्हणजे:
\begin{enumerate}
 \item धीर धरण्याच्या सरावाची संधी.
 \item इतरांची करुणेनं बाजू समजून घेण्याची संधी.
 \item दुसऱ्याने काय करावे ते आपण ठरवण्याच्या अपेक्षा सोडून देण्याची संधी.
 \item “हे असं नसावं” या विचाराला थांबवण्याची संधी.
 \item त्रासातही कृतज्ञ राहण्याची संधी.
\end{enumerate}
जर आपण प्रत्येक अवघड परिस्थितीकडे “शिक्षक” म्हणून पाहिलं, तर त्यांच्याकडून शिकण्यासारखं खूप काही सापडतं. तेव्हा हे लोक तुमच्या जीवनात अडचण बनून न येता उलट आशीर्वादच ठरतात.
%%%%%%%%%%%%%%%%%%%%%%%%%%%%%%%%%%%%%%%%%%%%%%%%%%%%%%
 \chapter{तुम्ही आधीच परिपूर्ण आहात}
बऱ्याचदा काहीजण स्वतःमधे काय बदल करावेत याची दिशा मिळावी किंवा ते ज्यातून जात आहेत त्यातून अजून कोणी जातंय का? आणि त्यातून बाहेर पाडण्यासाठी ते काय करत आहेत हे जाणून घेण्यासाठी व्यक्तिमत्व विकासाविषयी वाचन करीत असतात. याचं कारण बऱ्याच अंशी ते स्वतःच्या आयुष्यावर किंवा शरीरावर संतुष्ट नसतात, आणि त्यात बदल करण्याचा विचार सतत मनात घर करून असतो.
मी हे असे सांगतोय कारण मी स्वतः त्यातला एक होतो.
स्वतःला घडवायचं, जीवन घडवायचं, नवनवीन सुधारणा करायच्या या धडपडीनेच तर झेन हॅबिट्स (Zen Habits) ही संकल्पना जन्माला आली. ती लोकं ज्या वाटेवर आहेत त्या वाटेवरून मी चाललो आहे, आणि त्या अनुभवावरून सांगतो की,  त्या सततच्या प्रयत्नांनी माणूस थकतो, स्वतःबद्दल सतत नाराज राहतो, आणि जीवनात कधीच समाधानी वाटत नाही.
मला आयुष्याचं जे सगळ्यात मोठं गमक कळलेलं आहे ते अगदी साधं सोपं आहे.  तू आहेस तसाच, जात्याच पुरेसा आहेस, तुझ्याकडे आधीच पुरेसं आहे. खरं तर तू आधीपासूनच परिपूर्ण आहेस.
कितीही कृत्रिम किंवा वरवरचं वाटलं तरी हे वाक्य एकदा मनातल्या मनात म्हणून बघा - "मी आधीच परिपूर्ण आहे." हे म्हणताना जर काही सलत, बोचत नसेल आणि अगदी मनस्वी पटत आहे (तर इथून पुढचा भाग वाचायचं थांबा आणि लिहायला सुरु करा) का कुठेतरी आंतरिक आवाज असं सांगतोय की स्वतःमधे अजून बरेच बदल घडवायचे आहेत?
पण मी एक गोष्ट शिकलोय, आणि ती काहीतरी नवीनच शोधलेला तोटका वगैरे नाहीये तर जुनाच मंत्र आहे; ती अशी की तुम्ही कोण आहात आणि आयुष्यात कुठंवर पोचला आहात या बाबतीत जर तुम्ही सुखी समाधानी असाल, तर आयुष्याचं गणितच बदलून जातं.
त्यामुळे काय बदलतं ते बघा:
\begin{itemize}
 \item स्वतःबद्दलची कुरकुर थांबते. जीवनाबद्दलचा असमाधानाचा भाव नाहीसा होतो.
 \item सतत "काहीतरी बदललं पाहिजे" म्हणून धावपळ करत बसायची गरज उरत नाही.
 \item इतरांशी तुलना करणं थांबतं. "तो बघ किती चांगला, मी नाही" ही छळणारी स्पर्धा उरत नाही.
 \item तुमच्याभोवती काहीही घडलं तरी आनंदी राहता येतं. कारण आनंद हा तुमच्या मनात असतो, बाहेरच्या गोष्टींवर अवलंबून रहात नाही.
 \item स्वतःला बदलण्यात वेळ वाया घालवण्याऐवजी इतरांना मदत करण्याकडे लक्ष देता येतं.
 \item "जीवन सुधारण्यासाठी" अमुक गोष्ट लागेल, तमुक गोष्ट लागेल म्हणून करतो ती फालतू खरेदी थांबेल.
 \item आणि हो, जरा माझ्यासारखे आत्मसंतुष्टही होता येईल! (ठीक आहे. हे थोडं गंमतीत घ्या, पण बाकी सगळं खरं आहे.)
 \end{itemize}
आणि अजून एक महत्वाची जाणीव,  "तुज आहे तुजपाशी - अगदी आत्ता या घडीलासुद्धा."
तुझ्याकडे डोळस दृष्टी आहे का? मग तुला आकाशाचा निळाई, हिरवाई, माणसांचे चेहरे, पाण्याचे सौंदर्य या सगळ्यांचं रसग्रहण करता येतं. तुझ्याकडे कान आहेत का? मग तुला पावसाची टिपटिप, मित्रांचं हसणे, संगीतातील नाद अनुभवता येतो, ग्रहण करता येतो. तुला स्पर्शाची जाणीव आहे का? मग तुला वाऱ्याची थंडगार झुळूक, अनवाणी पावलांना जाणवणारे गवतावरचे दवबिंदू, डेनिमचा खरखरीतपणा या सगळ्यांचा अनुभव घेता येतो. तुला वास घेता येतो? मग तुला कॉफीचा दरवळ, फुलांचा सुगंध, कापलेल्या गवताचा गंध अनुभवता येतो. तुला चव कळते? मग तू आंबट जांभुळ, झणझणीत मिरची, गोड चॉकलेटचा आनंद घेऊ शकतोस.
हा सगळा खरं तर चमत्कार आहे, पण आपण त्याला गृहीत धरतो. आणि काय करतो? तर सतत हव्यास करतो.   आपल्याला महागडे कपडे, आधुनिक साधने (कूल गॅजेट्स), अजून मोठं घर, आलिशान कार असं सगळं हवं असतं! या भौतिक सुखाच्या मागे आपण ठार वेडे झालेलो असतो.
शहाणपण यातच आहे की यापैकी काहीच खरं तर लागत नाही. स्वतःला सुधारायची काहीही गरज नाही, कारण आपण आधीपासूनच परिपूर्ण आहोत.
ही गोष्ट एकदा मान्य केली की एक वेगळंच स्वातंत्र्य अनुभवायला मिळतं.
मग तुम्ही जे काही कराल, ते "मी अजून चांगला दिसायला पाहिजे म्हणून" करणार नाही. तुम्ही ते कराल कारण तुम्हाला ते आवडतं, त्यात आनंद मिळतो. आणि जे हवं ते करता येणं हा चमत्कारच नाही का?
खरं तर परिपूर्णता ही इतर कुणी सांगितलेली संकल्पना नाही. "परफेक्ट" म्हणजे तू जसा आहेस तसाच. कुणी ठरवलेल्या चौकटीत बसायची गरज नाही.
आज तू परिपूर्ण आहेस. उद्या तू बदलशील, आणि तरीही परिपूर्णच असशील.
म्हणून हे वाचन इथे थांबवा… आणि आनंदी व्हा.

%%%%%%%%%%%%%%%%%%%%%%%%%%%%%%%%%%%%%%%%%%%%%%%%%%%%%%
 \chapter{हे पुस्तक कृतीत आणताना}
गुंतागुंतीचे आणि संघर्षमयी जीवन जगणाऱ्या आणि अवघड परिस्थितींना सामोरे जाणाऱ्या एखाद्या व्यक्तीला हे पुस्तक थोडंसं समजायला आणि पचायला जड वाटू शकतं. पण असं असलं तरीही हे पुस्तक साधं-सोपं असावं, पटकन कळावं आणि रुचावं म्हणूनच लिहिलेलं आहे.
कदाचित कुठून सुरु करावं हे तुम्हाला कळत नसेल, किंवा आपल्या स्वतः मध्ये मूलभूत बदल कसे करावेत हे समजत नसेल. किंवा बदलाची भीती सुद्धा वाटत असेल. पण लक्षात ठेवा,  हा प्रवासही ताणतणावाचा किंवा खूप मोठ्या धडपडीचा नाही. खरं तर एफर्टलेसनेस (Effortlessness),  म्हणजेच सहजतेने जगण्याची कला,  शिकण्यासाठी फार मोठे कष्ट लागत नाहीत.
सुरुवात साध्या गोष्टीने करा. जरा हळूहळू, सहजपणे. दिवसभरातल्या छोट्या-छोट्या क्षणांमध्ये, थोड्या वेळासाठी, एखादी छोटी सवय जाणीवपूर्वक अंगीकारा.
प्रवासाची खरी सुरुवात नेहमी एकाच पावलाने होते. एक पाऊल टाकलं, मग दुसरं,  असं करत करतच संपूर्ण वाटचाल घडते.
या पुस्तकातील शिकवणी प्रत्यक्ष जीवनात आणायची असेल, तर सराव हाच उपाय आहे.
 \begin{itemize}
 \item इतरांकडून अपेक्षा न ठेवण्याचा सराव करा.
 \item "हे असं का झालं, वेगळं असायला हवं होतं. माझ्याच बाबतीत असं का घडतं" अशा मनातील हळहळीची जाणीव ठेवा आणि तिला अलगद सोडा.
 \item कुरकुर करण्याऐवजी कृतज्ञतेचा सराव करा.
 \item संघर्ष वाढतोय असं वाटलं की थोडं मागे सरा, शांत व्हा, आणि त्यातूनही काही मार्ग निघाला नाही तर त्या संघर्षालाच सोडून द्या.
 \item जीवन ठरवून ठेवलेल्या आराखड्याशिवाय, ठराविक परिणामांची अपेक्षा न ठेवता जगा. दिवसागणिक जे बदल घडणार आहेत त्यांना लवचिकतेने स्वीकारा.
 \end{itemize}
ही प्रत्येक गोष्टीचा स्वतंत्रपणे सराव करा. एकावेळी एकाच गोष्टीला हात घाला. प्रत्येक वेळेला सराव केल्यावर तुम्ही त्यात थोडे अधिक तरबेज व्हाल. याला इंग्रजीमध्ये खूप छान शब्द आहे “इन्क्रिमेंटल चेंजेस” (incremental changes) किंवा आर्थिक भाषेत बोलायचं झाला तर हे चक्रवाढ व्याजासारखं आहे. जितके दिवस कराल तितकं त्याच्यावरचं व्याज वाढत जाईल आणि थोड्याच दिवसांत तुम्हाला कळेल कि तुम्ही या कला लीलया आत्मसात केल्या आहेत!
या पुस्तकातली सगळीच तत्त्वं तुमच्या जीवनाला लागू पडतील असं नाही आणि त्यात काही वावगं नाहीये. हे पुस्तक म्हणजे लाईफ मॅन्युअल (Life Manual) नाही. ही काही नियमावली नाही की जसंच्या तसं, तंतोतंत पाळायलाच हवं. हे म्हणजे ढोबळ मार्गदर्शन आहे, ज्यातून तुम्हाला जे उपयोगी वाटेल, रास्त वाटेल, अनुकरणीय वाटेल ते घ्यायचं आहे. मला जी तत्त्वं उपयोगी पडली, ती तुम्हालाही पडतीलच असं नाही. प्रत्येक माणूस वेगळा असतो. म्हणूनच महत्त्वाचं काय तर स्वतः अनुभव घेऊन, स्वतः तपासून, तुम्हाला जे पटेल ते निवडून घ्या. उरलेलं हलक्या हाताने सोडून द्या.
माझ्या स्वतःच्या आयुष्यात असं घडलं आहे की, एखादी कल्पना आधी मला अगदी अशक्य, अव्यवहार्य वाटली. पण काही काळाने परत त्या कल्पनेवर आलो, आणि तीच कल्पना तेव्हा अगदी योग्य वाटली. म्हणजेच, काही गोष्टी आपल्या जीवनात वेगवेगळ्या टप्प्यांवरच लागू पडतात.
म्हणूनच लवचिक राहा. स्वतःला माफ करा, प्रत्येक गोष्टीला तराजूमधे तोलून मापून पाहू नका. रोज सराव करा. स्वतःला चुका करायची परवानगी द्या. भरपूर चुका करा! कारण त्या चुकांमधूनच तुम्ही शिकता. माझ्याही बाबतीत असंच झालं आहे. खरं तर मी अजून बऱ्याच चुका करीन, कारण शिकण्याचा प्रवास हा अखंड चालतच राहतो.
%%%%%%%%%%%%%%%%%%%%%%%%%%%%%%%%%%%%%%%%%%%%%%%%%%%%%%
 \chapter{सहज लेखन आणि हेच पुस्तक}
जी तत्त्वं या पुस्तकात उद्धृत केलेली आहेत त्याच आधारे मी हे पुस्तक लिहिले आहे. जे मी जगलो आणि जसा जगलो तीच तत्वे तुम्हाला या पुस्तकाच्या प्रत्येक पानावर दिसतील. 
कुठलंही विशिष्ट उद्दिष्ट ठेवून मी हे पुस्तक लिहिले नाहीये. तर माझे सहज जीवनाबद्दलचे अनुभव, शिकवणी आणि कल्पना; ज्या मी अनुभवल्या किंवा इतरत्र पहिल्या (कारण कधी कधी स्वतः अनुभव घ्यायच्या ऐवजी “पुढच्यास ठेच मागचा शहाणा” हे सूत्र पण वापरावं) त्या सगळ्यांसोबत वाटून घेण्यासाठी प्रेरित आणि उत्सुक होतो म्हणून हा पुस्तक लिहिण्याचा घाट घातला. 
मी आंतरजालावर (ऑनलाईन हा शब्द किती अंगवळणी पडलाय ना?) गूगल डॉक उघडलं आणि मनात येईल ते तिथे उतरवायला सुरुवात केली. कुठलेही उद्दिष्ट डोळ्यासमोर नाही, कुठलीही खास शैली नाही, फक्त लिहीत जायचं या तत्वावर सगळा खेळ चालू होता. लिहिता लिहिता मेंदूतल्या स्वैर विचारांनी लगेच “हे लिखाण जर सार्वजनिक केलं, आणि इतरांना पण यात सहभागी व्हायला सांगितलं तर?” असं सुचवलं. आणि लगेच ते अमलात आणलं सुद्धा. ज्याला वाटेल त्याने यात काही बदल करावेत किंवा एखादा विचार अजून चांगला फुलवून लिहावा अशी मांडणी केली. बघूया, यातून काही विलक्षण हाती लागतंय का?
ऐकायला जरी भयानक वाटलं तरी हा विचार मुक्त करणारा होता. मी नियंत्रणाची गरज सोडून दिली. घटनांना जसं घडायचं होतं तसं घडू दिलं. कॉपीराईट (Copyright) ही संकल्पनाच सोडली, म्हणजेच या मजकुरावरचा हक्क सोडून दिला. आणि मग माणसातल्या करुणेवर, बुद्धिमत्तेवर विश्वास ठेवला.
माझ्या मुलीनं विचारलं, “हे भीतीदायक नाही का?”
मी शांतपणे हसलो आणि म्हटलं, “वाईटात वाईट काय होऊ शकतं?”
या पद्धतीनं लिहिणं म्हणजे एक वेगळंच साहस होतं. एकट्याने डोके खाऊन, बंद दरवाजामागे लिहिण्याचा एकाकी अनुभव अचानक सार्वजनिक झाला. हे जणू एखाद्या सादरीकरणासारखं, परफॉर्मन्स आर्ट (Performance Art) सारखं वाटू लागलं. फक्त माझ्या लेखणीचा (आणि मनाचा) खेळ न राहता, हा एक सामूहिक अनुभव झाला. लेखकाचा हुकूम गेला, आता ही लोकांच्या उत्कटतेची, सामूहिक उर्जेची निर्मिती होती.
लेखन अगदी सहज घडलं. कारण?
 \begin{itemize}
 \item मला या विषयाबद्दल मनापासून उत्कटता होती.
 \item कुठलेही ठरवलेले आराखडे किंवा अपेक्षा नव्हत्या.
 \item घाई नव्हती; वेळेचं ओझं नव्हतं.
 \item जागरूकपणे, माईंडफुली (Mindfully) मी लिहित होतो.
 \item आणि महत्त्वाचं म्हणजे, इतरांनी मदत केली. संपादनात त्यांनी हातभार लावला, त्यामुळे मी अनावश्यक श्रम वाचवले.
 \end{itemize}
आजवरचा प्रत्येक क्षण या लेखनाचा मला मनापासून प्रिय वाटला. आणि म्हणूनच मी मनापासून म्हणतो,  धन्यवाद मित्रांनो, या प्रवासाचा भाग बनल्याबद्दल.

%%%%%%%%%%%%%%%%%%%%%%%%%%%%%%%%%%%%%%%%%%%%%%%%%%%%%%
 \chapter{योगदानकर्ते}
हे पुस्तक एकट्या माझ्या श्रमाचं फलित नाही. शेकडो लोकांनी आपापल्या पद्धतीनं यात हातभार लावला आहे. ही एक सामूहिक मेहनत आहे, आणि तीही किती सुंदर, किती प्रेरणादायी!
म्हणूनच या पुस्तकाचं श्रेय मी एकट्याने घेऊच शकत नाही. खरंतर हे पुस्तक म्हणजे असंख्य विचारांची, संपादनांची, सूचनांची आणि प्रोत्साहनाची एकत्रित शिदोरी आहे. ज्यांनी ज्यांनी आपलं ज्ञान, वेळ आणि ऊर्जेचा काही अंश जरी दिला असेल, त्यांचे मी मनःपूर्वक आभार मानतो.
या पुस्तकाच्या लेखनात आणि संपादनात प्रत्यक्ष किंवा अप्रत्यक्षपणे मदत करणाऱ्या अनेकांचा सहभाग आहे. त्यातील बऱ्याच जणांची नावे इथे नोंदवणं शक्य झालेलं नाही, पण त्यामुळे त्यांच्या योगदानाचं महत्त्व कमी होत नाही.
\begin{itemize}
 \item काहींनी शब्द न शब्द सुधारून दिला.
 \item काहींनी नव्या कल्पना सुचवल्या.
 \item काहींनी उणिवा दाखवल्या, चुका हळुवारपणे दुरुस्त करून दिल्या.
 \item तर काहींनी फक्त एक वाक्य म्हटलं,  "लिहित जा, छान होतंय!" आणि तेवढं प्रोत्साहनही पुरेसं ठरलं.
 \end{itemize}
मूळ इंग्रजीतील पुस्तकाच्या शेवटी त्या पुस्तकाच्या प्रमुख योगदानकर्त्यांची नावे आहेत. या मराठी अनुवादात भाषांतर करण्यात आणि तपासणीमध्ये योगेश कुलकर्णी यांचा सहभाग होता तसेच त्यांची कुटुंबिय, अंजली अनुष्का आणि अनुजा यांचा लेख तपासणी मध्ये सहभाग होता. प्रसाद कुलकर्णी यांनी यातील अनेक धड्यांचे भाषांतर आणि तपासणी केलेली आहे.
या सर्वांचा मी मनःपूर्वक ऋणी आहे. खरंतर प्रत्येक योगदान, लहान असो वा मोठं, या पुस्तकाला अधिक संपन्न आणि अर्थपूर्ण बनवत गेलं.
म्हणूनच इथे एक वाक्य पुन्हा ठामपणे सांगतो,  हे पुस्तक केवळ माझं नाही, हे सर्वांचं आहे.

