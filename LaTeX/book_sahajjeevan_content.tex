\chapter*{प्रस्तावना}

जीवन कठीण आहे. किंवा आपल्याला असे वाटते.
सत्य हे आहे की जीवन तेवढेच कठीण आहे जेवढे आपण ते बनवतो.
आपल्यापैकी बहुतेक दररोज अनेक कामे आणि धावपळीत व्यस्त राहतो, समस्यांवर तडजोड करतो आणि नाट्यांना सामोरे जातो. या संघर्षांपैकी बहुतेक काल्पनिक असतात.

आपण साधे प्राणी आहोत. अन्न, निवारा, कपडे आणि नातेसंबंध हे आपल्याला आनंदी राहण्यासाठी आवश्यक आहे. अन्न साधे आणि नैसर्गिकपणे वाढते. निवारा म्हणजे साधे छत. कपडे म्हणजे फक्त कापड. साधे नातेसंबंध म्हणजे अपेक्षांशिवाय एकमेकांच्या सहवासाचा आनंद घेणे.

या साध्या गरजांच्या पलीकडे, आपण काल्पनिक गरजा जोडल्या आहेत: करिअर, बॉस आणि सहकारी; नवीन यंत्रे, सॉफ्टवेअर आणि सोशल मीडिया; गाड्या आणि छान कपडे आणि पर्स आणि लॅपटॉप बॅग आणि टेलिव्हिजन आणि बरेच काही.

मी असे म्हणत नाही की आपल्याला गुहावासी काळात परत जावे, पण हे लक्षात ठेवणे महत्त्वाचे आहे की काय आवश्यक आहे आणि काय काल्पनिक आहे.

जेव्हा आपल्याला कळते की काहीतरी काल्पनिक आहे, तेव्हा आपण त्या गरजेला नाकारण्याचे निवडू शकतो; जर ती चांगल्या उद्देशाची पूर्तता करत नसेल, जर ती जीवन अधिक कठीण बनवत असेल, तर ती जाऊ शकते!

जीवन कठीण बनवणाऱ्या गोष्टी काढून टाकल्याने, आपल्याकडे सहज जीवन राहते.

मी एक महत्त्वाचा धडा शिकलो जेव्हा मला चांगला पोहणारा बनायचे होते - मला वाटत होते की अधिक आणि वेगाने पोहणे म्हणजे फक्त अधिक प्रयत्न करणे, अधिक कष्ट घेणे. मी पाण्यात वेडसरपणे धडपडत होतो, पण थकून जात होतो. जेव्हा मी शिकलो की पाणी तुम्हाला वर ढकलू शकते आणि तरंगण्यास मदत करू शकते, तेव्हा त्यातून सरकणे खूप सोपे झाले. मी आराम केला, गोष्टींना जबरदस्तीने करण्याचा प्रयत्न थांबवला आणि कमी प्रयत्नाने चांगले पोहायला शिकलो.

जीवन असेच आहे. जीवन हे पाण्यासारखे आहे, आणि आपण खूप जोराने ढकलतो, धडपडतो, गोष्टींना जबरदस्ती करतो, संघर्ष करतो. त्याऐवजी, तरंगायला शिका, गोष्टींना सहज होऊ द्या. तुम्ही अधिक पुढे जाल आणि जीवन तेवढे अधिक आनंददायक होईल.

\chapter{ सहज जीवन म्हणजे काय?}

अशा जीवनाची कल्पना करा जिथे तुम्ही जागे होता आणि तुम्हाला आवडणारे काम करता. तुम्ही आवडत्या लोकांसोबत वेळ घालवता आणि त्या वेळेचा पूर्ण आनंद घेता. तुम्ही या क्षणात जगता, भविष्याची चिंता न करता, भूतकाळातील चुका विसरून.

कल्पना करा की तुमचे काही जवळचे मित्र आणि कुटुंबीय आहेत, आणि तुम्ही त्यांच्यासोबत भरपूर वेळ घालवता. त्यांच्याकडून तुमच्या कोणत्या अपेक्षा नाहीत, म्हणून, ते तुम्हाला निराश करत नाहीत, आणि खरे तर, ते जे काही करतात ते परिपूर्ण आहे. तुम्ही त्यांच्यावर त्यांच्या व्यक्तिमत्त्वाबद्दल प्रेम करता, आणि तुमचे नातेसंबंध गुंतागुंतीचे राहत नाहीत.

तुम्हाला एकटे वेळ घालवण्यात आनंद वाटतो - तुमच्या विचारांसोबत, निसर्गासोबत, पुस्तकासोबत, आणि कदाचित निर्माण करताना.

हे एक साधे, सहज जीवन आहे. "प्रयत्न नाही" असे नाही, पण ते सहज वाटते, आणि तेच महत्त्वाचे आहे. आणि हे पूर्णपणे शक्य आहे.

सहज जीवनाच्या मार्गात फक्त मन येते.

\chapter{सहज जीवनासाठी मार्गदर्शक तत्त्वे}

ही कठोर नियम नाहीत. आणि ते नकारात्मक स्वरूपात आहेत कारणासाठी: हे मार्गदर्शन तुम्हाला काय करावे सांगत नाही. हे तुम्हाला काय करू नये सांगते, जेणेकरून तुम्ही अनावश्यक प्रयत्न निर्माण करू नका. तुम्ही काय करता हे तुमच्यावर सोडलेले आहे.

मार्गदर्शक तत्त्वे:
\begin{itemize}
\item कोणती हानी करू नका.
\item कोणतेही निश्चित लक्ष्य किंवा योजना ठेवू नका.
\item कोणत्या अपेक्षा ठेवू नका.
\item खोटी गरजा निर्माण करू नका.
\item आवडत नसलेले काम करू नका.
\item घाई करू नका.
\item अनावश्यक कृती निर्माण करू नका.
\end{itemize}

काही संभाव्य सकारात्मक मार्गदर्शक तत्त्वे:
\begin{itemize}
\item दयाळू राहा.
\item उत्साही राहा.
\item समाधान शोधा.
\item हळू जा.
\item धैर्य धरा.
\item उपस्थित राहा.
\item वजाबाकीला प्राधान्य द्या.
\end{itemize}

\chapter{ वू वेई आणि काहीच न करणे}

ताओवादात एक संकल्पना आहे जी पश्चिमी मनाला कठीण वाटते: वू वेई, जी अनेकदा "न-करणे" किंवा "क्रिया नसणे" असे अनुवादित केली जाते. मी त्याचा अर्थ कधी कृती न करावी हे जाणून घेणे आणि कधी कृती करणे योग्य आहे हे जाणून घेणे असा करू इच्छितो.

हे आपल्या "करणे" च्या पश्चिमी परंपरेत कठीण आहे. आपली संस्कृती कृतीला महत्त्व देते, आणि निष्क्रियता चिंता निर्माण करते. तथापि, जगण्याचा हा मार्ग आपल्या जीवनातील अनेक अडचणींचे मूळ आहे - आपण "न करणे" च्या स्थितीत अस्वस्थ असल्यामुळे अनावश्यक प्रयत्न करतो.

काहीच न करणे शक्य आहे का? शब्दशः नाही - आपण कृती करत नसलो तरी, आपण बसतो किंवा पडून असतो किंवा उभे राहतो. पण कृती करणे म्हणजे सहसा एखादी क्रिया करणे, अनेकदा लक्ष्याच्या दिशेने आणि हेतूने. जर आपण लक्ष्य किंवा हेतू काढून टाकला तर? मग ती कृती अनावश्यक आहे, आणि ती घेणे गोष्टींना अनावश्यकपणे कठीण बनवेल.

त्यामुळे, लक्ष्ये काढून टाकणे आणि हेतू साधे करणे अनेक कृतींची गरज नाही करते.

हा विचार स्वीकारणे आपल्यासाठी अत्यंत कठीण आहे. आपल्याला उत्पादक बनायचे आहे. "निष्क्रिय" या शब्दाचे इतके नकारात्मक अर्थ आहेत की आपण काहीच न करण्यापासून दूर राहतो. आपली संस्कृती आळशीपणाला तुच्छ मानते. म्हणून आपण अनावश्यक गोष्टी करतो, आणि आपल्याला असे वाटते म्हणून मनमानी लक्ष्ये तयार करतो.

आपण आपली किंमत आपल्या कामगिरीवरून मोजणे थांबवले तर काय? आपण काय करतो त्यापेक्षा आपण कोण आहोत हे नेहमीच महत्त्वाचे असेल. काहीच न करण्याचा प्रयत्न करा. फक्त पाच मिनिटांसाठी. आपण चिंताग्रस्त होतो आणि नवीन टॅब उघडायची, ईमेल तपासायची, बातम्या वाचायच्या, कोणाशी तरी बोलायची, काही काम करायचे इच्छा होते. आणि हे फक्त पाच मिनिटांसाठी - जर आपण सारा दिवस काहीच केले नाही तर काय?

जर आपण खोटी गरजा, लक्ष्ये, अपेक्षा आणि हेतू काढून टाकले, तर आपण जे काही करतो त्यापैकी बरेच काही करण्याची गरज नाही होते. मग आपल्याकडे एक रिकामेपणा राहू शकते जे फक्त आवश्यक असलेल्या गोष्टींनी, नैसर्गिक गोष्टींनी, सुंदर गोष्टींनी भरले जाऊ शकते.

\chapter{खरी गरजा, साधी गरजा}

तर खरोखर काय आवश्यक आहे? मी वर नमूद केले की आपल्या मूलभूत गरजा कमी आहेत: अन्न, कपडे, निवारा, नातेसंबंध.

या गरजांपैकी कोणतीही गुंतागुंतीची नाही.

तुम्ही असा युक्तिवाद करू शकता की अन्न मिळवणे गुंतागुंतीचे असू शकते, पण मसानोबू फुकुओकाचे "वन स्ट्रॉ रिव्होल्यूशन" वाचा - तो दाखवतो की आपण एका एकरावर कुटुंबासाठी पुरेसे उगवू शकतो, निसर्गात कमीत कमी हस्तक्षेप करून. तण वाढू द्या, कीटकनाशक वापरू नका, जमीन नांगरू नका, प्राणी आणि कीडे आणि सरडे शेतात मोकळे राहू द्या. हे गुंतागुंतीचे नाही.

याचा अर्थ असा नाही की आपण सगळे उद्या जमिनीकडे परततील, पण हे लक्षात ठेवणे महत्त्वाचे आहे की आपल्या खऱ्या गरजा आपण निर्माण केलेल्या समाजामुळे गुंतागुंतीच्या झाल्या आहेत, आणि अन्न आणखी एक स्टेटस सिम्बल बनण्याची गरज नाही. आणि अशाप्रकारे आपल्याकडे वजाबाकीने काहीतरी सोपे निर्माण करण्याची क्षमता आहे.

निवाराही गुंतागुंतीचा बनवला गेला आहे - घर हा अनेक लोकांसाठी सर्वात मोठा खर्च आहे, आणि सुंदर घर आता एक महाग स्टेटस सिम्बल आहे. पण त्याच्या सर्वात मूलभूत स्वरूपात, निवारा म्हणजे एक छत जे आपल्याला हवामानापासून संरक्षण देते. ते एका माणसाचे लीन-टू असू शकते, किंवा अनेक कुटुंबांसाठी मोठे आश्रयस्थान असू शकते. आपल्याला हवे तसे साधे असू शकते.

कपडेही खूप गुंतागुंतीचे बनवले गेले आहेत. ते इतक्या क्लिष्ट पद्धतीने स्टेटस सिम्बल बनवले गेले आहेत की ते खऱ्या गरजेपासून कितीतरी दूर गेले आहेत. आपल्याला फक्त आपले शरीर झाकायचे आहे, आणि गांधीजींनी दाखवल्याप्रमाणे, तुम्हाला फक्त थोडेसे हाताने कातलेले कापड हवे. पुन्हा, आपण कदाचित लवकरच कौपीन घालणार नाही, पण आपल्या कपड्यांपैकी किती खऱ्या गरजेची पूर्तता करतात आणि किती काल्पनिक आहे हे लक्षात ठेवूया.

नातेसंबंध कदाचित आपल्या गरजांपैकी सर्वात गुंतागुंतीचे आहेत, कारण माणसे गुंतागुंतीचे प्राणी आहेत जे सहजपणे साधे केले जाऊ शकत नाहीत. आपल्याला अपनत्वाची भावना हवी आहे. आपल्याला आपल्या समवयस्कांच्या नजरेत चांगले दिसायचे आहे, इतरांना आकर्षक वाटायचे आहे. म्हणून नातेसंबंध परस्परसंवाद आणि भावना आणि अपेक्षांचे इतके गुंतागुंतीचे जाळे बनले आहेत की त्यांना सहजपणे सोडवले जाऊ शकत नाही.

इतके कठीण असण्याची गरज नाही. मी एका मित्राशी भेटतो, उर्वरित जग विरघळू देतो, आणि उपस्थित राहण्यावर लक्ष केंद्रित करतो. आपण बोलतो, आपण विनोद करतो, आणि आपल्यात एकमेकांकडून कोणत्या अपेक्षा नाहीत. आपण दुखावले न जाणे आणि पुन्हा कधी एकत्र येणार याची काळजी न करता निघून जातो.

माझे लग्न आणि माझ्या मुलांसोबतचे नातेसंबंध त्यापेक्षा गुंतागुंतीचे आहेत, पण मी अपेक्षा आणि गरजा कमी करायला शिकत आहे, जेणेकरून जे राहते ते प्रत्येक कुटुंबीयाचा शुद्ध आनंद आहे त्यांच्या व्यक्तिमत्त्वाबद्दल. मी अजून तिथे नाही, पण शिकत आहे. वजाबाकी केल्याने फक्त सार राहतो, नातेसंबंधांमधून आपल्याला जे हवे आहे ते राहते.

समाजातील आपले योगदान अर्थातच तितकेच गुंतागुंतीचे होऊ शकते. यात सहसा आपली नोकरी समाविष्ट असते, आणि ते आपल्या जीवनाचा बहुमत भाग घेते आणि आपल्या ताण आणि निराशेचा बहुमत भाग बनवते. पण याचे एक कारण म्हणजे काल्पनिक गरजांना आधार देण्यासाठी आपल्याला करावे लागणारे लांब तास. जर आपण आपल्या गरजा कमी केल्या आणि कमीत काममे समाधान करायला शिकलो, तर आपल्याला जगण्यासाठी कमी काम करावे लागेल.

यामुळे आपल्याकडे खूप मुक्त वेळ राहते समाजात अगदी साध्या मार्गाने योगदान देण्यासाठी. आपण धर्मादाय संस्थांमध्ये स्वयंसेवा करू शकतो, काहीतरी अद्भुत निर्माण करू शकतो, आपल्या शेजारी इतरांना मदत करू शकतो. आपण चांगले काम करू शकतो आणि त्याला सोडून देऊ शकतो, बक्षीस, पेमेंट किंवा प्रशंसाची अपेक्षा न करता. किंवा आपण फक्त उपलब्ध राहू शकतो जेणेकरून जेव्हा इतरांना आपली गरज असेल, तेव्हा आपण नेहमी आपल्या स्वतःच्या लक्ष्यांकडे धावत नसू.

या आपल्या गरजा आहेत, आणि त्या साध्या आहेत.

\chapter{आपल्या गरजा कमी करा}

मी म्हटल्याप्रमाणे, आपल्या खऱ्या गरजा अगदी साध्या आहेत. आधुनिक समाजात, आपण अधिक गरजा निर्माण केल्या आहेत: तुम्हाला तुमच्या घर आणि कपडे आणि गाडी आणि कॉम्प्युटर आणि इंधन आणि वीज आणि अन्न आणि बाहेर जाणे आणि मनोरंजन आणि शिक्षण आणि बरेच काही यासाठी नोकरीची गरज आहे.

जर तुम्ही तुमच्या गरजा कमी केल्या आणि कमीत काममे समाधान करायला शिकला, तर तुम्हाला अगदी कमी करावे लागेल. तुमच्या कमी गरजांचा परिणाम कमी प्रयत्नात होतो.

जर तुमच्या कमी गरजा आहेत, तर तुमचे कमी खर्च आहेत, आणि मग तुम्हाला कामाची कमी गरज आहे. तुम्ही कमी काम करू शकता आणि अधिक खेळू शकता. तुम्ही आवडत्या कामात गुंतू शकता, कारण त्या उत्कट कामातून तुम्हाला फार कमावावे लागणार नाही, आपल्याला थोडा वेळ अपयशी होण्याचे स्वातंत्र्य मिळते.

जर तुमच्या कमी गरजा आहेत, तर तुमच्यावर यशस्वी होण्याचा कमी दबाव आहे आणि तुम्ही अधिक आराम करू शकता. तुम्ही जास्त काळजी करत नाही, कारण काळजी करण्यासारखे कमी आहे.

तुमच्या गरजा कमी करणे एक हळू आणि जागरूक प्रक्रिया आहे. तुम्हाला रातोरात सर्व काही कापावे लागत नाही. तुमच्या खर्चावर, तुम्ही दर आठवड्यात काय करता यावर लक्ष द्या, आणि स्वतःला विचारा की या गोष्टी आणि क्रियाकलाप खरोखर आवश्यक आहेत का.

हळूहळू कमी करण्यास सुरुवात करा, एक के एक महागडा क्रियाकलाप कापून टाका. तुम्हाला खरोखर दररोज त्या स्टारबक्स कॉफीची गरज आहे का, की तुम्ही स्वतः बनवू शकता, किंवा त्याऐवजी पाणी पिऊ शकता? तुम्हाला खरोखर महाग नाश्त्याची गरज आहे का, की तुम्ही फळे आणि नट्स खाऊ शकता? तुम्हाला खरोखर महाग मनोरंजनात भाग घेण्याची गरज आहे का, की तुम्ही तुमच्या मुलांसोबत खेळू शकता किंवा उद्यानात काही मित्रांसोबत वेळ घालवू शकता? तुम्हाला खरोखर जिम सदस्यत्वाची गरज आहे का, की तुम्ही तुमच्या जोडीदारासोबत फिरू शकता किंवा बाहेर पुश-अप्स करू शकता?

हळूहळू मोठ्या खर्चांकडे बघा: तुम्हाला खरोखर दोन गाड्यांची गरज आहे का? तुम्ही तुमची SUV लहान, कमी महाग, वापरलेल्या गाडीसाठी बदलू शकता का? तुम्ही तुमची गाडी सायकल किंवा सार्वजनिक वाहतुकीसाठी सोडू शकता का? तुम्हाला खरोखर इतक्या मोठ्या घराची गरज आहे का? तुम्ही लहान, कमी महाग, गरम किंवा थंड करण्यासाठी कमी खर्चाच्या जागी जाऊ शकता का? तुम्हाला खरोखर इतक्या महाग शिक्षणाची गरज आहे का, की तुम्ही स्वतःला मोफत शिक्षित करू शकता?

मी असे म्हणत नाही की तुम्हाला यापैकी काही किंवा सर्व गोष्टी सोडाव्या लागतील - मी सूचवितो की तुम्ही लक्ष द्या, आणि हळूहळू कमी करा जेणेकरून तुम्ही ज्यावर खर्च करता त्यापैकी बहुतेक मुख्य गरजांवर खर्च होईल.

तुम्हाला आनंदी बनवणाऱ्या गोष्टींना जास्त खर्च करावा लागत नाही. माझ्या काही आवश्यक गोष्टी:

\begin{itemize}
\item एक चांगले पुस्तक - जे ग्रंथालयात मिळू शकते.
\item लिहिण्यासाठी वही किंवा लॅपटॉप.
\item बाहेर फिरणे.
\item माझ्या पत्नीसोबत चहा.
\item माझ्या मुलांसोबत खेळणे.
\item मित्रासोबत धावणे.
\end{itemize}
मूलभूत गरजांच्या (अन्न, कपडे, निवारा, इ.) पलीकडे आनंदी राहण्यासाठी मला बहुतेक हेच हवे आहे. आणि या गोष्टींपैकी कोणत्याही गोष्टीला फार खर्च येत नाही.
तुमच्या गरजा कमी करा, कमीत काममे समाधान करा, आणि जीवनासाठी आवश्यक प्रयत्न मैलांनी कमी होतो.


\chapter{कोणती हानी करू नका, आणि दयाळू राहा}
हा माझा जगण्याचा मूलभूत नियम आहे, आणि त्याने माझी चांगली सेवा केली आहे. यामुळे माझे जीवन कमी कठीण झाले आहे:
\begin{itemize}
\item नातेसंबंध सोपे आणि अधिक फळदायी आहेत.
\item लोक माझ्याशी दयाळू असतात.
\item दयाळू व्यक्ती म्हणून ओळखले जाणे अधिक दरवाजे उघडते.
\item मी अधिक आनंदी आहे.
\item माझ्या आजूबाजूचे सर्वजण थोडे अधिक आनंदी आहेत.
\end{itemize}
सहज जीवनाचे पहिले मार्गदर्शक तत्त्व म्हणजे कोणती हानी करू नका. हे पहिले आहे कारण ते इतर सर्व गोष्टींवर परिणाम करते. जर "घाई करू नका" या मार्गदर्शक तत्त्वामुळे हानी होत असेल, तर तुम्ही "कोणती हानी करू नका" च्या बाजूने "घाई करू नका" नाकारावे.
जेव्हा तुम्ही हानी करता, तेव्हा ते समस्यांची लहरी निर्माण करते जी तुमच्यासाठी आणि ज्यांचे तुम्ही नुकसान केले आहे त्यांच्यासाठी जीवन अधिक कठीण बनवते. मग तुम्हाला तुमच्या चुका सुधारण्याची आणि क्षमा मागण्याची जबाबदारी येते, जी दोन्ही लांब आणि कंटाळवाणी कामे आहेत ज्या सहजपणे टाळल्या जाऊ शकल्या असत्या.
दैनंदिन जीवनात हे कसे घडते? येथे काही उदाहरणे आहेत:
\begin{itemize}
\item इतरांना हानी पोहोचेल तर त्यांच्यावर हिंसाचार करू नका किंवा त्यांना मारू नका.
\item प्रदूषण करू नका किंवा इतरांच्या आरोग्याला हानी पोहोचेल अशी गोष्टी करू नका.
\item मद्यपान करून गाडी चालवू नका, किंवा इतर निष्काळजीपणाचे काम करू नका ज्यामुळे इतरांना इजा होऊ शकेल.
\item प्राणी किंवा प्राणी उत्पादने खाऊ नका.
\item इतरांना दडपशाहीच्या पद्धतीने काम देऊ नका, किंवा दडपलेल्या मजुरांनी बनवलेली उत्पादने वापरू नका.
\item इतरांना हानी होईल अशी माहिती पसरवू नका.
\item इतरांना हानी होईल तर चोरी करू नका किंवा त्यांच्याकडून गोष्टी घेऊ नका.
\item इतरांना हानी होईल तर त्यांच्याकडून संसाधने थांबवू नका.
\item इतरांना हानी होताना बघितले तर शांत राहू नका किंवा उभे राहू नका.
\item तुमच्यासाठी जे अवडत नाही ते इतरांना करू नका.
\item तुमचे विश्वास इतरांवर जबरदस्ती लादू नका.
\item खोटे बोलू नका.
\item खरोखर गरज नसलेल्या गोष्टी विकत घेऊ नका (पर्यावरणाला हानी करू नका).
\end{itemize}
अनेकदा "कोणती हानी करू नका" मुळे कठीण निवडी करावे लागतात - कधीकधी तुम्हाला शोधून काढावे लागते की कोणती कृती (किंवा निष्क्रियता) कमी हानी करते.
या तत्त्वाची सकारात्मक बाजू म्हणजे "दयाळू राहा." यामध्ये अनेकदा आपल्या विचार करण्याच्या पद्धतीत पूर्ण बदल करावा लागतो. उदाहरणार्थ, इतरांचा न्याय करण्याऐवजी, दयाळू असणे म्हणजे त्यांना अधिक चांगल्या पद्धतीने समजून घेण्याचा प्रयत्न करणे, त्यांच्याशी सहानुभूती दाखवणे आणि त्यांच्या दुःखापासून मुक्ती मिळवण्याचा प्रयत्न करणे.
दयाळू जीवन हा विषय आहे ज्याला संपूर्ण पुस्तकाची गरज आहे - मी दलाई लामाचे "द आर्ट ऑफ हॅपिनेस" सुचवेन. थोडक्यात, यासाठी समजून घेणे, सहानुभूती आणि इतरांचे दुःख कमी करण्याची आणि आनंद वाढवण्याची इच्छा आवश्यक आहे.

\chapter{कोणतेही लक्ष्य किंवा निश्चित योजना ठेवू नका}
ठोस, साध्य करता येणारी लक्ष्ये असण्याची कल्पना आपल्या संस्कृतीत खोलवर रुजलेली दिसते. मला माहित आहे की मी अनेक वर्षे लक्ष्यांसोबत जगलो, आणि खरे तर, माझ्या पूर्वीच्या लेखनाचा मोठा भाग लक्ष्ये कशी ठेवावी आणि साध्य करावी याबद्दल आहे.
या दिवसांमध्ये, तथापि, मी बहुतेक वेळा लक्ष्यांशिवाय जगतो. हे मुक्तिदायक आहे, आणि तुम्हाला शिकवले गेले असेल त्या विरुद्ध, याचा अर्थ असा नाही की तुम्ही गोष्टी साध्य करणे थांबवता.
याचा अर्थ असा आहे की तुम्ही लक्ष्यांमुळे स्वतःला मर्यादित करणे थांबवता.
या सामान्य विश्वासाचा विचार करा: "तुम्ही कुठे जात आहात हे माहित नसेल तर तुम्ही कुठेही पोहोचणार नाही." हे सामान्य बुद्धीसारखे वाटते, तरीही तुम्ही खरोखर विचार केला तर हे स्पष्टपणे खोटे आहे. एक साधा प्रयोग करा: बाहेर जा आणि यादृच्छिक दिशेने चालत जा, आणि यादृच्छिकपणे दिशा बदलण्यास मोकळे राहा. वीस मिनिटांनंतर, एक तासानंतर... तुम्ही कुठेतरी असाल! तुम्हाला फक्त माहित नव्हते की तुम्ही तिथे पोहोचणार आहात.
आणि हीच खरी गोष्ट आहे: तुम्हाला तुमचे मन उघडे ठेवून अशा ठिकाणी जाण्यासाठी तयार असावे लागेल जिकडे तुमची अपेक्षा नव्हती. जर तुम्ही लक्ष्यांशिवाय जगलात, तर तुम्ही नव्या प्रदेशाचा शोध घ्याल. तुम्ही काही अनपेक्षित गोष्टी शिकाल. तुम्ही आश्चर्यकारक ठिकाणी पोहोचाल. हे या तत्त्वज्ञानाचे सौंदर्य आहे, पण हे देखील एक कठीण संक्रमण आहे.
आज, मी बहुतेक लक्ष्यांशिवाय जगतो. अधूनमधून मी एखादे लक्ष्य विचारू लागतो, पण मी त्यांना सोडत आहे. लक्ष्यांशिवाय जगणे हे माझे कधीच वास्तविक लक्ष्य नव्हते... हे फक्त एक गोष्ट आहे जी मी शिकत आहे ज्याचा मला आनंद होतो, जी मुक्त करणारी आहे, जी माझ्या आवडीचे काम करण्याच्या जीवनशैलीसोबत काम करते.

\chapter{तीन महत्त्वाची नोंदी}
अनेक लोकांना माझ्या "लक्ष्य नाही" प्रयोगावर आक्षेप आहे, म्हणून मी त्यात जाण्यापूर्वी तीन नोंदी करणार आहे:

\begin{itemize}
\item माझी "लक्ष्य" ची व्याख्या: मी लक्ष्याची व्याख्या "काहीही करायची इच्छा" अशी करत नाही. मी सर्व इच्छा संपवण्याबद्दल बोलत नाही. मी पूर्वनिर्धारित परिणाम सोडण्याबद्दल बोलत आहे. म्हणून "लक्ष्य" म्हणजे "पूर्वनिर्धारित परिणाम किंवा गंतव्य." जर तुम्ही कुठे जात आहात हे न जाणता चालू लागला, तर तुम्ही म्हणू शकता, "माझे चालण्याचे लक्ष्य आहे!" पण तुम्हाला माहित नाही तुम्ही कुठे जात आहात. त्याऐवजी जर तुम्ही दुकानात जाण्यासाठी चालू लागला, तर ते लक्ष्यासह चालणे आहे. जेव्हा लोक म्हणतात, "तुम्ही काहीतरी करत आहात, म्हणून तुमची लक्ष्ये आहेत!", तेव्हा माझे उत्तर असते, "होय, पण मला माहित नाही किंवा काळजी नाही ते मला कुठे घेऊन जाईल." (बीटीडब्लू, हे गॉचा सिंड्रोमचे लक्षण आहे, जिथे लोक खरोखर शिफारसी वापरून बघण्याऐवजी माझे दांभिकपणा दाखवण्याचा प्रयत्न करतात.)
\item तुम्हाला ते करून पाहण्याची गरज नाही. जर लक्ष्यांशिवाय जगणे तुम्हाला मूर्खपणाचे किंवा अतिपरिणामकारक वाटत असेल, तर ते करून पाहू नका. मला याबद्दल तुम्ही माझ्याशी असहमत आहात यावर काही फरक पडत नाही -- हे माझ्यासाठी काम करते, पण कदाचित तुमच्यासाठी काम करणार नाही. ते ठीक आहे. या पुस्तकातील इतर गोष्टी अजूनही उपयुक्त आहेत. आणि कोण जाणे, कदाचित एखाद्या दिवशी तुम्ही यावर परत याल आणि त्याचा विचार कराल.
\item मला सुरुवातीला लक्ष्यांची गरज होती का? अनेक लोक म्हणतात की मला आता लक्ष्यांची गरज नसणे ठीक आहे, पण हे फक्त यामुळे आहे की मी आधीच खूप काही साध्य केले आहे, आणि अशा स्थितीत पोहोचलो आहे जिथे मला लक्ष्यांची गरज नाही. ते ठीक आहे -- तुम्ही यावर विश्वास ठेवू शकता... किंवा, तुम्ही फक्त लक्ष्यांशिवाय जगून आणि काम करून बघू शकता, आणि काय होते ते पाहू शकता. मला माहित नाही मला सुरुवातीला लक्ष्यांची गरज होती का -- मी परत जाऊन ते तपासू कसे? माझा अंदाज असा आहे की जर मी या विचारांनी सुरुवात केली असती तर मी आता जिथे आहे तिथे नसतो, पण मी कुठेतरी छान जागी असतो.
\end{itemize}

\chapter{लक्ष्यांची समस्या}
भूतकाळात, मी वर्षासाठी एक किंवा तीन लक्ष्ये ठेवायचो, आणि मग प्रत्येक महिन्यासाठी उप-लक्ष्ये. मग मी प्रत्येक आठवडा आणि प्रत्येक दिवस कोणत्या कृतीच्या पायऱ्या घ्यायच्या याचा विचार करायचो, आणि त्या पायऱ्यांवर दिवसभराचे लक्ष केंद्रित करण्याचा प्रयत्न करायचो.
दुर्दैवाने, हे कधीही, कधीही इतक्या व्यवस्थितपणे काम करत नाही. तुम्हा सर्वांना हे माहित आहे. तुम्ही व्यस्त होता किंवा टाळाटाळ करता किंवा जीवन मध्ये अडथळे येतात, तुमची साप्ताहिक लक्ष्ये आणि मासिक लक्ष्ये मागे ढकलली जातात किंवा रांगेत राहतात, तुम्ही निराश होता कारण तुमच्यात शिस्त नाही. मग तुम्ही तुमच्या लक्ष्यांचा आढावा घेता आणि त्यांना पुन्हा सेट करता. तुम्ही उप-लक्ष्ये आणि कृती योजनांचा नवीन संच तयार करता.
कधीकधी तुम्ही लक्ष्य साध्य करता आणि मग तुम्हाला अद्भुत वाटते. पण बहुतेक वेळा तुम्ही ते साध्य करत नाही आणि तुम्ही स्वतःला दोष देता.
येथे गुपित आहे: समस्या तुमच्यात नाही, ती प्रणालीत आहे! लक्ष्य प्रणाली अपयशाची स्थापना आहे.
जरी तुम्ही सर्वकाही नीट केले तरीही, ते आदर्श नसते कारण लक्ष्ये तुमच्या शक्यता मर्यादित करतात. जेव्हा तुमचे काहीतरी करावेसे वाटत नाही तेव्हा तुम्हाला स्वतःला जबरदस्ती करावी लागते. तुमचा मार्ग निवडलेला असतो, म्हणून तुमच्याकडे नवीन प्रदेशाचा शोध घेण्यासाठी जागा नसते. तुम्हाला योजनेचे पालन करावे लागते, जरी तुम्ही दुसर्‍या गोष्टीबद्दल उत्कट असाल तरी.
काही लक्ष्य प्रणाली अधिक लवचिक आहेत, पण लक्ष्यांशिवायचे जीवनाइतके लवचिक काहीही नाही.

\chapter{लक्ष्यांशिवाय जगणे}
तर लक्ष्यांशिवायचे जीवन कसे दिसते? व्यवहारात, ते लक्ष्यांसह असलेल्यापेक्षा खूप वेगळे असते.
तुम्ही वर्षासाठी लक्ष्य ठेवत नाही, महिन्यासाठी, आठवड्यासाठी किंवा दिवसासाठी नाही. तुम्ही ट्रॅकिंग किंवा कृतीक्षम पायऱ्यांबद्दल वेड धरत नाही. तुम्हाला टू-डू लिस्टची गरजही नाही, जरी तुम्हाला हवे असल्यास आठवणी लिहिण्यात काही हरकत नाही.
मग तुम्ही काय करता? दिवसभर सोफ्यावर पडून राहता? नाही, तुम्ही एखादी गोष्ट शोधता ज्याबद्दल तुम्ही उत्कट आहात, आणि तुम्ही ती करता. फक्त तुमची लक्ष्ये नसल्यामुळे तुम्ही काहीच करत नाही असे नाही - तुम्ही निर्माण करू शकता, तुम्ही उत्पादन करू शकता, तुम्ही तुमच्या आवडीचे काम करू शकता.
आणि व्यवहारात, ही एक अद्भुत गोष्ट आहे: तुम्ही जागे होता आणि ज्या गोष्टीबद्दल तुम्ही उत्कट आहात ती करता. माझ्यासाठी, ते सहसा लिहिणे असते, पण ते इतरांना मदत करणे किंवा अविश्वसनीय लोकांशी संपर्क साधणे किंवा माझ्या पत्नीसोबत वेळ घालवणे किंवा माझ्या मुलांसोबत खेळणे असू शकते. कोणती मर्यादा नाही, कारण मी मुक्त आहे.
शेवटी मी सहसा लक्ष्ये असल्यापेक्षा अधिक साध्य करतो, कारण मी नेहमी अशा गोष्टीत व्यस्त असतो ज्याबद्दल मी उत्साहित आहे. पण मी साध्य करतो की नाही हा मुद्दा नाही: महत्त्वाचे हे आहे की मी नेहमी आवडते काम करतो.
मी अशा ठिकाणी पोहोचतो जी अद्भुत, आश्चर्यकारक, उत्कृष्ट असतात. मी फक्त सुरुवातीला माहित नव्हते की मी तिथे पोहोचणार आहे.
तुम्हाला जो काही मार्ग मिळतो, तुम्ही कुठे पोहोचता, ते सुंदर असते. कोणताही वाईट मार्ग नाही, कोणतेही वाईट गंतव्य नाही. ते फक्त वेगळे असते, आणि वेगळे म्हणजे अद्भुत. न्याय करू नका. फक्त अनुभव घ्या.
नेहमी लक्षात ठेवा: प्रवास हीच सर्वकाही आहे. गंतव्य हे बाजूला आहे.

\chapter{कोणत्या अपेक्षा ठेवू नका}
तुमचा ताण, निराशा, निराशा, राग, चिढ, वाईट मूड यापैकी किती एका छोट्या गोष्टीमुळे येतो?
या सर्वांचे जवळजवळ सर्व आपल्या अपेक्षांमधून येते, आणि, जेव्हा गोष्टी (अपरिहार्यपणे) आपल्या अपेक्षेप्रमाणे घडत नाहीत, तेव्हा गोष्टी वेगळ्या असाव्यात अशी इच्छा करण्यातून.
आपण आपल्या डोक्यात ही अपेक्षा तयार करतो की इतर लोकांनी काय करावे, आपले जीवन कसे असावे किंवा दिसावे, इतर ड्रायव्हर कसे वागावेत... आणि तरीही हे सर्व कल्पना आहे. हे खरे नाही.
आणि जेव्हा वास्तविकता आपल्या कल्पनांशी मेळ खात नाही, तेव्हा आपण जग वेगळे असावे अशी इच्छा करतो.
येथे एक साधा उपाय आहे:
तुमच्या अपेक्षा घेऊन समुद्रात फेकून द्या.
स्वतःसाठी, तुमच्या जीवनासाठी, तुमच्या जोडीदारासाठी, तुमच्या मुलांसाठी, तुमच्या सहकार्‍यांसाठी, तुमच्या नोकरीसाठी, जगासाठी तुमच्या सर्व अपेक्षांची कल्पना करा. त्या तुमच्या आतून काढून समुद्रात फेकून द्या. नदी किंवा तलाव देखील चालेल.
त्यांचे काय होते? ते तरंगतात. ते लाटांनी भोवती नेले जातात. प्रवाह त्यांना बाहेर नेतो, आणि ते वाहून जातात. त्यांना स्वच्छ पाण्याने धुवून नेऊ द्या, आणि त्यांना सोडून द्या.
आता त्यांशिवाय तुमचे जीवन जगा.
अपेक्षांशिवायचे जीवन कसे असते? तुम्ही वास्तविकता जशी आहे तशी स्वीकारता, आणि लोकांना जसे ते आहेत तसे, त्यांना तुम्ही निर्माण केलेल्या चौकटीत जबरदस्ती ठोकण्याचा प्रयत्न न करता. तुम्ही गोष्टी जश्या आहेत तश्या पाहता. तुम्हाला निराश किंवा निराश किंवा रागावणे गरजेचे नाही - किंवा जर तुम्ही आहात, तर तुम्ही ते स्वीकारता, आणि मग सोडून देता.
याचा अर्थ असा नाही की तुम्ही कधीच कृती करत नाही - तुम्ही तुमच्या मूल्यांनुसार कृती करू शकता आणि जगावर प्रभाव टाकू शकता, पण जग तुमच्या कृतींना कसा प्रतिसाद देईल याची कधीही अपेक्षा ठेवू नका.

जर तुम्ही काहीतरी चांगले केले, तर तुम्ही प्रशंसा किंवा कौतुकाची अपेक्षा ठेवणार नाही. त्या बक्षीस आणि प्रशंसेच्या अपेक्षांना लाटांसोबत वाहून जाऊ द्या. चांगले काम करा कारण तुम्हाला चांगले काम करायला आवडते, आणि त्यापलीकडे काहीही अपेक्षा करू नका.
तुमच्या विचारांकडे लक्ष द्या. तुमच्या अपेक्षा आहेत म्हणून स्वतःला मारहाण करू नका. फक्त त्यांना पहा. मग त्यांना समुद्रात फेकून द्या.
जर तुम्हाला गोष्टी ज्या प्रकारे आहेत त्यापेक्षा वेगळ्या असाव्यात अशी इच्छा सुरू झाली असे लक्षात आले तर. जर तुमची इच्छा असेल की इतर कोणीतरी काहीतरी करू नये असे वाटत असेल, तर त्याकडे लक्ष द्या. तुमच्या अपेक्षा आहेत, आणि तुमची इच्छा आहे की लोक किंवा जग त्या पूर्ण करावेत ते खरोखर काय करतात त्याऐवजी. त्या इच्छांनाही समुद्रात फेकून द्या. आता गोष्टी स्वीकारा, आणि पुढे जा.
जगाच्या पाण्याने आपल्याला स्वच्छ करू द्या, आणि आपल्या कल्पनांशिवाय आधीच अद्भुत असलेल्या जगात हलकेपणाने चालू द्या.

\chapter{नियंत्रणाचा भ्रम}
जेव्हा तुम्हाला वाटते की तुम्ही काहीतरी नियंत्रित करता, तेव्हा तुम्ही चुकता.
हे आश्चर्यकारक आहे की आपण किती वेळा आपण काहीतरी नियंत्रण करतो असे मानतो जेव्हा खरोखर आपण करत नाही.
नियंत्रण हा एक भ्रम आहे.
आपण सतत अशा योजना बनवतो ज्या कधीही आपल्या कल्पनेप्रमाणे घडत नाहीत. "जर तुम्हाला देवाला हसवायचे असेल तर योजना बनवा," एक जुनी म्हण आहे.
आपल्याला लक्ष्ये ठेवण्यासाठी आणि नंतर त्या लक्ष्यांकडे नेणाऱ्या कृतींवर काम करण्यासाठी प्रशिक्षित केले गेले आहे... आणि तरीही आपण किती वेळा त्या लक्ष्यांना पूर्ण करण्यात अपयशी ठरतो? आपण किती वेळा अशा भविष्याला नियंत्रित करण्याचा प्रयत्न करतो ज्याचा आपण अंदाज लावू शकत नाही?
तुम्हाला पाच वर्षांपूर्वी माहित होते का की जग असे निघेल - ओबामा राष्ट्रपती म्हणून निवडले जातील, स्टॉक मार्केट कोसळेल, आपण मंदीत खोल गेलो असू, भूकंप आणि त्सुनामी येतील, तुम्ही आज नक्की जे करत आहात ते करत असाल?
अर्थात नाही. आपल्याला भविष्य माहित नाही, तसेच आपण ते नियंत्रित करू शकत नाही. आपल्याला असे वाटायला आवडते की आपण करू शकतो, पण ते कधीच खरे ठरत नाही.
आणि तरीही आपण नियंत्रणाच्या भ्रमावर विश्वास ठेवत राहतो. आपण गोंधळलेल्या आणि गुंतागुंतीच्या जगाला तोंड देतो आणि जे काही शक्य आहे त्या पद्धतीने ते नियंत्रित करण्याचा प्रयत्न करतो.
जग नियंत्रित करण्याचे आपले प्रयत्न या पद्धतींद्वारे दिसू शकतात:
\begin{itemize}
\item आपली मुले कशी निघतील याला नियंत्रित करण्याचा प्रयत्न करणे, जणू काही आपण त्यांना माती च्या तुकड्यांसारखे आकार देऊ शकतो, जणू काही माणसे आपल्याला समजण्यापेक्षा अधिक गुंतागुंतीची नाहीत.
\item प्रत्येक छोट्या गोष्टीचा मागोवा घेणे, खर्चापासून व्यायामापासून आपण काय खातो ते आपण कोणती कामे करतो ते आपल्या साइटवर किती अभ्यागत आहेत ते आज आपण किती पावले उचलली आणि किती मैल धावलो यापर्यंत. जणू काही आपली निवडक ट्रॅकिंग परिणामांवर प्रभाव टाकणाऱ्या अनेक गुंतागुंतीच्या घटकांचा समावेश करू शकेल.
\item कर्मचाऱ्यांना नियंत्रित करण्याचा प्रयत्न - पुन्हा, गुंतागुंतीचे मानवी प्राणी ज्यांच्या अनेक प्रेरणा आणि लहरी आणि सवयी आहेत ज्या आपल्याला समजत नाहीत.
\item प्रकल्प, सहली, दिवस, पार्ट्यांची वेडसरपणे योजना करणे, जणू काही कार्यक्रमांचे परिणाम अशा गोष्टी आहेत ज्यांना आपण जगावरील आपल्या हाताळणीच्या शक्तींनी नियंत्रित करू शकतो.
\end{itemize}
जर आपण या भ्रमाला सोडू शकलो तर आपल्याकडे काय राहते? या गोंधळात आपण कसे जगू शकतो?
माशाचा विचार करा. मासा अशा गोंधळलेल्या समुद्रात पोहतो जे तो नियंत्रित करू शकत नाही - जसे आपण सर्व करतो. मासा, आपल्यापेक्षा वेगळा, असा भ्रम पाळत नाही की तो समुद्राला किंवा समुद्रातील इतर माशांना नियंत्रित करतो. मासा तो कुठे संपतो ते नियंत्रित करण्याचा प्रयत्नही करत नाही - तो फक्त पोहतो, एकतर प्रवाहासोबत जातो, किंवा प्रवाह येतो तसा त्याला तोंड देतो. तो खातो आणि लपतो आणि संभोग करतो, पण कोणतीही गोष्ट नियंत्रित करण्याचा प्रयत्न करत नाही.
आपण त्या माशापेक्षा चांगले नाही, तरीही आपले विचार करणे या भ्रमाची गरज निर्माण करते.
त्या विचार करण्याला सोडून द्या. मासा होण्यास शिका.
जेव्हा आपण गोंधळाच्या मधोमध असतो, तेव्हा त्याला नियंत्रित करण्याची गरज सोडून द्या. त्यात बुडून जा, त्या क्षणी त्याचा अनुभव घ्या; परिणाम नियंत्रित करण्याचा प्रयत्न करू नका पण प्रवाह येतो तसा त्याला तोंड द्या.
आपण असे आपले जीवन कसे जगतो? भ्रम सोडल्यावर हा पूर्णपणे वेगळा जगण्याचा मार्ग आहे:
\begin{itemize}
\item आपण लक्ष्ये ठेवणे थांबवतो, आणि त्याऐवजी आपल्याला जे उत्साहित करते ते करतो.
\item आपण योजना करणे थांबवतो, आणि फक्त करतो.
\item आपण भविष्याकडे बघणे थांबवतो, आणि या क्षणात जगतो.
\item आपण इतरांना नियंत्रित करण्याचा प्रयत्न थांबवतो, आणि त्याऐवजी त्यांच्याशी दयाळू राहण्यावर आणि प्रेम दाखवण्यावर लक्ष केंद्रित करतो.
\item आपण शिकतो की आपल्या मूल्यांवर विश्वास ठेवणे हे विशिष्ट परिणामांची इच्छा करणे आणि त्यांच्यासाठी प्रयत्न करण्यापेक्षा कृती करण्यासाठी अधिक महत्त्वाचे आहे.
\item आपण प्रत्येक पावल हलकेपणाने, संतुलनासह, या क्षणात, त्या मूल्यांनी आणि ज्या गोष्टीबद्दल आपण उत्कट आहोत त्याच्या मार्गदर्शनाखाली घेतो... पुढच्या १००० पावलांची आणि आपण कुठे संपणार याची योजना करण्याऐवजी.
\item आपण जग जसे आहे तसे स्वीकारायला शिकतो, त्याच्यामुळे कष्ट होणे, ताण होणे, राग येणे, निराशा होणे, किंवा आपल्याला हवे तसे बदलण्याचा प्रयत्न करण्याऐवजी.
\item गोष्टी कसे निघतात त्यामुळे आपण कधीही निराश होत नाही, कारण आपण कधीही काहीची अपेक्षा करत नाही - आपण फक्त जे येते ते स्वीकारतो.
\end{itemize}
काहींना हा जगण्याचा निष्क्रीय मार्ग वाटू शकतो, आणि हा आपल्या आक्रमक, उत्पादक, लक्ष्य-केंद्रित सांस्कृतिक स्वभावाविरुद्ध आहे. जर तुम्ही हा जगण्याचा मार्ग स्वीकारू शकत नसाल तर ते ठीक आहे - अनेक लोक नियंत्रणाच्या भ्रमासह आपले जीवन जगतात. त्यांना दुःखी किंवा निराश करणाऱ्या गोष्टींबद्दल अज्ञान राहणे हे सर्वात वाईट गोष्ट नाही.
पण जर तुम्ही अशा प्रकारे जगायला शिकू शकलात... तर हे जगातील सर्वात मुक्तिदायक गोष्ट आहे.


\chapter{गोंधळासोबत जगणे}
आपण लक्ष्ये, योजना, अपेक्षा सोडण्याबद्दल बोललो आहो. मी अजूनही शिकत आहे की नियंत्रणाचा भ्रम सोडल्यावर आणि शक्य तितकी कमी योजना केल्यावर काय करावे.
लक्ष्ये किंवा योजनांशिवायचे जीवन कसे असते? गोंधळाला आपण कसे तोंड देतो?
माझ्याकडे सर्व उत्तरे नाहीत, पण मी खूप काही शिकत आहे.
मी नुकतेच पोर्टलँडमधील वर्ल्ड डॉमिनेशन समिटला कमी योजनांसोबत गेलो होतो. माझे एक भाषण होते, काही छोट्या सत्रे घ्यायची होती, एक बाईक टूर ठरवलेला होता, विमानाचे तिकीट आणि हॉटेल रूम होते. पण आठवड्याचा मोठा भाग मी मोकळा ठेवला होता, कोणत्या योजनांशिवाय.
हे मुक्तिदायक होते. भाषणे देण्यात मला हरकत नव्हती, आणि टूरला मी प्रेम केले, पण अनपेक्षित अनोळखी लोकांशी भेटणे, मी कधीही न भेटलेल्या लोकांसोबत वेळ घालवणे, गर्दीच्या प्रवाहासोबत जाणे - मजा होती. मला खरोखर माहित नव्हते पुढे काय होणार आहे, आणि ते डरकारक आहे... पण विचित्रपणे मुक्तिदायक आहे.
मी नुकताच गुआममध्ये एका महिन्यासाठी गेलो होतो, आणि भेटण्यासाठी अनेक मित्र आणि कुटुंबीय होते. पण राहण्यासाठी जागा वगळून, आमच्या कोणत्या निश्चित योजना नव्हत्या. आम्हाला माहित नव्हते वाहतुकीसाठी काय करणार, दररोज काय करणार हे आम्हाला माहित नव्हते. डरकारक होते, पण आम्ही ठीक होतो.
गोंधळासोबत तुम्ही कसे जगता?
तुम्ही त्याला आलिंगन देण्यास शिकता.

\chapter{योजनांशिवाय दैनंदिन जगणे}
मी शक्य तितकी कमी वेळापत्रक ठेवण्याचा प्रयत्न करतो, आणि दररोजासाठी माझी कोणती लक्ष्ये नाहीत. मी जागा होतो आणि स्वतःला विचारतो, "आज मला कशामुळे उत्साह आहे?" आणि दररोज उत्तर वेगळे असते.
नक्कीच, माझी काही जबाबदाऱ्या आहेत ज्या मला पूर्ण करावी लागतात, पण बहुतेक त्या अशा गोष्टी आहेत ज्याबद्दल मी उत्साहित आहे. मी अजूनही त्या गोष्टी करेन ज्याबद्दल मी तितका उत्साहित नाही - जर मी त्या टाळू शकलो नाही तर.
पण प्रत्येक क्षणात मी जाणीवपूर्वक, या क्षणात जगण्याचा प्रयत्न करतो, आणि स्वतःला विचारतो... "मी कशाबद्दल उत्कट आहे? आणि माझ्या मूल्यांशी खरे राहून प्रत्येक क्षणाला कसे हाताळू शकतो?" हे "जागरूक" राहण्याबद्दल आहे. बहुतेक लोक जागरूकतेच्या अवस्थेत जगत नाहीत.
माझे मूल्य म्हणजे दया, जी विविध प्रकारांमध्ये येते: प्रेम, दयाळूपणा, सहानुभूती, कृतज्ञता. प्रत्येक वेळी परिस्थिती येते, मी स्वतःला विचारतो, "मी यास दयाळूपणाने कसे तोंड देऊ शकतो?" हा प्रश्न अधिक लोकांनी स्वतःला विचारावा.
मी अजूनही हे कसे करावे ते शिकत आहे. मी त्यावर प्रभुत्व मिळवले आहे असा दावा करत नाही, आणि कदाचित येत्या वर्षांमध्ये ते करण्याचे मार्ग शोधत राहीन.

\chapter{योजना का भ्रम आहेत}
योजनांशिवाय जगणे बहुतेक लोकांना मूर्खपणाचे किंवा अवास्तव वाटू शकते. ते ठीक आहे. पण जर तुम्हाला वास्तववादी व्हायचे असेल, तर तुम्हाला समजावे की तुम्ही बनवलेल्या योजना नियंत्रणाचे शुद्ध भ्रम आहेत.
एक साधे उदाहरण घेऊया. तुमची एक अहवाल (किंवा ब्लॉग पोस्ट किंवा पुस्तकाचा अध्याय) लिहिण्याची आणि नंतर सहकारी किंवा व्यावसायिक भागीदाराशी भेटण्याची योजना आहे. लेखन सकाळी ९ वाजता होणार आहे आणि भेट ११ वाजता आहे.
समजा या गोष्टी खरोखर योजनेप्रमाणे घडतात. अनेक दिवस, इतर गोष्टी येतात आणि नियंत्रणाचा भ्रम सहजपणे भंग पावतो. पण काही दिवस आपल्याला नशीब लागते आणि आपल्या योजना खरोखर आपल्या आशेप्रमाणे घडतात.
तर तुम्ही योजनेप्रमाणे लिहिण्यासाठी बसता. कदाचित तुम्ही तुमच्या लेखनाची रूपरेषा तयार केली असेल. पण जसे तुम्ही लिहिता, तुम्हाला अशा गोष्टींचा विचार येतो ज्यांची योजना नव्हती. तुम्हाला अशा समस्यांना सामोरे जावे लागते ज्यांचा तुम्ही लिहायला सुरुवात करण्यापूर्वी अंदाज लावू शकलात नसता. खरे तर, जर तुम्ही बारकाईने लक्ष दिले, तर हे स्पष्ट होते की तुम्ही लेखन आधीपासून नियोजित करू शकलात नसता - ते तुम्ही करताना उलगडत जाते, कारण फक्त तुम्ही करताना तुम्ही गोष्टींचा पूर्ण विचार करता, आणि स्वतःच्या विचार करण्याचा (इतरांच्या विचार करण्याची तर सोडाच) अंदाज लावण्याचा कोणताही मार्ग नाही.
आणि म्हणून आपल्या लेखनातून अशा गोष्टी उदयास येतात ज्यांची कधीही योजना करता आली नसती, आणि खरे तर, जर आपण त्यासाठी मुक्त असू, तर आपण काहीतरी पूर्णपणे प्रतिभावान लिहू शकू जे आपण कधीही अंदाज लावू शकलो नसतो. तथापि, जर आपण रूपरेषेला चिकटून राहण्याचा प्रयत्न केला, तर आपण उद्भवणाऱ्या प्रतिभावान शक्यतांकडे दुर्लक्ष करू शकतो.
तर आता सकाळी ११ वाजले आहेत आणि तुमच्या भेटीचा वेळ आहे. तुम्ही योजनेप्रमाणे तुमच्या सहकाऱ्याशी किंवा भागीदाराशी भेटता, आणि बोलायला सुरुवात करता. अर्थात, संभाषणांची योजना करता येत नाही, आणि तुम्ही बोलता तेव्हा काय उदयास येईल याचा अंदाज लावणे शक्य नाही. तुमच्याकडे अजेंडा देखील असू शकतो, पण तुम्ही अजेंडावरील गोष्टींबद्दल बोलता तेव्हा नवीन कल्पना येतात, आणि जेव्हा तुमच्यापैकी कोणी नवीन कल्पना सुचवतो, तेव्हा ती दुसऱ्या व्यक्तीमध्ये दुसरी कल्पना जन्माला घालते, आणि असेच चालू राहते - कल्पना जन्माला येतात, मागे-पुढे, ज्यांची योजना करता आली नसती.
आणि अशा प्रकारे या भेटीतून नवीन कल्पना आणि प्रकल्प आणि सहकार्य उदयास येते जे कधीही नियोजित करता आले नसते. हे एक छान गोष्ट आहे.
दोन नियोजित कार्यक्रम, जरी ते नियोजितप्रमाणे घडले, तरीही ते पूर्णपणे अंदाज न बांधता येणारे आणि अनियंत्रित होते. आपण या गोंधळाला जितके आलिंगन देतो, तितकी आपण उदयास येऊ शकणाऱ्या प्रतिभावान शक्यतांना आलिंगन देतो. आपण योजनांसह आपला दिवस आणि कृती नियंत्रित करण्याचा जितका प्रयत्न करतो, तितके आपण स्वतःला मर्यादित करतो.

\chapter{उलगडणाऱ्या क्षणासाठी मुक्त राहा}
आपण नियंत्रणाचा भ्रम धरून ठेवण्याचा प्रयत्न करतो, पण त्याऐवजी आपण गोंधळाला आलिंगन दिले तर काय? जर आपण स्वतःला बदलणाऱ्या, उलगडणाऱ्या क्षणासाठी आणि ज्या शक्यतांची आपण कधीही योजना करू शकलो नसतो त्यांसाठी मुक्त ठेवले तर काय?
हे सुंदर आहे.
त्याचा प्रयत्न करा. पुढच्या तासासाठी तुमच्या योजना फेकून द्या. क्षण-प्रति-क्षण काय होते ते पहा. तुम्हाला काय उत्साहित करते, तुमच्या मूल्यांशी कशाची जुळवणी होते याचा विचार करा. याबद्दल जाणीवपूर्वक राहा.
आणि तुम्ही अशा गोष्टी करायला सुरुवात केली ज्या तुम्हाला उत्साहित करतात, तुमच्या मूल्यांशी जुळतात... कोणत्या नवीन गोष्टी उदयास येतात ते पहा. निश्चित हेतूशिवाय लोकांशी बोला, आणि त्या संवादातून कोणत्या कल्पना उदयास येतात ते पहा. लोकांसोबत, कल्पनांसोबत, तुमच्या स्वतःच्या विचारांसोबत संवाद साधताना कोणत्या नवीन संधी विकसित होतात ते पहा.

हे अस्पष्ट वाटते, पण खरे तर हे इतर कोणत्याही गोष्टीइतकेच ठोस आहे. मी दाखवल्याप्रमाणे, जेव्हा आपण योजना बनवतो, तेव्हा आपल्याला वाटते की आपण गोष्टी ठोस कॉंक्रिटमध्ये ठेवत आहोत, पण जीवन नेहमी तरल असते - आपण फक्त स्वतःला विश्वास दिलवण्याचा प्रयत्न करतो की ते घट्ट कॉंक्रिटसारखे आहे.
जेव्हा आपण आपल्या जीवनाच्या तरलतेला मान्यता देतो, तेव्हा आपण त्या तरलतेचा आपल्या फायद्यासाठी वापर करायला शिकतो. आपण वाहतो. आपण बदलत्या प्रवाहांसाठी मुक्त असतो. आपण आपल्या योजना आणि लक्ष्यांशी जगाला जुळवण्याचा प्रयत्न करण्याऐवजी मुक्त डोळ्यांनी गोष्टी पाहतो.
माझ्याकडे सर्व उत्तरे नाहीत, आणि खरे तर, जर मी असे जगल्यावर काय होईल याचा अंदाज लावू शकतो असा दावा केला तर मी दांभिक ठरेन... किंवा जर कोणी असे जगले तर काय होईल.
मला माहित नाही काय होईल. त्या साध्या विधानाच्या अमर्याद शक्यतांचा विचार करा.

\chapter{खोटी गरजा निर्माण करू नका}
आपल्या जीवनात अशा गोष्टींनी भरलेले आहे ज्या आपल्याला करण्याची गरज आहे. जोपर्यंत आपण त्या गरजांकडे थोडे अधिक बारकाईने बघत नाही.
तुमच्या कोणत्या गरजा असू शकतात त्याचा विचार करा: दर १५ मिनिटांनी ईमेल तपासण्याची गरज, किंवा इनबॉक्स रिकामा करण्याची, किंवा तुमचे सर्व ब्लॉग वाचण्याची, किंवा काहीतरी पूर्णपणे व्यवस्थित ठेवण्याची, किंवा कामावर नवीनतम फॅशनमध्ये कपडे घालण्याची गरज. तुमच्या मुलांना गोष्टींबद्दल सतत त्रास देण्याची, किंवा सहकाऱ्यांना नियंत्रित करण्याची, किंवा भेटू इच्छिणाऱ्या कोणाशीही भेटण्याची, किंवा अधिकाधिक पैसे मिळवण्याची, किंवा छान गाडी बाळगण्याची गरज.
या प्रकारच्या गरजा कुठून येतात? त्या पूर्णपणे बनावट आहेत.
कधीकधी गरजा समाजाने निर्माण केल्या असतात: तुम्ही ज्या उद्योगात आहात त्यात तुम्हाला रात्री ९ पर्यंत काम करावे लागते किंवा निर्दोष सूट घालावे लागतात. तुमच्या शेजारच्या ठराविक मानके आहेत आणि जर तुमच्याकडे निष्कलंक हिरवळ आणि ड्राईव्हवेत दोन BMW नाहीत तर तुमचा न्याय केला जाईल. जर तुमच्याकडे नवीनतम iPhone नाही तर तुमची गीक क्रेड किंवा स्टेटस सिम्बल राहणार नाही, आणि तुम्हाला ते असलेल्यांचा हेवा वाटेल.
कधीकधी गरजा आपणच निर्माण करतो: आपल्याला आपल्या ईमेल किंवा RSS फीड किंवा बातम्या वेबसाइट किंवा टेक्स्ट मेसेज किंवा ट्विटर अकाउंट सतत तपासण्याची इच्छा होते, जरी आपण त्यांचा मागोवा ठेवला नाही तर कोणतेही नकारात्मक सामाजिक किंवा कामाचे परिणाम होत नाहीत. आपल्याला परिपूर्ण बनवलेला पलंग हवा असतो जरी दुसऱ्या कोणालाही त्याची पर्वा नसली. आपल्याला आयुष्यातील लक्ष्यांची, किंवा वर्षभराची यादी बनवायची आणि त्यापैकी प्रत्येक साध्य करायची इच्छा होते, जरी आपण त्यापैकी बहुतेक साध्य केले नाही तर काहीही वाईट होणार नाही.
यापैकी कोणत्याही बनावट गरजा नाकारल्या जाऊ शकतात. फक्त सोडण्याच्या इच्छेची गरज आहे.
तुमची एखादी बनावट गरज तपासा, आणि स्वतःला विचारा की ती इतकी महत्त्वाची गरज का आहे. विचारा की तुम्ही ती सोडली तर काय होईल. काय चांगले होईल? तुम्हाला अधिक मुक्त वेळ आणि लक्ष केंद्रित करण्यासाठी आणि निर्माण करण्यासाठी अधिक जागा मिळेल, किंवा कमी ताण आणि दररोज तपासण्यासाठी कमी गोष्टी मिळतील का? कोणत्या वाईट गोष्टी होतील - किंवा होऊ शकतात? आणि या गोष्टी होण्याची कितपत शक्यता आहे? आणि तुम्ही त्यांचा प्रतिकार कसा करू शकता?
या गरजा भीतीमुळे निर्माण होतात, आणि आपण या भीतींबद्दल जितके प्रामाणिक असू तितके चांगले. भीतींना सामोरे जा, आणि स्वतःला थोडा प्रयोग काळ द्या - स्वतःला गरज सोडण्याची परवानगी द्या, पण फक्त एक तासासाठी, किंवा एक दिवससाठी. फक्त एका आठवड्यासाठी. जर काहीही वाईट झाले नाही तर प्रयोग वाढवा, आणि या पद्धतीने हळूहळू तुम्हाला कळेल की ती गरज अजिबात गरज नव्हती.
सोडणे चांगले वाटू शकते, आणि सोडून देऊन, तुम्ही स्वतःला मुक्त करत आहात.

\chapter{उत्कट राहा आणि आवडत नसलेले काम करू नका}
आपल्या दिवसांपैकी किती भाग अशा गोष्टी करण्यात घालवला जातो ज्या आपल्याला आवडत नाहीत? आपल्यात हे रुजवलेले आहे की आपण आवडत नसलेल्या गोष्टी करायलाच हव्यात - या गोष्टी आवश्यक आहेत, आवडत नसलेल्या गोष्टी करणे हा एक सद्गुण आहे. मी असहमत आहे.
जर तुम्हाला काहीतरी करण्याचा तिरस्कार वाटत असेल तर ते करणे थांबवण्याचा मार्ग शोधा. हे कधीकधी अगदी सोपे असू शकते, पण कधीकधी याचा अर्थ अंततः आमूलाग्र जीवन बदल करणे होय. तो बदल करावा की नाही हे तुमच्यावर अवलंबून आहे.
कधीकधी मी फक्त काहीतरी करणे थांबवू शकतो, कधीकधी मला खरोखर आवश्यक वाटलेली गोष्ट सोडावी लागली आहे (माझी नोकरी, गुआममध्ये राहणे, इ.). आणि प्रत्येक वेळी मी कष्टकारक काम सोडले आहे, मला अधिक मुक्त वाटले आहे.
मी अनेक नोकऱ्या सोडल्या आहेत ज्यांचा मला तिरस्कार होता. मला गाडी चालवणे आवडत नाही, म्हणून मी सॅन फ्रान्सिस्कोला गेलो, आणि आता माझी पत्नी आणि सहा मुले आणि मी गाडी-मुक्त आहोत. मला बजेटिंग आवडत नाही, म्हणून मी माझे आर्थिक व्यवहार स्वयंचलित केले. मी कॉमेंट्सचे नियंत्रण करण्यामध्ये कंटाळलो, म्हणून मी ते काढून टाकले. मला जाहिरातदारांशी व्यवहार करणे आवडत नव्हते, म्हणून मी माझ्या साइटवरून जाहिराती काढून टाकल्या. जेव्हा मी असे पुस्तक वाचत असतो जे मला कंटाळवाणे वाटते, तेव्हा मी दुसरे निवडतो. मी माझ्या जीवनातील पुनरावृत्ती होणारी, कंटाळवाणी कामे स्वयंचलित करतो किंवा काढून टाकतो.
मला आवडत नसलेल्या गोष्टी सोडणे मला आवडत्या गोष्टी करण्यासाठी मुक्त करते. आता मी फक्त अशा गोष्टी करतो ज्याबद्दल मी उत्कट आहे. जर मला एखादा प्रकल्प आवडेनासा वाटायला लागला तर मी तो सोडून देईन. याचा अर्थ असा असू शकतो की मी सुरू केलेली प्रत्येक गोष्ट पूर्ण करत नाही, पण आपण सुरू केलेले पूर्ण करणे ही एक खोटी गरज आहे - माझ्या प्रयोगांमध्ये मला असे आढळले आहे की आपल्याला आवडते ते करणे ही खूप चांगली पद्धत आहे.
मी सर्वात प्रिय असलेल्या लोकांसोबत वेळ घालवतो. मी वाचतो, आणि धावतो, आणि लिहितो. मी इतरांना मदत करतो, आणि एकटेपणासाठी वेळ शोधतो. या गोष्टी मला आवडतात, आणि माझे जीवन त्यांनी भरलेले आहे.
मी आरोग्य आणि तंदुरुस्तीसाठी समान कल्पना लावतो: मला आवडणारे निरोगी पदार्थ शोधतो आणि ते खातो. मी खेळण्याचे मार्ग शोधतो, आणि म्हणून मला आवडत्या गोष्टी करताना मी सक्रिय आणि तंदुरुस्त राहतो - धावणे आणि उडी मारणे आणि गोष्टी फेकणे आणि माझ्या मुलांसोबत खेळणे आणि चढाई आणि टेकड्यांवर धावणे आणि पोहणे आणि बास्केटबॉल खेळणे. मला आवडत्या गोष्टी करताना मी तंदुरुस्त होतो, आणि ते सहज होते.
जर तुम्ही तुमच्या आवडत नसलेल्या गोष्टींमध्ये स्वतःला ढकलण्यात इतकी शक्ती खर्च न केलात तर तुमचे सर्वोत्तम काम किती अधिक कुशल, भावपूर्ण आणि उपयुक्त होऊ शकेल?

\chapter{घाई करू नका, हळू जा आणि उपस्थित राहा}
घाई करू नका. हळू जा. उपस्थित राहा. आपल्या दिवसांमधून घाईघाईने जाणे अडचणी आणि अतिरिक्त प्रयत्न निर्माण करते.
आपण घाईघाईने फिरतो, सर्व काही पटकन करतो, आपल्या दिवसांमध्ये मानवी शक्यतेइतकी कामे ठोकतो. याचा अर्थ असा आहे की आपल्याकडे थोडा विश्रांतीचा वेळ आहे, कामे आणि कार्यक्रमांमधली थोडी जागा आहे, थोडी विश्रांती आहे. आणि याचा अर्थ असा आहे की आपण काम करताना फारच क्वचितच उपस्थित असतो, ज्याचा अर्थ आपण जीवनाचा, अन्नाचा, लोकांचा आनंद घेत नाही.
याचा अर्थ असाही आहे की आपण अनावश्यक समस्या निर्माण करतो. घाई करणे अनेकदा अपघातांना कारणीभूत ठरते - उदाहरणार्थ, वेगाने गाडी चालवणे हे मोटार अपघातांचे सर्वात मोठे कारण आहे. कामाच्या ठिकाणी घाईघाईने फिरणे अपघातांना कारणीभूत ठरते. काम घाईघाईने करणे चुकांना कारणीभूत ठरते. आपण घाई करताना जागरूक नसतो, ज्याचा अर्थ आपण गोष्टी चुकवतो, समस्या जवळ येताना दिसत नाहीत, आपण स्वतःला आणि इतरांना हानी पोहोचवतो.
घाई करणे आपल्या आजूबाजूच्या इतरांनाही अधिक तणावग्रस्त बनवते. जेव्हा मी उशीर होऊ नये म्हणून माझ्या कुटुंबाला घाईने दारातून बाहेर काढण्याचा प्रयत्न करतो, तेव्हा माझी पत्नी (जी कोणत्याही गोष्टीसाठी तयार होण्यासाठी वेळ घेते) तणावग्रस्त होते कारण मी तिला घाई करायला लावतो. जेव्हा आपण ऑफिसमध्ये घाईघाईने काम करतो, तेव्हा आपल्या सहकाऱ्यांना स्वतःला अधिक घाईत वाटायला लागते. हे आपल्या जीवनातील प्रत्येक कार्यक्रमात एक अतिरिक्त, अनावश्यक दबाव जोडते.
त्याऐवजी, हळू जाण्याचा प्रयत्न करा. हा सहज जीवनाचा खरा ताल आहे, आणि विडंबना म्हणजे, अनेक लोकांसाठी हे सोपे नाही. ऑफिसमध्ये किंवा घरात दोन गोष्टींमधून हळूहळू चालणे ही आपल्यापैकी बहुतेकांसाठी परकी संकल्पना आहे.
अधिक हळूहळू खाण्याचा प्रयत्न करा. फक्त खाणे करण्याचा प्रयत्न करा - वाचन किंवा इंटरनेट ब्राउझिंग किंवा टेलिव्हिजन पाहणे किंवा इतरांशी बोलणे नाही. जर तुम्हाला याची सवय नसेल तर हे कठीण आहे. पण परिणाम असा होतो की आपण आपल्या अन्नाबद्दल अधिक जागरूक होतो - त्याचा चव आणि पोत, ते कुठून आले, आपण किती खातो, आपण किती भरलेले आहोत. वजन कमी करण्यासाठी, आपल्याकडे जे आहे त्याबद्दल कृतज्ञ राहण्यासाठी, तुम्ही खात असलेल्या अन्नाची पूर्ण प्रशंसा करण्यासाठी हा एक चांगला मार्ग आहे.
अधिक हळूहळू गाडी चालवण्याचा प्रयत्न करा. तुम्ही अधिक सुरक्षित असाल, कमी हानी कराल, कमी तणावग्रस्त असाल, तुमची ड्राइव्हचा अधिक आनंद घ्याल.
हळू जीवन जगणे म्हणजे अनावश्यक लक्ष्ये, योजना, कृती काढून टाकणे, जेणेकरून तुमच्याकडे अधिक श्वास घेण्याची जागा असेल. ही वजाबाकी करण्याची प्रक्रिया वेळ घेऊ शकते. हळूहळू वजाबाकी करण्यास मोकळे राहा.

\chapter{अनावश्यक कृती निर्माण करू नका}
आपण जे बरेच काही करतो ते अनावश्यक आहे. हे एक धाडसी विधान आहे, पण निरीक्षणातून मला हे खरे वाटले आहे.
मसानोबू फुकुओकाचा विचार करा, क्रांतिकारी जपानी शेतकरी ज्याचा मी आधीच्या "खरी गरजा, साधी गरजा" या अध्यायात उल्लेख केला आहे. त्याने पारंपरिक आणि आधुनिक शेतीचा अभ्यास केला, आणि दीर्घ आणि काळजीपूर्वक निरीक्षणानंतर निष्कर्ष काढला की शेतकरी (आधुनिक किंवा पारंपरिक) जे बरेच काही करतात ते अनावश्यक आहे: नांगरणे आणि मशागत आणि तण काढणे आणि खत घालणे आणि छाटणी आणि कीटकनाशकांचा वापर. त्याने या अनावश्यक कृती काढून टाकल्या आणि करण्यासाठी खूप कमी राहिले.
समान तत्त्व आपण करत असलेल्या प्रत्येक गोष्टीला लागू होते. आपण जे बरेच काही करतो ते फक्त रूढीमुळे, आपल्याला ते आवश्यक वाटते म्हणून, किंवा आपण घेतलेल्या इतर कृतींमुळे निर्माण झालेल्या समस्यांमुळे त्या कृतींची गरज निर्माण केल्यामुळे केले जाते. आपण घेत असलेल्या प्रत्येक कृतीचा काळजीपूर्वक विचार करून, आपण अनावश्यक कृती करणे थांबवू शकतो.
तुम्हाला तपशील हवेत, फक्त सामान्यीकरण नाही. म्हणून, येथे काही उदाहरणे आहेत:
\begin{itemize}
\item प्रत्येक ईमेल किंवा फेसबुक मेसेज किंवा ट्विटला उत्तर देणे अनावश्यक आहे. आपल्याला असे करण्याची इच्छा होते, कदाचित, कारण आपल्याला उद्धट वाटायचे नाही; पण मला असे आढळले आहे की मी उत्तर न दिल्यास फार कमी लोकांना खरोखर राग येतो. सर्वात आवश्यक उत्तरे कोणती आहेत हे शोधा, आणि तीच द्या.
\item जेव्हा आपण आपल्या जीवनात खूप सामान आणतो तेव्हा आपण अनावश्यक साफसफाई आणि देखभाल निर्माण करतो. त्या अनावश्यक वस्तू काढून टाकून (डी-क्लटरिंग) आणि आपल्या जीवनात अधिक आणून न घेतल्याने, आपल्याकडे कमी साफसफाई आणि देखभाल आणि साठवण्याचे काम राहते.
\item पालक म्हणून, आपण आपल्या मुलांसाठी आणि त्यांच्यासोबत खूप जास्त करतो. खरे तर, आपण आपल्या मुलांना बऱ्यापैकी स्वावलंबी बनायला शिकवू शकतो, आणि आपण त्यांना खेळण्यासाठी आणि निर्माण करण्यासाठी जागा देऊ शकतो त्यांच्या दिवसाचा प्रत्येक मिनिट भरण्यासाठी आपली (आणि इलेक्ट्रॉनिक्स) गरज न पडता. आणि म्हणून, पालनपोषणाच्या कृती कमी करून, आपण कमी करू शकतो, पण मुलांना स्वतः वाढण्यासाठी आणि शिकण्यासाठी अधिक जागा देऊ शकतो.
\item जर तुम्ही जंगली वनस्पती नैसर्गिकरित्या वाढू द्या आणि तणांमध्ये भाज्या पेरा तर अंगणातील काम अनावश्यक होऊ शकते. नक्कीच, ते शेजारच्या नियमांनुसार नाही, पण गोष्टी कश्या पद्धतीने केल्या जातात ते आपण कसे बदलू शकतो याचे उदाहरण आहे.
\item जर तुम्ही तुमचे डोके मुंडन केले तर केसांच्या देखभालीच्या अनेक कृती नाकारल्या जाऊ शकतात.
\item जर तुम्ही घरून काम करू शकता, किंवा तुमच्या कामाजवळ राहू शकता, तर तुम्ही प्रवास नाकारू शकता.
\item जर तुम्ही तुमच्या ब्लॉगवरून कॉमेंट्स काढून टाकले तर तुम्हाला कॉमेंट्सचे नियंत्रण करण्याची गरज नाही.
\end{itemize}
अर्थातच असंख्य उदाहरणे आहेत, पण "अनावश्यक काहीही करू नका" हे मार्गदर्शक तत्त्व तुम्ही तुमचा दिवस घालवताना लक्षात ठेवावे.

\chapter{समाधान शोधा}

जवळजवळ प्रत्येकजण, मला माहित असलेले लोक, नेहमी काहीतरी चांगले शोधत असतात.  
त्यांना हवे असते चांगले जीवन, चांगले कपडे, चांगली गाडी, चांगली नोकरी,  
राहायला चांगली जागा. आणि मला हे समजते, कारण माझ्या आयुष्याचा बराचसा भाग  
मी ह्याच शोधात होतो.  

परंतु, जेव्हा मी हळूहळू समाधान शोधायला शिकलो, तेव्हाच माझे जीवन  
प्रत्यक्षात सुधारायला लागले:

\begin{itemize}
\item जेव्हा मला जाणवले की माझ्या पत्नीबरोबर, मुलांबरोबर आणि स्वतःसोबत वेळ घालवणे हेच माझ्यासाठी पुरेसे आहे,  
तेव्हा मला वेगळ्या प्रकारचे मनोरंजन किंवा खरेदीची गरज उरली नाही.  
मी कमी खर्च करू लागलो आणि कर्जमुक्त झालो.  

\item जेव्हा मी घरी केलेल्या जेवणात समाधान मानायला शिकलो,  
तेव्हा मला सतत बाहेर खाण्याची गरज उरली नाही (कधी कधी अजूनही खातो),  
आणि माझे वजनही कमी झाले.  

\item जेव्हा मी माझ्या आजूबाजूच्या गोष्टी शोधायला आणि त्यात चकित व्हायला शिकलो,  
तेव्हा मला सतत गाडी चालवण्याची गरज उरली नाही. आता मी गाडी सोडली आहे.  
यामुळे जागतिक तापमानवाढीत माझा वाटा कमी झाला, आणि चालणे व सायकल चालवण्याने मी तंदुरुस्तही झालो.  

\item सर्वांत महत्त्वाचे म्हणजे, मी “आणखी हवे”, “आणखी चांगले हवे” या अखंड चक्रातून मुक्त झालो.  
मला जाणवले की माझ्याकडे आधीपासूनच सर्व काही आहे.  
आज मी खूप आनंदी आहे.  
\end{itemize}

समाधान मिळवणे हे सहसा एका रात्रीत घडत नाही.  
ते थोड्या थोड्या प्रमाणात घडते.  
आजपासून तुम्ही ह्या काही गोष्टी करू शकता जेणेकरून समाधान अनुभवायला शकाल:

\begin{itemize}
\item आत्ता आपल्या आजूबाजूला बघा, किंवा कदाचित जेव्हा तुम्ही घरी बसलेले असाल.  
जाणून घ्या की तुमच्या आजूबाजूला असलेले सर्व काही तुमच्या आनंदासाठी पुरेसे आहे.  
आनंदासाठी तुम्हाला काय लागतं? अन्न, निवारा, कपडे, इतर माणसे, काहीतरी अर्थपूर्ण काम, आणि समाधानाची वृत्ती.  

\item तुम्हाला अर्थपूर्ण काम करायचे आहे का? नोकरी बदलण्याची गरज नाही —  
इतरांना मदत करा, जशी जमेल तशी. सहकाऱ्यांना यशस्वी होण्यासाठी मदत करा.  
मित्रांना त्यांच्या गरजेच्या वेळी साथ द्या. जवळच्यांबरोबर वेळ घालवा आणि त्यांना प्रोत्साहन द्या.  
गरजूंसाठी स्वयंसेवा करा. आपल्या समाजात लहान बदल घडवा.  

\item तुम्हाला इतर लोकांची गरज आहे का? शेजाऱ्याशी मैत्री करा.  
स्वयंसेवा करा आणि मैत्रीपूर्ण राहा. सहकाऱ्यांबरोबर वेळ घालवा.  
प्रत्येक व्यवहारात विचारशील, सकारात्मक आणि सौम्य रहा.  

\item आपल्या आशीर्वादांची गणना सुरू करा — म्हणजे तुमच्याकडे आभारी होण्यासारख्या सर्व गोष्टी.  

\item जेव्हा तुम्ही स्वतःला “मला आणखी हवे आहे” असे विचार करताना सापडता,  
तेव्हा रोज जे काही आहे त्याचे कौतुक करा.  

\item तुम्ही जे काही करता त्यात अधिक जागरूक व्हा — खाणे, अंघोळ करणे, चालणे, काम करणे, भांडी धुणे, बोलणे, लिहिणे, वाचणे आणि इतरांसोबत वेळ घालवणे.  

\item दररोज थोडा वेळ बसून ध्यान केल्याने तुमची जागरूकता वाढेल.  
\end{itemize}

जेव्हा तुम्ही समाधान शोधता, तेव्हा तुम्हाला जाणवते की खूप थोडीच गोष्ट आवश्यक आहे,  
आणि फारसे काही करण्याची गरज नसते. जीवन सोपे आणि अधिक चांगले होते.


\chapter{यश आणि मान्यता मिळवण्याची गरज सोडा}

आज आपण सर्व अशा जगात जगतो जिथे यशाची व्याख्या 
संपत्ती, घरं, गाड्या, पदव्या, पुरस्कार, नोकरीतील उंच स्थान 
किंवा इतर बाह्य गोष्टींनी केली जाते. 

आपण सतत अशा लोकांकडे पाहतो जे या गोष्टी मिळवतात 
आणि त्यांच्यासारखे होण्याचा प्रयत्न करतो. 
पण खरे सांगायचे तर, यश म्हणजे काय? 
आणि ही व्याख्या कोण ठरवतो? 

जर आपण आपले संपूर्ण आयुष्य इतरांच्या व्याख्येप्रमाणे यश शोधण्यात घालवले, 
तर शेवटी आपल्याला रिकामेपण आणि असमाधान मिळेल. 
कारण जसे आपण एक लक्ष्य साध्य करतो, 
तसे लगेच पुढचे लक्ष्य डोळ्यासमोर उभे राहते. 
आपण कधीच थांबत नाही, कधीच समाधानी राहत नाही. 

यापेक्षा चांगला मार्ग म्हणजे, 
यश आणि मान्यतेच्या गरजेपासून स्वतःला मुक्त करणे. 

\begin{itemize}
\item स्वतःला विचारा: 
मी खरंच काय करू इच्छितो, इतर काय म्हणतील याकडे दुर्लक्ष करून? 

\item जर उद्या मला कोणतीही पदवी, पगार, पुरस्कार किंवा ओळख मिळणार नसती, 
तरी मी कोणते काम आनंदाने करत राहिलो असतो? 

\item तुम्ही करत असलेले काम जर तुमच्या मूल्यांशी आणि आवडीशी जुळते, 
तर तेच तुमचे यश आहे. 
\end{itemize}

आणखी एक अडथळा म्हणजे इतरांच्या मान्यतेची गरज. 
आपण सर्वांना आवडावे, आपल्याला टाळ्या मिळाव्यात, 
हे आपल्याला वाटत राहते. 

माझ्या आयुष्यात बराच काळ मी इतरांच्या मान्यतेवर अवलंबून होतो. 
जर लोकांनी कौतुक केले तर मी आनंदी व्हायचो, 
आणि जर कुणी टीका केली तर दिवस खराब व्हायचा. 
पण जेव्हा मी हळूहळू हे सोडायला शिकलो, 
तेव्हाच मी खऱ्या स्वातंत्र्याचा अनुभव घेतला. 

\begin{itemize}
\item इतरांनी काहीही म्हटलं तरी त्याचा तुमच्या खऱ्या मूल्यांशी काही संबंध नाही. 

\item तुम्ही जे आहात ते पुरेसे आहे. 

\item जर एखाद्या व्यक्तीने तुमच्यावर प्रेम केलं, 
तर ते छान आहे — पण ते आवश्यक नाही. 
जर कुणी टीका केली तरी, त्याने तुम्ही कमी होत नाही. 
\end{itemize}

म्हणूनच: 

\begin{enumerate}
\item स्वतःच्या मूल्यांनुसार यशाची व्याख्या ठरवा. 
\item तुमच्या अंतःकरणात आनंद मिळेल अशी कामं करा. 
\item बाह्य मान्यता किंवा टाळ्या यांच्यावर अवलंबून राहू नका. 
\end{enumerate}

जसेच तुम्ही या गरजा सोडता, 
तसे तुमच्यावरचा ताण नाहीसा होतो, 
आणि तुमच्या आतली खरी शांतता उलगडते.



\chapter{सध्याच्या क्षणात जगा}

आपले मन बहुतेक वेळा दोन ठिकाणी फिरत असते: 
भूतकाळातील आठवणींमध्ये किंवा भविष्याबद्दलच्या काळजीत. 

आपण कधी कुणी काय बोललं, आपण काय चूक केली, 
किंवा कोणत्या गोष्टी अधिक चांगल्या करता आल्या असत्या, 
याबद्दल विचार करतो. 
किंवा मग उद्या काय होईल, पुढील आठवड्यात काय करायचं, 
काही चूक होईल का, लोक काय म्हणतील, 
याची चिंता करत राहतो. 

परिणाम असा की आपण जे एकमेव खरे आहे, 
तो — हा क्षण — पूर्णपणे गमावतो. 

\section*{सध्याचा क्षणच खरा आहे}
भूतकाळ गेला आहे. 
तो आता परत येणार नाही. 
भविष्य अजून आलेलं नाही. 
ते अनिश्चित आहे. 

आपल्याकडे प्रत्यक्षात जे आहे, ते फक्त हा क्षण आहे. 
आणि जर आपण हा क्षण गमावला, तर आपण आयुष्यच गमावतो. 

\section*{जागरूकतेचा सराव}
सध्याच्या क्षणात जगण्यासाठी, 
आपण जागरूकतेचा सराव केला पाहिजे. 
जागरूकता म्हणजे, 
आत्ता जे काही घडत आहे त्याकडे पूर्ण लक्ष देणे. 

\begin{itemize}
\item जेव्हा तुम्ही खात आहात, तेव्हा फक्त खा. 
प्रत्येक घासाची चव, वास, स्पर्श अनुभवून बघा. 
\item जेव्हा तुम्ही चालत आहात, तेव्हा फक्त चला. 
प्रत्येक पाऊल जमिनीवर कसं पडतंय, 
हवेचा स्पर्श कसा आहे, 
शरीर कसं हलतंय — हे जाणून घ्या. 
\item जेव्हा तुम्ही एखाद्याशी बोलत आहात, 
तेव्हा पूर्ण लक्ष त्याच्यावर द्या. 
फोन, टीव्ही किंवा इतर विचार विसरा. 
फक्त त्या व्यक्तीचं बोलणं ऐका. 
\end{itemize}

\section*{भूतकाळ आणि भविष्य सोडा}
हे सोपं नाही. 
आपले मन पुन्हा पुन्हा भूतकाळाकडे किंवा भविष्याकडे पळतं. 
पण प्रत्येक वेळी जेव्हा तुम्ही स्वतःला तसं करताना पकडता, 
तेव्हा शांतपणे श्वासावर लक्ष द्या, 
आणि स्वतःला पुन्हा या क्षणात आणा. 

\section*{ध्यानाचा उपयोग}
दररोज काही मिनिटे शांत बसून श्वासावर लक्ष देणे, 
हे सध्याच्या क्षणात राहण्याचा सर्वोत्तम सराव आहे. 
श्वास आत येतो, बाहेर जातो — 
फक्त त्याकडे लक्ष द्या. 

हळूहळू तुम्हाला कळेल की तुम्ही तुमच्या विचारांचे गुलाम नाही. 
तुम्ही तुमचे विचार पाहू शकता, 
आणि त्यांना सोडून देऊ शकता. 

\section*{आनंदाचा खरा मार्ग}
खरा आनंद हा भूतकाळात किंवा भविष्यकाळात नाही. 
तो इथेच आहे, या क्षणात. 
जर तुम्ही हा क्षण पूर्णपणे अनुभवायला शिकलात, 
तर तुम्हाला जाणवेल की तुम्हाला अजून काही लागणार नाही. 
हा क्षणच पुरेसा आहे. 



\chapter{स्वतःवर विश्वास ठेवा}

आपण अनेकदा स्वतःला कमी लेखतो. 
आपल्याला वाटतं की आपण पुरेसे चांगले नाही, 
इतर लोक आपल्यापेक्षा जास्त सक्षम आहेत, 
त्यांच्याकडे जास्त कौशल्यं, पैसा, संधी किंवा ओळख आहे. 

पण खरी गोष्ट अशी आहे की, 
तुम्ही जसे आहात तसेच पुरेसे आहात. 
तुमच्यात जे काही आवश्यक आहे ते आधीपासूनच आहे. 
तुम्हाला फक्त स्वतःवर विश्वास ठेवण्याची गरज आहे. 

\section*{इतरांवर अवलंबून राहू नका}
आपण बराच वेळ इतरांनी आपल्याबद्दल काय विचार केला, 
यावर अवलंबून राहतो. 
जर त्यांनी कौतुक केलं, तर आपण आनंदी होतो. 
जर त्यांनी टीका केली, तर आपण दुःखी होतो. 

पण जेव्हा आपला आत्मविश्वास इतरांच्या मतांवर अवलंबून असतो, 
तेव्हा आपण कधीही खऱ्या अर्थाने मुक्त राहत नाही. 
आपलं सुख त्यांच्या हातात असतं. 

\section*{स्वतःची किंमत जाणून घ्या}
स्वतःला विचारा: 
जर उद्या जगात कुणीही माझं कौतुक केलं नाही, 
कुणीही मला ओळखलं नाही, 
तरी मी कोणत्या गोष्टी आनंदाने करत राहीन? 

तुमच्या अंतःकरणात तुम्हाला माहिती आहे की तुम्ही काय करू शकता. 
तुम्हाला स्वतःची किंमत ओळखायला हवी. 
आणि ती किंमत इतरांनी ठरवलेली नसते, 
ती तुम्ही स्वतः ठरवता. 

\section*{चुका करण्याची भीती सोडा}
आपण अनेकदा स्वतःवर विश्वास ठेवत नाही कारण 
आपल्याला चुका करण्याची भीती वाटते. 
जर मी अपयशी ठरलो तर काय होईल? 
लोक काय म्हणतील? 

पण सत्य असं आहे की प्रत्येकजण चुका करतो. 
चुका हा शिकण्याचा भाग आहे. 
अपयश हा यशाकडे जाण्याचा एक टप्पा आहे. 

म्हणून स्वतःला चुका करण्याची परवानगी द्या. 
कारण प्रत्येक चूक तुम्हाला शिकवते, 
आणि पुढच्या वेळी जास्त मजबूत बनवते. 

\section*{लहान पावलांनी सुरुवात करा}
स्वतःवर विश्वास वाढवण्यासाठी, 
लहान पावलांनी सुरुवात करा. 
एखादं लहान काम निवडा आणि ते पूर्ण करा. 
तुम्हाला जाणवेल की तुम्ही ते करू शकता. 
हळूहळू आत्मविश्वास वाढेल. 

\section*{निष्कर्ष}
तुमच्यात जे आवश्यक आहे ते आधीपासूनच आहे. 
तुम्ही पुरेसे आहात. 
इतरांच्या मतांवर अवलंबून राहू नका. 
चुका करण्यास घाबरू नका. 
लहान लहान पावलं टाका. 

आणि सर्वांत महत्वाचं म्हणजे: 
स्वतःवर विश्वास ठेवा. 



\chapter{नियंत्रण सोडा}

आपण बराच वेळ आणि उर्जा आपल्या आजूबाजूच्या गोष्टींवर नियंत्रण ठेवण्याचा प्रयत्न करण्यात घालवतो.  
आपल्याला वाटतं की सर्व काही आपल्या मनाप्रमाणे घडलं पाहिजे:  
लोकांनी जसं आपण अपेक्षित करतो तसं वागलं पाहिजे,  
परिस्थिती आपल्या योजनांप्रमाणे घडली पाहिजे.  

पण वास्तव असं आहे की,  
आपल्याला इतर लोकांवर किंवा बाह्य परिस्थितींवर नियंत्रण ठेवता येत नाही.  
आपण कितीही प्रयत्न केला तरी,  
जीवन अनपेक्षित असतं आणि ते नेहमी आपल्या इच्छेनुसार चालत नाही.  

\section*{नियंत्रणाची भ्रामक भावना}
आपण नियंत्रणात आहोत असं वाटतं,  
पण ते खरं नाही.  
जग आपल्या अपेक्षेपेक्षा खूप मोठं आणि गुंतागुंतीचं आहे.  
आपण फक्त आपल्या कृतींवर नियंत्रण ठेवू शकतो,  
बाकी सर्व आपल्या हाताबाहेर आहे.  

\section*{नियंत्रण सोडल्याचे स्वातंत्र्य}
जेव्हा आपण हे समजतो की सर्व काही आपल्या हातात नाही,  
आणि आपण ते स्वीकारतो,  
तेव्हा खरे स्वातंत्र्य मिळते.  

\begin{itemize}
\item जर एखादी परिस्थिती आपल्या अपेक्षेप्रमाणे नसेल, तर रागावण्याऐवजी तिला स्वीकारा.  
\item जर एखाद्याचं वागणं तुमच्या मनासारखं नसेल, तर त्याला बदलण्याचा प्रयत्न करू नका.  
\item फक्त तुमच्या स्वतःच्या प्रतिसादावर लक्ष द्या.  
\end{itemize}

\section*{अहंकार सोडा}
नियंत्रणाची गरज हा अनेकदा अहंकाराचा भाग असतो.  
आपल्याला वाटतं की आपणच योग्य आहोत,  
म्हणून इतरांनी आपल्यासारखं वागलं पाहिजे.  

पण जेव्हा आपण हा अहंकार सोडतो,  
तेव्हा आपण अधिक नम्र, दयाळू आणि समजूतदार होतो.  
आपल्याला जाणवतं की प्रत्येकाचा आपला मार्ग आहे,  
आणि ते ठीक आहे.  

\section*{अनिश्चिततेला मिठी मारा}
जीवन अनिश्चित आहे — आणि तेच त्याचं सौंदर्य आहे.  
जर सर्व काही आपल्या योजनेप्रमाणेच झालं असतं,  
तर जीवन कंटाळवाणं झालं असतं.  

अनिश्चिततेला घाबरण्याऐवजी तिला मिठी मारा.  
तिला एक साहस समजा.  
प्रत्येक दिवस एक नवीन संधी आहे,  
काहीतरी वेगळं, काहीतरी शिकण्याचं.  

\section*{निष्कर्ष}
आपण इतरांवर किंवा जगावर नियंत्रण ठेवू शकत नाही.  
आपण फक्त आपल्या कृतींवर आणि प्रतिसादांवर नियंत्रण ठेवू शकतो.  

म्हणून नियंत्रणाची गरज सोडा.  
अहंकार सोडा.  
जीवनाच्या अनिश्चिततेला स्वीकारा.  

आणि मगच तुम्हाला खरी शांती आणि स्वातंत्र्य मिळेल.  



\chapter{तुमचा दिवस साधा करा}

आजचा जग इतका व्यस्त, गोंधळलेला आणि आवाजाने भरलेला आहे की  
आपल्याला शांत, स्थिर आणि साधेपणाने जगायला शिकावं लागतं.  
आपला दिवस अनावश्यक गोष्टींनी भरलेला असतो:  
निरुपयोगी कामं, असंख्य ईमेल्स,  
सतत वाजणारे फोन,  
आणि सतत येणारे नवनवीन संदेश.  

हे सगळं आपला वेळ आणि उर्जा खातं.  
आपल्याला खरंच जे महत्वाचं आहे त्यासाठी वेळ उरत नाही.  

\section*{गोंधळ ओळखा}
सर्वप्रथम, तुम्ही तुमच्या दिवसातला गोंधळ ओळखा.  
कोणती कामं अनावश्यक आहेत?  
कोणत्या गोष्टी फक्त वेळ वाया घालवतात?  

उदा.:  
\begin{itemize}
\item सतत सोशल मीडियावर स्क्रोल करणं.  
\item दिवसभर ईमेल चेक करत राहणं.  
\item निरर्थक मीटिंग्स.  
\item टीव्ही किंवा इंटरनेटवर तासन्‌तास वेळ घालवणं.  
\end{itemize}

\section*{महत्वाच्या गोष्टींवर लक्ष द्या}
एकदा गोंधळ ओळखला की,  
त्याला कमी करा किंवा पूर्णपणे काढून टाका.  

मग स्वतःला विचारा:  
आजच्या दिवसात खरंच सर्वात महत्वाच्या ३ कामं कोणती आहेत?  

फक्त त्या ३ कामांवर लक्ष केंद्रित करा.  
बाकी सर्व दुय्यम आहे.  

\section*{दिवसाची सुरुवात साधेपणाने करा}
सकाळी उठल्यावर,  
फोन, ईमेल किंवा सोशल मीडियाकडे धावू नका.  
त्याऐवजी थोडा वेळ शांत बसा,  
श्वासावर लक्ष द्या,  
आणि तुमच्या दिवसासाठी साधं नियोजन करा.  

\section*{एकावेळी एकच काम}
आपण अनेकदा मल्टिटास्किंग करण्याचा प्रयत्न करतो.  
पण सत्य असं आहे की मल्टिटास्किंगमुळे आपण थकतो,  
आणि कामं नीट होत नाहीत.  

त्याऐवजी एकावेळी एकच काम करा.  
ते पूर्ण लक्षपूर्वक करा,  
आणि मग पुढच्या कामाकडे वळा.  

\section*{रिकाम्या जागेला महत्व द्या}
आपला दिवस प्रत्येक क्षणाने भरून टाकण्याची गरज नाही.  
थोडी रिकामी जागा ठेवा.  
शांततेसाठी वेळ ठेवा.  
फक्त श्वास घ्या, चालायला जा, किंवा बसून राहा.  

यामुळे तुम्हाला अधिक उर्जा, स्पष्टता आणि आनंद मिळेल.  

\section*{निष्कर्ष}
तुमचा दिवस साधा करा:  
\begin{enumerate}
\item गोंधळ काढून टाका.  
\item महत्वाच्या गोष्टींवर लक्ष द्या.  
\item दिवसाची सुरुवात शांतपणे करा.  
\item एकावेळी एकच काम करा.  
\item रिकाम्या जागेला महत्व द्या.  
\end{enumerate}

जेव्हा तुम्ही साधेपणाने जगता,  
तेव्हा प्रत्येक दिवस शांत, आनंदी आणि समाधानकारक होतो.  



\chapter{नवीन सवयी निर्माण करा}

आपलं आयुष्य आपल्या सवयींनी बनलेलं आहे.  
ज्या सवयी चांगल्या आहेत, त्या आपल्याला यश, आरोग्य आणि आनंद देतात.  
आणि ज्या सवयी वाईट आहेत, त्या आपलं आयुष्य कठीण करतात.  

जर तुम्हाला तुमचं आयुष्य बदलायचं असेल,  
तर पहिला टप्पा म्हणजे — तुमच्या सवयी बदलणं.  

\section*{एकावेळी एक सवय}
आपण अनेकदा एकाच वेळी खूप बदल करण्याचा प्रयत्न करतो.  
उदा.:  
जिमला जाणं, लवकर उठणं,  
आरोग्यदायी खाणं, ध्यान करणं — हे सगळं एकत्र सुरू करतो.  

पण हे फार अवघड होतं.  
आणि लवकरच आपण थकलो जातो आणि हार मानतो.  

त्याऐवजी, एकावेळी फक्त एक सवय घ्या.  
त्या एका सवयीवर ३० दिवस लक्ष केंद्रित करा.  
जेव्हा ती घट्ट रुजते,  
मग पुढची सवय सुरू करा.  

\section*{लहान सुरुवात करा}
मोठ्या बदलांपेक्षा लहान बदल करणं सोपं असतं.  
उदा.:  
\begin{itemize}
\item दररोज ३० मिनिटं व्यायाम करण्याऐवजी, फक्त ५ मिनिटं सुरू करा.  
\item दररोज १० पाने वाचण्याऐवजी, फक्त २ पाने वाचा.  
\item २० मिनिटं ध्यान करण्याऐवजी, फक्त २ मिनिटं बसा.  
\end{itemize}

लहान सवयी टिकवायला सोप्या असतात,  
आणि हळूहळू त्या मोठ्या बदलात वाढतात.  

\section*{स्मरणपत्रं तयार करा}
नवीन सवयी विसरू नयेत म्हणून स्मरणपत्रं ठेवा.  
\begin{itemize}
\item फोनवर अलार्म लावा.  
\item सवयीचं काम रोज एकाच वेळी करा.  
\item आधीपासून असलेल्या सवयीशी नवीन सवय जोडा  
(उदा.: सकाळी दात घासल्यानंतर लगेच २ मिनिटं ध्यान करा).  
\end{itemize}

\section*{सार्वजनिक वचन द्या}
तुमची नवीन सवय कुणाला तरी सांगा.  
मित्र, कुटुंबीय किंवा सोशल मीडियावर जाहीर करा.  
जेव्हा इतरांना माहिती असतं,  
तेव्हा आपण अधिक जबाबदार राहतो.  

\section*{लहान यश साजरे करा}
प्रत्येकवेळी तुम्ही तुमची सवय पूर्ण करता,  
तेव्हा स्वतःचं कौतुक करा.  
छोटीशी सवय पाळली तरी,  
स्वतःला शाबासकी द्या.  

यामुळे उत्साह टिकून राहतो,  
आणि हळूहळू सवय नैसर्गिक बनते.  

\section*{निष्कर्ष}
नवीन सवयी निर्माण करणं अवघड नाही,  
जर तुम्ही हळूहळू आणि सातत्याने प्रयत्न केला तर.  

\begin{enumerate}
\item एकावेळी एकच सवय निवडा.  
\item लहान सुरुवात करा.  
\item स्मरणपत्रं तयार करा.  
\item सार्वजनिक वचन द्या.  
\item लहान यश साजरे करा.  
\end{enumerate}

असं केल्याने,  
हळूहळू तुमच्या सवयी बदलतील,  
आणि त्यासोबत तुमचं संपूर्ण आयुष्य बदलेल.  



\chapter{तुमचा आवडता मार्ग शोधा}

आपण बराच वेळ अशा गोष्टी करत घालवतो  
ज्यात आपल्याला खरी आवड नसते.  
नोकरी, जबाबदाऱ्या, अपेक्षा —  
या सगळ्यामुळे आपण आपल्या मनाला विचारायलाच विसरतो:  
"मला खरंच काय आवडतं?"  

जर तुम्हाला आयुष्यात आनंद आणि समाधान हवं असेल,  
तर तुम्हाला तुमचा खरा आवडता मार्ग शोधावा लागेल.  

\section*{आवडीकडे लक्ष द्या}
स्वतःला विचारा:  
\begin{itemize}
\item कोणती कामं करताना वेळ कसा गेला हे कळत नाही?  
\item कोणत्या गोष्टींबद्दल तुम्ही अखंड बोलू शकता?  
\item लहानपणी तुम्हाला काय करायला आवडायचं?  
\item जर पैशाची किंवा यशाची चिंता नसती,  
तर तुम्ही कोणतं काम निवडलं असतं?  
\end{itemize}

ही प्रश्नं तुमच्या खऱ्या आवडीकडे नेतात.  

\section*{लहान पावलांनी सुरुवात करा}
एकदा तुम्हाला तुमची आवड लक्षात आली की,  
ताबडतोब मोठे बदल करण्याची गरज नाही.  

लहान पावलं टाका:  
\begin{itemize}
\item दररोज ३० मिनिटं तुमच्या आवडीवर काम करा.  
\item शनिवार-रविवार थोडा वेळ त्यासाठी काढा.  
\item मित्र किंवा गट शोधा ज्यांना तीच आवड आहे.  
\end{itemize}

हळूहळू ती आवड तुमच्या आयुष्यात मोठा भाग घेईल.  

\section*{भीतीवर मात करा}
आपल्याला अनेकदा वाटतं:  
"जर मी माझी आवड पूर्ण करण्याचा प्रयत्न केला,  
तर मी अपयशी ठरलो तर?"  

पण भीतीवर मात करणं महत्वाचं आहे.  
कारण खरी भीती अपयशाची नसून,  
कधीच प्रयत्न न करण्याची आहे.  

\section*{प्रवासाचा आनंद घ्या}
आवड शोधणं म्हणजे अंतिम गंतव्य गाठणं नाही.  
तो एक प्रवास आहे.  
कधी कधी आवड बदलते,  
कधी नवे मार्ग उघडतात.  

प्रत्येक पाऊलाचा आनंद घ्या.  
प्रक्रियेला महत्व द्या,  
फक्त परिणामाला नाही.  

\section*{आवड आणि काम}
कधी कधी तुम्हाला तुमची आवड नोकरीत रूपांतरित करता येईल.  
कधी कधी ती फक्त छंद म्हणून राहील.  
दोन्ही ठीक आहेत.  

महत्वाचं म्हणजे तुम्ही ती आवड जगता आहात,  
तिच्यासाठी वेळ काढता आहात,  
आणि ती तुमचं जीवन समृद्ध करत आहे.  

\section*{निष्कर्ष}
तुमचा आवडता मार्ग शोधणं म्हणजे  
तुमचं खरं जीवन शोधणं आहे.  

\begin{enumerate}
\item आवडीकडे लक्ष द्या.  
\item लहान पावलांनी सुरुवात करा.  
\item भीतीवर मात करा.  
\item प्रवासाचा आनंद घ्या.  
\item आवडेला नोकरी किंवा छंद — जसं योग्य वाटेल तसं ठेवा.  
\end{enumerate}

आवडीशिवाय जीवन रिकामं असतं.  
आवड मिळाली की जीवनाला अर्थ आणि आनंद मिळतो.  


\chapter{तुमची उद्दिष्टं साधी करा}

आपल्यापैकी अनेकांच्या आयुष्यात खूप साऱ्या उद्दिष्टांची यादी असते.  
आम्ही एकाच वेळी सगळं साध्य करण्याचा प्रयत्न करतो:  
नवीन करिअर, उत्तम आरोग्य, आर्थिक स्वातंत्र्य,  
नवीन छंद, प्रवास, आणि अजून बरेच काही.  

पण जेव्हा खूप सगळी उद्दिष्टं असतात,  
तेव्हा आपली उर्जा तुकड्यात विभागली जाते.  
आणि शेवटी फारसं काही साध्य होत नाही.  

\section*{कमी म्हणजे जास्त}
यशाचं रहस्य म्हणजे —  
सगळं करण्याचा प्रयत्न न करता,  
फक्त काही महत्वाच्या उद्दिष्टांवर लक्ष केंद्रित करणं.  

३ ते ४ उद्दिष्टं पुरेशी आहेत.  
कदाचित फक्त एक मोठं उद्दिष्ट.  
जेव्हा तुम्ही तुमची उर्जा काही मोजक्या गोष्टींवर देता,  
तेव्हा यशाची शक्यता खूप वाढते.  

\section*{स्पष्टता आणा}
तुमची उद्दिष्टं स्पष्ट करा.  
“मला आरोग्य सुधारायचं आहे” असं म्हणणं पुरेसं नाही.  
त्याऐवजी म्हणा:  
“मी ६ महिन्यात ५ किलो वजन कमी करणार आहे,  
दररोज ३० मिनिटं चालणार आहे.”  

स्पष्ट, मोजता येईल असं उद्दिष्ट असलं की,  
ते साध्य करणं सोपं होतं.  

\section*{लहान टप्प्यांमध्ये विभागा}
मोठं उद्दिष्ट एकदम गाठणं कठीण असतं.  
म्हणून ते लहान टप्प्यांमध्ये विभागा.  

उदा.:  
\begin{itemize}
\item उद्दिष्ट: “१ वर्षात पुस्तक लिहायचं.”  
\item टप्पे:  
\begin{enumerate}
\item पहिल्या महिन्यात अध्यायाची रूपरेषा लिहा.  
\item पुढच्या ३ महिन्यात पहिला मसुदा लिहा.  
\item त्यानंतर संपादन करा.  
\end{enumerate}
\end{itemize}

लहान टप्प्यांमुळे प्रगती दिसते,  
आणि प्रेरणा टिकून राहते.  

\section*{एकावेळी एकच}
जसं सवयींमध्ये आहे,  
तसंच उद्दिष्टांमध्येही —  
एकावेळी एका मोठ्या उद्दिष्टावर लक्ष द्या.  

बाकी सर्व उद्दिष्टं नंतरसाठी ठेवा.  
एक साध्य झालं की, पुढचं घ्या.  

\section*{पुनरावलोकन}
दर आठवड्याला किंवा दर महिन्याला थोडा वेळ काढा.  
तुमच्या उद्दिष्टांकडे पाहा.  
तुम्ही कुठपर्यंत आलात?  
काय बदलण्याची गरज आहे?  

नियमित पुनरावलोकन केल्याने  
उद्दिष्टं मनात ताजी राहतात  
आणि दिशा चुकत नाही.  

\section*{निष्कर्ष}
उद्दिष्टं साधी करणं म्हणजे  
तुमच्या जीवनाला स्पष्ट दिशा देणं आहे.  

\begin{enumerate}
\item खूप उद्दिष्टं न ठेवता फक्त काही महत्वाची निवडा.  
\item ती स्पष्ट आणि मोजता येण्यासारखी करा.  
\item त्यांना लहान टप्प्यांमध्ये विभागा.  
\item एकावेळी एका उद्दिष्टावर लक्ष केंद्रित करा.  
\item नियमित पुनरावलोकन करा.  
\end{enumerate}

असं केल्याने तुम्ही गोंधळातून बाहेर पडाल,  
आणि हळूहळू पण नक्कीच तुमची स्वप्नं पूर्ण कराल.  


\chapter{सोपं आणि विश्वासार्ह तंत्र}

आपल्या डोक्यात सतत विचारांची गर्दी असते:  
करायची कामं, अपॉइंटमेंट्स, कल्पना, योजना,  
आणि अजून बरीच माहिती.  

हे सगळं मनात ठेवण्याचा प्रयत्न केल्यामुळे  
आपण तणावग्रस्त होतो,  
आणि महत्वाची कामं विसरतो.  

यावर उपाय म्हणजे —  
एक साधं आणि विश्वासार्ह तंत्र वापरणं,  
ज्यात आपण सगळं बाहेर लिहून ठेवतो.  

\section*{का आवश्यक आहे?}
जेव्हा तुमच्याकडे एक विश्वासार्ह तंत्र असतं,  
तेव्हा तुम्हाला माहित असतं की  
सगळ्या कामांची, कल्पनांची, जबाबदाऱ्यांची नोंद तिथे आहे.  

म्हणून तुम्ही विसरणार नाही,  
आणि तुमचं मन शांत राहू शकतं.  

\section*{तंत्राची मूलभूत तत्वं}
सोपं आणि विश्वासार्ह तंत्र असण्यासाठी फक्त काही घटक पुरेसे आहेत:  

\begin{enumerate}
\item \textbf{संकलन} —  
प्रत्येक काम, कल्पना किंवा जबाबदारी  
ताबडतोब एका ठिकाणी लिहा.  
(उदा.: छोटा वही, डिजिटल अॅप).  

\item \textbf{प्रक्रिया} —  
नियमितपणे त्या यादीकडे पाहा,  
आणि ठरवा:  
काय लगेच करायचं आहे,  
काय नंतरसाठी ठेवायचं,  
काय दुसऱ्याला द्यायचं,  
काय टाकून द्यायचं.  

\item \textbf{यादी} —  
वेगवेगळ्या प्रकारच्या कामांसाठी साध्या यादी ठेवा:  
\begin{itemize}
\item “करायची कामं” यादी.  
\item “कधीतरी/कदाचित” यादी.  
\item “अपेक्षित” (इतरांकडून) यादी.  
\end{itemize}

\item \textbf{कॅलेंडर} —  
ज्या गोष्टींना विशिष्ट तारीख किंवा वेळ आहे,  
त्या कॅलेंडरमध्ये ठेवा.  

\item \textbf{साप्ताहिक पुनरावलोकन} —  
आठवड्यातून एकदा सगळ्या यादी आणि कॅलेंडर तपासा.  
प्रलंबित कामं अपडेट करा.  
पुढच्या आठवड्याचं नियोजन करा.  
\end{enumerate}

\section*{ते साधं ठेवा}
सिस्टम जटिल करण्याचा मोह होऊ शकतो:  
खूप वेगवेगळ्या अॅप्स, टॅग्स, फोल्डर्स,  
अनेक स्तरांचं नियोजन.  

पण जितकं जास्त जटिल,  
तितकं ते टिकत नाही.  

म्हणून तंत्र साधं ठेवा.  
जास्तीत जास्त कमी साधनं वापरा.  
जेवढं खरंच आवश्यक आहे तेवढंच ठेवा.  

\section*{विश्वास तयार करा}
सिस्टमवर विश्वास ठेवणं सर्वात महत्वाचं आहे.  
जर तुम्हाला खात्री नसेल की कामं योग्यरित्या लिहिली आहेत,  
किंवा यादी वेळेवर तपासली जाते,  
तर तुम्ही पुन्हा डोक्यातच सगळं ठेवायला लागाल.  

म्हणून:  
\begin{itemize}
\item प्रत्येक गोष्ट ताबडतोब सिस्टममध्ये लिहा.  
\item दररोज यादी तपासा.  
\item आठवड्यातून एकदा पुनरावलोकन नक्की करा.  
\end{itemize}

\section*{निष्कर्ष}
एक सोपं आणि विश्वासार्ह तंत्र म्हणजे:  
\begin{enumerate}
\item सगळं बाहेर लिहा.  
\item साध्या याद्या आणि कॅलेंडर वापरा.  
\item नियमित पुनरावलोकन करा.  
\end{enumerate}

असं तंत्र तुमचं मन मोकळं करेल,  
तणाव कमी करेल,  
आणि तुम्हाला खरंच महत्वाच्या गोष्टींवर लक्ष केंद्रित करायला मदत करेल.  


\chapter*{निष्कर्ष}

% \addcontentsline{toc}{chapter}{निष्कर्ष}

\textit{Zen To Done} (ZTD) हे कोणतं जटिल तंत्र नाही.  
ते साधेपणा, एकाग्रता आणि शांततेबद्दल आहे.  

\section*{लहान सुरुवात}
तुम्हाला एकदम सगळ्या १० सवयी अंगीकारायची गरज नाही.  
सुरुवात करा फक्त एका सवयीपासून.  
३० दिवस ती सवय जपा.  
नंतर दुसरी सवय जोडा.  

हळूहळू, थोड्या वेळातच  
तुम्ही संपूर्ण प्रणाली आत्मसात कराल.  

\section*{साधेपणाचं सौंदर्य}
GTD सारखी अनेक प्रणाली उपयुक्त असतात,  
पण कधी कधी जास्त क्लिष्ट वाटतात.  

ZTD तुम्हाला त्याचं सोपं रूप देतं:  
\begin{itemize}
\item कामं संकलित करा.  
\item त्यावर लगेच कृती ठरवा.  
\item महत्वाचं निवडा.  
\item सवयी जोडा.  
\item साधं आणि शांत ठेवा.  
\end{itemize}

\section*{तुमच्या आयुष्यात बदल}
ZTD अवलंबल्यावर काय होतं?  
\begin{itemize}
\item तणाव कमी होतो.  
\item लक्ष केंद्रित राहतं.  
\item महत्वाची कामं पूर्ण होतात.  
\item सवयी हळूहळू आयुष्याचा भाग बनतात.  
\item तुम्ही तुमच्या आवडीप्रमाणे जगू लागता.  
\end{itemize}

\section*{प्रवास चालू ठेवा}
ही एक सततची प्रक्रिया आहे.  
प्रत्येक दिवस नवा सराव आहे.  
कधी चुका होतील, कधी मार्ग चुकाल.  
पण हार मानू नका.  

साधेपणाकडे, शांततेकडे,  
आणि एकाग्रतेकडे परतत राहा.  

\section*{शेवटचा विचार}
\begin{quote}
“हे सर्व तंत्र, सवयी आणि साधनं  
फक्त एक उद्देशासाठी आहेत —  
तुमचं जीवन साधं, स्पष्ट, आणि आनंदी बनवण्यासाठी.”  
\end{quote}

आता तुमच्याकडे साधनं आहेत.  
त्यांचा वापर करा.  
आणि तुमचं जीवन बदला.  


\chapter*{माझे विचार - धडे }

\chapter{मिनिमलिझम — सर्वसाधारण जीवनशैली}

\textbf{अवलोकन :}  
मिनिमलिझम म्हणजे फक्त कमी वस्तू ठेवणं नाही, तर \textit{कमी पण योग्य} गोष्टी निवडून आपल्या उर्जेचं आणि वेळेचं उत्तम नियोजन करणं. Cal Newport, Matt D’Avella आणि “The Minimalists” यांच्या प्रेरणेवर आधारित हा अध्याय तुमचं जीवन साधं, स्पष्ट आणि हलकं बनवण्यासाठी मदत करेल.  

\section*{🌱 मनोवृत्ती आणि तत्त्वज्ञान (Week-Zero)}

\begin{itemize}
  \item स्वतःला विचारा: “मी मिनिमलिस्ट जीवन का जगू इच्छितो?”  
  \item डायरी लिहा: कोणत्या गोष्टींनी खरोखर मूल्य वाढतं? कोणत्या गोष्टी उर्जा शोषतात?  
  \item \textbf{Non-negotiables} ठरवा: Derek Sivers यांचं तत्त्व वापरा — \textit{“HELL YEAH else NO”} — आवश्यक काही निवडा, बाकी काढून टाका.  
  \item उच्चस्तरीय तत्त्वं ठरवा:  
  \begin{itemize}
    \item “Less is more focus.”  
    \item “Quantity पेक्षा Quality.”  
    \item “Clutter = Dis-traction.”  
  \end{itemize}
\end{itemize}

\section*{🧹 अव्यवस्था काढा (Daily Life Declutter)}

\subsection*{(अ) भौतिक वस्तू}
\begin{itemize}
  \item \textbf{Possession = Value?} — जर एखादी वस्तू मूल्य देत नसेल तर ती दान करा, रीसायकल करा किंवा टाका.  
  \item \textbf{30-Day Box Test} — कमी वापरल्या जाणाऱ्या वस्तू एका बॉक्समध्ये ठेवा. ३० दिवसांत न लागल्यास, सोडा.  
  \item \textbf{One-in-One-out Rule} — नवीन वस्तू आणली तर जुनी वस्तू जावीच लागेल.  
\end{itemize}

\subsection*{(ब) डिजिटल अव्यवस्था}
\begin{itemize}
  \item श्रेणीनुसार साफसफाई: ईमेल, फोटो, अॅप्स.  
  \item वर्षभर जुने फोटो/व्हिडिओ — क्लाउडवर आर्काईव्ह करा किंवा ऑफलोड करा.  
  \item Desktop आणि Downloads दर आठवड्याला साफ करा.  
  \item “Digital Spring-Clean” साठी रविवार संध्याकाळी ३० मिनिटं ठेवा.  
\end{itemize}

\section*{📵 स्क्रीन टाइम आणि डिजिटल वापर}

\begin{itemize}
  \item झोपण्याच्या आधी आणि उठल्या उठल्या — १ तास स्क्रीन वापर नाही. त्याऐवजी पुस्तक, डायरी, ध्यान.  
  \item सोशल मीडिया आणि वैयक्तिक ईमेल्स फक्त लॅपटॉपवर — दिवसातून दोनदा, जास्तीत जास्त ३० मिनिटं.  
  \item मोबाईल तपासण्याची वारंवारता नोंदवा. प्रत्येक ३० मिनिटांनी हातात मोबाईल गेला तर तुम्ही अडकलात.  
  \item आठवड्यातून अर्धा दिवस किंवा पूर्ण दिवस स्क्रीन-फास्ट करा. फोन फक्त अत्यावश्यक कॉलसाठी.  
  \item Digital Wellbeing किंवा Forest सारखी अॅप्स वापरा; वेळेची मर्यादा ठेवा.  
\end{itemize}

\section*{🗓️ दिनचर्या आणि जीवन डिझाईन}

\begin{itemize}
  \item दररोज किमान एक चांगली सवय: पुस्तक वाचन, मित्र/कुटुंबाशी संवाद, डायरी, चालणं, एखादं छंद.  
  \item रोज किमान १ तास \textbf{एकटं राहण्याचा वेळ}: फोन न घेता फक्त स्वतःसोबत.  
  \item कधीकधी फिरायला जाताना फोन घरी ठेवा. बेचैनी वाटली तर ती व्यसनाची खूण आहे.  
  \item तुमची उपलब्धता सौम्यपणे कमी करा: मित्र/सहकारी यांना सांगा की फक्त संध्याकाळी कॉल करा, तातडीशिवाय नाही.  
\end{itemize}

\section*{🧠 डिजिटल विश्वासार्ह प्रणाली}

\begin{itemize}
  \item पासवर्ड मॅनेजमेंट: सुरक्षित “crypto-form” वापरा, नियमित अपडेट करा.  
  \item ईमेल व्यवस्थापन: मोबाईलवर फक्त कामाचे ईमेल, तेही दोनदा. बाकी लॅपटॉपवर.  
  \item नोटिफिकेशन्स: फक्त आवश्यक — कॉल्स, मेसेजेस. सोशल अॅप्स बंद.  
\end{itemize}

\section*{📅 आठवड्याचे आणि महिन्याचे रीफ्रेशर्स}

\begin{itemize}
  \item \textbf{शनिवार सकाळी Review} (३०–६० मिनिटं): बॅकअप, साफसफाई, प्रगती टिपा.  
  \item आठवड्याचा Digital Audit: नको असलेले अॅप्स काढा, Downloads साफ करा, वेळेच्या मर्यादा पाहा.  
  \item महिन्याला Values Check: तुमच्या सवयी अजूनही “का?” शी जुळतात का? काही जुन्या सवयी परत येत आहेत का?  
\end{itemize}

\section*{😂 मजेशीर आठवणी}

\begin{itemize}
  \item तुमचं जीवन ३२GB पेनड्राईव्हमध्ये बसलं पाहिजे.  
  \item शनिवार सकाळी Review म्हणजे सोपी तपासणी — तणावाचा कार्यक्रम नाही.  
  \item तुमचं डिजिटल जीवन Netflix सारखं ठेवा: एकच प्लेलिस्ट निवडा आणि त्यावर ठेवा.  
  \item शंका आली तर कल्पना करा — Marie Kondo तुमचा फोन तपासत आहे!  
\end{itemize}


\chapter{आर्थिक मिनिमलिझम}

\textbf{टीप:} मी SEBI-प्रमाणित आर्थिक सल्लागार नाही.  
इथे दिलेलं वाचून तुम्हाला जे पटेल ते घ्या,  
जे पटणार नाही ते सोडा.  

हे प्रकरण तुम्हाला तुमचं \textbf{आर्थिक जीवन साधं, स्पष्ट आणि व्यवस्थित} ठेवायला मदत करेल.  
जास्तीचे खाते, गुंतवणुकींची गर्दी, कर्जाचं ओझं,  
यामुळे आपण नेहमी गोंधळलेले, तणावाखाली असतो.  

\textbf{मिनिमलिझम} म्हणजे अनावश्यक गोष्टी काढून टाकणे,  
आणि आवश्यक गोष्टींवर लक्ष केंद्रीत करणे.  
आर्थिक मिनिमलिझम म्हणजे — \textbf{तुमच्या पैशांचा गोंधळ कमी करणे}.  


\section*{१. खाते सोपं करा}
\begin{enumerate}
\item सर्व बँक खाते, ब्रोकरेज खाते, विमा, क्रेडिट कार्ड,  
गुंतवणूक खाते यांची यादी करा.  
त्यात: खाते क्रमांक, IFSC, नॉमिनी, पासवर्ड (सांकेतिक स्वरूपात),  
सल्लागाराचा संपर्क लिहा.  

\item २–३ आवश्यक बँक खाती ठेवा:  
\begin{itemize}
\item एक सरकारी बँक — विश्वास आणि धोरणांसाठी.  
\item एक खाजगी, तंत्रज्ञान-सज्ज बँक — ऑटो-पे, उच्च व्याजासाठी.  
\item एक स्थानिक/समुदाय बँक — प्रत्यक्ष भेटींसाठी.  
\end{itemize}

\item डुप्लिकेट खाती बंद करा.  
अनेक ब्रोकरेज, अनेक बचत खाते, न वापरलेली कार्ड्स यांची गरज नाही.  

\item 🎯 \textbf{लक्ष्य:} शनिवारी सकाळी २० मिनिटांत तुमची सगळी खाती पाहून पूर्ण आढावा मिळाला पाहिजे.  
\end{enumerate}

\section*{२. गुंतवणूक: कमी पण परिणामकारक}
\begin{enumerate}
\item प्रमाणित किंवा विश्वासार्ह आर्थिक सल्लागारासोबत  
तुमचा \textbf{जोखीम प्रोफाइल} तपासा.  

\item कमी संख्येतील, सोपी आणि ट्रॅक करता येणारी गुंतवणूक ठेवा:  
\begin{itemize}
\item शेअर्स (मर्यादित, थेट मार्केट नको असल्यास म्युच्युअल फंड).  
\item म्युच्युअल फंड (विशेषतः इंडेक्स फंड, ETF).  
\item FDs किंवा सरकारी साधनं.  
\item थोडं रिअल इस्टेट, सोनं.  
\end{itemize}

\item एक \textbf{गुंतवणूक धोरण विधान} लिहा:  
वाटपाचं प्रमाण, केव्हा पुनरावलोकन करायचं (उदा. वर्षातून एकदा),  
केव्हा विकायचं/खरेदी करायचं.  

\item सगळं एका \textbf{माईंडमॅप} किंवा स्प्रेडशीटमध्ये लिहा.  
त्यात: खाते क्रमांक, मॅच्युरिटी डेट्स, नॉमिनी, पासवर्ड (एन्क्रिप्टेड).  
विश्वासार्ह व्यक्तीला एन्क्रिप्टेड कॉपी शेअर करा.  
एक प्रिंट काढून सुरक्षित ठिकाणी ठेवा.  

\item 🎯 \textbf{लक्ष्य:} तुमचं संपूर्ण पोर्टफोलिओ एका ३२ GB पेनड्राईव्हमध्ये बसलं पाहिजे.  
\end{enumerate}

\section*{३. कर्ज आणि आपत्कालीन निधी}
\begin{enumerate}
\item सर्वप्रथम उच्च व्याजाचं कर्ज फेडा.  
विशेषतः क्रेडिट कार्डचं.  

\item \textbf{आपत्कालीन निधी} तयार करा:  
६–१२ महिन्यांचा खर्च बँकेत.  
आय उत्पन्न अनियमित असल्यास, किमान १ वर्षाचा निधी ठेवा.  

\item प्रत्येक पगारातून ठराविक रक्कम ऑटोमॅटिक या निधीत जमा होईल याची सोय करा.  

\item कर्ज फेडून, आपत्कालीन निधी तयार झाल्यावर  
अतिरिक्त रक्कम गुंतवणुकीत वळवा.  

\item 🎯 \textbf{लक्ष्य:} तुमचा नेट वर्थ चार्ट एक साधा, स्थिर वर जाणारा ग्राफ दिसला पाहिजे.  
\end{enumerate}


\section*{४. सुरक्षितता आणि वारसा}
\begin{enumerate}
\item तुमचा आर्थिक माईंडमॅप एन्क्रिप्टेड PDF मध्ये जतन करा.  
विश्वासार्ह व्यक्तीला ईमेलद्वारे एक कॉपी द्या.  

\item एक प्रिंट काढून लॅमिनेट करा आणि लॉकर/सेफमध्ये ठेवा.  

\item त्यात लिहा: पासवर्ड डिक्रिप्ट कसा करायचा,  
बँक शाखांचे संपर्क, नॉमिनी तपशील, सल्लागाराचा पत्ता.  

\item 🎯 \textbf{लक्ष्य:} जर तुम्ही उपलब्ध नसाल,  
तर तुमच्या जवळच्या व्यक्तीने ३० मिनिटांत सगळी माहिती उलगडून  
आर्थिक बाबी हाताळल्या पाहिजेत.  
\end{enumerate}


\section*{५. वार्षिक पुनरावलोकन}
\begin{enumerate}
\item दरवर्षी एक ठराविक आठवड्यात \textbf{फायनान्शियल चेक-अप} करा.  
कर्ज स्थिती, आपत्कालीन निधी, गुंतवणूक वाटप तपासा.  

\item जर वाटप ५% पेक्षा जास्त बदललं असेल तर पुन्हा संतुलन साधा.  

\item पासवर्ड, एन्क्रिप्शन स्कीम अपडेट करा.  
नवीन कॉपी शेअर करा.  

\item न वापरलेली खाती/कार्ड्स बंद करा.  

\item 🎯 \textbf{लक्ष्य:} वार्षिक आढावा एका रविवारच्या ब्रंचसारखा  
आनंदी, जलद, आणि सवयीचा वाटला पाहिजे.  
\end{enumerate}


\section*{६. मनोवृत्ती आणि सवयी}
\begin{enumerate}
\item \textbf{कमी पण चांगलं} हे तत्त्व पाळा.  
नवीन गुंतवणूक घेण्याआधी स्वतःला विचारा:  
यामुळे खरंच मूल्य वाढतंय का?  

\item आत्यंतिक आकर्षक गुंतवणुकींना (shiny objects) बळी पडू नका.  
सोपं, व्यापक आणि ट्रॅक करण्यासारखं निवडा.  

\item शक्य तितकं \textbf{ऑटोमेशन} करा — पेमेन्ट्स, गुंतवणूक, बचत.  

\item तुमचं \textbf{डॅशबोर्ड साधं ठेवा}:  
फक्त नेट वर्थ, आपत्कालीन निधी, वाटप —  
बाकी सर्व आवाज काढून टाका.  

\item 🎯 \textbf{लक्ष्य:} तुमचं मासिक आर्थिक डॅशबोर्ड पाहायला ५ मिनिटं पुरेशी,  
आणि तरीही खात्री द्यायला सक्षम.  
\end{enumerate}


\section*{७. अतिरिक्त मुद्दे}
\begin{itemize}
\item \textbf{बजेट ट्रॅकिंग}:  
दर महिन्याला उत्पन्न विरुद्ध खर्चाचं स्प्रेडशीट ठेवा.  
अनावश्यक खर्च काटून टाका.  

\item \textbf{विमा तपासणी}:  
फक्त आवश्यक विमा ठेवा — जीवन, आरोग्य, महत्वाची मालमत्ता.  
डुप्लिकेट पॉलिसीज रद्द करा.  

\item \textbf{कर नियोजन}:  
सोप्या आणि कमी साधनांचा वापर करा,  
जेणेकरून प्रशासनिक ओझं कमी राहील.  

\item \textbf{डिजिटल डीक्लटर}:  
जाहिरातींचे ईमेल थांबवा.  
फक्त आवश्यक स्टेटमेंट्स ठेवा.  
जुने स्टेटमेंट्स योग्य फोल्डरमध्ये हलवा.  

\item \textbf{सस्टेनेबिलिटी चेक}:  
वापरत नसलेल्या गुंतवणुकी,  
जुन्या निष्क्रिय खाती, क्रेडिट कार्ड्स काढून टाका.  
\end{itemize}


\section*{समारोप}
\begin{quote}
“आर्थिक मिनिमलिझम म्हणजे —  
गोंधळ कमी करून तुमच्या पैशांवर नियंत्रण मिळवणे.  
कमी खाते, सोप्या गुंतवणुका, कर्जमुक्त जीवन,  
आणि विश्वासार्ह दस्तऐवजीकरण.”  
\end{quote}

जेव्हा तुमचं आर्थिक जग साधं होतं,  
तेव्हा तुमचं मन मोकळं होतं.  
आणि खरी स्वातंत्र्याची चव मिळते.  


\chapter{डिजिटल मिनिमलिझम}

\textbf{अवलोकन :}  
आपलं डिजिटल जीवन जितकं साधं आणि स्पष्ट असेल तितकं लक्ष, शांतता आणि खोलवर काम (Deep Work) करण्याची क्षमता वाढते. “Digital Minimalism” हा सराव म्हणजे तंत्रज्ञान सोडून देणे नाही, तर त्याचं योग्य नियोजन आणि उपयोग. चला तर मग एक स्वच्छ, साधं आणि हलकं डिजिटल जीवन उभारण्यासाठी चेकलिस्ट बघूया.  

\section*{१. उद्दिष्टं आणि नियम निश्चित करा}

\begin{itemize}
  \item आधी स्वतःला विचारा: “मी डिजिटल मिनिमलिझम का करतोय?” — जास्त लक्ष, कमी तणाव, अधिक मोकळा वेळ यासाठी का?  
  \item काही मूलभूत नियम लिहा: \textit{हा साधन खरंच आवश्यक आहे का?} नसेल तर काढून टाका.  
  \item मर्यादा ठरवा: उदा. मोबाईलवर सोशल मीडिया अॅप्स नाहीत, ईमेल फक्त लॅपटॉपवर.  
  \item “Digital Sabbath” ठरवा: आठवड्यातून एक दिवस इंटरनेट पूर्णपणे बंद, किंवा किमान जेवणाच्या वेळेस.  
\end{itemize}

\section*{२. डिजिटल ऑडिट : काय आहे ते पाहा आणि वर्गीकृत करा}

\subsection*{(अ) डिव्हाइस आणि स्टोरेज}
\begin{itemize}
  \item लॅपटॉप, मोबाईल, क्लाउड ड्राईव्ह, हार्डडिस्क, पेनड्राईव्ह — सगळं यादीत लिहा.  
  \item “Private” आणि “Public” फाईल्स वेगळ्या ठेवा. महत्वाच्या (खाजगी) फाईल्स ३२GB पेनड्राईव्हमध्ये पुरल्या पाहिजेत.  
  \item जुन्या डुप्लिकेट फाईल्स, ट्युटोरियल्स, निरुपयोगी बॅकअप काढून टाका.  
  \item जुने ड्रायव्हर्स, सॉफ्टवेअर, “ghost files” साफ करा.  
\end{itemize}

\subsection*{(ब) अकाउंट्स आणि अॅप्स}
\begin{itemize}
  \item सगळ्या ऑनलाईन अकाउंट्स, क्लाउड सेवा, सोशल लॉगिन्स लिहून ठेवा.  
  \item न वापरलेले अकाउंट्स बंद करा. नको त्या ईमेल्स/न्यूजलेटरमधून “Unsubscribe” करा.  
  \item मोबाईलवरून सोशल मीडिया अॅप्स काढा (जर अत्यावश्यक नसतील तर).  
  \item अनावश्यक “फॉलो” कमी करा. १५० पेक्षा जास्त लोक/पेजेस फॉलो करू नका.  
\end{itemize}

\section*{३. व्यवस्थित करा आणि अव्यवस्था काढा}

\subsection*{(अ) फाईल्स, डेस्कटॉप आणि क्लाउड}
\begin{itemize}
  \item Downloads, Trash, Desktop रिकामं करा.  
  \item महत्वाच्या फाईल्स ठराविक फोल्डर स्ट्रक्चरमध्ये ठेवा: \textit{Personal, Work, Money} इ.  
  \item फाईल्सची नावं स्पष्ट आणि ठराविक ठेवा.  
  \item क्लाउड आणि बाह्य हार्डड्राईव्हवर बॅकअप ठेवा.  
\end{itemize}

\subsection*{(ब) ईमेल आणि इनबॉक्स}
\begin{itemize}
  \item रोज Inbox शून्य करा: उत्तर द्या, Archive करा किंवा Delete करा.  
  \item ईमेल्सचे फोल्डर्स करा: Personal, Work, Money.  
  \item Auto-Unsubscribe आणि Auto-Clean साठी टूल्स वापरा.  
\end{itemize}

\subsection*{(क) मोबाईल अॅप्स}
\begin{itemize}
  \item न वापरलेले अॅप्स काढा. “कदाचित लागेल” असं वाटणारेही काढा.  
  \item फक्त आवश्यक: ईमेल, बँकिंग, अलार्म, ऑथेन्टिकेटर.  
  \item Notifications बंद करा (फक्त कॉल, मेसेज, कॅलेंडर राहू द्या).  
  \item जुने फोटो, नोट्स, प्लेलिस्ट्स, कॅश डेटा डिलीट करा.  
\end{itemize}

\subsection*{(ड) ब्राउझर आणि बुकमार्क्स}
\begin{itemize}
  \item बुकमार्क्स फोल्डर्समध्ये लावा: Work, Entertainment, View-Later.  
  \item मासिक ब्राउझर हिस्टरी आणि कुकीज साफ करा.  
\end{itemize}

\section*{४. सवयी डिझाईन करा आणि निगा राखा}

\begin{itemize}
  \item Screen Time / Digital Wellbeing टूल्स लावा. दररोजचा वापर मर्यादा ठेवा (उदा. २ तासांपेक्षा कमी).  
  \item एकाग्रतेच्या वेळेस मोबाईल “ग्रे-स्केल” मोडवर ठेवा.  
  \item सोशल मीडिया/मनोरंजन अॅप्स लॅपटॉपवरच वापरा, मोबाईलवर नाही.  
  \item रोज १० मिनिटं ईमेल Inbox पाहा.  
  \item शनिवारी सकाळी : बॅकअप, स्टोरेज क्लीन, अॅप्स डिलीट.  
  \item महिन्याला: अकाउंट्स, सबस्क्रिप्शन्स, बुकमार्क्स तपासा.  
\end{itemize}

\section*{५. डिजिटल डिटॉक्स आणि रीसेट}

\begin{itemize}
  \item ३० दिवसांचा Digital Declutter करा: फक्त आवश्यक अॅप्स ठेवा. नंतर हळूहळू आवश्यक गोष्टी परत आणा.  
  \item ऑफलाईन सवयी लावा: फिरणं, वाचन, डायरी लिहिणं.  
  \item आठवड्यातून एक दिवस पूर्णपणे “नेट-मुक्त” ठेवा.  
\end{itemize}

\section*{६. सुरक्षा आणि गोपनीयता}

\begin{itemize}
  \item पासवर्ड मॅनेजर वापरा.  
  \item दोन-स्टेप ऑथेन्टिकेशन नियमित तपासा.  
  \item जुने अकाउंट्स/सेशन्स बंद करा.  
\end{itemize}

\section*{७. टिकाऊ सीमारेषा}

\begin{itemize}
  \item सकाळी उठल्या उठल्या आणि झोपण्याआधी स्क्रीन वापरू नका.  
  \item संध्याकाळी काम बंद — काम आणि विश्रांती वेगळं ठेवा.  
  \item Deep Work च्या वेळी मोबाईल पूर्णपणे बाजूला ठेवा.  
\end{itemize}

\section*{📋 शनिवार सकाळी “डिजिटल रिफ्रेश” चेकलिस्ट}
\begin{enumerate}
  \item Personal फाईल्सचा क्लाउड आणि ३२GB पेनड्राईव्ह बॅकअप.  
  \item Downloads, Trash, Desktop रिकामं.  
  \item Inbox शून्य करा.  
  \item मोबाईलवरील न वापरलेले अॅप्स काढा.  
  \item ब्राउझर हिस्टरी आणि कुकीज साफ करा.  
  \item Screen Time आकडेवारी तपासा.  
\end{enumerate}

\begin{quote}
\textbf{Tagline :} “तुमचं संपूर्ण डिजिटल जीवन ३२GB पेनड्राईव्ह आणि काही क्लाउड फोल्डर्समध्ये बसलं पाहिजे — त्यापेक्षा जास्त नको.”  
\end{quote}


\chapter*{शेवटी }

\chapter{निष्कर्ष : साधेपणातून समृद्धीकडे}

या पुस्तकाचा प्रवास “Zen to Done” पासून सुरु होऊन डिजिटल मिनिमलिझम आणि सर्वसाधारण मिनिमलिझमपर्यंत आला. प्रत्येक प्रकरणात आपण साधेपणाचं तत्त्व वेगवेगळ्या संदर्भांत पाहिलं: कामकाज, तंत्रज्ञान, वस्तू, वेळ आणि सवयी. आता शेवटी, सर्व शिकलेल्या गोष्टींचा आढावा घेऊया आणि पुढे कसं जायचं यावर काही स्पष्ट दिशा ठरवूया.  

\section*{१. शिकलो ते संक्षेपात}

\subsection*{Zen to Done}  
Zen to Done आपल्याला शिकवतं की आपले दिवस तुकड्या-तुकड्यांमध्ये गेला तरी आपण त्याला शिस्तबद्ध, शांत आणि स्पष्टतेनं भरलेलं बनवू शकतो. थोड्या सवयींवर लक्ष केंद्रित करून, कार्यसूची साधी ठेवून, छोट्या कामांचं नियोजन करून आपण गोंधळातून शांततेकडे जाऊ शकतो.  

\subsection*{Digital Minimalism}  
डिजिटल मिनिमलिझम आपल्याला दाखवतो की मोबाईल, अॅप्स, सोशल मीडिया, आणि क्लाउड सेवांमध्ये आपण किती हरवून जातो. डिजिटल ऑडिट, ईमेल साफसफाई, अॅप डिलीट, स्क्रीन टाइम मर्यादा आणि डिजिटल सब्बाथ यांमुळे आपण तंत्रज्ञानाचा गुलाम न राहता त्याचा उपयोगकर्ता बनतो. “तुमचं डिजिटल जीवन ३२GB पेनड्राईव्हमध्ये बसलं पाहिजे” ही आठवण आपल्या डिजिटल साधेपणाचं प्रतीक आहे.  

\subsection*{Minimalism in General}  
सर्वसाधारण मिनिमलिझम म्हणजे भौतिक आणि मानसिक गोंधळ दोन्ही साफ करणं. कमी वस्तूंमध्ये, कमी स्क्रीनमध्ये, पण जास्त दर्जेदार क्षणांमध्ये आपण खरा आनंद शोधतो. “HELL YEAH else NO” हा फिल्टर आपल्याला दाखवतो की प्रत्येक गोष्ट आवश्यक नसते. जे खरंच मूल्य वाढवतं ते ठेवा; बाकी सोडा.  

\section*{२. मुख्य धडे}

\begin{enumerate}
  \item \textbf{स्पष्टता मिळवा} — आपल्या “का?” ला ओळखा. साधेपणाचं कारण वैयक्तिक असेल: जास्त फोकस, कमी चिंता, अधिक खोल काम.  
  \item \textbf{मर्यादा ठेवा} — तंत्रज्ञान, वस्तू, कामं यांना मर्यादा नसतील तर गोंधळ वाढतो. योग्य नियमांनी साधेपणा टिकतो.  
  \item \textbf{आवश्यक गोष्टींना प्राधान्य द्या} — Quantity पेक्षा Quality. काही निवडक गोष्टी तुमच्या जीवनाचं केंद्र असू द्या.  
  \item \textbf{सवयींवर भर द्या} — मिनिमलिझम ही एकदाच केलेली साफसफाई नाही, तर दररोजची सवय आहे: ईमेल साफ करणं, डेस्कटॉप स्वच्छ ठेवणं, शनिवार सकाळी Review.  
  \item \textbf{अव्यवस्था = उर्जा चोरी} — वस्तू, नोटिफिकेशन्स, अर्धवट कामं यामुळे मन तुटतं. अव्यवस्था काढून टाका म्हणजे मन आणि वेळ दोन्ही मोकळं होतं.  
\end{enumerate}

\section*{३. पुढील टप्पे}

\begin{itemize}
  \item \textbf{लहान सुरुवात करा} — एकदम सर्व काही बदलण्याऐवजी एखाद्या एका क्षेत्रापासून सुरुवात करा. उदा. आठवड्यातून एकदा डिजिटल सब्बाथ.  
  \item \textbf{साप्ताहिक Review} — शनिवार सकाळी ३० मिनिटं काढून फाइल्स, ईमेल, वेळेचा वापर तपासा.  
  \item \textbf{मासिक मूल्य तपासणी} — महिन्याला एकदा स्वतःला विचारा: माझ्या सवयी माझ्या “का?” शी जुळत आहेत का?  
  \item \textbf{सोशल मर्यादा} — मित्र, सहकारी यांना कळवा की तुम्ही ठरावीक वेळेतच उपलब्ध आहात.  
  \item \textbf{सततचा सराव} — मिनिमलिझम म्हणजे गंतव्य नाही, तर प्रवास आहे. दररोज थोडी सुधारणा म्हणजेच खरी प्रगती.  
\end{itemize}

\section*{४. अखेरचं स्मरण}

तुम्ही किती कामं केली, किती वस्तू ठेवल्या किंवा किती ईमेलला उत्तर दिलं, यापेक्षा महत्त्वाचं म्हणजे तुम्ही किती शांत, समाधानकारक आणि लक्ष केंद्रित जीवन जगलंत. मिनिमलिझम आणि Zen to Done एकत्र येऊन सांगतात:  

\begin{quote}
“कमी गोष्टी ठेवा, पण योग्य गोष्टी ठेवा.  
कमी कामं करा, पण योग्य कामं करा.  
कमी विचलन ठेवा, पण अधिक खोल अनुभव घ्या.”  
\end{quote}

म्हणून पुढच्या वेळेस तुम्ही एखादी वस्तू हातात घेतली, मोबाईल अॅप उघडलं किंवा नवं काम स्वीकारलं तर एकच प्रश्न विचारा:  
\textbf{“हे माझ्या जीवनात मूल्य वाढवतं का?”}  

जर उत्तर \textit{हो} असेल तर ठेवा. नसेल तर सोडा.  
याच्यातच तुमचं साधेपण, यश आणि शांतता दडलेली आहे.  
