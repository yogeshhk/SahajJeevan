%%%%%%%%%%%%%%%%%%%%%%%%%%%%%%%%%%%%%%%%%%%%%%%%%%%%%%%%%%%%%%%%%%%%%%%%%%%%%%%%%%%%%%%%%%%%%%%%%
\chapter{नवल रविकांत  यांचे विचार }

% --------------------
\section*{Life, Happiness \& Peace of Mind}
\begin{itemize}
  \item A fit body, a calm mind, a house full of love. These things cannot be bought — they must be earned. \\
  \textit{तंदुरुस्त शरीर, शांत मन आणि प्रेमाने भरलेले घर – हे विकत घेता येत नाहीत, कमवावे लागतात.}

  \item This is such a short and precious life that it’s really important that you don’t spend it being unhappy. \\
  \textit{हे आयुष्य खूप छोटे आणि मौल्यवान आहे, ते दु:खी होऊन वाया घालवू नका.}

  \item Happiness is a choice and a skill and you can dedicate yourself to learning that skill and making that choice. \\
  \textit{आनंद ही एक निवड आणि कौशल्य आहे, ते शिकून तुम्ही तो निवडू शकता.}

  \item The most important trick to be happy is to realize that happiness is a choice that you make and a skill that you develop. \\
  \textit{आनंदी होण्याचा सर्वात मोठा मंत्र म्हणजे आनंद ही तुमची निवड आहे हे ओळखणे.}

  \item Happiness is a state where nothing is missing. \\
  \textit{आनंद म्हणजे जिथे काहीही कमी नाही अशी अवस्था.}

  \item Reality is neutral. Our reactions reflect back and create our world. \\
  \textit{वास्तव तटस्थ आहे. आपली प्रतिक्रिया आपला जग निर्माण करते.}

  \item A rational person can find peace by cultivating indifference to things outside of their control. \\
  \textit{तर्कशुद्ध व्यक्ती नियंत्रणाबाहेरील गोष्टींकडे उदासीनता ठेवून शांतता शोधू शकते.}

  \item The enemy of peace of mind is expectations drilled into you by society and other people. \\
  \textit{मन:शांतीचा शत्रू म्हणजे समाज आणि लोकांनी रुजवलेल्या अपेक्षा.}

  \item Anger is a hot coal that you hold in your hand while waiting to throw it at someone else. \\
  \textit{राग म्हणजे हातात धरलेले जळते निखारे – दुसऱ्यावर फेकायच्या प्रतीक्षेत तुम्ही स्वतः भाजता.}

  \item Caught in a funk? Use meditation, music, and exercise to reset your mood. \\
  \textit{खिन्न वाटतंय? ध्यान, संगीत आणि व्यायामाने मनःस्थिती बदला.}

  \item Be present above all else. \\
  \textit{सर्वप्रथम वर्तमानात रहा.}

  \item Every moment has to be complete in and of itself. \\
  \textit{प्रत्येक क्षण स्वतःमध्ये पूर्ण असावा.}

  \item You can have the mind or you can have the moment. \\
  \textit{तुमच्याकडे मन असू शकतं किंवा क्षण – दोन्ही एकत्र नाही.}

  \item A busy mind accelerates the perceived passage of time. Buy more time by cultivating peace of mind. \\
  \textit{गोंधळलेले मन वेळ जलद जातोय असे वाटते. मन:शांती मिळवा म्हणजे वेळ वाढेल.}

  \item Life hack: when in bed, meditate. Either you will have a deep meditation or fall asleep. Victory either way. \\
  \textit{जीवनमंत्र: पलंगावर झोपताना ध्यान करा – गाढ ध्यान लागेल किंवा झोप लागेल, दोन्ही फायदेशीर.}

  \item Meditation is intermittent fasting for the mind. \\
  \textit{ध्यान म्हणजे मनासाठी आंतरायिक उपवास.}

  \item Enlightenment is the space between your thoughts. \\
  \textit{प्रबोधन म्हणजे विचारांमधील रिक्त जागा.}

  \item You can change it, you can accept it, or you can leave it. What is not a good option is to sit around wishing. \\
  \textit{बदलू शकता, स्वीकारू शकता किंवा सोडू शकता – पण फक्त इच्छा करत बसणे हा वाईट पर्याय आहे.}
\end{itemize}

% --------------------
\section*{Wealth, Work \& Entrepreneurship}
\begin{itemize}
  \item Earn with your mind, not your time. \\
  \textit{तुमच्या वेळेने नव्हे तर बुद्धीने कमवा.}

  \item You’re never going to get rich renting out your time. \\
  \textit{वेळ भाड्याने देऊन कोणी श्रीमंत होत नाही.}

  \item Wealth is assets that earn while you sleep. \\
  \textit{संपत्ती म्हणजे तुम्ही झोपलेले असतानाही कमावणारी साधने.}

  \item You must own equity to gain your financial freedom. \\
  \textit{आर्थिक स्वातंत्र्यासाठी मालकी हक्क (इक्विटी) असणे आवश्यक आहे.}

  \item Apply specific knowledge with leverage and eventually, you will get what you deserve. \\
  \textit{विशेष ज्ञान आणि साधनांचा उपयोग करा – शेवटी योग्य फळ मिळेल.}

  \item Play long-term games with long-term people. \\
  \textit{दीर्घकालीन लोकांबरोबर दीर्घकालीन खेळ खेळा.}

  \item Embrace accountability and take business risks under your own name. Society will reward you with responsibility, equity, and leverage. \\
  \textit{जबाबदारी स्वीकारा आणि स्वतःच्या नावाने धोका घ्या – समाज तुम्हाला मालकी आणि लाभ देईल.}

  \item Forget rich versus poor, white-collar versus blue. It’s now leveraged versus un-leveraged. \\
  \textit{श्रीमंत-गरीब किंवा नोकरी-व्यवसाय नाही, आजचा फरक म्हणजे ज्यांच्याकडे साधने आहेत आणि ज्यांच्याकडे नाहीत.}

  \item Who you do business with is just as important as what you choose to do. \\
  \textit{तुम्ही कोणाबरोबर व्यवसाय करता हे, तुम्ही काय करता याइतकेच महत्त्वाचे आहे.}

  \item If you can’t see yourself working with someone for life, don’t work with them for a day. \\
  \textit{ज्यांच्याशी आयुष्यभर काम करणार नाही असं वाटतं, त्यांच्याबरोबर एक दिवसही काम करू नका.}

  \item Don’t partner with cynics and pessimists; their beliefs are self-fulfilling. \\
  \textit{नकारात्मक लोकांशी भागीदारी करू नका – त्यांचे विश्वास स्वतः सत्य ठरतात.}

  \item Escape competition through authenticity. \\
  \textit{प्रामाणिकपणातून स्पर्धा टाळा.}

  \item Above ``product-market fit'' is ``founder-product-market fit.'' \\
  \textit{``उत्पादन-बाजार जुळणी'' पेक्षा महत्त्वाची म्हणजे ``संस्थापक-उत्पादन-बाजार जुळणी.''}

  \item I would rather be a failed entrepreneur than someone who never tried. \\
  \textit{प्रयत्न न करणाऱ्या व्यक्तीपेक्षा अपयशी उद्योजक असणे पसंत करीन.}

  \item Grind and sweat, toil and bleed, face the abyss. It’s all part of becoming an overnight success. \\
  \textit{कष्ट, संघर्ष आणि वेदना – हे सर्व एका क्षणात मिळालेल्या यशाचा भाग आहेत.}

  \item If you see a get rich quick scheme, that’s someone else trying to get rich off of you. \\
  \textit{जलद श्रीमंतीचे योजनेत दुसरा तुमच्यावरून श्रीमंत होतो.}

  \item Retirement is when you stop sacrificing today for an imaginary tomorrow. \\
  \textit{निवृत्ती म्हणजे काल्पनिक उद्यासाठी आजचा त्याग थांबवणे.}

  \item Forty hour workweeks are a relic of the Industrial Age. Knowledge workers function like athletes — train and sprint, then rest and reassess. \\
  \textit{४० तासांचा आठवडा ही औद्योगिक काळाची परंपरा आहे. ज्ञानकर्मी खेळाडूसारखे – प्रशिक्षण, धाव, मग विश्रांती.}
\end{itemize}

\section*{Work, Wealth \& Success}
\begin{itemize}
    \item ``Forty hour workweeks are a relic of the Industrial Age. Knowledge workers function like athletes — train and sprint, then rest and reassess.'' \\
    \textbf{मराठी:} चाळीस तासांचा कामाचा आठवडा ही औद्योगिक युगाची परंपरा आहे. ज्ञानकामगार खेळाडूप्रमाणे काम करतात — प्रशिक्षण, जोरदार प्रयत्न, मग विश्रांती व पुनर्मूल्यांकन.

    \item ``Earn with your mind, not your time.'' \\
    \textbf{मराठी:} आपल्या वेळेऐवजी आपल्या बुद्धीने कमवा.

    \item ``You’re never going to get rich renting out your time.'' \\
    \textbf{मराठी:} आपला वेळ विकून तुम्ही श्रीमंत होऊ शकत नाही.

    \item ``Wealth is assets that earn while you sleep.'' \\
    \textbf{मराठी:} संपत्ती म्हणजे ती मालमत्ता जी तुम्ही झोपेत असतानाही उत्पन्न देते.

    \item ``You must own equity to gain your financial freedom.'' \\
    \textbf{मराठी:} आर्थिक स्वातंत्र्यासाठी तुम्हाला इक्विटीची मालकी असणे आवश्यक आहे.

    \item ``Play long-term games with long-term people.'' \\
    \textbf{मराठी:} दीर्घकालीन लोकांसोबत दीर्घकालीन खेळ खेळा.

    \item ``Most of the gains in life come from suffering in the short term so you can get paid in the long term.'' \\
    \textbf{मराठी:} आयुष्यातील बहुतांश फायदा हा अल्पकालीन दुःखातून येतो, जेणेकरून दीर्घकाळात त्याचे फळ मिळेल.

    \item ``All the returns in life, whether in wealth, relationships, or knowledge, come from compound interest.'' \\
    \textbf{मराठी:} आयुष्यातील सर्व परतावे — संपत्ती, नातेसंबंध किंवा ज्ञान — हे चक्रवाढ व्याजातून येतात.
\end{itemize}

\section*{Learning \& Knowledge}
\begin{itemize}
    \item ``The most important skill for getting rich is becoming a perpetual learner.'' \\
    \textbf{मराठी:} श्रीमंत होण्यासाठीची सर्वात महत्त्वाची कौशल्य म्हणजे कायम शिकत राहणे.

    \item ``Your most important skill isn’t even what you majored in... it’s just knowing how to learn.'' \\
    \textbf{मराठी:} तुमचे सर्वात महत्त्वाचे कौशल्य म्हणजे तुम्ही काय शिकलात हे नाही, तर कसे शिकायचे हे आहे.

    \item ``Even today, what to study and how to study it are more important than where to study it...'' \\
    \textbf{मराठी:} आजही काय शिकायचे आणि कसे शिकायचे हे कुठे शिकायचे यापेक्षा महत्त्वाचे आहे.

    \item ``Read what you love until you love to read.'' \\
    \textbf{मराठी:} तुम्हाला जे वाचायला आवडते ते वाचा, जोपर्यंत वाचनाची आवड निर्माण होत नाही.

    \item ``It’s better to read a great book slowly than to fly through a hundred books quickly.'' \\
    \textbf{मराठी:} शंभर पुस्तके झपाट्याने वाचण्यापेक्षा एक महान पुस्तक हळूहळू वाचणे चांगले.

    \item ``Knowledge is a skyscraper. You can take a shortcut... or build slowly upon a steel frame of understanding.'' \\
    \textbf{मराठी:} ज्ञान म्हणजे गगनचुंबी इमारत आहे. तुम्ही शॉर्टकटने नाजूक पाया बांधू शकता किंवा हळूहळू समजुतीच्या मजबूत चौकटीवर बांधू शकता.
\end{itemize}

\section*{Habits \& Self-Mastery}
\begin{itemize}
    \item ``Humans are basically habit machines… learning how to break habits is a very important meta skill.'' \\
    \textbf{मराठी:} माणसे म्हणजे सवयींच्या यंत्रांसारखी आहेत... सवयी मोडण्याचे कौशल्य खूप महत्त्वाचे आहे.

    \item ``The power to make and break habits and learning how to do that is really important.'' \\
    \textbf{मराठी:} सवयी लावणे आणि मोडणे तसेच ते कसे करायचे हे शिकणे फार महत्त्वाचे आहे.

    \item ``The enemy of peace of mind is expectations drilled into you by society and other people.'' \\
    \textbf{मराठी:} मनःशांतीचा शत्रू म्हणजे समाज आणि इतरांनी तुमच्यात रुजवलेल्या अपेक्षा.

    \item ``A busy mind accelerates the perceived passage of time. Buy more time by cultivating peace of mind.'' \\
    \textbf{मराठी:} व्यस्त मनामुळे वेळ जलद जातो असे वाटते. मनःशांती वाढवून अधिक वेळ मिळवा.
\end{itemize}

\section*{Philosophy \& Clarity}
\begin{itemize}
    \item ``Happiness is a choice and a skill and you can dedicate yourself to learning that skill and making that choice.'' \\
    \textbf{मराठी:} आनंद हा एक निवड आहे आणि कौशल्य आहे. हे शिकण्यासाठी आणि निवडण्यासाठी तुम्ही स्वतःला समर्पित करू शकता.

    \item ``Desire is a contract that you make with yourself to be unhappy until you get what you want.'' \\
    \textbf{मराठी:} इच्छा ही स्वतःसोबत केलेली अशी करार आहे की जी पूर्ण होईपर्यंत तुम्ही दुःखी राहाल.

    \item ``Be present above all else.'' \\
    \textbf{मराठी:} सर्वात महत्त्वाचे म्हणजे वर्तमानात रहा.

    \item ``Reality is neutral. Our reactions reflect back and create our world.'' \\
    \textbf{मराठी:} वास्तव तटस्थ असते. आपली प्रतिक्रिया परत प्रतिबिंबित होऊन आपला जग तयार करते.

    \item ``A rational person can find peace by cultivating indifference to things outside of their control.'' \\
    \textbf{मराठी:} विवेकी माणूस आपल्या नियंत्रणाबाहेरील गोष्टींकडे उदासीनता ठेवून शांती मिळवू शकतो.
\end{itemize}

\section*{Society \& Politics}
\begin{itemize}
    \item ``Wealth creation is an evolutionarily recent positive-sum game. Status is an old zero-sum game.'' \\
    \textbf{मराठी:} संपत्ती निर्माण करणे ही अलीकडची सकारात्मक बेरीज आहे. प्रतिष्ठा ही जुनी शून्य बेरीज आहे.

    \item ``Politics is sports writ large — pick a side, rally the tribe, exchange stories confirming bias, hurl insults.'' \\
    \textbf{मराठी:} राजकारण म्हणजे मोठ्या प्रमाणावर खेळ आहे — बाजू निवडा, गट जमवा, पूर्वग्रह दृढ करणाऱ्या कथा सांगा, शिवीगाळ करा.

    \item ``Don’t debate people in the media when you can debate them in the marketplace.'' \\
    \textbf{मराठी:} माध्यमांमध्ये वाद घालण्यापेक्षा बाजारपेठेत वाद घाला.

    \item ``Signaling virtue is a vice.'' \\
    \textbf{मराठी:} गुण दाखवण्याचा प्रयत्न करणे ही एक उणीव आहे.
\end{itemize}

\section*{Technology \& Progress}
\begin{itemize}
    \item ``Technology is not only the thing that moves the human race forward, but it’s the only thing that ever has.'' \\
    \textbf{मराठी:} मानवजातीला पुढे नेणारी गोष्ट म्हणजे तंत्रज्ञान आहे आणि तेच एकमेव साधन आहे.

    \item ``The democratization of technology allows anyone to be a creator, entrepreneur, scientist. The future is brighter.'' \\
    \textbf{मराठी:} तंत्रज्ञानाचे लोकशाहीकरण कोणालाही निर्माता, उद्योजक, शास्त्रज्ञ बनण्याची संधी देते. भविष्य उज्ज्वल आहे.

    \item ``The Internet allows you to scale any niche obsession.'' \\
    \textbf{मराठी:} इंटरनेट तुम्हाला कोणत्याही खास आवडीला मोठे करण्याची संधी देते.
\end{itemize}

\section*{Miscellaneous Wisdom}
\begin{itemize}
    \item ``Don’t take yourself so seriously. You’re just a monkey with a plan.'' \\
    \textbf{मराठी:} स्वतःला इतके गंभीरतेने घेऊ नका. तुम्ही फक्त एका योजनेसह माकड आहात.

    \item ``Love is given, not received.'' \\
    \textbf{मराठी:} प्रेम हे दिले जाते, घेतले जात नाही.

    \item ``Anger is a hot coal that you hold in your hand while waiting to throw it at someone else.'' \\
    \textbf{मराठी:} राग म्हणजे एक गरम निखारा आहे जो तुम्ही इतरांवर फेकण्यासाठी स्वतःच्या हातात धरून ठेवता.

    \item ``My 1 repeated learning in life: ‘There Are No Adults.’ Everyone’s making it up as they go along.'' \\
    \textbf{मराठी:} माझे आयुष्यातील एक शिकणे: ‘इथे कोणीही प्रौढ नाही.’ सगळे आपापल्या पद्धतीने शिकत चालले आहेत.
\end{itemize}

