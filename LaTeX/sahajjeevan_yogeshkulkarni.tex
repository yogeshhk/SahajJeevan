%%%%%%%%%%%%%%%%%%%%%%%%%%%%%%%%%%%%%%%%%%%%%%%%%%%%%%%%%%%%%%%%%%%%%%%%%%%
\chapter*{डॉ. योगेश हरिभाऊ कुलकर्णी  यांचे विचार }

%%%%%%%%%%%%%%%%%%%%%%%%%%%%%%%%%%%%%%%%%%%%%%%%%%%%%%%%%%%%%%%%%%%%%%%%%%%
\chapter{मिनिमलिझम — सर्वसाधारण जीवनशैली}

मिनिमलिझम (अल्पता)  म्हणजे केवळ कमी वस्तू ठेवणं नाही. मिनिमलिझम म्हणजे कमी पण योग्य गोष्टी निवडणं. या निवडीमुळे आपण आपल्या उर्जेचं आणि वेळेचं उत्तम नियोजन करू शकतो. कार्ल न्यूपोर्ट आणि ‘द मिनिमलिस्ट’  यांची शिकवण जीवन साधं, स्पष्ट आणि हलकं करण्याचा मार्ग दाखवते. त्याचा आधार घेऊन व स्वत:च्या विचारातून खालील काही मुद्दे मांडत आहे. 

\section*{मनोवृत्ती आणि तत्त्वज्ञान}

मिनिमलिझम सुरू करण्यापूर्वी स्वतःला स्पष्ट प्रश्न विचारा की “मी मिनिमलिस्ट जीवन का जगू इच्छितो?” या प्रश्नाचं उत्तर तुमच्या प्रवासाला दिशा देईल. पाहली गोष्ट करायची म्हणजे यादी करा. कोणत्या गोष्टींमुळे खरोखर मूल्य वाढतं आणि कोणत्या गोष्टी उर्जा शोषतात हे स्पष्ट करा. तुमच्या आयुष्यातील काही गोष्टी नॉन-निगोशिएबल म्हणजेच अनिवार्य ठरवा. डेरेक सिव्हर्स यांचं तत्त्व वापरा: “HELL YEAH else NO”. गमतीत म्हणायचं झालं  तर ‘खाईन तर तुपाशी, नाहीतर उपाशी’. आवश्यक गोष्टी ठेवा. बाकी हळूहळू काढून टाका.

मिनिमलिझमसाठी तत्त्वं ठरवा: 
\begin{itemize}
\item ‘लेस इज मोर’ (कमी म्हणजेच जास्त’) हे तत्त्व अंगीकारा.
\item क्वांटिटी (संख्या) पेक्षा  क्वालिटी (दर्जा) या तत्त्वावर विश्वास ठेवा.
\item क्लटर (पसारा, अव्यवस्था, गोंधळ) म्हणजे डिस्ट्रॅक्शन (लक्ष विचलित होणे) हे नेहमी लक्षात ठेवा.
\end{itemize}

\section*{डिक्लटर (अव्यवस्था कमी करा))}

\subsection*{भौतिक वस्तू}

आपल्या वस्तूंवर प्रश्न विचारायला शिका. “ही वस्तू माझ्यासाठी मूल्य निर्माण करते का?” हा प्रश्न प्रत्येक वस्तूसाठी विचारा. उत्तर “नाही” असेल तर ती वस्तू दान करा, रीसायकल (पुनर्वापर) करा किंवा टाकून द्या.

३० दिवसांचा चाचणी पद्धती वापरा. ज्या वस्तूंचा कमी वापर होतो त्या एका बॉक्समध्ये ठेवा. ३० दिवसांत जर त्या लागल्या नाहीत, तर त्यांना निरोप द्या.

एक साधं नियम पाळा. नवीन वस्तू आणली तर एक जुनी वस्तू घरातून बाहेर जायलाच हवी.

\subsection*{ डिजिटल अव्यवस्था}

डिजिटल अव्यवस्था देखील कमी करा. ईमेल्स, फोटो, ऍप्स यांची नियमित श्रेणीवार साफसफाई करा. एक वर्ष जुने फोटो आणि व्हिडिओ क्लाउडवर आर्काईव्ह करा किंवा बाहेर हलवा. प्रत्येक आठवड्याला कॉम्पुटर वरील डेस्कटॉप आणि डाउनलोडस साफ करा. रविवार संध्याकाळी डिजिटल साफ-सफाईसाठी अर्धा तास ठेवा.

\section*{स्क्रीन टाइम आणि डिजिटल वापर}

डिजिटल वापरासाठी स्पष्ट नियम ठेवा. झोपण्याच्या आधी आणि उठल्या उठल्या एक तास स्क्रीन टाळा. त्या वेळेस पुस्तक वाचा, डायरी लिहा किंवा ध्यान करा.

सोशल मीडिया आणि वैयक्तिक ईमेल फक्त लॅपटॉपवर तपासा. दिवसातून जास्तीत जास्त दोनदा आणि तीस मिनिटांपेक्षा जास्त नाही.

तुम्ही मोबाईल कितीदा हातात घेता याची नोंद ठेवा. प्रत्येक तीस मिनिटांनी हात मोबाईलकडे गेला तर ते व्यसनाचं लक्षण आहे.

आठवड्यातून किमान अर्धा दिवस डिजिटल उपवास, स्क्रीन-फास्ट करा. फोन फक्त अत्यावश्यक कॉलसाठीच वापरा.

‘डिजिटल वेलबीइंग’ सारखी ऍप्स वापरा आणि वेळेची मर्यादा पाळा.

\section*{दिनचर्या आणि जीवन डिझाईन}

दररोज एक चांगली सवय जोडा. पुस्तक वाचन, मित्रांशी संवाद, कुटुंबासोबत वेळ, डायरी लिहिणं, चालणं किंवा छंद जोपासणं या पैकी काही निवडा.

दररोज किमान एक तास स्वतःसोबत वेळ घालवा. फोन न घेता फक्त स्वतःसोबत राहा. कधीकधी बाहेर फिरायला जाताना फोन घरी ठेवा. जर बेचैनी वाटली तर ते डिजिटल व्यसनाचं लक्षण आहे.

तुमची उपलब्धता कमी करा. मित्र आणि सहकारी यांना सांगा की फक्त संध्याकाळी कॉल घ्या. तातडीशिवाय इतर वेळी उपलब्ध राहू नका.

\section*{डिजिटल विश्वासार्ह प्रणाली}

पासवर्डसाठी सुरक्षित पद्धती वापरा. नियमित अपडेट करा. ईमेल व्यवस्थापन साधं ठेवा. मोबाईलवर फक्त कामाचे ईमेल ठेवा आणि तेही दिवसातून दोनदा. बाकीचे ईमेल फक्त लॅपटॉपवर पाहा.

नोटिफिकेशन्स मर्यादित ठेवा. फक्त कॉल आणि मेसेजेस ठेवा. सोशल ऍप्सचे नोटिफिकेशन्स बंद करा.

\section*{आठवड्याचे आणि महिन्याचे रीफ्रेशर्स}

शनिवारी सकाळी ३० ते ६० मिनिटं सर्व प्रणालीची तपासणी (रिव्ह्यू ) करा. त्यात बॅकअप, साफसफाई आणि प्रगतीची नोंद करा.

आठवड्याला डिजिटल ताळेबंद (ऑडिट)  करा. नको असलेले ऍप्स काढा, डाउनलोडस साफ करा आणि वेळेच्या मर्यादा तपासा.

महिन्याला तत्वांची अंमलबजावणी होते आहे की नाही त्याची तपासणी करा. तुमच्या सवयी अजूनही “का” शी जुळतात का हे पाहा. जुन्या वाईट सवयी परत येत आहेत का ते तपासा.

\section*{मजेशीर सल्ले}

तुमचं जीवन ३२ जीबी पेनड्राईव्हमध्ये बसावं इतकं साधं ठेवा. शनिवारी सकाळी ही तपासणी सोपी आणि हलकी वाटली पाहिजे. तुमचं डिजिटल जीवन नेटफ्लिक्स सारखं ठेवा. फक्त एक प्लेलिस्ट निवडा आणि त्यावर ठाम राहा.
शंका आली तर कल्पना करा की ‘मारी कांडो’ तुमचा फोन तपासत आहे.

छान! मग आता आपण पुढील लेख “आर्थिक मिनिमलिझम” बघुयात.


%%%%%%%%%%%%%%%%%%%%%%%%%%%%%%%%%%%%%%%%%%%%%%%%%%%%%%%%%%%%%%%%%%%%%%%%%%%
\chapter{आर्थिक मिनिमलिझम}

महत्वाची सूचना : मी सेबी-प्रमाणित आर्थिक सल्लागार नाही. म्हणून या लेखातील विचार हे केवळ मार्गदर्शनासाठी आहेत. जे तुम्हाला पटेल ते स्वीकारा आणि जे पटणार नाही ते सहज सोडून द्या.

आर्थिक मिनिमलिझम म्हणजे पैशांचं जीवन साधं, स्पष्ट आणि व्यवस्थित करणं. खूप खाती, गुंतवणुकींची गर्दी, जास्तीचं कर्ज या सगळ्यामुळे आपला गोंधळ वाढतो. या गोंधळामुळे आपण नेहमी तणावाखाली जगतो आणि खरी स्वातंत्र्य हरवतं. मिनिमलिझम म्हणजे अनावश्यक गोष्टी काढून टाकणं आणि आवश्यक गोष्टींवर लक्ष केंद्रीत करणं. आर्थिक मिनिमलिझम म्हणजे पैशांच्या व्यवस्थेतला गोंधळ कमी करणं आणि स्पष्टता मिळवणं.



\section*{खात्यांचे सुलभीकरण}

सर्वप्रथम आपल्या सर्व बँक खाती, ब्रोकरेज खाती, विमा पॉलिसीज, क्रेडिट कार्ड्स आणि गुंतवणूक खाती यांची संपूर्ण यादी तयार करा. त्या यादीत खाते क्रमांक, आय-एफ-एस-सी कोड, नॉमिनीची माहिती, पासवर्डचं सांकेतिक स्वरूप आणि आर्थिक सल्लागाराचा संपर्क क्रमांक लिहून ठेवा.

यानंतर आपल्या खात्यांची संख्या कमी करा. फक्त दोन ते तीन बँक खाती ठेवा. एक सरकारी बँक खाते विश्वास आणि धोरणात्मक कारणांसाठी ठेवा. एक खाजगी तंत्रज्ञान-सज्ज बँक खाते ऑटो-पे आणि उच्च व्याज दरांसाठी ठेवा.
आवश्यक असल्यास एक स्थानिक किंवा सहकारी बँक खाते ठेवा, जे प्रत्यक्ष भेटीसाठी उपयोगी पडेल.

जुनी, निष्क्रिय किंवा डुप्लिकेट खाती बंद करा. न वापरलेली क्रेडिट कार्ड्स आणि अनेक ब्रोकरेज खाती यांची गरज नाही. लक्ष्य ठेवा की शनिवारी सकाळी केवळ वीस मिनिटांत तुम्हाला तुमच्या सर्व खात्यांचा पूर्ण आढावा मिळाला पाहिजे.



\section*{गुंतवणूक: कमी पण परिणामकारक}

गुंतवणूक करण्यापूर्वी प्रमाणित आर्थिक सल्लागारासोबत तुमचा जोखीम प्रोफाइल तपासा. तुमच्या उत्पन्न, खर्च आणि मानसिकतेनुसार किती जोखीम स्वीकारू शकता ते जाणून घ्या.

गुंतवणूक कमी ठेवा पण सोपी आणि ट्रॅक करण्यास सोपी ठेवा.थोडेच पण निवडक साधनांमध्ये गुंतवणूक करा.
उदाहरणार्थ, मर्यादित प्रमाणात शेअर्स, म्युच्युअल फंड (विशेषतः इंडेक्स फंड किंवा इ-टी -एफ), फिक्स्ड डिपॉझिट्स किंवा सरकारी साधनं. थोड्या प्रमाणात रिअल इस्टेट किंवा सोनं ठेवा.

आपलं गुंतवणूक धोरण स्पष्टपणे लिहून ठेवा. त्यात तुमचं वाटपाचं प्रमाण, पुनरावलोकनाची वेळ (उदा. वर्षातून एकदा), आणि खरेदी-विक्रीचे निकष स्पष्ट करा.

तुमच्या सगळ्या गुंतवणुकींची माहिती एका माईंडमॅप किंवा स्प्रेडशीटमध्ये लिहा.त्यात खाते क्रमांक, मॅच्युरिटी डेट्स, नॉमिनी आणि पासवर्डची सुरक्षित माहिती समाविष्ट करा.ही माहिती विश्वासार्ह व्यक्तीला एन्क्रिप्टेड स्वरूपात शेअर करा आणि एक प्रिंट काढून सुरक्षित ठिकाणी ठेवा.

तुमचं संपूर्ण पोर्टफोलिओ ३२ जीबी पेनड्राईव्हमध्ये बसलं पाहिजे, हे एक साधं आणि स्पष्ट ध्येय ठेवा.


\section*{कर्ज आणि आपत्कालीन निधी}

आर्थिक स्थैर्यासाठी सर्वप्रथम कर्ज कमी करणं आवश्यक आहे. विशेषतः उच्च व्याज असलेलं कर्ज तातडीने फेडा.
क्रेडिट कार्डचं कर्ज हे सर्वांत घातक असतं, ते प्राधान्याने संपवा.

यानंतर आपत्कालीन निधी तयार करा. सहा ते बारा महिन्यांचा खर्च बँकेत सुरक्षित ठेवा. जर उत्पन्न अनियमित असेल तर किमान एका वर्षाचा निधी ठेवा.

प्रत्येक पगारातून ठराविक रक्कम आपोआप या निधीत जमा होईल अशी व्यवस्था करा.
कर्ज फेडून आणि आपत्कालीन निधी तयार केल्यानंतरच गुंतवणुकीत अतिरिक्त रक्कम वळवा.

लक्ष्य ठेवा की तुमचा नेट वर्थ चार्ट साधा आणि स्थिर वर जाणारा ग्राफ दाखवला पाहिजे.

\section*{सुरक्षितता आणि वारसा}

तुमची संपूर्ण आर्थिक माहिती एक सुरक्षित आणि एन्क्रिप्टेड डॉक्युमेंट मध्ये जतन करा. ही माहिती विश्वासार्ह व्यक्तीला ईमेलद्वारे एक कॉपी पाठवा.एक प्रिंट काढून लॅमिनेट करा आणि लॉकरमध्ये ठेवा.

या दस्तऐवजात पासवर्ड डिक्रिप्ट करण्याची पद्धत, बँक शाखांचे संपर्क, नॉमिनीचे तपशील आणि सल्लागाराचा पत्ता लिहा.

लक्ष्य ठेवा की जर तुम्ही उपलब्ध नसाल तर तुमच्या जवळच्या व्यक्तीला तीस मिनिटांत सर्व माहिती समजली पाहिजे आणि त्याने आर्थिक बाबी हाताळल्या पाहिजेत.

\section*{वार्षिक पुनरावलोकन}

दरवर्षी एक ठराविक आठवडा निवडा आणि त्यात पूर्ण फायनान्शियल चेक-अप करा. कर्जाची स्थिती, आपत्कालीन निधी आणि गुंतवणूक वाटप तपासा.

जर वाटप पाच टक्क्यांपेक्षा जास्त बदललं असेल तर पुन्हा संतुलन साधा. पासवर्ड्स आणि एन्क्रिप्शन योजना अपडेट करा आणि नवीन कॉपी शेअर करा.

न वापरलेली खाती आणि कार्ड्स बंद करा.वार्षिक आढावा एका रविवारच्या ब्रंचसारखा आनंदी, जलद आणि सवयीचा वाटला पाहिजे.



\section*{मनोवृत्ती आणि सवयी}

“कमी पण चांगलं” हे तत्त्व नेहमी पाळा. नवीन गुंतवणूक करण्यापूर्वी स्वतःला विचारा: यामुळे खरंच मूल्य वाढतं का?

अत्यंत आकर्षक पण गुंतागुंतीच्या गुंतवणुकींना बळी पडू नका. सोप्या, व्यापक आणि ट्रॅक करता येण्याजोग्या साधनांवर विश्वास ठेवा.

शक्य तितकी व्यवस्था ऑटोमेशनद्वारे करा. पेमेन्ट्स, गुंतवणूक आणि बचत यांना स्वयंचलित करा.

तुमचं आर्थिक डॅशबोर्ड साधं ठेवा. फक्त नेट वर्थ, आपत्कालीन निधी आणि वाटप या तीन गोष्टी ठेवा.
बाकी सर्व अनावश्यक तपशील वगळा.

लक्ष्य ठेवा की तुमचं मासिक डॅशबोर्ड पाहायला फक्त पाच मिनिटं लागावीत आणि तरीही पूर्ण खात्री मिळावी.



\section*{अतिरिक्त मुद्दे}

दर महिन्याला उत्पन्न आणि खर्चाचं स्प्रेडशीट ठेवा. अनावश्यक खर्च लगेच कमी करा.

विमा तपासणी करा. फक्त जीवन, आरोग्य आणि महत्वाची मालमत्ता यांचेच विमे ठेवा. डुप्लिकेट पॉलिसीज रद्द करा.

कर नियोजन साधं ठेवा. कमी साधनांचा वापर करा, जेणेकरून प्रशासनिक ओझं कमी राहील.

डिजिटल डीक्लटर करा. अनावश्यक जाहिरातींचे ईमेल थांबवा. फक्त आवश्यक स्टेटमेंट्स ठेवा आणि जुनी स्टेटमेंट्स योग्य फोल्डरमध्ये हलवा.

\section*{समारोप}

“आर्थिक मिनिमलिझम म्हणजे गोंधळ कमी करून पैशांवर नियंत्रण मिळवणं. कमी खाते, सोप्या गुंतवणुका, कर्जमुक्त जीवन आणि विश्वासार्ह दस्तऐवजीकरण.”


जेव्हा तुमचं आर्थिक जीवन साधं होतं, तेव्हा तुमचं मन हलकं होतं. आणि खरी स्वातंत्र्याची चव मिळते.


%%%%%%%%%%%%%%%%%%%%%%%%%%%%%%%%%%%%%%%%%%%%%%%%%%%%%%%%%%%%%%%%%%%%%%%%%%%
\chapter{डिजिटल मिनिमलिझम}


डिजिटल जीवन जितकं साधं आणि स्पष्ट असेल तितकं लक्ष केंद्रित होतं. मन अधिक शांत होतं आणि खोलवर काम (डीप वर्क) करण्याची क्षमता वाढते. डिजिटल मिनिमलिझम म्हणजे तंत्रज्ञान पूर्णपणे सोडून देणं नव्हे. त्याचा योग्य नियोजन करून, आवश्यक तितकाच आणि अर्थपूर्ण वापर करणं हाच यामागचा उद्देश आहे.

हा लेख तुम्हाला एक स्वच्छ, साधं आणि हलकं डिजिटल जीवन उभारण्याचा मार्ग दाखवेल.

\section*{उद्दिष्टं आणि नियम निश्चित करा}

सर्वप्रथम स्वतःला प्रश्न विचारा: “मी डिजिटल मिनिमलिझम का करतोय?” हे लक्ष वाढवण्यासाठी आहे का, तणाव कमी करण्यासाठी आहे का, की अधिक मोकळा वेळ मिळवण्यासाठी आहे? कारण स्पष्ट झाल्यावर त्यानुसार तुमचे निर्णय ठाम होतात.

काही मूलभूत नियम लिहून ठेवा. प्रत्येक साधनाबद्दल स्वतःला विचारा: “हे खरंच आवश्यक आहे का?” नसेल तर ते साधन काढून टाका.

मर्यादा ठरवा. उदाहरणार्थ, मोबाईलवर सोशल मीडिया ऍप्स ठेवू नका. ईमेल्स फक्त लॅपटॉपवरच पाहा.

आठवड्यातून एक दिवस “डिजिटल उपवास” ठरवा. त्या दिवशी पूर्णपणे इंटरनेट बंद ठेवा किंवा किमान जेवणाच्या वेळेस स्क्रीन टाळा.



\section*{डिजिटल ऑडिट : काय आहे ते पाहा आणि वर्गीकृत करा}

\subsection*{डिव्हाइस आणि स्टोरेज}

सर्व डिव्हाइस आणि स्टोरेज साधनांची यादी करा. लॅपटॉप, मोबाईल, क्लाउड ड्राईव्ह, हार्डडिस्क, पेनड्राईव्ह या सगळ्याचं दस्तऐवजीकरण करा.

“वैयक्तिक (प्रायव्हेट)” आणि “सार्वजनिक (पब्लिक)” फाईल्स वेगळ्या ठेवा. महत्वाच्या फाईल्स ३२ जीबी पेनड्राईव्हमध्ये सुरक्षित ठेवल्या पाहिजेत.

जुन्या डुप्लिकेट फाईल्स, नको असलेले ट्युटोरियल्स आणि निरुपयोगी बॅकअप काढून टाका. जुने ड्रायव्हर्स, सॉफ्टवेअर आणि छुप्या (घोस्ट) फाइल्स साफ करा.

\subsection*{अकाउंट्स आणि ऍप्स}

तुमची सगळी ऑनलाईन अकाउंट्स, क्लाउड सेवा आणि सोशल लॉगिन्स यादीत लिहा. न वापरलेले अकाउंट्स बंद करा. नको असलेले ईमेल्स आणि न्यूजलेटर्समधून “बाहेर पडा  (अनसबस्क्राईब) करा.

मोबाईलवरून सोशल मीडिया ऍप्स काढून टाका, जर ते अत्यावश्यक नसतील तर. जास्त फॉलो करणे थांबवा.
१०० पेक्षा जास्त लोक किंवा पेजेस फॉलो करू नका.

\section*{व्यवस्थित करा आणि अव्यवस्था काढा}

\subsection*{फाईल्स, डेस्कटॉप आणि क्लाउड}

डाउनलोडस, ट्रॅश आणि डेस्कटॉप रिकामे ठेवा. महत्वाच्या फाईल्स ठराविक फोल्डर स्ट्रक्चरमध्ये ठेवा.
उदाहरणार्थ:  वैयक्तिक (पर्सनल), काम (वर्क), पैसे (मनी). 

फाईल्सची नावं स्पष्ट ठेवा. क्लाउड आणि बाह्य हार्डड्राईव्हवर बॅकअप ठेवा.

\subsection*{ईमेल आणि इनबॉक्स}

दररोज इनबॉक्स शून्य ठेवा. जे ईमेल आवश्यक आहेत त्यांना उत्तर द्या. बाकी अर्काइव्ह किंवा डिलीट करा.

\subsection*{(मोबाईल ऍप्स }

न वापरलेले ऍप्स काढून टाका. “कदाचित लागेल” असं वाटणारेही काढून टाका. फक्त आवश्यक ऍप्स ठेवा. उदा. ईमेल, बँकिंग, अलार्म आणि ऑथेन्टिकेटर.

नोटिफिकेशन्स मर्यादित ठेवा. फक्त कॉल्स, मेसेजेस आणि कॅलेंडर राहू द्या.

जुने फोटो, नोट्स, प्लेलिस्ट्स आणि कॅश डेटा डिलीट करा.

\subsection*{ ब्राउझर आणि बुकमार्क्स}

बुकमार्क्स व्यवस्थित फोल्डर्समध्ये ठेवा. उदा. काम (वर्क), करमणूक (एंटरटेनमेंट), नंतर-बघू (व्ह्यू लेटर).

मासिक ब्राउझर हिस्टरी आणि कुकीज साफ करा.



\section*{सवयी डिझाईन करा आणि निगा राखा}

एकाग्रतेच्या वेळेस मोबाईल “ग्रे-स्केल” मोडवर ठेवा. सोशल मीडिया आणि मनोरंजन ऍप्स फक्त लॅपटॉपवर वापरा.
मोबाईलवर वापरू नका.

दररोज दहा मिनिटं ईमेल इनबॉक्स तपासा. शनिवारी सकाळी बॅकअप घ्या, स्टोरेज क्लीन करा आणि न वापरलेले ऍप्स डिलीट करा.महिन्याला अकाउंट्स, सबस्क्रिप्शन्स आणि बुकमार्क्स तपासा.


\section*{डिजिटल डिटॉक्स आणि रीसेट}

३० दिवसांचा डिजिटल डिक्लटर  करा.त्या काळात फक्त अत्यावश्यक ऍप्स ठेवा.उर्वरित नंतर हळूहळू परत आणा.

ऑफलाईन सवयी लावा.फिरणं, पुस्तक वाचन, डायरी लिहिणं यांसारख्या गोष्टी नियमित करा.

आठवड्यातून एक दिवस पूर्णपणे “नेट-मुक्त” ठेवा.



\section*{सुरक्षा आणि गोपनीयता}

पासवर्ड मॅनेजर वापरा.दोन-स्टेप ऑथेन्टिकेशन नियमित तपासा. जुने अकाउंट्स आणि सेशन्स बंद करा.



\section*{टिकाऊ सीमारेषा}

सकाळी उठल्यावर आणि झोपण्याआधी स्क्रीन वापरू नका. संध्याकाळी ठराविक वेळी काम बंद करा. काम आणि विश्रांती यांच्यात स्पष्ट रेषा ठेवा.

गहन काम (डीप वर्क)करताना मोबाईल पूर्णपणे बाजूला ठेवा.



\section*{शनिवार सकाळी “डिजिटल रिफ्रेश” चेकलिस्ट}

\begin{itemize}
\item  वैयक्तिक (पर्सनल) फाईल्सचा क्लाउड आणि ३२ जीबी  पेनड्राईव्हवर बॅकअप घ्या.
\item  डाउनलोडस, ट्रॅश  आणि डेस्कटॉप रिकामे करा.
\item इनबॉक्स शून्य करा.
\item मोबाईलवरील न वापरलेले ऍप्स काढा.
\item ब्राउझर हिस्टरी आणि कुकीज साफ करा.
\item स्क्रीन टाइम आकडेवारी तपासा.
\end{itemize}


“तुमचं संपूर्ण डिजिटल जीवन ३२ जीबी  पेनड्राईव्ह आणि काही क्लाउड फोल्डर्समध्ये बसलं पाहिजे. त्यापेक्षा जास्त नको.”
