Sahaj Jeevan by Yogesh Kulkarni 

ToDos
- Add my notes, quotes ready ref latex articles etc into articles for sahaj jeevan chapters, from publication repo, various gems , give those chapters names, basically my principles

Content

%%%%%%%%%%%%%%%%%%%%%%%%%%%%%%%%%%%%%%%%%%%%%%%%%%%%%%%%%%%%%%%%%%%%%%%%%%%
\chapter*{डॉ. योगेश हरिभाऊ कुलकर्णी  यांचे विचार }

%%%%%%%%%%%%%%%%%%%%%%%%%%%%%%%%%%%%%%%%%%%%%%%%%%%%%%%%%%%%%%%%%%%%%%%%%%%
\chapter{मिनिमलिझम — सर्वसाधारण जीवनशैली}

मिनिमलिझम (अल्पता)  म्हणजे केवळ कमी वस्तू ठेवणं नाही. मिनिमलिझम म्हणजे कमी पण योग्य गोष्टी निवडणं. या निवडीमुळे आपण आपल्या उर्जेचं आणि वेळेचं उत्तम नियोजन करू शकतो. कार्ल न्यूपोर्ट आणि ‘द मिनिमलिस्ट’  यांची शिकवण जीवन साधं, स्पष्ट आणि हलकं करण्याचा मार्ग दाखवते. त्याचा आधार घेऊन व स्वत:च्या विचारातून खालील काही मुद्दे मांडत आहे. 

\section*{मनोवृत्ती आणि तत्त्वज्ञान}

मिनिमलिझम सुरू करण्यापूर्वी स्वतःला स्पष्ट प्रश्न विचारा की “मी मिनिमलिस्ट जीवन का जगू इच्छितो?” या प्रश्नाचं उत्तर तुमच्या प्रवासाला दिशा देईल. पाहली गोष्ट करायची म्हणजे यादी करा. कोणत्या गोष्टींमुळे खरोखर मूल्य वाढतं आणि कोणत्या गोष्टी उर्जा शोषतात हे स्पष्ट करा. तुमच्या आयुष्यातील काही गोष्टी नॉन-निगोशिएबल म्हणजेच अनिवार्य ठरवा. डेरेक सिव्हर्स यांचं तत्त्व वापरा: “HELL YEAH else NO”. गमतीत म्हणायचं झालं  तर ‘खाईन तर तुपाशी, नाहीतर उपाशी’. आवश्यक गोष्टी ठेवा. बाकी हळूहळू काढून टाका.

मिनिमलिझमसाठी तत्त्वं ठरवा: 
\begin{itemize}
\item ‘लेस इज मोर’ (कमी म्हणजेच जास्त’) हे तत्त्व अंगीकारा.
\item क्वांटिटी (संख्या) पेक्षा  क्वालिटी (दर्जा) या तत्त्वावर विश्वास ठेवा.
\item क्लटर (पसारा, अव्यवस्था, गोंधळ) म्हणजे डिस्ट्रॅक्शन (लक्ष विचलित होणे) हे नेहमी लक्षात ठेवा.
\end{itemize}

\section*{डिक्लटर (अव्यवस्था कमी करा))}

\subsection*{भौतिक वस्तू}

आपल्या वस्तूंवर प्रश्न विचारायला शिका. “ही वस्तू माझ्यासाठी मूल्य निर्माण करते का?” हा प्रश्न प्रत्येक वस्तूसाठी विचारा. उत्तर “नाही” असेल तर ती वस्तू दान करा, रीसायकल (पुनर्वापर) करा किंवा टाकून द्या.

३० दिवसांचा चाचणी पद्धती वापरा. ज्या वस्तूंचा कमी वापर होतो त्या एका बॉक्समध्ये ठेवा. ३० दिवसांत जर त्या लागल्या नाहीत, तर त्यांना निरोप द्या.

एक साधं नियम पाळा. नवीन वस्तू आणली तर एक जुनी वस्तू घरातून बाहेर जायलाच हवी.

\subsection*{ डिजिटल अव्यवस्था}

डिजिटल अव्यवस्था देखील कमी करा. ईमेल्स, फोटो, ऍप्स यांची नियमित श्रेणीवार साफसफाई करा. एक वर्ष जुने फोटो आणि व्हिडिओ क्लाउडवर आर्काईव्ह करा किंवा बाहेर हलवा. प्रत्येक आठवड्याला कॉम्पुटर वरील डेस्कटॉप आणि डाउनलोडस साफ करा. रविवार संध्याकाळी डिजिटल साफ-सफाईसाठी अर्धा तास ठेवा.

\section*{स्क्रीन टाइम आणि डिजिटल वापर}

डिजिटल वापरासाठी स्पष्ट नियम ठेवा. झोपण्याच्या आधी आणि उठल्या उठल्या एक तास स्क्रीन टाळा. त्या वेळेस पुस्तक वाचा, डायरी लिहा किंवा ध्यान करा.

सोशल मीडिया आणि वैयक्तिक ईमेल फक्त लॅपटॉपवर तपासा. दिवसातून जास्तीत जास्त दोनदा आणि तीस मिनिटांपेक्षा जास्त नाही.

तुम्ही मोबाईल कितीदा हातात घेता याची नोंद ठेवा. प्रत्येक तीस मिनिटांनी हात मोबाईलकडे गेला तर ते व्यसनाचं लक्षण आहे.

आठवड्यातून किमान अर्धा दिवस डिजिटल उपवास, स्क्रीन-फास्ट करा. फोन फक्त अत्यावश्यक कॉलसाठीच वापरा.

‘डिजिटल वेलबीइंग’ सारखी ऍप्स वापरा आणि वेळेची मर्यादा पाळा.

\section*{दिनचर्या आणि जीवन डिझाईन}

दररोज एक चांगली सवय जोडा. पुस्तक वाचन, मित्रांशी संवाद, कुटुंबासोबत वेळ, डायरी लिहिणं, चालणं किंवा छंद जोपासणं या पैकी काही निवडा.

दररोज किमान एक तास स्वतःसोबत वेळ घालवा. फोन न घेता फक्त स्वतःसोबत राहा. कधीकधी बाहेर फिरायला जाताना फोन घरी ठेवा. जर बेचैनी वाटली तर ते डिजिटल व्यसनाचं लक्षण आहे.

तुमची उपलब्धता कमी करा. मित्र आणि सहकारी यांना सांगा की फक्त संध्याकाळी कॉल घ्या. तातडीशिवाय इतर वेळी उपलब्ध राहू नका.

\section*{डिजिटल विश्वासार्ह प्रणाली}

पासवर्डसाठी सुरक्षित पद्धती वापरा. नियमित अपडेट करा. ईमेल व्यवस्थापन साधं ठेवा. मोबाईलवर फक्त कामाचे ईमेल ठेवा आणि तेही दिवसातून दोनदा. बाकीचे ईमेल फक्त लॅपटॉपवर पाहा.

नोटिफिकेशन्स मर्यादित ठेवा. फक्त कॉल आणि मेसेजेस ठेवा. सोशल ऍप्सचे नोटिफिकेशन्स बंद करा.

\section*{आठवड्याचे आणि महिन्याचे रीफ्रेशर्स}

शनिवारी सकाळी ३० ते ६० मिनिटं सर्व प्रणालीची तपासणी (रिव्ह्यू ) करा. त्यात बॅकअप, साफसफाई आणि प्रगतीची नोंद करा.

आठवड्याला डिजिटल ताळेबंद (ऑडिट)  करा. नको असलेले ऍप्स काढा, डाउनलोडस साफ करा आणि वेळेच्या मर्यादा तपासा.

महिन्याला तत्वांची अंमलबजावणी होते आहे की नाही त्याची तपासणी करा. तुमच्या सवयी अजूनही “का” शी जुळतात का हे पाहा. जुन्या वाईट सवयी परत येत आहेत का ते तपासा.

\section*{मजेशीर सल्ले}

तुमचं जीवन ३२ जीबी पेनड्राईव्हमध्ये बसावं इतकं साधं ठेवा. शनिवारी सकाळी ही तपासणी सोपी आणि हलकी वाटली पाहिजे. तुमचं डिजिटल जीवन नेटफ्लिक्स सारखं ठेवा. फक्त एक प्लेलिस्ट निवडा आणि त्यावर ठाम राहा.
शंका आली तर कल्पना करा की ‘मारी कांडो’ तुमचा फोन तपासत आहे.

छान! मग आता आपण पुढील लेख “आर्थिक मिनिमलिझम” बघुयात.


%%%%%%%%%%%%%%%%%%%%%%%%%%%%%%%%%%%%%%%%%%%%%%%%%%%%%%%%%%%%%%%%%%%%%%%%%%%
\chapter{आर्थिक मिनिमलिझम}

महत्वाची सूचना : मी सेबी-प्रमाणित आर्थिक सल्लागार नाही. म्हणून या लेखातील विचार हे केवळ मार्गदर्शनासाठी आहेत. जे तुम्हाला पटेल ते स्वीकारा आणि जे पटणार नाही ते सहज सोडून द्या.

आर्थिक मिनिमलिझम म्हणजे पैशांचं जीवन साधं, स्पष्ट आणि व्यवस्थित करणं. खूप खाती, गुंतवणुकींची गर्दी, जास्तीचं कर्ज या सगळ्यामुळे आपला गोंधळ वाढतो. या गोंधळामुळे आपण नेहमी तणावाखाली जगतो आणि खरी स्वातंत्र्य हरवतं. मिनिमलिझम म्हणजे अनावश्यक गोष्टी काढून टाकणं आणि आवश्यक गोष्टींवर लक्ष केंद्रीत करणं. आर्थिक मिनिमलिझम म्हणजे पैशांच्या व्यवस्थेतला गोंधळ कमी करणं आणि स्पष्टता मिळवणं.



\section*{खात्यांचे सुलभीकरण}

सर्वप्रथम आपल्या सर्व बँक खाती, ब्रोकरेज खाती, विमा पॉलिसीज, क्रेडिट कार्ड्स आणि गुंतवणूक खाती यांची संपूर्ण यादी तयार करा. त्या यादीत खाते क्रमांक, आय-एफ-एस-सी कोड, नॉमिनीची माहिती, पासवर्डचं सांकेतिक स्वरूप आणि आर्थिक सल्लागाराचा संपर्क क्रमांक लिहून ठेवा.

यानंतर आपल्या खात्यांची संख्या कमी करा. फक्त दोन ते तीन बँक खाती ठेवा. एक सरकारी बँक खाते विश्वास आणि धोरणात्मक कारणांसाठी ठेवा. एक खाजगी तंत्रज्ञान-सज्ज बँक खाते ऑटो-पे आणि उच्च व्याज दरांसाठी ठेवा.
आवश्यक असल्यास एक स्थानिक किंवा सहकारी बँक खाते ठेवा, जे प्रत्यक्ष भेटीसाठी उपयोगी पडेल.

जुनी, निष्क्रिय किंवा डुप्लिकेट खाती बंद करा. न वापरलेली क्रेडिट कार्ड्स आणि अनेक ब्रोकरेज खाती यांची गरज नाही. लक्ष्य ठेवा की शनिवारी सकाळी केवळ वीस मिनिटांत तुम्हाला तुमच्या सर्व खात्यांचा पूर्ण आढावा मिळाला पाहिजे.



\section*{गुंतवणूक: कमी पण परिणामकारक}

गुंतवणूक करण्यापूर्वी प्रमाणित आर्थिक सल्लागारासोबत तुमचा जोखीम प्रोफाइल तपासा. तुमच्या उत्पन्न, खर्च आणि मानसिकतेनुसार किती जोखीम स्वीकारू शकता ते जाणून घ्या.

गुंतवणूक कमी ठेवा पण सोपी आणि ट्रॅक करण्यास सोपी ठेवा.थोडेच पण निवडक साधनांमध्ये गुंतवणूक करा.
उदाहरणार्थ, मर्यादित प्रमाणात शेअर्स, म्युच्युअल फंड (विशेषतः इंडेक्स फंड किंवा इ-टी -एफ), फिक्स्ड डिपॉझिट्स किंवा सरकारी साधनं. थोड्या प्रमाणात रिअल इस्टेट किंवा सोनं ठेवा.

आपलं गुंतवणूक धोरण स्पष्टपणे लिहून ठेवा. त्यात तुमचं वाटपाचं प्रमाण, पुनरावलोकनाची वेळ (उदा. वर्षातून एकदा), आणि खरेदी-विक्रीचे निकष स्पष्ट करा.

तुमच्या सगळ्या गुंतवणुकींची माहिती एका माईंडमॅप किंवा स्प्रेडशीटमध्ये लिहा.त्यात खाते क्रमांक, मॅच्युरिटी डेट्स, नॉमिनी आणि पासवर्डची सुरक्षित माहिती समाविष्ट करा.ही माहिती विश्वासार्ह व्यक्तीला एन्क्रिप्टेड स्वरूपात शेअर करा आणि एक प्रिंट काढून सुरक्षित ठिकाणी ठेवा.

तुमचं संपूर्ण पोर्टफोलिओ ३२ जीबी पेनड्राईव्हमध्ये बसलं पाहिजे, हे एक साधं आणि स्पष्ट ध्येय ठेवा.


\section*{कर्ज आणि आपत्कालीन निधी}

आर्थिक स्थैर्यासाठी सर्वप्रथम कर्ज कमी करणं आवश्यक आहे. विशेषतः उच्च व्याज असलेलं कर्ज तातडीने फेडा.
क्रेडिट कार्डचं कर्ज हे सर्वांत घातक असतं, ते प्राधान्याने संपवा.

यानंतर आपत्कालीन निधी तयार करा. सहा ते बारा महिन्यांचा खर्च बँकेत सुरक्षित ठेवा. जर उत्पन्न अनियमित असेल तर किमान एका वर्षाचा निधी ठेवा.

प्रत्येक पगारातून ठराविक रक्कम आपोआप या निधीत जमा होईल अशी व्यवस्था करा.
कर्ज फेडून आणि आपत्कालीन निधी तयार केल्यानंतरच गुंतवणुकीत अतिरिक्त रक्कम वळवा.

लक्ष्य ठेवा की तुमचा नेट वर्थ चार्ट साधा आणि स्थिर वर जाणारा ग्राफ दाखवला पाहिजे.

\section*{सुरक्षितता आणि वारसा}

तुमची संपूर्ण आर्थिक माहिती एक सुरक्षित आणि एन्क्रिप्टेड डॉक्युमेंट मध्ये जतन करा. ही माहिती विश्वासार्ह व्यक्तीला ईमेलद्वारे एक कॉपी पाठवा.एक प्रिंट काढून लॅमिनेट करा आणि लॉकरमध्ये ठेवा.

या दस्तऐवजात पासवर्ड डिक्रिप्ट करण्याची पद्धत, बँक शाखांचे संपर्क, नॉमिनीचे तपशील आणि सल्लागाराचा पत्ता लिहा.

लक्ष्य ठेवा की जर तुम्ही उपलब्ध नसाल तर तुमच्या जवळच्या व्यक्तीला तीस मिनिटांत सर्व माहिती समजली पाहिजे आणि त्याने आर्थिक बाबी हाताळल्या पाहिजेत.

\section*{वार्षिक पुनरावलोकन}

दरवर्षी एक ठराविक आठवडा निवडा आणि त्यात पूर्ण फायनान्शियल चेक-अप करा. कर्जाची स्थिती, आपत्कालीन निधी आणि गुंतवणूक वाटप तपासा.

जर वाटप पाच टक्क्यांपेक्षा जास्त बदललं असेल तर पुन्हा संतुलन साधा. पासवर्ड्स आणि एन्क्रिप्शन योजना अपडेट करा आणि नवीन कॉपी शेअर करा.

न वापरलेली खाती आणि कार्ड्स बंद करा.वार्षिक आढावा एका रविवारच्या ब्रंचसारखा आनंदी, जलद आणि सवयीचा वाटला पाहिजे.



\section*{मनोवृत्ती आणि सवयी}

“कमी पण चांगलं” हे तत्त्व नेहमी पाळा. नवीन गुंतवणूक करण्यापूर्वी स्वतःला विचारा: यामुळे खरंच मूल्य वाढतं का?

अत्यंत आकर्षक पण गुंतागुंतीच्या गुंतवणुकींना बळी पडू नका. सोप्या, व्यापक आणि ट्रॅक करता येण्याजोग्या साधनांवर विश्वास ठेवा.

शक्य तितकी व्यवस्था ऑटोमेशनद्वारे करा. पेमेन्ट्स, गुंतवणूक आणि बचत यांना स्वयंचलित करा.

तुमचं आर्थिक डॅशबोर्ड साधं ठेवा. फक्त नेट वर्थ, आपत्कालीन निधी आणि वाटप या तीन गोष्टी ठेवा.
बाकी सर्व अनावश्यक तपशील वगळा.

लक्ष्य ठेवा की तुमचं मासिक डॅशबोर्ड पाहायला फक्त पाच मिनिटं लागावीत आणि तरीही पूर्ण खात्री मिळावी.



\section*{अतिरिक्त मुद्दे}

दर महिन्याला उत्पन्न आणि खर्चाचं स्प्रेडशीट ठेवा. अनावश्यक खर्च लगेच कमी करा.

विमा तपासणी करा. फक्त जीवन, आरोग्य आणि महत्वाची मालमत्ता यांचेच विमे ठेवा. डुप्लिकेट पॉलिसीज रद्द करा.

कर नियोजन साधं ठेवा. कमी साधनांचा वापर करा, जेणेकरून प्रशासनिक ओझं कमी राहील.

डिजिटल डीक्लटर करा. अनावश्यक जाहिरातींचे ईमेल थांबवा. फक्त आवश्यक स्टेटमेंट्स ठेवा आणि जुनी स्टेटमेंट्स योग्य फोल्डरमध्ये हलवा.

\section*{समारोप}

“आर्थिक मिनिमलिझम म्हणजे गोंधळ कमी करून पैशांवर नियंत्रण मिळवणं. कमी खाते, सोप्या गुंतवणुका, कर्जमुक्त जीवन आणि विश्वासार्ह दस्तऐवजीकरण.”


जेव्हा तुमचं आर्थिक जीवन साधं होतं, तेव्हा तुमचं मन हलकं होतं. आणि खरी स्वातंत्र्याची चव मिळते.


%%%%%%%%%%%%%%%%%%%%%%%%%%%%%%%%%%%%%%%%%%%%%%%%%%%%%%%%%%%%%%%%%%%%%%%%%%%
\chapter{डिजिटल मिनिमलिझम}


डिजिटल जीवन जितकं साधं आणि स्पष्ट असेल तितकं लक्ष केंद्रित होतं. मन अधिक शांत होतं आणि खोलवर काम (डीप वर्क) करण्याची क्षमता वाढते. डिजिटल मिनिमलिझम म्हणजे तंत्रज्ञान पूर्णपणे सोडून देणं नव्हे. त्याचा योग्य नियोजन करून, आवश्यक तितकाच आणि अर्थपूर्ण वापर करणं हाच यामागचा उद्देश आहे.

हा लेख तुम्हाला एक स्वच्छ, साधं आणि हलकं डिजिटल जीवन उभारण्याचा मार्ग दाखवेल.

\section*{उद्दिष्टं आणि नियम निश्चित करा}

सर्वप्रथम स्वतःला प्रश्न विचारा: “मी डिजिटल मिनिमलिझम का करतोय?” हे लक्ष वाढवण्यासाठी आहे का, तणाव कमी करण्यासाठी आहे का, की अधिक मोकळा वेळ मिळवण्यासाठी आहे? कारण स्पष्ट झाल्यावर त्यानुसार तुमचे निर्णय ठाम होतात.

काही मूलभूत नियम लिहून ठेवा. प्रत्येक साधनाबद्दल स्वतःला विचारा: “हे खरंच आवश्यक आहे का?” नसेल तर ते साधन काढून टाका.

मर्यादा ठरवा. उदाहरणार्थ, मोबाईलवर सोशल मीडिया ऍप्स ठेवू नका. ईमेल्स फक्त लॅपटॉपवरच पाहा.

आठवड्यातून एक दिवस “डिजिटल उपवास” ठरवा. त्या दिवशी पूर्णपणे इंटरनेट बंद ठेवा किंवा किमान जेवणाच्या वेळेस स्क्रीन टाळा.



\section*{डिजिटल ऑडिट : काय आहे ते पाहा आणि वर्गीकृत करा}

\subsection*{डिव्हाइस आणि स्टोरेज}

सर्व डिव्हाइस आणि स्टोरेज साधनांची यादी करा. लॅपटॉप, मोबाईल, क्लाउड ड्राईव्ह, हार्डडिस्क, पेनड्राईव्ह या सगळ्याचं दस्तऐवजीकरण करा.

“वैयक्तिक (प्रायव्हेट)” आणि “सार्वजनिक (पब्लिक)” फाईल्स वेगळ्या ठेवा. महत्वाच्या फाईल्स ३२ जीबी पेनड्राईव्हमध्ये सुरक्षित ठेवल्या पाहिजेत.

जुन्या डुप्लिकेट फाईल्स, नको असलेले ट्युटोरियल्स आणि निरुपयोगी बॅकअप काढून टाका. जुने ड्रायव्हर्स, सॉफ्टवेअर आणि छुप्या (घोस्ट) फाइल्स साफ करा.

\subsection*{अकाउंट्स आणि ऍप्स}

तुमची सगळी ऑनलाईन अकाउंट्स, क्लाउड सेवा आणि सोशल लॉगिन्स यादीत लिहा. न वापरलेले अकाउंट्स बंद करा. नको असलेले ईमेल्स आणि न्यूजलेटर्समधून “बाहेर पडा  (अनसबस्क्राईब) करा.

मोबाईलवरून सोशल मीडिया ऍप्स काढून टाका, जर ते अत्यावश्यक नसतील तर. जास्त फॉलो करणे थांबवा.
१०० पेक्षा जास्त लोक किंवा पेजेस फॉलो करू नका.

\section*{व्यवस्थित करा आणि अव्यवस्था काढा}

\subsection*{फाईल्स, डेस्कटॉप आणि क्लाउड}

डाउनलोडस, ट्रॅश आणि डेस्कटॉप रिकामे ठेवा. महत्वाच्या फाईल्स ठराविक फोल्डर स्ट्रक्चरमध्ये ठेवा.
उदाहरणार्थ:  वैयक्तिक (पर्सनल), काम (वर्क), पैसे (मनी). 

फाईल्सची नावं स्पष्ट ठेवा. क्लाउड आणि बाह्य हार्डड्राईव्हवर बॅकअप ठेवा.

\subsection*{ईमेल आणि इनबॉक्स}

दररोज इनबॉक्स शून्य ठेवा. जे ईमेल आवश्यक आहेत त्यांना उत्तर द्या. बाकी अर्काइव्ह किंवा डिलीट करा.

\subsection*{(मोबाईल ऍप्स }

न वापरलेले ऍप्स काढून टाका. “कदाचित लागेल” असं वाटणारेही काढून टाका. फक्त आवश्यक ऍप्स ठेवा. उदा. ईमेल, बँकिंग, अलार्म आणि ऑथेन्टिकेटर.

नोटिफिकेशन्स मर्यादित ठेवा. फक्त कॉल्स, मेसेजेस आणि कॅलेंडर राहू द्या.

जुने फोटो, नोट्स, प्लेलिस्ट्स आणि कॅश डेटा डिलीट करा.

\subsection*{ ब्राउझर आणि बुकमार्क्स}

बुकमार्क्स व्यवस्थित फोल्डर्समध्ये ठेवा. उदा. काम (वर्क), करमणूक (एंटरटेनमेंट), नंतर-बघू (व्ह्यू लेटर).

मासिक ब्राउझर हिस्टरी आणि कुकीज साफ करा.



\section*{सवयी डिझाईन करा आणि निगा राखा}

एकाग्रतेच्या वेळेस मोबाईल “ग्रे-स्केल” मोडवर ठेवा. सोशल मीडिया आणि मनोरंजन ऍप्स फक्त लॅपटॉपवर वापरा.
मोबाईलवर वापरू नका.

दररोज दहा मिनिटं ईमेल इनबॉक्स तपासा. शनिवारी सकाळी बॅकअप घ्या, स्टोरेज क्लीन करा आणि न वापरलेले ऍप्स डिलीट करा.महिन्याला अकाउंट्स, सबस्क्रिप्शन्स आणि बुकमार्क्स तपासा.


\section*{डिजिटल डिटॉक्स आणि रीसेट}

३० दिवसांचा डिजिटल डिक्लटर  करा.त्या काळात फक्त अत्यावश्यक ऍप्स ठेवा.उर्वरित नंतर हळूहळू परत आणा.

ऑफलाईन सवयी लावा.फिरणं, पुस्तक वाचन, डायरी लिहिणं यांसारख्या गोष्टी नियमित करा.

आठवड्यातून एक दिवस पूर्णपणे “नेट-मुक्त” ठेवा.



\section*{सुरक्षा आणि गोपनीयता}

पासवर्ड मॅनेजर वापरा.दोन-स्टेप ऑथेन्टिकेशन नियमित तपासा. जुने अकाउंट्स आणि सेशन्स बंद करा.



\section*{टिकाऊ सीमारेषा}

सकाळी उठल्यावर आणि झोपण्याआधी स्क्रीन वापरू नका. संध्याकाळी ठराविक वेळी काम बंद करा. काम आणि विश्रांती यांच्यात स्पष्ट रेषा ठेवा.

गहन काम (डीप वर्क)करताना मोबाईल पूर्णपणे बाजूला ठेवा.



\section*{शनिवार सकाळी “डिजिटल रिफ्रेश” चेकलिस्ट}

\begin{itemize}
\item  वैयक्तिक (पर्सनल) फाईल्सचा क्लाउड आणि ३२ जीबी  पेनड्राईव्हवर बॅकअप घ्या.
\item  डाउनलोडस, ट्रॅश  आणि डेस्कटॉप रिकामे करा.
\item इनबॉक्स शून्य करा.
\item मोबाईलवरील न वापरलेले ऍप्स काढा.
\item ब्राउझर हिस्टरी आणि कुकीज साफ करा.
\item स्क्रीन टाइम आकडेवारी तपासा.
\end{itemize}


“तुमचं संपूर्ण डिजिटल जीवन ३२ जीबी  पेनड्राईव्ह आणि काही क्लाउड फोल्डर्समध्ये बसलं पाहिजे. त्यापेक्षा जास्त नको.”

%%%%%%%%%%%%%%%%%%%%%%%%%%%%%%%%%%%%%%%%%%%%%%%%%%%%%%%%%%%%%%%%%%%%%%%%%%%
\chapter{कृत्रिम बुद्धिमत्ता आणि स्वयंचलन : काही विचार}
आजच्या युगात खऱ्या कृत्रिम बुद्धिमत्तेचे लक्षण म्हणजे शक्य तितक्या प्रक्रिया स्वयंचलित करणे. मानवी हस्तक्षेपावर कमी अवलंबून राहणे आणि मशीन स्वतः निर्णय घेऊ शकतील अशा पातळीवर जाणे, ही खरी क्रांती आहे. साध्या चैटबॉटपासून ते नैसर्गिक भाषा प्रक्रिया (NLP) प्रणालीपर्यंत, अशा साधनांमध्ये आपल्याला खऱ्या एआयचे दर्शन घडते.
एआय किंवा मशीन लर्निंग म्हणजे केवळ काही अल्गोरिदम लिहिणे नाही. या अल्गोरिदमना चालना मिळण्यासाठी प्रचंड प्रमाणात व्यवस्थित माहिती आवश्यक असते. त्यामुळे माहिती संकलन, तिचे लेबलिंग आणि काटेकोरपणे वर्गीकरण करणे ही तितकीच महत्त्वाची पायरी ठरते. अनेक वेळा या प्रक्रियेत मानवी हस्तक्षेप आवश्यक असतो. यालाच (Human in the Loop) असे म्हटले जाते.
व्यवसायांचे भवितव्य आता कृत्रिम बुद्धिमत्तेकडे सरकत आहे. अनेक कंपन्या आपले कामकाज अधिक स्वयंचलित करत आहेत. मानवी निर्णयांपेक्षा प्रणाली स्वतः निर्णय घेऊन पुढे जाणार आहेत. यामुळे कार्यक्षमता वाढते आणि विस्तारक्षमता मिळते.
वैद्यकीय क्षेत्रात याचे स्पष्ट दर्शन घडते. एखाद्या डॉक्टरसाठी दरवर्षी प्रकाशित होणाऱ्या पाच हजार संशोधनपत्रांचा अभ्यास करणे अशक्य आहे. हजारो घटकांतील परस्परसंबंध लक्षात ठेवणेही त्याच्या क्षमतेच्या पलीकडे आहे. अशा वेळी एआय मदतीला धावून येते. प्रचंड डेटामधून योग्य निष्कर्ष काढणे हीच त्याची ताकद आहे. कायदा किंवा न्यायालयीन मदतीसारख्या क्षेत्रातही हाच उपयोग होतो.
कृत्रिम बुद्धिमत्तेचा खरा अर्थ म्हणजे व्यवसाय. जो कोणी त्याचा योग्य वापर करतो, तो आर्थिकदृष्ट्या प्रगती करतो. व्यवस्थित आयोजन करणे, प्रक्रिया स्वयंचलित करणे आणि त्या प्रणाली संस्थात्मक करणे हे यशाचे सूत्र आहे.
स्वयंचलनाचा आणखी एक महत्त्वाचा पैलू म्हणजे गुंतवणुकीवरील परतावा. माणूस कितीही मेहनत घेतला तरी उपलब्ध माहितीच्या महासागराशी स्पर्धा करू शकत नाही. परंतु एकदा स्वयंचलन झाले की माहितीची प्रक्रिया जलदगतीने आणि कार्यक्षमतेने होते. त्यामुळे परतावा मोठ्या प्रमाणात वाढतो.
अशा प्रकारे एआय म्हणजे केवळ तंत्रज्ञान नसून, तो आजच्या युगाचा आर्थिक आणि सामाजिक परिवर्तनाचा पाया ठरत आहे.

%%%%%%%%%%%%%%%%%%%%%%%%%%%%%%%%%%%%%%%%%%%%%%%%%%%%%%%%%%%%%%%%%%%%%%%%%%%
\chapter{अंगीकारण्यासाठीची तत्वे}

``यशाच्या मागे धावायचे नाही, तर अद्भूतता मिळवायची आहे.'' (``Not looking for success, but for Wonder!!'')

\section*{लक्ष्य}
मनुष्याने आपल्या जीवनातील उद्दिष्टे निश्चित करताना केवळ धावपळ किंवा पदव्या मिळवण्याचा विचार न करता, शांतता, आनंद आणि दृढता या गुणांचा अंगीकार करावा. समस्या आल्या की त्यांचा निराकरण करण्याची तयारी ठेवली पाहिजे. प्रत्येक कृतीत वेगळेपणा असावा, ती कृती केवळ योग्यच नाही तर सुंदर देखील असावी. जे काही करायचे ते आंतरराष्ट्रीय दर्जाचे असावे, ज्यासाठी आधीच चोख तयारी आवश्यक आहे. सतत सत्याचा शोध घ्यावा, घाईगडबडीत नव्हे तर हळूहळू, खोलवर विचार करून.

\section*{कोऽहम?}
आपण कोण आहोत याचे आत्मपरीक्षण आवश्यक आहे. व्यक्तिमत्वाच्या दृष्टीने मी एक ``INFJ Advocate’’ म्हणजेच अंतर्मुख (इन्ट्रोव्हर्ट), मनकवडा (इंट्यूटिव्ह), भावनाशील (फिलिंग) आणि निर्णयक्षम (जजमेंट) असा प्रकार आहे. जीवनशैलीत मी साधेपणावर आणि मिनिमलिझमवर भर देतो. माझ्या स्वभावात वात प्रधानता आहे. लोकांना मार्गदर्शन करणे, शिक्षकासारखे ज्ञान वाटणे, हा माझा स्वभावधर्म आहे, जणू ज्ञानेश्वरांनी सांगितल्याप्रमाणे. माझा सर्वात मोठा गुण म्हणजे समजूतदारपणा आणि सहानुभूती. मन शांत आणि आनंदी ठेवण्याचा मी प्रयत्न करतो, तर शरीर सुदृढ व आकर्षक ठेवण्याचा संकल्प करतो. कला, प्रोग्रॅमिंग, अध्यापन आणि योग यात मला नैसर्गिकपणे कौशल्य आहे.

\section*{काय करावे, करू नये}
दैनंदिन जीवनात कोणत्या गोष्टी टाळायच्या हे जाणून घेणे महत्त्वाचे असते. अप्रासंगिक क्रिया, अनावश्यक संपर्क, फालतू बातम्या, सततचे सोशल मिडिया यापासून दूर राहावे. जे गरजेचे नाही ते इतरांना दान करावे. कुठल्याही कौशल्यावर प्रभुत्व मिळवायचे असेल तर सतत सराव करावा, लहान प्रयोग करावेत आणि ते वारंवार पुनरावृत्ती करावी. विचार जसे असतात तसेच आपण बनतो, त्यामुळे चांगले विचार, चांगल्या इच्छा, चांगली वाणी आणि चांगली कृती यांचा अंगीकार करावा. समजून घेण्यात स्पष्टता यावी यासाठी वाचन हळूहळू व सखोल करणे आवश्यक आहे. काही विचार विसरले गेले तरी चिंता नको, विसरणे हाही एक स्वाभाविक भाग आहे. हातातील कामावर लक्ष केंद्रित करावे आणि ते पूर्ण होईपर्यंत चिकाटी ठेवावी. जबाबदारी इतरांवर ढकलू नये, तर जवळच्या माणसांशी मुक्तपणे संवाद साधावा.

शांत राहणे, ध्यानधारणा करणे, जीवनाचा आनंद घेणे हीसुद्धा कृतीच आहेत. शरीर मजबूत असेल तर मनही मजबूत राहते. समस्यांचा सामना करताना आक्रमकपणे नव्हे तर सकारात्मकतेने आणि शांतपणे करावा. एखादी गोष्ट पूर्ण न होण्याची भीती बाळगू नये; अपूर्णतेतही एक टप्पा पूर्ण होतो, पुढे हळूहळू प्रगती होते, हेच खरे समाधान आहे. विश्रांतीत मन शांत असावे आणि कृतीत शरीर सक्रिय असावे. अनपेक्षित अडचणी येऊ शकतात, त्यांचा सामना करण्यासाठी शरीर तंदुरुस्त, मन शांत आणि वृत्ती सकारात्मक ठेवावी. कपील गुप्ता यांच्या म्हणन्यानुसार अंतर्दृष्टी मौनात प्राप्त होते (``Insights are achieved in silence’’)

\section*{क्षेत्र निवडीचे तत्त्व}
आपल्या जीवनातील कार्यक्षेत्र निवडताना सर्वप्रथम एक गोष्ट लक्षात ठेवली पाहिजे की केवळ साधारण ज्ञान पुरेसे नसते. खरे महत्त्वाचे असते ते विशिष्ट ज्ञान. असे ज्ञान जे दुर्मीळ असते, जे कुणालाही थोड्या प्रशिक्षणाने किंवा अभ्यासाने शिकता येत नाही. विशिष्ट ज्ञान हे बहुधा प्रत्यक्ष कामाच्या अनुभवातून निर्माण होते. त्यामागे खरी जिज्ञासा आणि खोलवरची आवड असते. म्हणूनच हे ज्ञान तुमचे वेगळेपण दाखवते.
विशिष्ट ज्ञान मिळवले की त्यातून जणू एक प्रकारचे एकाधिकाराचे सामर्थ्य येते. कारण इतर कोणीही सहजपणे ते आत्मसात करू शकत नाही. या एकाधिकारातूनच इकीगाई निर्माण होते. ‘इकीगाई’ म्हणजे आपल्या जीवनाचा उद्देश. हे चार घटक एकत्र आल्यावर तयार होते – तुम्हाला जे करायला आवडते, जगाला जे खरोखर गरजेचे आहे, ज्यात तुम्ही खरंच कुशल आहात आणि ज्यासाठी मोबदला मिळू शकतो. या चौघांच्या संगमातून दीर्घकाळ टिकणारा मार्ग सापडतो.
क्षेत्र निवडताना स्पर्धा कमी असलेले, किंवा अजिबात स्पर्धा नसलेले क्षेत्र शोधणे शहाणपणाचे ठरते. कारण जिथे गर्दी असते तिथे तुमचे वेगळेपण हरवते. परंतु जिथे लोक कमी आहेत तिथे तुमची ओळख, तुमचे काम झळकते.
आर्थिक यश साध्य करायचे असेल तर तीन स्तंभ आवश्यक असतात: विशेषीकरण, लाभकारी साधने आणि जबाबदारी. विशेषीकरण म्हणजे तुमच्या क्षेत्रातील असा गुण, अशी कौशल्यांची सांगड, जी इतरांकडे नाही. हे एक प्रकारचे अन्यायकारक पण खरेच लाभदायी असे फायद्याचे साधन आहे. त्यात कष्ट लागतात, पण ते एकदा मिळाले की ते टिकते.
दुसरा स्तंभ म्हणजे ‘लिव्हरेज’ किंवा उपयोगी साधने. याला ‘फोर्स मल्टिप्लायर’ असे म्हणतात. जसे की संगणकावर लिहिलेला एक कार्यक्रम तुम्ही झोपेत असतानाही काम करतो. अशा साधनांमुळे तुमचे काम दुप्पट, तिप्पट वेगाने वाढते.
तिसरा स्तंभ म्हणजे जबाबदारी. जबाबदारी घेणारा मनुष्यच खरोखर मूल्य निर्माण करतो. इतरांना दिलेले वचन पूर्ण करणे, विश्वासार्ह राहणे आणि आपल्या कार्यामुळे स्वतःची ओळख एक ब्रँड म्हणून उभी करणे हे खरे यशाचे लक्षण आहे.
या सर्व प्रक्रियेत संचय महत्त्वाचा आहे. लहान-लहान प्रयत्न आणि सातत्यपूर्ण कृती हळूहळू मोठे फळ देतात. जणू पाण्याच्या थेंबातून तलाव भरतो तशी प्रगती दिसते. म्हणून क्षेत्र निवडताना अल्पकालीन मोह टाळून, दीर्घकालीन संचयी परिणामांचा विचार करणे अत्यंत आवश्यक आहे.
\section*{माणसांशी वागण्याची तंत्रे}
माणसांशी संवाद साधताना काही सोपी पण परिणामकारक तंत्रे उपयोगी ठरतात. कुणाला तुमच्याबद्दल आवड नसली तरी त्याच्याकडून मदत मागितली तर तो हळूहळू मऊ होतो. कारण मदत केल्याने त्याला स्वतःबद्दल चांगले वाटते आणि तुमच्याबद्दलचा विरोध कमी होतो.
एखादे काम करून घ्यायचे असेल तर थोडे युक्तीने वागावे लागते. जर तुम्हाला कुणाकडून ‘एक्स’ काम करून घ्यायचे असेल, तर सुरुवातीला ‘पाच एक्स’ मागा. तो नक्कीच नकार देईल. पण मग तुम्ही खरी मागणी म्हणजे ‘एक्स’ ठेवाल, तेव्हा त्याला ती फार अवघड वाटणार नाही. उलट, पहिल्यापेक्षा सोपी असल्याने तो ती मागणी मान्य करेल.
लोकांना त्यांच्या नावाने संबोधणे हा एक साधा पण प्रभावी उपाय आहे. प्रत्येकालाच आपले नाव गोड वाटते. नावाबरोबर जर सन्मानाची पदवी वापरली, जसे की “डॉ. अमुक” किंवा “प्राध्यापक तमुक”, तर त्यांना विशेष आदर वाटतो आणि संवाद अधिक मोकळा होतो.
‘मिररिंग’ म्हणजे समोरच्याच्या वागण्याची, बोलण्याची शैली आपल्यात परावर्तित करणे. जेव्हा आपण दुसऱ्यासारखेच वागतो, तेव्हा त्याला आपलेपणाची भावना होते आणि त्याच्याकडून काम करून घेणे सुलभ होते.
तसेच कौतुक करणे हेही एक शक्तिशाली तंत्र आहे. मनुष्याला आपल्या गुणांची दखल घेतली गेली की त्याला श्रेष्ठत्वाची जाणीव होते. त्यातून तो आनंदी होतो आणि तुम्हाला मदत करण्यासाठी तत्पर होतो.
शेवटी हे लक्षात ठेवावे की आपण ज्या लोकांमध्ये राहतो, त्यांचा प्रभाव आपल्यावर पडतो. आपण आपल्या सभोवतालच्या लोकांचा सरासरी बनतो. त्यामुळे आपला संग विचारपूर्वक निवडणे हेच माणसांशी वागण्याचे सर्वोत्तम तंत्र आहे.

\section*{शिकवण}
जीवन प्रवासात अनेक शिकवणी वेळोवेळी अनुभवायला मिळतात. काही थेट अनुभवातून तर काही परंपरेतून किंवा पूर्वजांच्या विचारांतून. प्रत्येक शिकवण मनाला दिशा देते आणि पुढील कृतीसाठी आधार ठरते.
सर्वप्रथम लक्षात येते ते म्हणजे “चमत्काराशिवाय नमस्कार नाही.” म्हणजेच विशेष काही केले तरच लोक आदर देतात. साध्या गोष्टींनी कोणी प्रभावित होत नाही. त्यामुळे स्वतःच्या मर्यादेपलीकडे जाऊन, स्वतःची “औकात पार करनी पडेगी”  आपली क्षमता विस्तारून दाखवली पाहिजे.
ज्ञान मिळवण्यासाठी अभ्यास हा एकमेव मार्ग आहे. अभ्यासातूनच ज्ञान प्रकट व्हावे, अन्यथा ते दडून राहणे श्रेयस्कर. कारण अपुरे ज्ञान प्रकट झाले तर ते लवकर नष्ट होते. त्यामुळे अध्ययन हे गंभीरपणे, सातत्याने आणि नम्रतेने केले पाहिजे.
इतर जसे करतात तसेच केले तर आपण इतरांसारखेच होतो. म्हणून वेगळे करणे, नवीन करणे आणि अधिक चांगले करणे हीच प्रगतीची गुरुकिल्ली आहे. “Out of sight, out of mind” ही म्हण खरी आहे. जी गोष्ट महत्त्वाची आहे ती सतत डोळ्यासमोर ठेवली पाहिजे, नाहीतर ती विस्मरणात जाते.
लढाईत कमी रक्त सांडायचे असेल तर प्रशिक्षणात जास्त घाम गाळावा लागतो. याचा अर्थ म्हणजे कठोर परिश्रम व तयारी यामुळे संकटांचा सामना सोपा होतो. भीतीबाबतही हाच धडा लागू होतो. खरी भीती वाघासारख्या प्राण्यापासून किंवा आगीपासून असते, बाकी गोष्टींची भीती ही केवळ मनाची कल्पना असते.
जीवनात दुर्मिळ व्हा, मौल्यवान व्हा आणि सर्वसाधारण नव्हे तर विलक्षण बना. स्वतःला जाणणे हे अत्यंत आवश्यक आहे. आपल्या स्वभावाचे, प्रकृतीचे, जसे की वात प्रधान आणि पित्त गौण आहे, हे जाणून घेणे महत्त्वाचे ठरते. नकारात्मकता ही सहज येते, ती नैसर्गिक आहे, पण तिच्यावर मात करण्यासाठी प्रचंड मानसिक ताकद लागते.
एक महत्त्वाची जाणीव अशी की, कुणीही दुसऱ्याच्या दृष्टीने खूप महान नसतो. प्रत्येक जण आपापल्या नजरेतच मोठा असतो. म्हणून इतरांकडून सतत महानतेची अपेक्षा ठेवणे व्यर्थ आहे. जीवनात सर्व गोष्टी करता येणार नाहीत, त्यामुळे आपल्या मूलभूत मूल्यांशी विसंगत क्रिया टाळल्या पाहिजेत. जेव्हा योग्य वाटत नाही तेव्हा “हो” न म्हणता नकार देण्याचे धैर्य ठेवावे.
आपली उर्जा शोषून घेणाऱ्या, थकवणाऱ्या लोकांपासून दूर राहणे आवश्यक आहे. कारण जीवन ही उर्जेचीच शर्यत आहे. सर्जनशीलता म्हणजे आधीच अस्तित्वात असलेल्या गोष्टींना नव्या पद्धतीने एकत्र आणणे. ही सांगड घालण्याची कला नवीन संधी निर्माण करते. प्रतिकारशक्ती म्हणजेच ताकद. शारीरिक असो किंवा मानसिक, प्रतिकूलतेशी झुंज दिल्यावरच खऱ्या अर्थाने सामर्थ्य वाढते.
यशस्वी व्हायचे असेल तर एका वेळेस एका क्षेत्रावर दीर्घकाळ लक्ष केंद्रित करावे लागते. एका दिवसातही संपूर्ण लक्ष एकाच गोष्टीवर केंद्रित झाले की त्यातून अद्वितीय परिणाम साधता येतात. जीवनात एकच सर्वोत्तम गोष्ट असते असे नाही, अनेक वेगवेगळ्या क्षेत्रांत “सर्वोत्तम” साध्य करता येते.
नेहमी वेगळे आणि अधिक चांगले व्हा. काहीवेळा जीवनाला अर्थच नाही, उद्देशच नाही असे वाटते. कारण विश्व अनिश्चित आहे, आणि बर्‍याच गोष्टी केवळ मनाच्या कल्पना आहेत. श्वासावर नियंत्रण ठेवले की मनावरही नियंत्रण मिळते. प्राणायाम हा मन आणि शरीर यांच्यातील पूल आहे. ध्यानामुळे जागरूकता वाढते आणि त्यावर आधारलेली धारणा म्हणजे लक्ष केंद्रित करणे सुलभ होते.
या सर्व शिकवणी एकत्रित केल्या तर स्पष्ट दिसते की जीवन हे शिस्त, सर्जनशीलता, स्व-ओळख, सातत्यपूर्ण परिश्रम आणि शांततेच्या सरावावर उभे असते.
\section*{तत्त्वे}
जीवन जगताना काही मूलभूत तत्त्वे आपल्याला दिशा देतात. ही तत्त्वे सोपी वाटतात, पण त्यांच्या आचरणातूनच खरी प्रगती आणि समाधान मिळते.
सर्वप्रथम, वाढीसाठी अस्वस्थतेला सामोरे जायला शिका. सुखसोयीच्या बाहेर पाऊल टाकल्याशिवाय खरी प्रगती होत नाही. हा प्रवास कधी कडू तर कधी गोड असतो, पण त्यातून घडणारे व्यक्तिमत्त्व अधिक बळकट असते.
विचार वेगळा ठेवा, अधिक चांगला ठेवा आणि समस्या सोडवण्यासाठी नवीन मार्ग शोधा. जगाला हवे असते ते असे लोक, जे इतके चांगले काम करतात की त्यांना दुर्लक्षित करणे अशक्य होते. म्हणून स्वतःला त्या पातळीपर्यंत घडवा.
खरी लढाई मनाशी असते. मनावर नियंत्रण मिळवले की खरे सामर्थ्य निर्माण होते. समस्या सोडवताना चातुर्य, विनोदबुद्धी आणि शहाणपण वापरा. आयुष्यात मिळालेल्या देणग्यांची जाणीव ठेवा, निरोगी शरीर, सुखद अनुभव, साथ देणारे लोक आणि त्याबद्दल कृतज्ञ राहा.
“सर्वात वाईट काय होऊ शकते?” हा प्रश्न स्वतःला विचारा. भीती कमी होते. आणि मग “डोन्ट वरी, बी हॅपी” ही वृत्ती स्वीकारा. तक्रार कधीही करू नका; तक्रार म्हणजे दुर्बलतेचे लक्षण.
आपले प्रत्येक काम उत्तम करा. सर्वोत्तम करण्याचा प्रयत्न हाच प्रामाणिकपणाचा पाया आहे. एकावेळी एकच काम करा (single-tasking) ही खरी कार्यक्षमता आहे.
नेहमी सत्याचा शोध घ्या. जीवनात साधेपणाला जागा द्या. जितके कमी, तितके हलके, तितके स्पष्ट. चेहऱ्यावर स्मित ठेवा. आणि अखेरीस  फक्त करा. विचारांत अडकून राहू नका, कृती करा.
ही काही सोपी पण प्रभावी तत्त्वे जीवनाचा पाया मजबूत करतात.
\section*{आता आलोच आहोत तर \ldots}
जीवनात कधी तरी प्रश्न उभा राहतो.  मी येथे का आहे? माझ्या अस्तित्वाचा उद्देश काय? मी फक्त एका स्वार्थी जीनचा (Richard Dawkins यांच्या भाषेत) वाहक आहे का, ज्याचा एकमेव हेतू म्हणजे वंशविस्तार? या विश्वात माझे वेगळे स्थान आहे का, की मी केवळ योगायोगाने येथे आलो आहे?
उत्तर ठोस मिळणे कठीण आहे. पण एक गोष्ट नक्की. आता आपण येथे आलो आहोत, तर काय करू शकतो, हे ठरवणे आपल्या हातात आहे.
सर्वप्रथम, जीवनाचा आस्वाद घ्यावा. आराम करावा, अनावश्यक ओझे घेऊ नये. फसवाफसवी न करता, जे आवडते तेच करावे. आणि वेळ व परिस्थितीने संधी दिली तर इतरांना मदत करावी. कारण या जगात खूप लोक दुर्दैवी परिस्थितीत जगत आहेत. नैसर्गिक निवड (natural selection) हा अल्गोरिदम नेहमी न्याय्य ठरलेला नाही. तो अनेकदा अव्यवस्थित, अन्यायकारक आणि निव्वळ योगायोगांनी चालणारा वाटतो.
या जीवनात आपण जे काही मिळवले आहे ते मोठ्या प्रमाणात केवळ नशिबाचे देणे आहे. हो, कष्ट आणि प्रयत्न यांचा वाटा नक्कीच आहे, पण शेवटी परिस्थिती, संधी आणि वेळ यांचा प्रभाव सर्वांत जास्त असतो. प्रत्येक क्षणी सर्व काही नाहीसे होऊ शकते. अस्तित्वच इतके नाजूक आहे.
म्हणूनच काहीतरी निर्माण करणे, बांधणे, अर्थपूर्ण, सकारात्मक आणि उपयोगी असे काहीतरी घडवणे ही खूप मोठी गोष्ट आहे. अनंत संधींच्या आणि अपयशांच्या या महासागरात, अशा यशस्वी प्रयत्नांची शक्यता खूपच कमी असते. त्यामुळे जेव्हा असे घडते, तेव्हा त्याचे मोल अपरंपार असते.
तुम्ही हे वाचत आहात, म्हणजे तुम्ही नशीबवान आहात. होय, फारच भाग्यवान. कदाचित विशेषाधिकार प्राप्तसुद्धा. म्हणूनच – चांगले जगा, निवांत राहा, जीवनाचा आनंद घ्या. आणि कधीतरी संधी मिळाली की, इतरांना देखील या अवस्थेत पोहोचण्यासाठी मदत करा.
\section*{S’s}
\begin{itemize}
\item Strength
\item Stamina
\item Serenity
\item Sun Salutation 
\item Savor
\item Suppleness
\item Sleep 
\item Style
\item Sketches
\item Smile
\item Slow 
\item Simplicity
\end{itemize}

\section*{पालकत्व}

पालकत्व म्हणजे केवळ शब्दांनी मुलांना शिकवणे नव्हे, तर स्वतःच्या कृतीतून उदाहरण घालून देणे. मुलं आपले बोलणे फार कमी लक्षात ठेवतात, परंतु आपली कृती आणि तिचे परिणाम त्यांच्यावर खोलवर परिणाम करतात. त्यामुळे कृतीत सचोटी असावी आणि त्यातून सातत्याने स्पष्ट संदेश द्यावा. मुलांच्या भावनांचा विचार करणेही आवश्यक आहे, कारण भावना हाच कृतीचा खरा उगम असतो. भीती, संधी गमावण्याची चिंता, वर्ष वाया जाण्याची काळजी या भावनांमुळे मुलांची दिशा ठरते. या भावना समजून घेऊन त्यांना योग्य मार्गदर्शन करणे हे पालकांचे कार्य आहे.

मुलांना आपल्याबद्दल नेहमीच प्रेम वाटेलच असे नाही, परंतु त्यांना आपल्या वागण्यामुळे आणि प्रामाणिकपणामुळे आदर वाटला पाहिजे. जेव्हा आपण एखादी धमकी देतो, तेव्हा ती खरी केली पाहिजे; अन्यथा मुलं आपल्याला गंभीरतेने घेत नाहीत. एखाद्या दिवशी जेवण न केल्याने मुलांना काही हानी होत नाही, त्यामुळे अति लाड पुरवू नयेत. हट्ट, थयथयाट किंवा अनावश्यक चालूगिरी सहन केली जाऊ नये. घरातील कामे करण्याची जबाबदारी मुलांनी पेलली पाहिजे, कारण स्वावलंबनाची शिकवण लहानपणापासूनच द्यावी लागते. आपण मोठे झाल्यामुळे मदत करणे शक्य असते, परंतु आयुष्यात पायावर उभे राहण्याची जबाबदारी मुलांचीच असते.

आहाराच्या बाबतीतही स्पष्ट शिस्त असावी. रोजचे जेवण नियमित खाल्ले पाहिजे. जंक फूड, तयार अन्न किंवा साखर यांचा वापर टाळावा. केवळ मुलांनीच नव्हे तर मोठ्यांनीही हे नियम पाळावेत. घर हे दुकान नसते, म्हणून मुलांशी वाटाघाटी करू नयेत. घरात पालकांचे नियमच चालतील ही स्पष्ट जाणीव असावी. अशा प्रकारे प्रेम, आदर, शिस्त आणि स्वावलंबन यांच्या संतुलनातूनच खरी पालकत्वाची भूमिका पूर्ण होते.

\section*{शरीरावरील शिस्त}

शरीर हा जीवनाचा पाया आहे. मन निरोगी ठेवायचे असेल तर शरीर निरोगी असणे अपरिहार्य आहे. वय वाढत असताना स्नायूंचा क्षय टाळण्यासाठी वजन उचलण्याचा व्यायाम करणे आवश्यक आहे. मजबूत शरीर आपल्याला केवळ तंदुरुस्त ठेवत नाही तर आत्मविश्वासही वाढवते. मजबूत बाहू आणि सपाट पोट ही फक्त सौंदर्याची लक्षणे नसून आत्मशिस्तीचेही द्योतक आहेत.

मनातील चिंता दूर करण्यासाठी ध्यानधारणा उपयुक्त ठरते, तर पाठदुखी कमी करण्यासाठी योग सर्वोत्तम आहे. शरीरातील पेशींचे शुद्धीकरण व्हावे म्हणून सकाळचे उपाशीपोटी राहणे, म्हणजेच नाश्ता टाळणे, हे ‘ऑटोफॅजी’ प्रक्रियेसाठी उपयोगी ठरते. धावणे किंवा डोंगर चढणे यामुळे शरीरात सहनशक्ती वाढते. अशा प्रकारे योग, व्यायाम, आहार आणि विश्रांती यांच्या समतोलातून शरीर आणि मन दोन्ही सशक्त राहतात.


\section*{यात माझ्यासाठी काय?}
एखाद्याकडून काही काम करून घ्यायचे असेल, तर सर्वात आधी स्वतःला एक प्रश्न विचारावा – “या व्यक्तीला यात काय मिळणार आहे?” म्हणजेच what’s in it for them?
कारण कुणीही विनाकारण वेळ, श्रम किंवा लक्ष घालवणार नाही. प्रत्येक निर्णयामागे एक Return on Investment (RoI) असतो. हा RoI नक्कीच पैशांच्या स्वरूपात असू शकतो, पण तो केवळ आर्थिकच असेल असे नाही. तो प्रतिष्ठा वाढवण्याचा असू शकतो, नवीन संधी मिळवण्याचा असू शकतो, शिकण्याचा अनुभव असू शकतो किंवा भविष्यातील वाढीच्या शक्यता निर्माण करण्याचा असू शकतो.
निश्चितच, कुटुंब आणि अतिशय जवळच्या मित्र-मैत्रिणींशी संबंध ठेवताना हे गणित आवश्यक नसते. पण व्यावसायिक, सामाजिक किंवा विस्तारित संपर्कांमध्ये विनंती करण्यापूर्वी हे गणित मांडणे आवश्यक आहे.
सामान्यतः आपण विनंती करताना केवळ आपली गरज पूर्ण होते की नाही यावर लक्ष केंद्रित करतो. पण जोपर्यंत समोरच्याला त्याच्या कष्टाचे काही मोल मिळत नाही, तोपर्यंत तुमची विनंती न्याय्य ठरत नाही. म्हणून विनंती करताना त्या व्यक्तीला काय लाभ होईल, हे स्पष्टपणे सांगणे महत्त्वाचे आहे. आर्थिक बाब असेल तर मोबदला स्पष्ट असावा. जर ठोस लाभ सांगता आला नाही, तर निदान एक प्रश्न विचारला तरी चालतो – “मी तुमच्यासाठी काय करू शकतो?”
यासाठी एक साधा लिटमस टेस्ट वापरता येतो – “मी हेच काम कुणासाठी विनामूल्य करीन का?” जर उत्तर नसेल, तर तुम्ही ज्या प्रकारे परतफेड अपेक्षित कराल, त्याच प्रकारे समोरच्यालाही मूल्य द्यायला हवे.
शेवटी, सोपा नियम – There is no free lunch. कुणी काही करते म्हणजे त्याला काहीतरी परत हवे असते.
(\textit{टीप: हे विचार सतत येणाऱ्या विनंत्यांच्या पार्श्वभूमीवर लिहिले आहेत – चर्चांना बोलावणे, सेमिनारमध्ये सहभागी होणे, कल्पना तपासून देणे, समस्यांचे निराकरण करणे, व्यवसाय प्रचार करणे, पुनरावलोकन करणे – अशा अनेक मागण्यांमुळे एक मुद्दा नेहमी ठळक झाला: “What’s in it for me?”})










