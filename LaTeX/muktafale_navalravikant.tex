%%%%%%%%%%%%%%%%%%%%%%%%%%%%%%%%%%%%%%%%%%%%%%%%%%%%%%%%%%%%%%%%%%%%%%%%%%%%%%%%%%%%%%%%%%%%%%%%%
\chapter{नवल रविकांत}

% --------------------
\section*{Life, Happiness \& Peace of Mind}
\begin{itemize}
  \item A fit body, a calm mind, a house full of love. These things cannot be bought, they must be earned. तंदुरुस्त शरीर, शांत मन आणि प्रेमाने भरलेले घर – हे विकत घेता येत नाहीत, कमवावे लागतात.

  \item This is such a short and precious life that it’s really important that you don’t spend it being unhappy. आपलं आयुष्य इतकं छोटे आणि मौल्यवान आहे, ते दु:खी होण्यात वाया घालवू नका.

  \item Happiness is a choice and a skill and you can dedicate yourself to learning that skill and making that choice. आनंदी असणं स्वतःची इच्छा आणि कौशल्य आहे. आपलं आयुष्य हे त्या इच्छेसाठी आणि कौशल्य विकसित करण्यासाठी खर्ची घाला.

  \item The most important trick to be happy is to realize that happiness is a choice that you make and a skill that you develop. आनंदी होण्याचा सर्वात मोठा मंत्र म्हणजे आनंद ही तुमची निवड आहे हे ओळखणे आणि त्या निवडीचं कौशल्य जपण्यात आहे. 

  \item Happiness is a state where nothing is missing. आनंद म्हणजे अशी अवस्था आहे जिथे कमतरतेचा मागमूसही नसतो. 

  \item Reality is neutral. Our reactions reflect back and create our world. वास्तव हे निर्विकार आहे. आपली प्रतिक्रिया या वास्तवात आपलं जग निर्माण करते. 

  \item A rational person can find peace by cultivating indifference to things outside of their control. तर्कशुद्ध व्यक्ती त्याच्या नियंत्रणाबाहेरील गोष्टींबद्दल उदासीन राहून (दुर्लक्ष करून) शांतता मिळवते. 

  \item The enemy of peace of mind is expectations drilled into you by society and other people. मन:शांतीचा शत्रू म्हणजे समाज आणि लोकांनी रुजवलेल्या अपेक्षा होत. 

  \item Anger is a hot coal that you hold in your hand while waiting to throw it at someone else. राग म्हणजे दुसऱ्यावर फेकायचे जळते निखारे स्वतःच्या हातात ठेवणे.

  \item Caught in a funk? Use meditation, music, and exercise to reset your mood. अस्वस्थ वाटतंय? तर मनःस्थिती बदलण्यासाठी ध्यान, संगीत किंवा व्यायाम याचा वापर करा

  \item Be present above all else. सर्वप्रथम वर्तमानात रहा. 

  \item Every moment has to be complete in and of itself. प्रत्येक क्षण वर्तमानात राहून अनुभवावा. 

  \item You can have the mind or you can have the moment. एकतर तुम्ही हा क्षण कसा जगायचा हा विचार करत बसू शकता किंवा तो क्षण जगू शकता. दोन्ही एकत्र नाही होऊ शकत. 

  \item A busy mind accelerates the perceived passage of time. Buy more time by cultivating peace of mind. व्यस्त असलेल्या मनाला काळ वेळाचं भान राहत नाही. जास्त वेळ मिळावा असं वाटत असेल तर मन:शांती चा मार्ग धरा.  

  \item Life hack: when in bed, meditate. Either you will have a deep meditation or fall asleep. Victory either way. जीवनमंत्र: झोपताना ध्यान करा. एक तर ध्यान चांगलं लागेल किंवा छान झोप लागेल, दोन्ही फायदेशीर. 

  \item Meditation is intermittent fasting for the mind. ध्यान म्हणजे मनासाठीचा उपवास. 

  \item Enlightenment is the space between your thoughts. ज्ञान म्हणजे तुमच्या विचारांमधील रिक्त जागा.

  \item You can change it, you can accept it, or you can leave it. What is not a good option is to sit around wishing. एक तर तुम्ही बदलू शकता, स्वीकारू शकता किंवा सोडू शकता – पण फक्त इच्छा करत बसणे निरर्थक आहे. 
\end{itemize}

% --------------------
\section*{Wealth, Work \& Entrepreneurship}
\begin{itemize}
  \item Earn with your mind, not your time. तुमच्या वेळेने नव्हे तर बुद्धीने कमवा.

  \item You’re never going to get rich renting out your time. वेळ भाड्याने देऊन कोणी श्रीमंत होत नाही.

  \item Wealth is assets that earn while you sleep. धन म्हणजे तुम्ही झोपलेले असतानाही कमावणारी साधने.

  \item You must own equity to gain your financial freedom. आर्थिक स्वातंत्र्यासाठी मालकी हक्क (इक्विटी) असणे आवश्यक आहे.

  \item Apply specific knowledge with leverage and eventually, you will get what you deserve. विशेष ज्ञान आणि साधनांचा उपयोग करा – शेवटी योग्य फळ मिळेल.

  \item Play long-term games with long-term people. दीर्घकालीन खेळायचं असेल तर दूरदृष्टीच्या लोकांसोबत राहा/खेळा .

  \item Embrace accountability and take business risks under your own name. Society will reward you with responsibility, equity, and leverage. जबाबदारी स्वीकारा आणि स्वतःच्या नावाने धोका स्वीकारा, समाज तुम्हाला मालकी आणि लाभ देईल.

  \item Forget rich versus poor, white-collar versus blue. It’s now leveraged versus un-leveraged. 
गरीब की श्रीमंत, पांढरपेशी कि रोजंदारीवाले या फरकांवर आजकाल काही अवलंबून नाहीये. आत्ताची स्पर्धा  ज्यांच्याकडे साधने आहेत आणि ज्यांच्याकडे नाहीत यांच्यात आहे.

  \item Who you do business with is just as important as what you choose to do. तुम्ही कोणाबरोबर व्यवसाय करता हे, तुम्ही काय करता याइतकेच महत्त्वाचे आहे.

  \item If you can’t see yourself working with someone for life, don’t work with them for a day. ज्यांच्याशी आयुष्यभर काम करू शकत नाही असं वाटतं, त्यांच्याबरोबर एक दिवसही काम करू नका.

  \item Don’t partner with cynics and pessimists; their beliefs are self-fulfilling. निंदक आणि निराशावादी लोकांशी भागीदारी करू नका, त्यांची श्रद्धा स्वतः पुरतीच असते.

  \item Escape competition through authenticity. प्रामाणिकपणातून स्पर्धा टाळा.

  \item Above ``product-market fit'' is ``founder-product-market fit.'' ``उत्पादन-बाजार जुळणी'' पेक्षा महत्त्वाची म्हणजे ``संस्थापक-उत्पादन-बाजार जुळणी.''

  \item I would rather be a failed entrepreneur than someone who never tried. प्रयत्न न करणाऱ्या व्यक्तीपेक्षा अपयशी उद्योजक असणे पसंत करीन.

  \item Grind and sweat, toil and bleed, face the abyss. It’s all part of becoming an overnight success. कष्ट, संघर्ष आणि वेदना – हे सर्व एका क्षणात मिळालेल्या यशाचा भाग आहेत.

  \item If you see a get rich quick scheme, that’s someone else trying to get rich off of you. जलद श्रीमंतीच्या योजनेत तुमचा वापर करून दुसरा श्रीमंत होतो.

  \item Retirement is when you stop sacrificing today for an imaginary tomorrow. निवृत्ती म्हणजे काल्पनिक उद्यासाठी आजचा त्याग थांबवणे.

  \item Forty hour workweeks are a relic of the Industrial Age. Knowledge workers function like athletes, train and sprint, then rest and reassess. ४० तासांचा आठवडा ही औद्योगिक काळाची परंपरा आहे. ज्ञानकर्मी खेळाडूसारखे – प्रशिक्षण मग जलदगतीने धावणे, आणि मग विश्रांती घेत मागोवा घेणे. 
\end{itemize}


\section*{Work, Wealth \& Success}
\begin{itemize}
    \item ``Most of the gains in life come from suffering in the short term so you can get paid in the long term.'' आयुष्यातील बहुतांश फायदा हा अल्प-काळात सोसलेल्या त्रासातून येतो, जेणेकरून दीर्घकाळात त्याचे फळ मिळते.

    \item ``All the returns in life, whether in wealth, relationships, or knowledge, come from compound interest.'' 
आयुष्यातील सर्व परतावे, संपत्ती, नातेसंबंध किंवा ज्ञान, हे चक्रवाढ व्याजातून येतात.
\end{itemize}

\section*{Learning \& Knowledge}
\begin{itemize}
    \item ``The most important skill for getting rich is becoming a perpetual learner.'' श्रीमंत होण्यासाठीची सर्वात महत्त्वाची कौशल्य म्हणजे कायम शिकत राहणे.

    \item ``Your most important skill isn’t even what you majored in... it’s just knowing how to learn.'' तुमचे सर्वात महत्त्वाचे कौशल्य म्हणजे तुम्ही काय शिकलात हे नाही, तर कसे शिकायचे हे आहे.

    \item ``Even today, what to study and how to study it are more important than where to study it...'' आजही काय शिकायचे आणि कसे शिकायचे हे कुठे शिकायचे यापेक्षा महत्त्वाचे आहे.

    \item ``Read what you love until you love to read.'' जोपर्यंत वाचनाची आवड निर्माण होत नाही तुम्हाला जे वाचायला आवडते ते वाचा. .

    \item ``It’s better to read a great book slowly than to fly through a hundred books quickly.'' शंभर पुस्तके झपाट्याने वाचण्यापेक्षा एक चांगलं पुस्तक हळूहळू वाचणे चांगले.

    \item ``Knowledge is a skyscraper. You can take a shortcut... or build slowly upon a steel frame of understanding.'' ज्ञान म्हणजे गगनचुंबी इमारत आहे. ती तुम्ही शॉर्टकटने नाजूक पायावर बांधू शकता किंवा हळूहळू समजुतीच्या मजबूत चौकटीवर बांधू शकता.
\end{itemize}

\section*{Habits \& Self-Mastery}
\begin{itemize}
    \item ``Humans are basically habit machines… learning how to break habits is a very important meta skill.'' माणसे म्हणजे सवयीचे गुलाम आहेत... सवयी बदलण्याचे कौशल्य खूप महत्त्वाचे आहे.

    \item ``The power to make and break habits and learning how to do that is really important.'' सवयी लावणे आणि मोडणे तसेच ते कसे करायचे हे शिकणे फार महत्त्वाचे आहे.

    \item ``The enemy of peace of mind is expectations drilled into you by society and other people.'' मनःशांतीचा शत्रू म्हणजे समाज आणि इतरांनी तुमच्यात रुजवलेल्या अपेक्षा.

    \item ``A busy mind accelerates the perceived passage of time. Buy more time by cultivating peace of mind.'' व्यस्त असलेल्या मनाला काळ वेळाचं भान राहत नाही. जास्त वेळ मिळावा असं वाटत असेल तर मन:शांती चा मार्ग धरा.
\end{itemize}

\section*{Philosophy \& Clarity}
\begin{itemize}
    \item ``Happiness is a choice and a skill and you can dedicate yourself to learning that skill and making that choice.'' आनंदी असणं स्वतःची इच्छा आणि कौशल्य आहे. आपलं आयुष्य हे त्या इच्छेसाठी आणि कौशल्य विकसित करण्यासाठी खर्ची घाला.

    \item ``Desire is a contract that you make with yourself to be unhappy until you get what you want.'' इच्छा/वासना/अपेक्षा - स्वतःसोबत केलेला असा करार आहे की जो पूर्ण होईपर्यंत तुम्ही दुःखी राहाल.

    \item ``Be present above all else.'' सर्वात महत्त्वाचे म्हणजे वर्तमानात रहा.

    \item ``Reality is neutral. Our reactions reflect back and create our world.''वास्तव हे निर्विकार आहे. आपली प्रतिक्रिया या वास्तवात आपलं जग निर्माण करते.

    \item ``A rational person can find peace by cultivating indifference to things outside of their control.'' तर्कशुद्ध व्यक्ती त्याच्या नियंत्रणाबाहेरील गोष्टींबद्दल उदासीन राहून (दुर्लक्ष करून) शांतता मिळवते.
\end{itemize}

\section*{Society \& Politics}
\begin{itemize}
    \item ``Wealth creation is an evolutionarily recent positive-sum game. Status is an old zero-sum game.'' संपत्ती निर्माण करणे ही अलीकडची सर्वहिताचा खेळ आहे. प्रतिष्ठा ही जुनी एकाचा नफा तर दुसऱ्याचा तोटा असा खेळ आहे.

    \item ``Politics is sports writ large, pick a side, rally the tribe, exchange stories confirming bias, hurl insults.'' राजकारण म्हणजे मोठ्या प्रमाणावरचा खेळ आहे, बाजू निवडा, गट जमवा, पूर्वग्रह दृढ करणाऱ्या कथा सांगा, शिवीगाळ करा.

    \item ``Don’t debate people in the media when you can debate them in the marketplace.'' माध्यमांवर लोकांशी वाद का घालायचा जर त्यांना आपण बाजारपेठेत आणून वाद घालू शकतो.

    \item ``Signaling virtue is a vice.'' गुण दाखवण्याचा प्रयत्न करणे हा एक दुर्गुण आहे.
\end{itemize}


\section*{Technology \& Progress}
\begin{itemize}
    \item ``Technology is not only the thing that moves the human race forward, but it’s the only thing that ever has.'' मानवजातीला पुढे नेणारी एक आणि एकमेव गोष्ट म्हणजे तंत्रज्ञान.

    \item ``The democratization of technology allows anyone to be a creator, entrepreneur, scientist. The future is brighter.'' तंत्रज्ञानाचे लोकशाहीकरण कोणालाही निर्माता, उद्योजक, शास्त्रज्ञ बनण्याची संधी देते. भविष्य उज्ज्वल आहे.

    \item ``The Internet allows you to scale any niche obsession.'' इंटरनेट तुम्हाला कोणत्याही खास आवडीला मोठे करण्याची संधी देते.
\end{itemize}

\section*{Miscellaneous Wisdom}
\begin{itemize}
    \item ``Don’t take yourself so seriously. You’re just a monkey with a plan.'' स्वतःला इतके गंभीरतेने घेऊ नका. तुम्ही फक्त एका योजना असलेलं माकड आहात.

    \item ``Love is given, not received.'' प्रेम हे दिले जाते, मिळवले जात नाही.

    \item ``Anger is a hot coal that you hold in your hand while waiting to throw it at someone else.'' राग म्हणजे दुसऱ्यावर फेकायचे जळते निखारे स्वतःच्या हातात ठेवणे.

    \item ``My 1 repeated learning in life: ‘There Are No Adults.’ Everyone’s making it up as they go along.'' माझ्या आयुष्यातला एक धडा: ‘इथे कोणीही प्रौढ नाही.’ सगळे आपापल्या पद्धतीने शिकत चालले आहेत.
\end{itemize}
