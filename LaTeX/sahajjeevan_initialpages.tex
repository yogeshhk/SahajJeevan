% Copyright page
\thispagestyle{empty}
% \null\vfill

\begin{center}
\includegraphics[width=0.2\linewidth,keepaspectratio]{YHK_Color_OutOfTheBox_tight} \\[1.5em]

\textbf{\Huge सहज जीवन}\\ [0.5em]
{\small(मराठीतील पहिले मुक्त-स्रोत, सार्वजनिक, सहकारी आणि अद्ययावत  पुस्तक)}\\[0.5em]
{\small(GitHub repository: https://github.com/yogeshhk/SahajJeevan)}\\[0.5em]

अनुवादक-लेखक: \textbf{{\large \ldots  आणि  डॉ. योगेश हरिभाऊ कुलकर्णी}}\\[1.5em]
\end{center}

\vspace{1.5em}

\begin{flushleft}

प्रकाशक: डॉ. योगेश हरिभाऊ कुलकर्णी (self-published at Notion Press)\\
पत्ता:  पाषाण ,  पुणे ८ \\
फोन:  +91 9890251406\\
ईमेल: yogeshkulkarni@yahoo.com\\[1.5em]

\vspace{0.5em}

प्रथम आवृत्ती: २०२५\\[2.5em]

% \includegraphics[width=0.3\linewidth,keepaspectratio]{sahajjeevan_isbn} \\ [0.5em]
% ISBN-13 ‏ : ‎ XXX-XXXXXXX\\[1.5em]

कॉपीराइट-मुक्त © २०२५ डॉ. योगेश हरिभाऊ कुलकर्णी\\[0.5em]

{\textit{सर्व हक्क सार्वजनिक. या पुस्तकाचा कोणताही भाग प्रकाशकाच्या लेखी परवानगीशिवाय कोणत्याही स्वरूपात पुनर्मुद्रित किंवा पुनर्प्रकाशित करता येईल.}}\\[1.5em]

{\large Legal Notice:}\\
{\textit{This entire work is uncopyrighted. No rights reserved. Any part of this publication may be reproduced, distributed, or transmitted in any form or by any means, including photocopying, recording, or other electronic or mechanical methods, without the prior written permission of the publisher.}}
\end{flushleft}
\vfill\null
\clearpage

\begin{dedication}
मुक्त-स्रोत (`ओपन-सोर्स') चळवळीस  समर्पित  
\end{dedication}

\clearpage

\chapter*{पुस्तकाविषयी}
हे पुस्तक प्रामुख्याने दोन भागात आहे.  पहिल्या भागात लीओ बाबाउटा यांच्या 'द एफर्टलेस लाईफ' या पुस्तकाचा स्वैर अनुवाद आहे.  हे मूळ पुस्तक जगभरातील लेखकांच्या-वाचकांच्या मदतीने सार्वजनिक पद्धतीने लिहिले गेले होते.  मूळ इंग्रजीतील पुस्तकप्रमाणेच  हे मराठी भाषांतरसुद्धा  कोणत्याही प्रकाशन हक्काधिकाराशिवाय (Uncopyrighted) उपलब्ध ठेवले आहे.  दुसऱ्या भागात पुस्तक स्वरूपात डॉ. योगेश हरिभाऊ कुलकर्णी यांच्यासह इतरांचे विचार आहेत.  हा दुसरा भाग ऑनलाईन वर अद्ययावत (live) असणार आहे.  म्हणजेच GitHub repository मध्ये इतर लेखक-वाचकांचे विचार सुद्धा सम्मिलीत केले जात आहेत. याचा सर्व प्रपंचाचा उद्देश मराठी वाचकांना ``सहज जीवन'' जगण्यासाठी मार्गदर्शन मिळावे हा आहे. 
