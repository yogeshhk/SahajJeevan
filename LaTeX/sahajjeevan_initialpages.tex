% Copyright page
\thispagestyle{empty}
% \null\vfill

\begin{center}
\includegraphics[width=0.2\linewidth,keepaspectratio]{YHK_Color_OutOfTheBox_tight} \\[1.5em]

\textbf{\Huge सहज जीवन}\\ [0.5em]
{\small(लीओ बाबाउटा यांच्या 'द एफर्टलेस लाईफ' या पुस्तकाचा स्वैर अनुवाद )}\\[0.5em]

लेखक: \textbf{{\large डॉ. योगेश हरिभाऊ कुलकर्णी}}\\[1.5em]
\end{center}

\vspace{1.5em}

\begin{flushleft}

प्रकाशक: डॉ. योगेश हरिभाऊ कुलकर्णी (self-published at Notion Press)\\
पत्ता:  पाषाण ,  पुणे ८ \\
फोन:  +91 9890251406\\
ईमेल: yogeshkulkarni@yahoo.com\\[1.5em]

\vspace{0.5em}

प्रथम आवृत्ती: २०२५\\[0.5em]

\includegraphics[width=0.3\linewidth,keepaspectratio]{sahajjeevan_isbn} \\ [0.5em]
ISBN-13 ‏ : ‎ XXX-XXXXXXX\\[1.5em]

कॉपीराइट-मुक्त © २०२५ डॉ. योगेश हरिभाऊ कुलकर्णी\\[0.5em]

{\textit{सर्व हक्क सार्वजनिक. या पुस्तकाचा कोणताही भाग प्रकाशकाच्या लेखी परवानगीशिवाय कोणत्याही स्वरूपात पुनर्मुद्रित किंवा पुनर्प्रकाशित करता येईल.  या पुस्तकात व्यक्त केलेली मते लेखकाची व्यक्तिगत नाहीत तर मूळ लेखकाची असून येथे फक्त अनुवादित केलेली आहेत.}}\\[1.5em]

{\large Legal Notice:}\\
{\textit{This entire work is uncopyrighted. No rights reserved. Any part of this publication may be reproduced, distributed, or transmitted in any form or by any means, including photocopying, recording, or other electronic or mechanical methods, without the prior written permission of the publisher.}}
\end{flushleft}
\vfill\null
\clearpage

\begin{dedication}
मुक्त-स्रोत (`ओपन-सोर्स') चळवळीस  समर्पित  
\end{dedication}

\clearpage

\chapter*{मुक्त-स्रोत अनुवादाविषयी}
लीओ बाबाउटा यांच्या 'द एफर्टलेस लाईफ' या पुस्तकाचा स्वैर अनुवाद आहे.  मूळ पुस्तक जगभरातील वाचकांच्या मदतीने इंग्रजीत सार्वजनिक पद्धतीने लिहिले गेले होते.  आता मी त्याचे मराठी भाषांतर केले आहे.   मूळ पुस्तकप्रमाणेच  हे भाषांतरसुद्धा  कोणत्याही हक्काधिकाराशिवाय (Uncopyrighted) उपलब्ध आहे.  
याचा उद्देश म्हणजे मराठी वाचकांना ``सहज जीवन'' जगण्यासाठी मार्गदर्शन मिळावे हा आहे.  हे एक संक्षिप्त पण अर्थपूर्ण मार्गदर्शक आहे.

% \chapter*{शीर्षकाविषयी }
% ``एआय-रूपे'' या शीर्षकात `रूपे' शब्दाचे तीन अर्थ सूचित होतात. एक म्हणजे `रूप' चे बहुवचन. एआय (AI, आर्टिफिशिअल इंटेलिजन्स, कृत्रिम बुद्धिमत्ता) ची अनेक रूपे. दुसरा अर्थ त्याचे प्रकट होणे, जसे ``--- रूपे प्रकट झाले'',  विविध क्षेत्रांत प्रकट होणे. तिसरा अर्थ `रूप'  शब्दाची सप्तमी विभक्ती, म्हणजे `रूपामध्ये'.  हे शीर्षक एआयमध्ये काय दडले आहे याचा उहापोहसुद्धा या पुस्तकात आहे हे सुचवत आहे .  