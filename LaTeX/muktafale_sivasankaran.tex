%%%%%%%%%%%%%%%%%%%%%%%%%%%%%%%%%%%%%%%%%%%%%%%%%%%%%%%%%%%%%%%%%%%%%%%%%%%
\chapter*{सी. शिवशंकरन}

सी. शिवशंकरन (एअरसेलचे माजी संस्थापक) हे भारतातील आरोग्य संवर्धन आणि दीर्घायुष्याच्या क्षेत्रातील एक अग्रगण्य आवाज आहेत. अनेक दशकांच्या अनुभवासह, ते विज्ञान आणि प्राचीन शहाणपणाचे मिश्रण करून चांगले आणि दीर्घ जगण्याचे व्यावहारिक, सिद्ध मार्ग सांगतात.

त्यांनी सांगितलेले काही महत्त्वाचे मुद्दे येथे आहेत:

\section*{रोगमुक्त दीर्घायुष्य शक्य आहे}

त्यांच्या मते, ९५ वर्षापर्यंत रोगाशिवाय जगणे हे स्वप्न नाही, तर अगदी शक्य आहे. येथे "रोगमुक्त" म्हणजे खोकला-सर्दी नसणे असे नाही, तर मधुमेह, कर्करोग, हृदयविकार यांसारखे गंभीर आजार नसणे. आणि हे सर्व दैनंदिन निवडींवर अवलंबून असते. आपण दररोज काय करतो, काय खातो, कसे राहतो - यावरच आपले भविष्यातील आरोग्य ठरते.

\section*{पवित्र त्रिकूट (होली ट्रिनिटी): खा, श्वास घे, झोप}

जर तुम्ही या तीन गोष्टी योग्य प्रकारे केल्यात, तर तुम्ही आधीच जिंकत आहात. हळू, लांब श्वास सोडणे म्हणजे तुमच्या नर्व्हस सिस्टमला शांत करणे. थंड, अंधाऱ्या खोलीत सात तास झोप घेणे म्हणजे तुमच्या शरीराला पूर्ण विश्रांती देणे. जागरूकपूर्वक खाणे म्हणजे फक्त काय खातो हे नाही, तर कधी खातो हेही महत्त्वाचे आहे.

\section*{कॉफी आणि कार्बोहायड्रेट्स, पुन्हा विचार करा}

काळी कॉफी (साखर नाही, दूध नाही) योग्य प्रकारे घेतली तर दीर्घायुष्यासाठी फायदेशीर असू शकते. आणि कार्बोहायड्रेट्सच्या बाबतीत, वेळ ही सर्व काही आहे. बहुतेक आहारपद्धती या गोष्टीला खूप सोपे बनवून सांगतात, पण प्रत्यक्षात हे अधिक जटिल आहे.

\section*{छोटे बायोहॅक्स महत्त्वाचे आहेत}

तुम्हाला महागडे तंत्रज्ञानाची गरज नाही. फक्त सकाळी तीस सेकंद सूर्यप्रकाश घेणे, आंघोळीचा शेवट थंड पाण्याने करणे, किंवा तुम्ही कशासाठी कृतज्ञ आहात ते लिहून ठेवणे - या सगळ्या छोट्या गोष्टींचा एकत्रित परिणाम मोठा होतो.

\section*{अंदाज लावू नका, तपासणी करा}

ते डेटावर जास्त भर देतात. रक्त तपासणी करा, काय कमतरता आहे (जसे व्हिटॅमिन डी, मॅग्नेशियम, ओमेगा-३) ते पहा, आणि त्याच आधारावर पूरक आहार घ्या, मग पुन्हा तपासणी करा. अंधारात तीर मारण्यापेक्षा, माहितीच्या आधारे निर्णय घेणे चांगले.

\section*{स्नायू = दीर्घायुष्याचा विमा}

फक्त पाच किलो स्नायू वाढवा, आणि तुमचे वृद्धत्व कसे येते यात मोठा फरक पडू शकतो. विशेषतः जेव्हा आपण ४० च्या पुढे जातो तेव्हा हे अत्यंत महत्त्वाचे आहे. स्नायू म्हणजे फक्त दिसण्यासाठी नाही, तर आरोग्यासाठी एक गुंतवणूक आहे.

\section*{तुमचे जीन्स हे नियती नाहीत}

आनुवंशिकता कदाचित बंदूक भरत असेल, पण तुमची जीवनशैली ट्रिगर दाबते. आणि जीवनशैली सुमारे ९३% परिणामांवर नियंत्रण ठेवते. म्हणजेच तुमच्या हातात सत्तर टक्के नियंत्रण आहे.

\section*{आरोग्य हे भारतासाठी काम करायला हवे}

आपल्याला अशे उपाय हवेत जे येथे काम करतात. आणि बरीच उत्तरे आपल्या स्वयंपाकघरातच आधीपासून आहेत - जसे की उडद डाळ (सालीसह), शेवगा, अंडी, रताळे, जांभळ्या रंगाच्या भाज्या. आपले आरोग्य संकट अटळ नाही, ते सोडवता येण्यासारखे आहे.

\section*{झोप ही एक महासत्ता आहे}

हा मुद्दा वारंवार येत राहिला. चांगली झोप सर्व काही प्रभावित करते. त्यांनी सांगितलेल्या काही व्यावहारिक टिप्स: झोपण्यापूर्वी कीवी फळ किंवा ग्लायसिन घेऊन पहा, तोंडाला टेप लावून झोपा (हो, खरोखरच!), आणि स्क्रीन लवकर बंद करा. झोप ही फालतू वेळाची बर्बादी नाही, तर तुमच्या आरोग्याची पायाभूत गरज आहे.