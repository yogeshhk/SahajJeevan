\chapter{योगदानकर्ते}

हे पुस्तक Leo Babauta यांनी \textbf{सार्वजनिकरीत्या} लिहिलं.  
जगभरातील वाचकांना आमंत्रित केलं गेलं की ते यामध्ये योगदान देतील —  
लेखन, संपादन, सूचना किंवा प्रोत्साहनाच्या रूपाने.  

या सहकार्यामुळेच हे पुस्तक अधिक समृद्ध झालं आहे.  
हे केवळ एका व्यक्तीचं काम नाही, तर अनेकांच्या सामूहिक प्रयत्नांचं फलित आहे.  

\section*{जगभरातील योगदानकर्ते}

\begin{itemize}
  \item ज्यांनी आपले विचार, लेखन किंवा संपादनासाठी वेळ दिला.  
  \item ज्यांनी लहानसहान दुरुस्त्या, शब्दप्रयोग आणि शैली सुधारली.  
  \item ज्यांनी या पुस्तकाच्या तत्त्वांचा स्वतःच्या आयुष्यात प्रयोग केला आणि अनुभव शेअर केले.  
  \item ज्यांनी या पुस्तकाचा मुक्त-स्रोत (open source) उपक्रम म्हणून प्रसार केला.  
\end{itemize}

\section*{कृतज्ञता}

प्रत्येक योगदानकर्त्याबद्दल मी आभारी आहे.  
तुमच्या मदतीशिवाय हे पुस्तक एवढं जिवंत, सहज आणि मुक्त होऊ शकलं नसतं.  

या पुस्तकाचं वैशिष्ट्य म्हणजे —  
ते एकाच लेखकाचं नाही, तर संपूर्ण जगाचं आहे.  

\section*{सततचा प्रवास}

"सहज जीवन" हा एक सतत बदलणारा प्रवास आहे.  
आज ज्या लोकांनी योगदान दिलं, त्यांना धन्यवाद.  
आणि उद्या जे लोक या विचारांना पुढे नेतील, त्यांचेही स्वागत आहे.  

या प्रवासात प्रत्येक वाचक एक सहप्रवासी आहे.  
