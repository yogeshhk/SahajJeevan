%%%%%%%%%%%%%%%%%%%%%%%%%%%%%%%%%%%%%%%%%%%%%%%%%%%%%%%%%%%%%%%%%%%%%%%%%%%
\chapter*{साधना}


\section*{सकाळची साधना (ईशा~४५ मिनिटे)}

सकाळच्या पहिल्या किरणांसोबत जीवनाची सुरुवात करण्याची ही पद्धत आहे. दिवसाची सुरुवात योग्य रीतीने केली तर संपूर्ण दिवस सुंदर जातो. ईशा योगाच्या पद्धतीनुसार हा क्रम ठरवलेला आहे जो साधारण ४५ मिनिटांत पूर्ण होतो.

सर्वप्रथम आसन करावेत (५ ते ७ मिनिटे). या आसनांमध्ये तितली आसन (बटरफ्लाय पोज), बाळ पाळणा आसन (बेबी क्रॅडल), आणि मांजर-गाय आसन (कॅट-काऊ) यांचा समावेश होतो. हे आसन शरीराला जागृत करून दिवसभराच्या कामकाजासाठी तयार करतात.

त्यानंतर प्राणायाम आणि बंध (१५ मिनिटे) करावेत. यामध्ये अनुलोम-विलोम दोन मिनिटे, ॐकार मंत्र २१ वेळा, जलद उथळ श्वास १२० वेळा, आणि तीन बंधांसह श्वास घेणे व तीन बंधांसह श्वास सोडणे यांचा समावेश आहे. हे श्वासक्रिया मनाला शांत करून ऊर्जा जागृत करतात.

शेवटी ध्यान अर्थात स्थिरता (६ ते ७ मिनिटे) करावे. या ध्यानात मन पूर्णपणे शांत होऊन स्वतःशी जुळवून घेते.

\section*{दिवसाची साधना}

शारीरिक आणि मुद्रा सक्रियता म्हणजे दिवसभराची ऊर्जा व्यवस्थित राखणे. जागे झाल्यावर लगेच कोमट पाणी प्यावे कारण हे पचनक्रिया सुरू करते आणि शरीरातील विषारी पदार्थ काढून टाकते. त्यानंतर प्राण लिंग मुद्रा ५ मिनिटे करावी जी जीवनशक्ती वाढवते. शक्तिशाली टाळी २ मिनिटे वाजवाव्यात जी रक्तसंचार वाढवून हाताच्या नाड्या उत्तेजित करते.

आत्मनियंत्रणासाठी (डिसिप्लिन) आठवड्यातून दोन दिवस डिजिटल आणि अन्नाचे उपवास करावेत. हे मनावर नियंत्रण मिळवून स्वतःशी जुळवून घेण्यास मदत करते.

मानसिक स्पष्टतेसाठी नवीनतम कृत्रिम बुद्धिमत्ता आणि संगणक प्रोग्रामिंग (एआय/कोड) चा अभ्यास करावा. हा अभ्यास एकाग्रतेने आणि हेतूने करावा जेणेकरून मनाची तीक्ष्णता वाढेल.

\section*{संध्याकाळची प्रवाह साधना (श्रीमत~६० ते ७० मिनिटे)}

संध्याकाळची वेळ आंतरिक शांतीसाठी वापरावी. श्रीमत पद्धतीनुसार हा क्रम साधारण ६० ते ७० मिनिटे चालतो.

प्राणायाम आणि अंतर्मुखी कार्याची सुरुवात भ्रामरी प्राणायामाने करावी (६ ते ७ मिनिटे). मधमाशीसारखा आवाज काढून मनाला एकाग्र करावे. त्यानंतर अंतर मौन (इनर सायलेन्स) २० मिनिटे करावे जेथे मन पूर्णपणे शांत राहून आंतरिक आवाजाला ऐकते. योग निद्रा ३० मिनिटे करावी जी खोल विश्रांती देऊन शरीर आणि मनाला पुनर्जीवित करते.

सात्त्विक संवर्धनासाठी बासरीचा सराव (लांब स्वर) १० मिनिटे करावा. संगीत हे मनाला सुंदर भावनांनी भरून देते. धोरण आणि अंतर्दृष्टी (स्ट्रॅटेजी/इनसाइट) पुस्तकांचे वाचन २० मिनिटे करावे जे जीवनाला दिशा देते.

\section*{रात्रीची शांतता}

रात्री झोपण्याआधी नाकपुड्यात आणि पायाच्या तळव्यात तूप लावावे. हे शरीराला पोषण देऊन झोपेची गुणवत्ता वाढवते. "ॐ श्री मात्रे नमः" या मंत्राचा जप हळूवारपणे झोप येईपर्यंत करावा.

सात तास खोल झोप घ्यावी. जर लवकर जागे झालात तर पोट आकुंचित करून राम नामाचे (रॅम नाम) जप ६३ वेळा करावे. हे संख्या विशेष महत्त्व असलेली आहे.

निष्क्रिय किंवा प्रतीक्षेच्या क्षणांमध्ये आनापानसती (अनापान-सती) करावी. हे म्हणजे श्वासाला मौनपणे आत-बाहेर येताना पाहणे. हा सराव कुठेही कधीही करता येतो.

\section*{मूळ संकल्प}

मी निवडतो: चांगली झोप, मजबूत संतुलित शरीर, स्पष्ट मन आणि गरिमामय उपस्थिति.

गुण: चांगली बुद्धी म्हणजे स्पष्टता (क्लॅरिटी), चांगली शक्ती म्हणजे बळ (स्ट्रेंथ), चांगले स्वरूप म्हणजे उपस्थिति (प्रेझन्स). हे तिन्ही गुण जीवनात असणे आवश्यक आहे.

आंतरिक मार्ग: शक्ती नंतर शांती नंतर ईश्वर. साक्षी म्हणजे अनासक्ती म्हणजे ईश्वर. साधना करणारा, संन्यासी जीवनशैली, आणि न्यूनतमवादी (मिनिमलिस्ट) दृष्टिकोन हा मार्ग आहे.

हा माझा अंतिम जन्म आहे. फक्त एकच गोष्ट पूर्णपणे करावी: योग. इंद्रियांच्या ओढाताणीत राहून जगावे. मनाचे विक्षेप स्वीकारावेत. समस्यांमध्ये स्थिर राहावे. नेहमी जागरूक राहावे.

या सर्व गोष्टी जीवनाला एक सुंदर आकार देतात आणि मनुष्याला त्याच्या खऱ्या स्वरूपाशी जोडतात.